Die theoretische Planung einer App weicht oft von der späteren praktischen Anwendung ab. Dies kann unterschiedliche Gründe haben, die hier zum Teil beleuchtet werden sollen.\\

\noindent
Bei der Entwicklung einer App werden Funktionen für gutmütige Benutzer implementiert. Auch wenn dieses Vorgehen unbewusst auftritt, ändert es nichts daran, dass es sich hierbei um eine naive Herangehensweise handelt. Die eingebauten Funktionen werden in der Regel so ausgelegt, dass sie bei sachgemäßer Benutzung problemlos funktionieren. Wie sie jedoch  in der Praxis angewendet werden, wird hierbei oft nicht bedacht. Die unsachgemäße Benutzung muss nicht mutwillig durch die Nutzer geschehen, kann jedoch durch einfache Bedienfehler Probleme verursachen. Beim Programmieren der Features sollte deshalb auf verschiedene Problemherde eingegangen werden.\\

\noindent
\hangindent1cm
\textbf{Falscheingaben:} Hierzu zählen freiwillige und unfreiwillige Falscheingaben. Unfreiwillige Angaben sind beispielsweise Tippfehler oder falsche Filmpräferenzierung. Ersteres wird teilweise bei der Eingabe überprüft, wie etwa die E-Mail-Adresse, die eine bestimmte Form aufweisen muss, und andere Angaben können später in den Einstellungen korrigiert werden. Falsche Filmbewertungen können in unserem Fall jedoch nicht rückgängig gemacht werden, sollten aber auf zwischenmenschlichem Wege nach einem Matching vom Nutzer gelöst werden können.\\
Unter freiwilligen Falscheingaben versteht man falsche Datenangaben, was bei einem falschen Namen zur Anonymisierung führt. Das Matchingverfahren wird über den angegebenen Wohnort gemacht, sodass durch einen Wohnortwechsel der potentielle Personenkreis geändert wird. Da StreamSwipe von einer hohen Nutzerdichte pro Wohnort profitiert, wirkt sich dies negativ auf den vom Nutzer verlassenen Wohnort aus. Es ist möglich den Nutzerstandort per GPS auszulesen, jedoch können so Nutzer in einem schwach besiedelten Gebiet nicht in die nächstgrößere Stadt wechseln und die Filterung bei der Matchberechnung wäre dann wesentlich feinmaschiger und rechenaufwändiger.\\

\noindent
\hangindent1cm
\textbf{Schließen\,der\,App:} Falls die App beispielsweise während einer Datenangaben geschlossen wird, können unerwartete Fehler auftreten. Die Auswirkungen  variieren,  abhängig davon, an welchem Zeitpunkt die App geschlossen wird. Im schlimmsten Fall sind die Benutzerdaten anschließend beschädigt und verursachen einen Absturz der App beim nächsten Start.\\

\noindent
\hangindent1cm
\textbf{Funktionenmissbrauch:} Ist der genaue Vorgang hinter einer Funktion bekannt, kann diese schnell missbraucht werden. Bei dem Bewertungsverfahren von StreamSwipe wird über eine große Anzahl von Filmen ein Präferenzprofil erstellt und eine Übereinstimmung mit anderen Profilen über einem gewissen Prozentsatz gesucht. Wird jedoch ein Film mit \glqq Superlike\grqq \, bewertet, so matcht das System alle Nutzer, die diesen Film ebenfalls mit \glqq Superlike\grqq \, bewertet haben. Außerdem fließt die Bewertung \glqq Superlike\grqq \, ebenfalls in die Erstellung des Präferenzprofils. Wenn also ein Nutzer ausschließlich Filme superliket, unabhängig ob ihm der Film gefällt, kann er eine sehr hohe Anzahl an Matches erreichen. Jedoch wird durch diese Art des Matchings nicht das ursprünglich geplante Ziel aus Kapitel \ref{sec:einleitung} erreicht und Nutzer, die die App ernsthaft für einen gelungenen Filmeabend benutzen, könnten sich betrogen fühlen und die App wieder deinstallieren.\\
Es ist bereits geplant die Anzahl der gespeicherten Superlikes pro Nutzer zu begrenzen, sodass ab einer gewissen Anzahl alte Superlikes gelöscht werden wenn Neue hinzukommen.  Nur die aktuell gespeicherten Superlikes werden auch zum Matching verwendet. Die Implementierung dieses Updates kann auch nach Abgabe dieser Dokumentation geschehen, da es in der Regel selbst nach der Veröffentlichung einer App eine Weile dauert bis solche Schlupflöcher gefunden werden.\\

\noindent
Die Praxis zeigt jedoch, dass auch bei guten anfänglichen Überlegungen nicht alle Probleme beseitigt werden können, da diese teilweise zu vielschichtig sind oder sich mit anderen gewollten Features überschneiden. Trotz umfangreicher Maßnahmenergreifung kann nie vorausgesagt werden wie sich etwas entwickelt. Uber wurde beispielsweise ursprünglich als Limousinenservice gegründet und ist heute eines der größten Taxiunternehmen weltweit. Deshalb sollte bei der Planung bedacht werden was die Nutzer eigentlich wollen und welche Funktionen in der Praxis wirklich genutzt werden. Bei einer Dating-App geht es darum möglichst einfach mit anderen Personen Kontakt aufzunehmen.  Das Swipen von Filmpostern sollte in den meisten Fällen kein Hindernis darstellen. Ein großer Prozentsatz der Nutzer einer Dating-App verfolgt bei der Benutzung ein sehr spezielles Ziel, auf das bei StreamSwipe keinen Fokus gelegt wurde. Jedoch existiert hierbei bereits zu Beginn des Matches zwischen den beteiligten Personen ein gemeinsames Thema und ein Filmabend ist für diese Absicht wahrscheinlich zielführender als manch anderer Vorwand. \\
Sicher lockt StreamSwipe auch Personen an, die gerne  neue Filme und Personen mit ähnlichem Geschmack kennenlernen würden. Hierfür bietet sich StreamSwipe optimal an, sobald Filmempfehlungen implementiert werden, mehr dazu in Kapitel \ref{sec:ausblick}.\\

\noindent
Ebenso relevant wie die bestehenden Bedürfnisse der Nutzer, sind die bisherigen Lösungen, die der Markt bietet. Hierzu müssen alle Apps betrachtet werden, die die selben Funktionen wie StreamSwipe anbieten. Schränkt man die Funktionen auf das wesentliche ein, sodass der Fokus auf dem Bewerten von Filmen ist wodurch neue Leute kennengelernt werden, ist der Markt noch absolut frei. Keine andere App bietet diese Funktionen an, jedoch existieren ähnliche Konzepte, die aber alle für Einzelpersonen oder bestehende Gruppen Filmvorschläge auf Basis der bewerteten Filme bieten. Das Matching mit neuen Leuten ist bisher noch nicht vertreten, was die perfekte Basis für einen Marktstart bietet.