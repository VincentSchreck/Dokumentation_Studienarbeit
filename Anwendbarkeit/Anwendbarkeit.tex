Die theoretische Planung einer App weicht oft von der späteren praktischen Anwendung ab. Dies kann unterschiedliche Gründe haben, die hier zum Teil beleuchtet werden sollen.\\

\noindent
Bei der Entwicklung einer App werden Funktionen für gutmütige Benutzer implementiert. Auch wenn dieses Vorgehen unbewusst auftritt, ändert es nichts daran, dass es sich hierbei um eine naive Herangehensweise handelt. Die eingebauten Funktionen werden in der Regel so ausgelegt, dass sie bei sachgemäßer Benutzung problemlos funktionieren. Wie sie jedoch  in der Praxis angewendet werden, wird hierbei nicht bedacht. Die unsachgemäße Benutzung muss nicht mutwillig durch die User geschehen, kann jedoch durch einfache Bedienfehler Probleme verursachen. Beim Programmieren der Features sollte deshalb auf verschiedene Problemherde eingegangen werden.\\

\noindent
\hangindent1cm
\textbf{Falscheingaben:} Hierzu zählen freiwillige und unfreiwillige Falscheingaben. Unfreiwillige Angaben sind beispielsweise Tippfehler oder falsche Filmpräferenzierung. Ersteres wird teilweise bei der Eingabe überprüft, wie etwa die E-Mail-Adresse, die eine bestimmte Form aufweisen muss, und andere Angaben können später in den Einstellungen korrigiert werden. Letzteres kann in unserem Fall nicht rückgängig gemacht werden, sollte aber auf zwischenmenschlichem Wege nach einem Matching vom User gelöst werden.\\
Unter freiwilligen Falscheingaben versteht man falsche Dateneingaben, was in unserem Fall zur Anonymisierung führt. Das Matchingverfahren wird über den angegebenen Wohnort gemacht, sodass durch einen Wohnortwechsel der potentielle Personenkreis geändert wird. Da StreamSwipe von einer hohen Userdichte pro Wohnort profitiert, wirkt sich dies negativ auf den vom User verlassenen Wohnort aus. Es ist möglich den Userstandort per GPS auszulesen, jedoch können so User in einem schwach besiedelten Gebiet nicht in die nächstgrößere Stadt wechseln und die Filterung bei der Matchberechnung wäre dann wesentlich feinmaschiger.\\

\noindent
\hangindent1cm
\textbf{Schließen der App:} Falls die App beispielsweise während einer Datenangaben geschlossen wird, können unerwartete Fehler auftreten. Die Auswirkungen  variieren,  abhängig davon, an welchem Zeitpunkt die App geschlossen wird. Im schlimmsten Fall sind die Benutzerdaten anschließend beschädigt und verursachen einen Absturz der App beim nächsten Start.\\

\noindent
\hangindent1cm
\textbf{Funktionenmissbrauch:} Ist der genaue Vorgang hinter einer Funktion bekannt, kann diese schnell missbraucht werden. Bei dem Bewertungsverfahren von StreamSwipe wird über eine große Anzahl von Filmen ein Präferenzprofil erstellt und eine Übereinstimmung mit anderen Profilen über einem gewissen Prozentsatz gesucht. Wird jedoch ein Film mit 'Superlike' bewertet, so matcht das System alle User, die diesen Film ebenfalls mit 'Superlike' bewertet haben. Außerdem fließt die Bewertung 'Superlike' ebenfalls in die Erstellung des Präferenzprofils. Wenn also ein User ausschließlich Filme superliket, unabhängig ob ihm der Film gefällt, kann er eine sehr hohe Anzahl an Matches erreichen. Jedoch wird durch diese Art des Matchings nicht das ursprünglich geplante Ziel aus Kapitel \ref{sec:einleitung} erreicht und User, die die App ernsthaft für einen gelungenen Filmeabend benutzen könnten sich betrogen fühlen und die App wieder deinstallieren.\\
Es ist bereits geplant die Anzahl der gespeicherten Superlikes pro User zu begrenzen, sodass ab einem gewissen Punkt alte Superlikes gelöscht werden wenn Neue hinzukommen.  Nur die aktuell gespeicherten Superlikes werden auch zum Matching verwendet. Die Implementierung dieses Updates wird jedoch erst nach Abgabe dieser Dokumentation geschehen, da es in der Regel selbst nach der Veröffentlichung einer App eine Weile dauert bis solche Schlupflöcher gefunden sind.\\

\noindent
Trotz umfangreicher Maßnahmenergreifung kann nie vorausgesagt werden wie sich etwas entwickelt. Uber wurde beispielsweise ursprünglich als Limousinenservice gegründet und ist heute eines der größten Taxiunternehmen weltweit. \\
Ein guter Ansatz so etwas einzuleiten 

- was will der user? was machen die user wirklich daraus\\
\hspace{1cm} - dating. alles im leben dreht sich ums Eine. wahrscheinlich bruachen wir einen größeren Kundenstamm an mädchen, also gezieltere werbung\\
\hspace{1cm} - viele matches $\rightarrow$ viele User und vor allem lokal hohe dichten. wahrscheinlich sinnvoll
\hspace{1cm} - neue filme kennenlernen $\rightarrow$  filmvorschläge\\
- wie sieht das technisch gesehen aus?\\
- wie sind wir auf dem markt vertreten?\\


Jedoch zeigt die Praxis, dass auch bei guten anfänglichen Überlegungen nicht alle Probleme beseitigt werden können, da diese teilweise zu vielschichtig sind oder sich mit anderen gewollten Features überschneiden.