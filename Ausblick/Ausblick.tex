Die Entwicklung und Programmierung der App war ein wichtiger und großer Schritt, jedoch ist er nicht der letzte. Für die nun kommende Zeit stehen bereits einige Aufgaben fest um das Projekt StreamSwipe aufrechtzuerhalten.\\
Wie bereits in Kapitel \ref{sec:probleme} erwähnt, konnten nicht alle gewünschten Features eingebaut werden. Sie waren während der Entwicklungsphase entweder von zu geringer Priorität, oder sind erst im späteren Stadium aufgekommen. Beispielsweise werden die Filmdaten bisher nur auf Englisch erhalten. In einer sonst deutschsprachigen App ist diese Umstellung jedoch schon in Planung. Es existieren deutsche Filmdatenbanken wie beispielsweise OFDb, aus denen die übersetzten Daten erhalten werden können. Diese ist leider nicht so umfangreich wie die bisher genutzte Datenbank, sodass dann von beiden parallel Daten erhalten und abgeglichen werden um die User Experience nicht zu mindern.  In diesem Zuge ist auch eine optionale In-App-Sprache geplant, die der Benutzer auswählen kann, da nun die Informationen auf Deutsch und auf Englisch vorliegen. Um die Sprache der App ändern zu können, müssen alle Textelemente durch Variablen ersetzt werden, die in einer zentralen Tabelle beschrieben werden. Für eine andere Sprache werden diese Variablen dann mit den übersetzten Formulierungen überschrieben. Ist die App angepasst und der Datenfluss der beiden Datenbanken synchronisiert, können auch mit weniger Aufwand auch andere Sprachen in das System aufgenommen werden.\\
Ein Update, bei dem Filmempfehlungen eingeführt werden, ist bereits geplant. Hierbei werden auf Basis der berechneten Filmpräferenz und den Überschneidungen innerhalb eines Matches eine Liste mit Filmvorschlägen generiert. Dieses Feature ist von großer Bedeutung wenn man bedenkt wie viel Zeit in die Suche eines geeigneten Filmes gesteckt wird.\\
Auch das Ressourcenproblem des Servers kann in absehbarer Zeit behoben werden. Da die Skripte auf dem Raspberry Pi bereits geschrieben sind, ist lediglich eine leistungsstärkere Hardware nötig um einen stabilen Server zu realisieren. Solange die Benutzerzahlen und somit auch die benötigte Leistung überwacht werden, kann frühzeitig ohne Datenverlust oder Performanceeinbußen auf ein größeres System umgestellt werden.\\
Ist die App dann für die Öffentlichkeit zugänglich, lohnt es sich ein Feedbacksystem zu nutzen, durch welches die User Lob und Kritik äußern können. Oft treten kleine Bugs nur in sehr speziellen Situationen auf, die während der Testphase nicht bedacht oder nicht realisiert werden. Der Wartungsaufwand nach Release einer Software wird oft im Verhältnis zum Entwicklungsaufwand als gering erachtet, sollte aber nicht unterschätzt werden. 
Werbung spielt einen entscheidenden Faktor in der Kundengewinnung. Mithilfe gezielter Produktplatzierungen, möglichst großflächiger Marketingstrategien und zufriedenen Bestandskunden kann innerhalb von wenigen Wochen ein beachtlicher Kundenstamm aufgebaut werden. Timing ist in diesem Fall sehr wichtig, da die angeworbenen Kunden möglichst gleichzeitig mit der App bekannt gemacht werden sollten. StreamSwipe basiert auf einem möglichst flächendeckenden Kundenkreis, da nur lokal gematcht wird und ein User ohne Matches nicht lange gehalten werden kann. Entsprechend ergibt es Sinn gezielte Werbung in einer lokalen Umgebung einzusetzen  und lieber eine hohe Benutzerdichte, als eine große Reichweite aufzubauen. Wird die App in diesem Umfeld genutzt und somit Matches generiert, verbreitet sie sich automatisch auch in umliegende Städte und beginnt so zu wachsen. Mit dem Benutzerradius sollte auch der Radius der geschalteten Werbung sukzessiv vergrößert werden.