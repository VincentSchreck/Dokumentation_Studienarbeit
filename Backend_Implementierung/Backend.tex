Unter dem Begriff "Backend" versteht man Komponenten eines digitalen Systems, die den Betrieb des Programms ermöglichen und vom Benutzer nicht ersichtlich sind. ...[TODO]
\newline
Eine Backend-Anwendung kann direkt mit dem Frontend interagieren und dessen Benutzerdienste unterstützen sowie die Schnittstellen mit allen erforderlichen Ressourcen anbieten.

\subsection{Datenbank}
\subsubsection{Bereitstellung der Datenbank}
Über die offizielle Seite der MongoDB-Hersteller wird das Community-Installationspaket heruntergeladen und ausgeführt. \footnote{MongoDB Download: \url{ https://www.mongodb.com/try/download/community}}
Ausserdem wird die Kommandozeilenanwendung MongDBShell\footnote{MongDBShell Download: \url{ https://www.mongodb.com/try/download/shell}} und die
graphische Benutzeroberfläche MongoDBCompass installiert\footnote{MongDBCompass  Download: \url{https://www.mongodb.com/try/download/compass}}. 

Nach erfolgreichem Initiieren eines Replica-Sets\footnote{Vergleiche Replica Set: \url{https://docs.mongodb.com/manual/tutorial/deploy-replica-set/}}
kann der mongod-Prozess unter Angabe des zu verwendenden Replica-Sets, des Ports und des Datenspeicherpfads gestartet werden. \newline

%\begin{lstlisting}[caption=Mongod-Aufruf, %label=lst:mongodcall]
%mongod --port 27017 --replSet rs0 --dbpath=”..\data\db0
%\end{lstlisting} %TODO

\noindent
Standardmäßig wird bei Installation von MongoDB eine Konfigurationsdatei erstellt, dessen Name und Verzeichnis anhängig vom benutzten Betriebssystem sind. Auf diese Datei wird bei fehlender Angabe im mongod-ausführendem Kommando zugegriffen\footnote{MongDB Konfiguration: \url{https://docs.mongodb.com/manual/reference/configuration-options/}}.

Sie ermöglicht das Konfigurieren der zu nutzen Storage-Engine. Dementsprechend wird als Storage Engine die WiredTiger-Engine eingestellt. 
Über die graphische Benutzeroberfläche MongoDBCompass wird nach erfolgreicher Verbindung mit der MongoDB-Datenbanksystem eine neue Datenbank hinzugefügt namens 'StreamSwipeDatabase'. Ihr werden die benötigten Collections hinzugefügt:

\begin{itemize}
\item movies
\item users
\item matches
\item swipes
\end{itemize}

\subsubsection{Importieren der Film- und Städtedaten}
Um Nutzer innerhalb einer Stadt miteinander matchen zu können, werden Städtenamendaten auf der Datenbank benötigt. Auf der Webseite "datenb"orse.net" wird eine Liste deutscher Städtenamen zur Verfügung gestellt\footnote{Städteverzeichnis: \url{https://www.datenborse.net/item/Liste\_von\_deutschen\_Staedtenamen\_.csv}}.
Die von uns genutzte Filmdatenbank TMDB bietet eine Liste der datenbankseitig vorhandenen Filme. Sie kann über die Filmdatenbankanbieter-API heruntergeladen werden\footnote{TMDB-API Daily File Exports: \url{https://developers.themoviedb.org/3/getting-started/daily-file-exports}}. 
Über MongoDBCompass werden die heruntergeladenen Städtenamen in die Datenbank als Collection 'cities' und die Filminformationen in die 'movies'-Collection importiert.

Die Serverzuständigkeiten lassen sich in folgende Aspekte zusammenfassen:
\begin{itemize}
\item Empfangen und Beantworten der Clientanfragen
\item Routing der Anfragen zu den entsprechenden Abhandlungsroutinen
\item Überprüfen der Identität des Clients
\item Kommunikation zur Datenbank für die persistente Speicherung der Zustände der Nutzer und ihrer Präferenzen, Swipes und Matches.
\item Matching-Algorithmus
\end{itemize} 

In den folgenden Unterkapitel wird auf die Einrichtung des Node.js-Webservers und der MongoDB-Datenbank, auf die Erstellung der sicheren Kommunikationsschnittstelle und auf die Implementierung der serverzuständigen Funktionalitäten eingegangen.

\subsection{Webserver}
Die Serverzuständigkeiten des StreamSwipe-Webservers lassen sich in folgende Aspekte zusammenfassen:
\begin{itemize}
\item Empfangen und Beantworten der Clientanfragen
\item Routing der Anfragen zu den entsprechenden Abhandlungsroutinen
\item Überprüfen der Identität des Clients
\item Kommunikation zur Datenbank für die persistente Speicherung der Zustände der Nutzer und ihrer Präferenzen, Swipes und Matches.
\item Matching-Algorithmus
\end{itemize} 

\noindent
In den folgenden Unterkapiteln wird auf die Einrichtung des Node.js-Webservers und der MongoDB-Datenbank, sowie auf die Erstellung der sicheren Kommunikationsschnittstelle und auf die Implementierung der serverzuständigen Funktionalitäten eingegangen. Der StreamSwipe-Webserver wird im weiteren Text in der Kurzform als Webserver bezeichnet. 

\subsubsection{Bereitstellung des Webservers}
Zunächst wird das Node.js-Installationspaket aus der offiziellen Seite der Hersteller heruntergeladen und ausgeführt, wie in Abbildung \ref{fig:installation_nodejs} dargestellt. Hierbei werden sowohl die Laufzeitumgebung für Node.JS, als auch der npm package manager installiert (siehe nächste Abbildung). 
Zusätzlich wird bei der Installation ausgewählt, dass Node.js sowie npm und dessen Module zu den Umgebungsvariablen hinzugefügt werden. Dabei werden Variablen unter ihrem Applikationsnamen gespeichert und ihre entsprechenden Datei-Pfade hinterlegt.
Über die Benutzung dieser Umgebungsvariablen ist ein schneller Zugriff über ein Terminal beziehungsweise einer anderen Applikation gewährleistet.


\begin{figure}[tbt]
\centering
\includegraphics[width=8cm]{images/nodejs_install.png}
\caption{Node.JS Installation}
\label{fig:installation_nodejs}
\end{figure}

\noindent
Nach der Installation von Node.js (siehe Abbildung \ref{fig:installation_nodejs}) kann das Projekt mithilfe des Befehls \glqq npm init\grqq \, im Terminal initialisiert werden. 
Hier werden nacheinander Input für relevante Projektaspekte wie dem Projektnamen, der Initialversion, der Startprogrammdatei oder dem GIT-Repository abgefragt.  
Im Anschluss wird im aktuellen Verzeichnis eine Datei \glqq package.json\grqq \, erstellt,  bei der es sich um eine Manifest-Datei im JSON-Format handelt, die unter anderem die benötigten Pakete sowie dessen Version, als auch projektspezifische Meta-Informationen wie den Projektnamen, die Projektversion, die Projektbeschreibung und den Autor enthält.
\newline
Im Anschluss an die Initialisierung werden die benötigten Pakete installiert. Dafür wird der Befehl \glqq npm install\grqq \, in Kombination mit dem angeforderten Modul genutzt. 
Nach der ersten Installation eines Moduls wird im Hauptverzeichnis des Projekts automatisch ein Ordner \glqq node\_modules\grqq \, erzeugt. Dieser enthält die Quelldaten der Node.js-Module. 
\newline
Da die Funktionalität, die nodemon bietet, nur in der Entwicklung benötigt wird, wird in der Datei \glqq package.json\grqq \, ein Entwicklungsskript \glqq devStart\grqq \, definiert. 
Skripte erlauben das automatische Starten von anderen Applikationen. Über \glqq npm run\grqq \, in Kombination mit dem auszuführenden Skript wird die Hauptapplikation über die Datei, die im package.json unter \glqq main \grqq \, hinterlegt ist, zusammen mit den Applikationen, welche im package.json unter dem entsprechenden Skript aufgezählt sind, gestartet.
\newline
Als Applikationsstartpunkt wird die Datei \glqq server.js\grqq \, erzeugt und im package.json unter main hinterlegt.\\

\begin{lstlisting}[caption=Datei package.json, label=lst:packagejson]
{
  "name": "StreamSwipeServer",
  "version": "1.0.0",
  "description": "Our Backend-Server for the StreamSwipe Mobile Application",
  "main": "server.js",
  "scripts": {
    "start": "",
    "devStart": "nodemon"
  },
  "author": "Robin Meckler, Vincent Schreck, Leon Gieringer",
  "license": "-",
  "dependencies": {
    "express": "^4.17.1",
    "firebase-admin": "^9.5.0",
    "mongoose": "^5.11.17",
    "node-cron": "^2.0.3",
    "dotenv": "^8.2.0"
  },
  "devDependencies":{
    "nodemon": "^2.0.7"
  }
}
\end{lstlisting}

\subsubsection{Geplante Architektur}
Die Software für den StreamSwipe-Server wird in Komponenten/Module aufgeteilt. Vorteile dieser Modularisierung sind, dass die einzelnen Module schneller verstanden und dementsprechend leichter überarbeitet werden können. Außerdem können Codeduplizierungen vermieden werden und Softwarekomponenten an verschiedenen Punkten im Code wiederverwendet werden. Abbildung \ref{fig:WebserverArchitektur} stellt die Architektur des Webservers dar. Die Route-Komponenten dienen als Empfangsschnittstelle für HTTPS-Anfragen. Die tatsächlichen Abhandlungsroutinen finden in den Controller-Komponenten statt, welche zum Zugriff auf die Datenbank auf die Service-Komponenten zugreifen und die HTTPS-Antworten zurückschicken.
\begin{figure}[tbt]
\centering

\includegraphics[width=13cm]{images/backendstruktur.PNG}
\caption{Webserver Architektur}
\label{fig:WebserverArchitektur}
\end{figure}

\subsubsection{Sichere Kommunikation}
\label{sec:SichereKommunikation}
Das http-Modul ermöglicht eine Kommunikation über das http-Protokoll.\\
 
\begin{lstlisting}[caption=Einfache Verbindung, label=lst:nodejs_easyconnection]
{
 const app = express();
 app.use(express.json()); 
 var httpServer = http.createServer(app);
 httpServer.listen(process.env.HTTP_PORT, () => 
 console.log("HTTP-Server started on " + process.env.HTTP_PORT));
}
\end{lstlisting}

\noindent
Dabei werden jedoch die Daten unverschlüsselt versendet. Um ausreichend Datenschutz zu gewährleisten, wird stattdessen das https-modul genutzt\footnote{Siehe Dokumentation: \url{https://nodejs.org/api/https.html}, letzter Zugriff: 24. April 2021}.
\newline
Benötigt für einen HTTPS-Server werden ein Sicherheitszertifikat und ein privater Schlüssel, die zunächst mithilfe des Tools OpenSSL erzeugt werden.  
Dabei ist zu beachten, dass während der Entwicklungsphase das Zertifikat nicht von einer zuständigen Zertfikatsstelle signiert wurde und somit von anderen Gegenstellen nicht akzeptiert wird.
\newline
In der Anwendung wird zunächst ein Objekt 'httpsOptions' erzeugt, das unter dem Attribut 'cert' das generierte Sicherheitszertifikat und unter dem Attribut 'key' den privaten Schlüssel enthält. Anschließend wird über die Funktion 'createServer' des https-Objekts der https-Server gestartet, woraufhin ein Objekt vom Typ https.Server zurückgegeben wird\footnote{Siehe Dokumentation:  \url{[https://nodejs.org/api/https.html\#https_class_https_server]}, letzter Zugriff: 24. April 2021}. 
Diesem Serverobjekt wird über seine Methode 'listen' aufgefordet, auf eingehende Nachrichten in dem als Parameter übergebenen Port einzugehen.\\

\begin{lstlisting}[caption=Gesicherte Verbindung, label=lst:nodejs_safeconnection]
{
 ...
 const https = require("https");
 const httpsOptions = {
  cert: fs.readFileSync('sslcert/server.crt', 'utf8'),
  key: fs.readFileSync('sslcert/server.key', 'utf8')
 }
 var httpsServer = https.createServer(httpsOptions, app);
 httpsServer.listen(process.env.HTTPS_PORT, () => {console.log("HTTPS - 	Server started on " + process.env.HTTPS_PORT)});
}
\end{lstlisting}


\subsubsection{Datenbankverbindung}
Wie bereits erwähnt, wird das Modul „mongoose“ für die Verbindung mit der MongoDB-Datenbank verwendet, wie im Beispiel \ref{lst:mongodbconnection} dargestellt.
Da der Quellcode in anderen Dateien hinterlegt ist, muss für den Zugriff auf dessen Funktionalitäten das entsprechende Modul zunächst inkludiert werden. Dazu wird die require-Methode aufgerufen, die ein Objekt zurückgibt, dass die aus dem Modul exportierten Methoden enthält und im Folgenden als Variabel mit dem Namen „mongoose“ gespeichert wird. 
\newline
Über die connect-Methode des zurückgelieferten Objekts wird nun bei Parameterübergabe der Datenbank-URL versucht, eine Verbindung aufzubauen.  
Dabei wird unter der Objekt-Membervariable  „connection“ ein Objekt vom Typ „Connection“ hinterlegt, über das bei erfolgreicher Verbindung mit der Datenbank kommuniziert werden kann und das nachfolgend unter der Variable „database“ abgespeichert ist.\\

\begin{lstlisting}[caption=Verbindung zur MongoDB-Datenbank, label=lst:mongodbconnection]
{
 const mongoose = require('mongoose');
 let database = null;

 async function startDatabase() {
  await mongoose.connect(process.env.DATABASE_URL, 
   {useNewUrlParser: true,
   useUnifiedTopology: true}); 
  database = mongoose.connection;
  database.on('error',(error) => console.log(error));
  	database.on('open',(error) => console.log('Connected to DB'))}

 async function getDatabase() {
 if (!mongoose.connection) await startDatabase();
  return database; }

 module.exports = {
  getDatabase,
  startDatabase }
}
\end{lstlisting}

%\subsection{Datenbank}
\subsubsection{Datenbankmodelle und Schemata}
Ein Model in Mongoose ist ein aus einer Schemadefinition erstellter Konstruktor, aus denen Objekte instanziiert werden können. Diese Instanzen stehen in direkter Verbindung zu den jeweiligen Collections der verbundenen Datenbank und enthalten Methoden für die persistente Speicherung, Bearbeitung oder Löschung.

%TODO Konzeptioneller Aufbau
\begin{figure}[tbt]
\centering
\includegraphics[width=\textwidth]{images/databasemodells.PNG}
\caption{MongoDB - Aufbau der Collections und Beziehungen}
\label{fig:databasemodells}
\end{figure}

\noindent
Folgender Code zeigt den Aufbau des Schemas für die Swipe-Collection. \\

\begin{lstlisting}[caption=Swipe Schema und Model, label=lst:modelswipe]
const mongoose = require('mongoose')

const swipeSchema = new mongoose.Schema({
 uid: {
  type: String,
  required: true
 },
 swipes :
 [{ movieid: { type: String },
    swipeaction: {type: Number}}]
 })

module.exports = mongoose.model('Swipe',swipeSchema)
\end{lstlisting}

\noindent
Die einzelnen Schemata wurden nach dem im Diagramm der Abbildung \ref{fig:databasemodells} dargestellten Aufbau für jede Collection in separaten Dateien unter dem Verzeichnis '/database/models' erstellt (siehe Abbildung \ref{fig:node_structure}). Jede Datei exportiert dabei das aus dem zugehörigen Schema erzeugten Model.

\begin{figure}[tbt]
\centering
\includegraphics[width=12cm]{images/modelsstruktur.PNG}
\caption{Node.js Server - Models Struktur}
\label{fig:node_structure}
\end{figure}


\subsubsection{Datenbankzugriff}                   
Für den Datenzugriff auf die Datenbank wurden zu jeder Collection Service-Module unter dem Verzeichnis '/services' erstellt, die entsprechenden Zugriff gewähren. Dafür wurden innerhalb der Service-Module die benötigten Zugriffsfunktionen implementiert (siehe Abbildung \ref{fig:node_service_structure}).

\begin{figure}[tbt]
\centering
\includegraphics[width=12cm]{images/serviceStruktur.PNG}
\caption{Node.js Server - Services Struktur}
\label{fig:node_service_structure}
\end{figure}


%
%              Movie
%                   Service
%

\paragraph{Movie Service}
Die Funktionen des Moduls movieService.js bieten innerhalb der Projektumgebung den Zugriff auf bestimmte Operationen, die auf die Collection movies angewandt werden. Das Funktionsspektrum begrenzt sich für diesen Service auf die Funktion 'FindMovieExcept', die in Listing \ref{lst:findmoviesexcept} zu sehen ist.\\

\noindent
\textbf{FindMoviesExcept:}
Diese Funktion erhält als Parameter 'excludedMovies' eine Liste von MovieID's und als Parameter 'amount' einen Integerwert.
Über die Find-Funktion des importierten Movie-Models wird eine über den Wert von 'amount' begrenzte Anzahl an Movie-Dokumenten ausgelesen, deren IDs nicht in der übergebenen 'excludedMovies'-Liste vorhanden sind. Für das Filtern wird die 'nin'-Operation verwendet\footnote{Siehe Dokumentation: \url{https://docs.mongodb.com/manual/reference/operator/query/nin/}, letzter Zugriff: 26. April 2021}.\\
Sollte die Datenbank bei der Datenauslese einen Fehler zurückgeben, wird dieser über den try-catch-Block gefangen und an im Aufrufstack liegende Funktion über das Schlüsselwort throw weitergeleitet.\\

\begin{lstlisting}[caption=movieService.js - FindMoviesExcept, label=lst:findmoviesexcept]
const Movie = require('../database/models/movie')

async function FindMoviesExcept(excludedMovies, amount) {
    var movies;

    try{ 
    movies = await Movie.find({ id: { $nin: excludedMovies }}).limit(amount); }
    }
    catch(err){throw err;}

    return movies;
}

module.exports.FindMoviesExcept = FindMoviesExcept;
\end{lstlisting}


%
%              User
%                   Service
%

\paragraph{User Service}
Die Funktionen des Moduls userService.js bieten innerhalb der Projektumgebung den Zugriff auf bestimmte Operationen, die auf die Collection User angewandt werden. Dieser Service import das User-Model.\\

\noindent
\textbf{CreateUser:}
Die in Listing \ref{lst:userservicecreateuser} dargestellte Funktion erhält sämtliche Eigenschaften, die im User-Schema beschrieben sind, als Parameter. Diese Funktion wird in der Projektumgebung in Zusammenhang mit mindestens einer weiteren datenbankzugreifenden Funktion aufgerufen. Im Sinne einer Transaktion müssen sie als atomare Operationen ausgeführt werden, um den Datenbestand konsistent zu halten. Daher wird ein Session-Objekt als Parameter mitgeliefert. Innerhalb der Funktion wird über die Create-Funktion des importierten User-Models ein neuer Eintrag in der User-Collection der Datenbank erstellt.\\

\begin{lstlisting}[caption=User Service - CreateUser, label=lst:userservicecreateuser]
async function CreateUser(uid, swipeid, matchid, city, malewanted, femalewanted, diversewanted, mygender, session) {
    try {
        return (await User.create([{
            _id: mongoose.Types.ObjectId(),
            uid: uid,
            _swipeid: swipeid,
            _matchid: matchid,

            city: city,
            malewanted: malewanted,
            femalewanted: femalewanted,
            diversewanted: diversewanted,
            mygender: mygender
        }
        ], { session: session }))[0];
    }
    catch (Exception) {
        throw Exception;
    }
}
\end{lstlisting}

\noindent
\textbf{CheckExistence:}
Wie in Listing \ref{lst:userservicecheckexistence} zu sehen prüft diese Funktion, ob innerhalb der User-Collection ein User mit entsprechendem Wert für die Eigenschaft 'uid', die als Parameter übergeben wird, existiert und gibt entsprechend den boolschen Wert 'true' bei Vorhandensein beziehungsweise 'false' bei Nicht-Vorhandensein zurück.\\

\begin{lstlisting}[caption=User Service - CheckExistence, label=lst:userservicecheckexistence]
async function CheckExistence(uid) {
    return await User.exists({ uid: uid });
}
\end{lstlisting}

\noindent
\textbf{GetCityFromUser:}
Diese Funktion gibt den Eintrag der Eigenschaft 'city' eines Dokuments zurück, dessen 'uid'-Attribut mit dem übergebenen 'uid'-Parameter übereinstimmt, wie in Listing \ref{lst:userservicecheckexistence} zu sehen ist.\\

\begin{lstlisting}[caption=User Service - CheckExistence, label=lst:userservicecheckexistence]
async function GetCityFromUser(uid) {
    try {
        var user = await User.findOne({ 'uid': uid });
        if (user) {
            return user.city;
        }
        else throw { message: "No user Found" + uid };
    }
    catch (err) { console.log(err); throw err; }
\end{lstlisting}

\noindent
\textbf{ChangeCityFromUser:}
Die in Listing \ref{lst:userservicechangecityfromuser} dargestellte Funktion führt ein Update auf einem User-Dokument aus, dessen 'uid'-Eigenschaft mit dem gleichnamigen übergebenem Parameter übereinstimmt. Dabei wird die 'city'-Eigenschaft innerhalb des User-Dokuments auf den Wert des gleichnamigen Parameters aktualisiert.\\

\begin{lstlisting}[caption=User Service - ChangeCityFromUser, label=lst:userservicechangecityfromuser]
async function ChangeCityFromUser(uid, city, session) {  
   try {
        if (CheckExistence(uid)) {
            await user.UpdateOne(
                { 'uid': uid },
                { city: city },
                { session: session });
        }
        else throw { message: "No User Found" + uid }
    }
    catch (err) { console.log(err); throw err; }
\end{lstlisting}

\noindent
\textbf{ChangeGenderWantedFromUser:}
Diese Funktion erfüllt die gleiche Funktionalität wie die ChangeCityFromUser-Funktion  mit dem Unterschied, dass statt der 'city'-Eigenschaft die Eigenschaften 'malewanted','femalewanted' und 'diverswanted' aktualisiert werden.\\

\noindent
\textbf{ChangeGenderFromUser:}
Die Funktion ChangeGenderFromUser erfüllt die gleiche Funktionalität wie die ChangeCityFromUser-Funktionalität mit dem Unterschied, dass statt der 'city'-Eigenschaft die 'mygender'-Eigenschaft aktualisiert wird.\\


%
%              Match
%                   Service
%


\paragraph{Match Service}
Die Funktionen des Moduls matchService.js bieten innerhalb der Projektumgebung den Zugriff auf bestimmte Operationen, die auf die Collection matches angewandt werden. Dieser Service import das Match-Model.

\noindent
\textbf{CreateUserMatchDocument:}
Die in Listing \ref{lst:matchserviceCreateUserMatchDocument} dargestellte Funktion überprüft zunächst, ob ein Match-Dokument mit übergebener 'uid'-Eigenschaft bereits existiert. Ist dies nicht der Fall, wird ein neues Match-Dokument in der matches-Collection erzeugt.\\

\begin{lstlisting}[caption=Match Service - CreateUserMatchDocument, label=lst:matchserviceCreateUserMatchDocument]
async function CreateUserMatchDocument(uid, session) {
    if (!(await Match.exists({ uid: uid }))) {
        try {
            return (await Match.create([{
                _id: mongoose.Types.ObjectId(),
                uid: uid,
                swipes: []}
            ], { session: session }))[0];
        }
        catch (Exception) {
            throw Exception;
        }
    }
    else 
      throw { message: "Match already exists for " + uid 		};
    
}
\end{lstlisting}

\noindent
\textbf{CheckMatchExists:}
Wie in Listing \ref{lst:matchserviceCreateUserMatchDocument} zu sehen, überprüft diese  Funktion ob ein Match-Dokument existiert, welches den übergebenen 'uid'-Eigenschaftswert hat sowie innerhalb seiner 'supermatches'- oder 'normalmatches'-Liste die übergebene 'matchedUid' enthält. Zurück wird ein entsprechender boolescher Wert geschickt. \\

\begin{lstlisting}[caption=Match Service - CreateUserMatchDocument, label=lst:matchserviceCreateUserMatchDocument]
async function CheckMatchExists(uid, matchedUid, session) {
    try {
        var match = await Match.findOne({ uid: uid, 'supermatches.uid': matchedUid }).session(session)

        if (match) { return true; }
        else {
            var match = await Match.findOne({ uid: uid, 'normalmatches.uid': matchedUid }).session(session)

            if (match) return true;
            else return false;
        }
    }
    catch (err) { console.log(err); throw err; }
}
\end{lstlisting}

\noindent
\textbf{AddNormalMatchToUser:}
Die Funktion AddNormalMatchToUser, welche in Listing \ref{lst:matchserviceAddNormalMatchToUser} zu sehen ist,  fügt der 'normalmatches'-Liste eines Match-Dokuments ein neues Objekt mit den Eigenschaften 'uid','matchUid' und 'movieid', die ihren Wert über die übergebenen Funktionsparameter erhalten. Des Weiteren wird das Attribut 'startedChat' und 'removed' jeweils mit dem booleschen Standardwert 'false' hinzugefügt. Außerdem wird die Eigenschaft 'newChanges' des Match-Dokuments auf true gesetzt.\\

\begin{lstlisting}[caption=Match Service - AddNormalMatchToUser, label=lst:matchserviceAddNormalMatchToUser]
async function AddNormalMatchToUser(uid, matchedUid, movieid, session) {
 try {
  if (Match.exists{ 'uid': uid }) {
   await Match.findOneAndUpdate(
    { 'uid': uid },
    { newChanges: true,
    $push: { normalmatches: { uid: matchedUid, 
                              movieid: movieid, 
                              startedChat: false, 
                              removed: false } }},
    { session: session });
    } else throw { message: "No Match Found" }; }
    catch (err) { console.log(err); throw err; }
}
\end{lstlisting}

\noindent
\textbf{AddSuperMatchToUser:}
Die Funktion unterscheidet sich von 'AddNormalMatchToUser' nur in dem Aspekt, dass ein neuer Eintrag in die 'supermatches'- statt der 'normalmatches'-Liste hinzugefügt wird.\\

\noindent
\textbf{GetMatches:}
Die Funktion empfängt eine 'uid' als Parameter und gibt ein entsprechendes Match-Dokument zurück, sofern es existiert.\\

\noindent
\textbf{SuperMatchMarkAsRemoved:}
Innerhalb der Funktion wird, wie in Listing \ref{lst:matchserviceSuperMatchMarkAsRemoved} dargestellt, anhand des übergebenen 'uid' und 'matchesUid' der Eintrag in der 'supermatches'-Liste angepasst. Dabei wird der Wert für 'removed' auf true gesetzt. Ausserdem wird 'newChanges' des betroffenen Match-Dokuments auf true gesetzt.\\

\begin{lstlisting}[caption=Match Service - SuperMatchMarkAsRemoved, label=lst:matchserviceSuperMatchMarkAsRemoved]
async function SuperMatchMarkAsRemoved(uid, matchUid, session) {
 try {
  var match = await Match.findOneAndUpdate({ uid: uid, "supermatches.uid": matchUid },
  { "$set": { newChanges: true,
             "supermatches.$.removed": true }},
            { session: session });
  if (!match) {
   throw { message: "No Match Found:" + 
           uid + "matching: " + matchUid };}
 } catch (err) { console.log(err); throw err; }
}
\end{lstlisting}

\noindent
\textbf{NormalMatchMarkAsRemoved:}
Die Funktion unterscheidet sich von 'SuperMatchMarkAsRemoved' nur in dem Aspekt, dass der  Eintrag in der 'supermatches'- statt der 'normalmatches'-Liste angepasst wird.\\

\noindent
\textbf{MatchesReceived:}
Anhand einer übergebenen 'uid' und 'matchUid' wird die Eigenschaft 'new"-Changes' eines entsprechenden Match-Dokuments in der Match-Collection auf den booleschen Wert false gesetzt. 



%
%              Swipe
%                   Service
%

\paragraph{Swipe Service}
Die Funktionen des Moduls swipeService.js bieten innerhalb der Projektumgebung den Zugriff auf bestimmte Operationen, die auf die Collection swipes angewandt werden. Dieser Service importiert das Swipe-Model.\\

\noindent
\textbf{CreateUserSwipeDocument:}
Die Funktion erfüllt die gleiche Funktionalität wie die 'Create\-UserMatchDocument'-Funktionalität des Match-Services mit dem Unterschied, dass anstelle eines Match-Dokuments ein Swipe-Dokument in der swipes-Collection erstellt wird.\\

\noindent
\textbf{AddSwipe:}
Die in Listing \ref{lst:swipeserviceaddswipe} dargestellte Funktion erstellt zunächst ein 'swipe'-Objekt mit den Eigenschaften 'movieid' und 'swipeaction' und entnimmt die Werte dafür aus den gleichnamigen übergebenen Parametern. 
Wenn ein Swipe-Dokument mit übergebener 'uid' existiert, wird überprüft ob das Dokument in der 'swipes'-Liste bereits ein Eintrag mit entsprechender movieid enthält. Ist dies der Fall, wird die 'swipeaction'-Eigenschaft auf den Wert des übergebenen gleichnamigen Parameters aktualisiert. Ansonsten wird der Liste das zu Beginn erstellte 'swipe'-Objekt hinzugefügt. Letzlich wird das Swipe-Dokument über die Save-Funktion des Save-Models gespeichert.\\

\begin{lstlisting}[caption=Swipe Service - AddSwipe, label=lst:swipeserviceaddswipe]
async function AddSwipe(uid, movieid, swipeaction) {
 var swipe = { movieid: movieid, swipeaction: swipeaction };
 var dbSwipe;
 if (Swipe.exists({ 'uid': uid })) {
  try {
   dbSwipe = await Swipe.findOne({ 'uid': uid });

   //Ueberpruefe Vorhandensein des Swipes
   var index = dbSwipe.swipes.findIndex(x => x.movieid === movieid);

   if (index >= 0) {
    if (dbSwipe.swipes[index].swipeaction != swipeaction)
      dbSwipe.swipes[index].swipeaction = swipeaction; }
    else { dbSwipe.swipes.push(swipe); }
    dbSwipe.save(); 
    } catch (err) { throw err; }
  } else { throw "No Swipe available for this uid " + uid;}
  dbSwipe.save();
  return swipe;
}
\end{lstlisting}

\noindent
\textbf{RequestSwipes:}
Die Funktion empfängt eine 'uid' als Parameter und gibt ein entsprechendes Match-Dokument zurück, sofern es existiert.\\

\noindent
\textbf{RequestSuperlikeSwipes:}
Innerhalb dieser in Listing \ref{lst:swipeserviceRequestSuperlikeSwipes} dargestellten Funktion kommt es zum Einsatz einer Aggregation. 
Dabei kommt es zum Einsatz mehrerer Pipeline-Operatoren.
% %TODO Link auf Tabelle 5 Aggregation Framework
Über den 'unwind'-Operator wird gesetzt, dass für jeden Listeneintrag innerhalb der 'swipes'-Liste neue Dokumente erzeugt werden, die in die nachfolgende Pipeline-Stufen weitergeleitet werden. Anhand des 'match'-Operators werden dann die neuen Dokumente gefiltert. Letztlich wird eine Liste von Swipes zurückgeschickt, die eine 'swipeaction' von 2 (repräsentativ für Superlike) aufweisen.\\

\begin{lstlisting}[caption=Swipe Service - RequestSuperlikeSwipes, label=lst:swipeserviceRequestSuperlikeSwipes]
async function RequestSuperlikeSwipes(uid) {
 var dbSwipe;
 if (Swipe.exists({ 'uid': uid })) {
  try {
   dbSwipe = await (await Swipe.aggregate([
    { $unwind: '$swipes' },
    { $match: { uid: uid,
                'swipes.swipeaction': 2 }},
    { $group: { _id: '$_id',
                swipes: { $push: { movieid: "$swipes.movieid",
                                  swipeaction: "$swipes.swipeaction" 
     }}}}]));
   if (dbSwipe.length > 0 && dbSwipe[0]) {
    return dbSwipe[0].swipes; }
   } catch (err) { throw err; } }
  else {throw { message: "Swipe uid not existing " + uid }; }
}
\end{lstlisting}

\noindent
\textbf{FindAllSwipedMoviesByUserID:}
Innerhalb dieser in Listing \ref{lst:swipeserviceFindAllSwipedMoviesByUserID} dargestellten Funktion werden für eine übergebene 'uid' sämtlich 'movieid's zurückgeben, die in der 'swipes'-Liste des entsprechenden Swipe-Dokuments vorhanden sind.\\

\begin{lstlisting}[caption=Swipe Service - FindAllSwipedMoviesByUserID, label=lst:swipeserviceFindAllSwipedMoviesByUserID]
var swipedMovieIDs = [];
    
if (Swipe.exists({"uid": uid)) {
 try { var dbSwipe = await FindOne(uid);
  await dbSwipe.swipes.forEach(x => swipedMovieIDs.push(x.movieid));
 } catch (err) { throw err; }}
return swipedMovieIDs;
\end{lstlisting}


%
%              City
%                   Service
%
\paragraph{City Service}
Die Funktionen des Moduls cityService.js bieten innerhalb der Projektumgebung den Zugriff auf bestimmte Operationen, die auf die Collection cities angewandt werden. Dieser Service import das Swipe-Model.\\

\noindent
\textbf{GetAllInhabitedCities:}
Diese Funktion wird im Matching-Algorithmus aufgerufen. Wie in Listing \ref{lst:cityServiceGetAllInhabitedCities} zu sehen liefert er sämtliche Städte, mit mindestens zwei Nutzern. \\

\begin{lstlisting}[caption=City Service - GetAllInhabitedCities, label=lst:cityServiceGetAllInhabitedCities]
async function GetAllInhabitedCities() {
    var cities;
    try{ cities = await City.find({ user : {$exists:true},$where:'this.user.length>1'}) }
    catch (err) { throw err; }
    return cities;
}
\end{lstlisting}

\noindent
\textbf{AddUserToCity:}
Diese Funktion fügt einem 'City'-Dokument eine 'uid' hinzu.\\

\noindent
\textbf{RemoveUserFromCity:}
Diese Funktion entfernt eine 'uid' aus einem 'City'-Dokument.


\subsubsection{Controller} 
Die Controller enthalten die Abhandlungsroutinen für die HTTPS-Anfragen (siehe Abbildung \ref{ControllerStruktur}). Die einzelnen Funktionen empfangen jeweils das Request- und das Response-Objekt der Anfrage. Das Füllen und Zurückschicken des Response-Objekts ist ebenfalls Aufgabe der Controller. Zum Zugriff auf die Datenbank greifen sie auf die Services zu.\\



\begin{figure}[tbt]
\centering
\includegraphics[width=12cm]{images/controllerStruktur.PNG}
\caption{Node.js Server - Controller Struktur}
\label{ControllerStruktur}
\end{figure}

\paragraph{Movie Controller}

Der Movie-Controller  aus Listing \ref{lst:movieController_js} nutzt den Movie-Service zum Zugriff auf die Datenbank. \\

\begin{lstlisting}[caption=movieController.js Imports und Funktionen, label=lst:movieController_js]
const SwipeService = require('../services/swipeService')
const MovieService = require('../services/movieService')
const FirebaseService = require('../services/firebaseService')

exports.RequestMovies = async function(req, res){
   ...
}
\end{lstlisting}

\noindent
Er enthält die Funktion 'RequestMovies', welche in engem Kontakt zum SwipeManager des Frontends steht und es Nutzern ermöglichen sollen, neue Filminformationen, die vom Nutzer noch nicht empfangen wurden, abzufragen.
\newline
Wie in Listing \ref{lst:controllerfirebaseauth} zu erkennen ist, erwartet die Funktion im 'body'-Objekt des als Parameter übergebenen Request-Objekts eine Eigenschaft 'uidtoken', dessen Wert eine aus Firebase generierte Token-Referenz zur eindeutigen Authentifizierung des Nutzers ist. Der Token wird an die 'GetUID'-Funktion des Firebase-Services weitergeleitet, die bei erfolgreicher Authentifizierung die entsprechende 'uid' zurückschickt. Bei einem aufgetretenen Fehler, wie beispielsweise einem ungültigen Token, wird das Response-Objekt mit dem Statuscode 401 sowie der aufgetretenen Fehlernachricht zurückgeschickt und die Funktion beendet. Der folgende Code kommt in weiteren Funktionen anderer Controller ebenfalls zum Einsatz, wenn eine Authentifizierung benötigen.

\begin{lstlisting}[caption=Controller Firebase-Authentifizierung, label=lst:controllerfirebaseauth]
    var uid; 
    const uidToken = req.body.uidtoken;
    try{ uid = await FirebaseService.GetUID(uidToken); }
    catch(Exception)
    { res.status(401).json({title: "TOKEN ERROR", message: Exception}); return; }
\end{lstlisting}

\noindent
Nach erfolgreicher Authentifizerung des Firebase-Tokens werden weitere Eigenschaftswerte aus dem Request-Body als Variabeln gespeichert. Erwartet wird ein Zahlenwert 'amount', der die Anzahl der abgefragten Filme darstellt. Über die lokale Funktion 'RestrictAmount' wird geprüft, dass die begrenzende Zahl den Wert 10 nicht übersteigt. Damit soll sichergestellt werden, dass die Datenbankabfrage mit den weit über 500.000 Filmen nicht ausgelastet wird. Des Weiteren werden in der Variable 'alreadyRequestedMovieIDs' eine Liste von   eindeutigen Identifizierern aus der Movie-Collection erwartet. Die Werte sollen jene Film-ID's wiederspiegeln, die bereits vom Nutzer abgefragt, aber noch nicht über einen Swipe-Request in der Datenbank hinterlegt wurden. Die vom Nutzer bereits getätigten Swipes werden über die 'FindAllSwipesByUserID'-Funktion des SwipeServices abgefragt. Zurück\-gegeben wird eine Liste von Film-ID's, die zusammen mit der Liste der 'alreadyRequestedMovieIDs' in die Variabel 'excludedMovieIDs' gespeichert werden.\\

\begin{lstlisting}[caption=MovieController - RequestMovie - Excluded Movies, label=lst:MovieControllerExcludedMovies]
    var amount = RestrictAmount(req.body.amount);
    var alreadyRequestedMovieIDs = req.body.alreadyRequestedMovieIDs;
    var excludedMovieIDs = [];
    var newMovies;
    
    // Frage bereits geswipete Filme ab
    try{ var swipedMovieIDs = await SwipeService.FindAllSwipesByUserID(uid) }
    catch(err){ res.status(400).json({message: err.message}); return; }
        excludedMovieIDs.push(...swipedMovieIDs); }

    // Speichere bereits abgefragte Filme ab
    if(alreadyRequestedMovieIDs !== undefined && alreadyRequestedMovieIDs != null)
    { await alreadyRequestedMovieIDs.forEach(element => excludedMovieIDs.push(element)); }
\end{lstlisting}

\noindent
Letzlich wird bei vorhandenen zu exkludierenden Filmen die Funktion 'FindMoviesExcept' des Movie-Services aus Listing \ref{lst:MovieControllerExcludedMovies}  aufgerufen.  Ist die Liste 'excludedMovieIDs' dagegen leer, so wird die 'Find\-ExactAmount'-Funktion aufgerufen. Bei Erfolg wird das Response-Objekt mit dem Statuscode 200 und den Movie-Dokumenten als JSON-Objekt im Body der Antwort zurückgeschickt.\\

\begin{lstlisting}[caption=MovieController - RequestMovie - Excluded Movies, label=lst:MovieControllerExcludedMovies]
    if(excludedMovieIDs.length > 0)
    {
        try{ newMovies = await MovieService.FindMoviesExcept(excludedMovieIDs,amount) } 
        catch(err){ res.status(400).json({message: err.message}); return; }
    } else {
        try{ newMovies = await MovieService.FindExactAmount(amount); } 
        catch(err){ res.status(400).json({message: err.message}); return; }
    }

    res.status(200).json(newMovies);
\end{lstlisting}




%
%			User	
%				Controller
%



\paragraph{User Controller}
Der User-Controller bietet Funktionen, die die users-Collection der Datenbank betreffen. Sie greift dafür auf das User-Service zu. Die folgenden Funktionen greifen teils auch auf andere Collections zu. Daher werden auch die entsprechenden weiteren Services importiert.\\

\noindent
\textbf{CreateUser:}
Die Funktion erstellt ein neues User-Dokument in der users-Collection. Dafür werden gleichnamige Eigenschaften des User-Models im Request-Body der eingehenden Anfrage erwartet. Nach erfolgreicher Authentifizierung des Firebase-Tokens und Überprüfung über die 'CheckExistence'-Funktion des User-Services, ob ein User-Dokument mit der gleichen 'uid' bereits existiert, wird für die weiteren Datenbankabfragen eine Transaktion gestartet.\\

\begin{lstlisting}[caption=UserController - Create User - Transaktionsstart, label=lst:UserControllertransaction]
    // 1. Starte Transaktion!
    const session = await mongoose.startSession();
    await session.startTransaction();
\end{lstlisting}

\noindent
Das dafür genutzte 'session'-Objekt wird in den weiteren Service-Funktionen übergeben. In Listing \ref{lst:UserControllerdocumentscreation} wird:
\begin{itemize}
\item ein Swipe-Dokument über die 'CreateUserSwipeDocument'-Funktion des Swipe-Services erstellt.
\item ein Match-Dokument über die 'CreateUserMatchDocument'-Funktion des Match-Services erstellt. 
\item ein User-Dokument über die 'CreateUser'-Funktion des User-Services mit entsprechender Parametrisierung erstellt.
\item die 'uid' der entsprechenen Stadt über die 'AddUserToCity'-Funktion des City-Services hinzugefügt.
 \end{itemize}
Nur wenn alle Operationen erfolgreich ausgeführt wurden, wird die Transaktion über die 'commit\-Transaction'-Methode ausgeführt. Damit soll sichergestellt sein, dass einzelne, zusammenhängende Dokumente und Informationen nur im Ganzen erstellt werden. Bei Misserfolg einer Operation wird die komplette Transaktion über die 'abortTransaction'-Methode abgebrochen.\\

\begin{lstlisting}[caption=UserController - Create User - Dokumente erstellen, label=lst:UserControllerdocumentscreation]
try {
        // 2. Erstelle SWIPE-Dokument
        var createdSwipe = await SwipeService.CreateUserSwipeDocument(uid, session);

        // 3. Erstelle MATCH-Dokument
        var createdMatch = await MatchService.CreateUserMatchDocument(uid, session);

        // 4. Erstelle USER-Dokument
        var createdUser = await UserService.CreateUser(uid, createdSwipe._id, createdMatch._id, city, malewanted, femalewanted, diversewanted, mygender, session);

        // 5. Fuege User zu City hinzu
        if (createdUser._id)
            await CityService.AddUserToCity(uid,city,session);
            
        // Transaktion erfolgreich abschliessen
        await session.commitTransaction();
        res.status(201).json();
    }
catch (Exception) {
        res.status(501).json({
         title: "Server-User Creation Error", message: Exception });
        // Fehler => Transaktion abbrechen
        await session.abortTransaction(); }
        
session.endSession();
\end{lstlisting}
   
\noindent
\textbf{ChangeUser:}
Diese Funktion erlaubt einem Nutzer, seine Eigenschaften innerhalb der Datenbank zu aktualisieren. Die Schritte sind in Listing \ref{lst:UserControlleruserchange} zu sehen. Dafür wird nach erfolgreicher Authentifizierung eine Transaktion gestartet. Im folgendem Try-Block werden die einzelnen Operationen dargestellt, die für eine erfolgreiche Transaktion ausgeführt werden. Über das User-Service werden die Methoden 'ChangeGenderWantedFromUser' und 'ChangeGenderFromUser' aufgerufen. Anschließend muss der Stadteintrag angepasst werden, welcher an  mehreren Stellen in der Datenbank geändert werden muss. 
So muss zunächst die 'uid' aus dem alten 'city'-Dokument entfernt (Schritt 3 und 4) und dem  Dokument hinzugefügt werden, dass dem aktualisierten Stadtwert entspricht (Schritt 5). Letzlich wird der Wert der 'city'-Eigenschaft über die 'ChangeCityFromUser' des User-Services angepasst.\\

\begin{lstlisting}[caption=UserController - Change User, label=lst:UserControlleruserchange]
//1. Aenderung an GenderWanted
await UserService.ChangeGenderWantedFromUser(uid, malewanted,femalewanted,diversewanted,session);

//2. Aenderung an Gender
await UserService.ChangeGenderFromUser(uid, mygender, session);

//3. Frage vorherige Stadt ab
var oldCity = await UserService.GetCityFromUser(uid,session);

//4. Loesche Nutzer aus vorheriger Stadt
await CityService.RemoveUserFromCity(uid, oldCity, session);

//5. Fuege Nutzer zu neuer Stadt hinzu
await CityService.AddUserToCity(uid, newCity,session);

//6. Aktualisiere den Stadteintrag beim Nutzer
await UserService.ChangeCityFromUser(uid, newCity,session);

await session.commitTransaction();
res.status(200).json();
\end{lstlisting}
  
\noindent
\textbf{InfoUser:}     
Diese Funktion dient dazu, nach erfolgreicher Authentifizierung die gespeicherten Eigenschaft und ihre Werte des abgefragten User-Dokuments zu erhalten. Dafür wird die 'GetInfoFromUser'-Methode des User-Services aufgerufen.
        

%
%			Match	
%				Controller
%


\paragraph{Match Controller}
Der Match-Controller bietet Funktionen, die die matches-Collection der Datenbank betreffen. Sie greift dafür vorrangig auf den Match-Service zu.\\

\noindent
\textbf{RequestMatches:} 
Das Frontend erlaubt es, Matches anzeigen zu lassen. Dafür bietet die in Listing \ref{lst:matchcontrollerrequestmatches} dargestellte Funktion Informationen über das Match-Dokument des jeweiligen Nutzers. Nach erfolgreicher Authentifizierung wird das zugehörige Match-Dokument über die 'GetMatches'-Methode abgefragt. Das Dokument enthält zwei Listen: supermatches und normalmatches. Beide enthalten jeweils eine Eigenschaft 'removed'. Ist diese auf true gesetzt, so soll impliziert werden, dass der Nutzer das Match entfernt hat. Folglich soll das gelöschte Match nicht mehr angezeigt werden. Daher werden die beiden Listen über die 'filter'-Funktion nach den nicht entfernten Matches gefiltert. Bei Erfolg wird ein JSON-Objekt mit beiden Listen und der Eigenschaft 'newChanges' aus dem Match-Dokument zurückgesendet.\\
 
\begin{lstlisting}[caption=MatchController - RequestMatches, label=lst:matchcontrollerrequestmatches]
var match = await MatchService.GetMatches(uid);
var filteredSupermatches =  match.supermatches.filter(match => match.removed == false)
var filteredNormalmatches =  match.normalmatches.filter(match => match.removed == false)
res.status(200).json({ newChanges: match.newChanges,
            supermatches: filteredSupermatches, 
            normalmatches: filteredNormalmatches} );
\end{lstlisting}

\noindent
\textbf{DeleteSupermatch:} 
Hier wird die 'SuperMatchMarkAsRemoved'-Methode des Match-Services aufgerufen, um die 'removed'-Eigenschaft des entsprechenden Supermatches auf true zu setzen. Dieser Supermatch wird folglich nicht mehr über die 'RequestMatches'-Funktion zurückgegeben.\\

\noindent
\textbf{DeleteNormalmatch:} 
Gleiches Prinzip wie 'DeleteSupermatch' mit Normalmatches.\\

\noindent
\textbf{Received:} 
Hier wird die 'newChanges'-Eigenschaft auf false gesetzt. Es wird impliziert, dass der Nutzer den aktuellsten Stand der Matches hat.\\

\noindent
\textbf{Trigger:} 
Diese Funktion ruft MatchManager.startMatching() auf. Sie ist vorerst nur für die Entwicklung gedacht, und soll es ermöglichen, über das Frontend den Matching-Algorithmus im Backend zu starten.\\

%
%			Swipe	
%				Controller
%


\paragraph{Swipe Controller}
Der Swipe-Controller bietet eine Funktion, die die swipes-Collection der Datenbank betrifft. Hierfür greift sie auf den Swipe-Service zu.
Sie bietet lediglich die Funktion \textbf{CreateSwipe} an. Sie ruft die SwipeService.AddSwipeToDB-Methode auf.



\subsubsection{Routing} 
%\begin{figure}[h]
%\centering
%\includegraphics[width=8cm]{images/routestruktur.PNG}
%\caption{Node.js Server - Controller Struktur}
%\end{figure}
In der Server.js werden dem Express-Objekt 'app' die einzelnen Routen für die Weiterleitung der HTTPS-Anfragen an die entsprechenden Controller hinzugefügt. Dabei werden die zu den Anfragen gehörenden Request- und Response-Objekte als Parameter an die Controller übergeben.

\begin{lstlisting}[caption=Routing in server.js, label=lst:routingserver]
//Movies
const moviesRouter = require('./routes/movies')
app.use('/movies', moviesRouter)

//Users
const usersRouter = require('./routes/users')
app.use('/users', usersRouter)

//Matches
const matchesRouter = require('./routes/matches')
app.use('/matches', matchesRouter)

//Swipes
const swipesRouter = require('./routes/swipes')
app.use('/swipes', swipesRouter)

\end{lstlisting}

\paragraph{Movie Router}
Innerhalb des Movie Routers wird die '/movies/request' an die Funktion RequestMovie des Movie-Controller weitergeleitet.
\begin{lstlisting}[caption=Routing in movieRouter.js, label=lst:routingmovie]
// Importiere benoetigtes Controllermodul.
const MovieController = require('../controllers/movieController')

// Leite Anfragen weiter an MovieController
router.post('/request', MovieController.RequestMovies)
\end{lstlisting}

\paragraph{User Router}
Der User Router leitet '/users/create', '/users/change' und '/users/info' an die entsprechenden Funktionen des User-Controllers weiter.
\begin{lstlisting}[caption=Routing in userRouter.js, label=lst:routinguser]
// Importiere benoetigtes Controllermodul.
const UserController = require('../controllers/userController')

// Leite Anfragen weiter an UserController
router.post('/create', UserController.CreateUser)
router.post('/change', UserController.ChangeUser)
router.post('/info', UserController.InfoUser)
\end{lstlisting}

\paragraph{Match Router}
Innerhalb des Match Routers werden die unten dargestellten URL's an die Funktion des Match-Controller weitergeleitet.

\begin{lstlisting}[caption=Routing in matchRouter.js, label=lst:routingmatch]
// Importiere benoetigtes Controllermodul.
const MatchController = require('../controllers/matchController')

// Leite Anfragen weiter an MatchController
router.post('/request', MatchController.RequestMatches)
router.post('/deleteSupermatch',MatchController.DeleteSupermatch)
router.post('/deleteNormalmatch',MatchController.DeleteNormalmatch)
router.post('/received', MatchController.Received)
router.post('/trigger', MatchController.Trigger)
\end{lstlisting}


\paragraph{Swipe Router}
Der Swipe Router leitet '/swipes/create' an die entsprechende Funktionen des Swipe-Controllers weiter.

\begin{lstlisting}[caption=Routing in swipeRouter.js, label=lst:routingswipe]
// Importiere benoetigtes Controllermodul.
const SwipeController = require('../controllers/swipeController')

// Leite Anfragen weiter an SwipeController
router.post('/create',  SwipeController.CreateSwipe )
\end{lstlisting}










\subsubsection{Weitere Backendfunktionalit"aten} 
Nachfolgend werden weitere Systemfunktionalitäten des Backends dargestellt.

\paragraph{Firebase-Service}
Um unberechtigte Zugriffe zu vermeiden, findet für die nutzerbezogenenen Anfragen eine Authentifizierung statt. Die erste Authentifizierung des Nutzers findet über den Login des Frontends in Firebase statt. Nachträglich muss bei Anfragen an den Webserver sichergestellt werden, dass der Nutzer weiterhin authentifiziert ist. Ohne diesen Vorgang könnte man sich über das Schicken einer willkürlich übermittelten Nutzer-Uid fälschlicherweise als anderer Nutzer ausgeben. Für den Zugriff auf die Firebase-Authentifizierungsfunktionen wird in die firebaseService.js-Datei das Modul 'firebase-Admin' importiert \footnote{Siehe Dokumentation: \url{https://firebase.google.com/docs/admin/setup}, letzter Zugriff: 3. April 2021}.\\

\noindent
%\hangindent1cm
\textbf{Register:}
Um die Anwendung bei Firebase zu registrieren, wird die Funktion 'initializeApp' des Firebase-Moduls ausgeführt. 
Ein aus Firebase generierter Authentifizierungsschlüssel wird dabei für den Zugriff auf die StreamSwipe-Umgebung mitübergeben.
Die Register-Methode wird anschließend nach außen exportiert und zu Beginn des Serverstarts in der Server.js-Datei ausgeführt 
\footnote{Siehe Dokumentation: \url{https://firebase.google.com/docs/admin/setup\#initialize-without-parameters}, letzter Zugriff: 3. April 2021}.\\
   
\begin{lstlisting}[caption=Firebase-Service Register, label=lst:firebaseService Register]
var serviceAccount = require("../sslcert/streamswipe-firebase-adminsdk-uiyci-80bc08a5b2.json");
firebaseAdmin.initializeApp({
      credential: admin.credential.cert(serviceAccount),
        databaseURL: "https://streamswipe.firebaseio.com"
    });
\end{lstlisting}

\noindent
\hangindent1cm
\textbf{UID/TokenID-Dictionary:}
Um Zugriffszeiten auf die Firebase-Schnittstelle, werden in einem lokalen Dictionary aus Schlüsselwertpaaren der Zusammenhang zwischen TokenID und den UID samt ihrem Ablaufsdatum zwischen\-gespeichert.\\

\noindent
%\hangindent1cm
\textbf{GetUID:}
Die Funktion erwartet einen Firebase Token als Parameter 'uidtoken', welcher an die Funktion 'verifyIdToken' des FirebaseAdmin-Objeekts weitergeleitet wird. Zurück wird ein Objekt gegeben, dass unter anderem die 'uid' des zum Token zugehörigen Nutzers und die Ablaufzeit schickt. Nach erfolgreichem Überprüfen, ob die 'uid' tatsächlich ein Wert übermittelt bekommen hat, wird das Paar aus UidToken und Uid samt Ablaufzeit in der UID/TokenID-Dictionary gespeichert.\\

\begin{lstlisting}[caption=Firebase-Service Register, label=lst:firebaseServiceRegister]
verifiedUid = await firebaseAdmin.auth().verifyIdToken(uidToken);
uid = verifiedUid.uid;
expireTime = verifiedUid.exp;
if(uid === undefined || uid == null) { throw {message: "No uid returned!"}; }
TokenIDDict[uidToken] = {uid,new Date(expireTime*1000)};
return uid;
... //Ende Try-Catch-Block
\end{lstlisting}
   
\noindent
%\hangindent1cm
\textbf{RefreshList:}
Diese Funktion wird aufgerufen, um abgelaufene Token in der UID/TokenID-Dictionary zu löschen. Sie wird über das Modul TimedEvents periodisch aufgerufen. Dabei wird zu jedem Paar die aktuelle Uhrzeit und die Ablaufszeit verglichen. Stellt die Ablaufszeit ein größeren Wert dar, wird das Schlüsselwertpaar aus der Dictionary entfernt.\\

\paragraph{Timed Events}
Über das 'node-cron'-Modul \footnote{Siehe Dokumentation: \url{//https://www.npmjs.com/package/node-cron}, letzter Zugriff: 26. April 2021}
können zeitlich definierte und periodische Funktionen ausgeführt werden. Dafür wird das 'node-cron'-Modul in die TimedEvents.js-Datei importiert. Die Funktionalität des Moduls wird beispielsweise für das periodische Aktualisieren der movies-Collection, das periodische Ausführen des Matching-Algorithmus und das Aufrufen der 'firebaseService.RefreshList'-Funktion zum Aktualisieren der UID/TokenID-Dictionary verwendet.\\

\paragraph{MatchManager}
\noindent
%\hangindent1cm
\textbf{StartMatching - Teil 1:}
Die aktuelle Implementierung des Matching-Algorithmus sucht für jede Stadt Nutzerpaare, die einen gleichen Film mit einem Superlike versehen haben. Dafür werden zunächst über die 'GetAllInhabitedCities'-Funktion des City-Services die Städte in einer Liste gespeichert, die mindestens zwei Nutzer aufweisen. Folglich finden eine Verschachtelung von Iterationsabläufen zum Ausführen von Programmcode auf jedem Element einer Liste statt.
In der ersten Iterationsstufe wird durch die Städte iteriert.
Die zweite Iterationsstufe vom ersten bis zum vorletzten Nutzereintrag ('uid') innerhalb der aktuell iterierten Stadt. Innerhalb des zugehörigen Codeblocks wird die 'RequestSuperlikeSwipes'-Funktion des SwipeServices aufgerufen mit der 'uid' des aktuell iterierten Nutzers als Parameterübergabe. Die zurückerhaltene Liste enthält sämtliche Film-ID's, die vom Nutzer mit einem Superlike versehen wurden. Die Liste wird samt dem Index des nächsten Users in der Liste, der aktuell iterierten Stadt und dem aktuell iterierten User. Ausserdem wird eine Referenz auf das 'foundSupermatches'-Objekt, dass später mit Informationen zu den errechneten Supermatches gefüllt wird, mitgegeben.\\

\begin{lstlisting}[caption=Match Manager - startMatching - Teil 1: Finde Matches, label=lst:findMatches]
var cities = await CityService.GetAllInhabitedCities();

// City - 1. Iterationsstufe
for (let cityIterator = 0; cityIterator < cities.length; cityIterator++) {
	// Fuer jeden Nutzer ausser den letzten
	var prelastSupermatchIndex = cities[cityIterator].user.length - 2;
           
    // User - 2. Iterationsstufe
    for (let user1Iterator = 0; user1Iterator <= prelastSupermatchIndex; user1Iterator++) {
    	var matchingUserNextIndex = user1Iterator + 1;
    	var User1 = cities[cityIterator].user[user1Iterator];
      
    	//SUPERMATCH-CHECK
    	var user1SuperlikedMovies = await SwipeService.RequestSuperlikeSwipes(User1.uid);
    	if (user1Superlikes) {
    	// Ueberpruefe Superlikes mit den nachfolgenden Usern
    	await checkSuperMatches(matchingUserNextIndex, cities[cityIterator], user1SuperlikedMovies, foundSupermatches, User1);
                }

    	//NORMALMATCH-CHECK
    	var user1swipes = await SwipeService.RequestSwipes(User1.uid);
    	if (user1swipes) {
    	await checkNormalMatches(matchingUserNextIndex, cities[cityIterator], user1swipes, normalmatches, User1);
    }}}}
    ... //end Try-Catch-Block
\end{lstlisting}

\noindent
%\hangindent1cm
\textbf{checkSuperMatches:} Die Funktion wird im Matching-Algorithmus von der 'findMatches'-Methode aufgerufen. Es wird ausgehened vom übergebenen 'matchingUserNextIndex', welches der Index des nächsten Users nach dem aktuell iterierten Users der 2. Iterationsstufe ist, durch die restliche Nutzerliste der übergebenen Stadt 'currentCity' iteriert. Dies entspricht der 3. Iterationsstufe.
Hierbei wird zunächst die lokale Funktion 'checkGenderPreference' aufgerufen, die anhand der übergebenen Nutzerobjekte prüft, ob jeweils das Geschlecht des einen Nutzers und das für das Matchen preferierte Geschlecht des anderen Nutzers übereinstimmen. Ist dies der Fall, so werden die Superlikes des zweiten Users angefragt. Abschließend werden die Superlikes-Listen beider Nutzer verglichen. Dabei wird überprüft, ob eines der MovieID's der einen Liste in der anderen vorhanden ist. Bei einem Treffer wird ein Objekt mit den 'uid' der bei Nutzer und die 'movieid' des zugehörigen Films in die 'supermatches'-Liste hinzugefügt. \\

\begin{lstlisting}[caption=Match Manager - checkSuperMatches, label=lst:checkSuperMatches]
async function checkSuperMatches(matchingUserNextIndex, currentCity, user1Superlikes, foundSupermatches, User1) {

 for (let user2Iterator = matchingUserNextIndex; user2Iterator <= currentCity.user.length - 1; user2Iterator++) {
   var User2 = currentCity.user[user2Iterator];
   
  // Ueberpruefe GenderPreferenz
  if (!(await checkGenderPreference(User1, User2))) return;
  
  // Frage Superlikes ab
  var user2Superlikes = await SwipeService.RequestSuperlikeSwipes(User2.uid);
  if (user2Superlikes) {
        
  //Vergleiche beide
   for (let superlikeIterator = 0; superlikeIterator < user1Superlikes.length; superlikeIterator++) {
    for (let superlike2Iterator = 0; superlike2Iterator < user2Superlikes.length; superlike2Iterator++) {
    
      if (user1Superlikes[superlikeIterator].movieid === user2Superlikes[superlike2Iterator].movieid) {
        foundSupermatches.push({ matchid1: User1.uid,
                            matchid2: User2.uid, 
                            movieid: user1Superlikes
                            [superlikeIterator].movieid });
                            continue;
}}}}}}
\end{lstlisting}

\noindent
%\hangindent1cm
\textbf{checkNormalMatches:}
Diese Funktion ist zum Zeitpunkt der Dokumentation nicht implementiert \\

\noindent
%\hangindent1cm
\textbf{StartMatching - Teil 2:}
Die 'startMatching'-Funktion wird beendet, nachdem die gefundenen Matchinformationen in den entsprechenden Match-Dokumenten gespeichert werden. Es wird durch die Liste 'foundSupermatches' durchiteriert. Dafür wird zunächst geprüft, dass ein Match der zwei Nutzer noch nicht in der Datenbank hinterlegt ist. Anschließend werden beiden zugehörigen Match-Dokumenten die gegenseitigen 'uid' in die 'supermatches'-Liste hinzugefügt.
Der Vorgang wird für die 'normalmatches'-Liste wiederholt mit entsprechender Speicherung in die 'normalmatches'-Listen.

\begin{lstlisting}[caption=Match Manager - startMatching - Teil 2: Speichere Matches, label=lst:startMatchingteil2]
await session.startTransaction();

for (let i = 0; i < foundSupermatches.length; i++) {        
 // Ueberpruefe, ob Match vorhanden ist
 if (!(await MatchService.CheckMatch(foundSupermatches[i].matchid1,   foundSupermatches[i].matchid2, session))) {
  // Fuege Match zu User1's Match-Dokument hinzu
  await MatchService.AddSuperMatchToUser(foundSupermatches[i].matchid1, foundSupermatches[i].matchid2, foundSupermatches[i].movieid, session);

  // Fuege Match zu User2's Match-Dokument hinzu
  await MatchService.AddSuperMatchToUser(foundSupermatches[i].matchid2,  foundSupermatches[i].matchid1, foundSupermatches[i].movieid, session);
}}
await session.commitTransaction();
\end{lstlisting}





\subsection{Firebase}
