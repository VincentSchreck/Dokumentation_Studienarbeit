Unter dem Begriff "Backend" versteht man Komponenten eines digitalen Systems, die den Betrieb des Programms ermöglichen und vom Benutzer nicht ersichtlich sind. ...[TODO]
\newline
Eine Backend-Anwendung kann direkt mit dem Frontend interagieren und dessen Benutzerdienste unterstützen sowie die Schnittstellen mit allen erforderlichen Ressourcen anbieten.

\subsection{Datenbank}
Hier steht mein BackendDatenbank Text.

\subsubsection{Einrichtung}
Hier steht mein Imlementierung Text.

Die Serverzuständigkeiten lassen sich in folgende Aspekte zusammenfassen:
\begin{itemize}
\item Empfangen und Beantworten der Clientanfragen
\item Routing der Anfragen zu den entsprechenden Abhandlungsroutinen
\item Überprüfen der Identität des Clients
\item Kommunikation zur Datenbank für die persistente Speicherung der Zustände der Nutzer und ihrer Präferenzen, Swipes und Matches.
\item Matching-Algorithmus
\end{itemize} 

In den folgenden Unterkapitel wird auf die Einrichtung des Node.js-Webservers und der MongoDB-Datenbank, auf die Erstellung der sicheren Kommunikationsschnittstelle und auf die Implementierung der serverzuständigen Funktionalitäten eingegangen.

\subsection{Webserver}
Hier steht mein BackendServer Text.

\subsubsection{Einrichtung}
Zunächst...


\subsubsection{Sicherheit}
Den Server gilt es zu schützen.

\subsubsection{Webserver}
Den Server gilt es zu schützen.



\subsection{Firebase}
