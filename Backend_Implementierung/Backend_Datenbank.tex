\subsubsection{Bereitstellung der Datenbank}
Über die offizielle Seite der MongoDB-Hersteller wird das Community-Installationspaket heruntergeladen und ausgeführt. []
%https://www.mongodb.com/try/download/community
Ausserdem wird die Kommandozeilenanwendung MongDBShell[] und die
%https://www.mongodb.com/try/download/shell
graphische Benutzeroberfläche MongoDBCompass installiert. []
%https://www.mongodb.com/try/download/compass
Nach erfolgreichem Initiieren eines Replica-Sets[]
%https://docs.mongodb.com/manual/tutorial/deploy-replica-set/
kann der mongod-Prozess unter Angabe des zu verwendenden Replica-Sets, des Ports und des Datenspeicherpfads gestartet werden. \newline

%\begin{lstlisting}[caption=Mongod-Aufruf, %label=lst:mongodcall]
%mongod --port 27017 --replSet rs0 --dbpath=”..\data\db0
%\end{lstlisting} %TODO


Standardmäßig wird bei Installation von MongoDB eine Konfigurationsdatei erstellt, dessen Name und Verzeichnis anhängig vom benutzten Betriebssystem sind. Auf diese Datei wird bei fehlender Angabe im mongod-ausführendem Kommando zugegriffen. [] 
%https://docs.mongodb.com/manual/reference/configuration-options/
Sie ermöglicht das Konfigurieren der zu nutzen Storage-Engine. Dementsprechend wird als Storage Engine die WiredTiger-Engine eingestellt. 
Über die graphische Benutzeroberfläche MongoDBCompass wird nach erfolgreicher Verbindung mit der MongoDB-Datenbanksystem eine neue Datenbank hinzugefügt namens 'StreamSwipeDatabase'.

\subsubsection{Importieren der Film- und Städtedaten}
Um Nutzer innerhalb einer Stadt miteinander matchen zu können, werden Städtenamendaten auf der Datenbank benötigt. Auf der Webseite "datenboerse.net" wird eine Liste deutscher Städtenamen zur Verfügung gestellt[].\newline
%https://www.datenbörse.net/item/Liste_von_deutschen_Staedtenamen_.csv


Die von uns genutzte Filmdatenbank TMDB bietet eine Liste der datenbankseitig vorhandenen Filme. Sie kann über Filmdatenbankanbieter-API heruntergeladen werden.[]
%TODO
\newline
Über MongoDBCompass werden die heruntergeladenen Städtenamen in die Datenbank als Collection 'cities' und die Filminformationen als 'movies' importiert.
\newline
=\> Todo Ausblick: Filme aktualisieren
