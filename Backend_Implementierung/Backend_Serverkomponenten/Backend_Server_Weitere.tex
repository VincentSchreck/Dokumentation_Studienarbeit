Nachfolgend werden weitere Systemfunktionalitäten des Backends dargestellt.

\paragraph{Firebase-Service}
Um unberechtigte Zugriffe zu vermeiden, findet für die nutzerbezogenenen Anfragen eine Authentifizierung statt. Die erste Authentifizierung des Nutzers findet über den Login des Frontends in Firebase statt. Nachträglich muss bei Anfragen an den Webserver sichergestellt werden, dass der Nutzer weiterhin authentifiziert ist. Ohne diesen Vorgang könnte man sich über das Schicken einer willkürlich übermittelten Nutzer-Uid fälschlicherweise als anderer Nutzer ausgeben. Für den Zugriff auf die Firebase-Authentifizierungsfunktionen wird in die firebaseService.js-Datei das Modul 'firebase-Admin' importiert.
[https://firebase.google.com/docs/admin/setup]

\subparagraph{Register}
Um die Anwendung bei Firebase zu registrieren, wird die Funktion 'initializeApp' des Firebase-Moduls ausgeführt. 
Ein aus Firebase generierter Authentifizierungsschlüssel wird dabei für den Zugriff auf die StreamSwipe-Umgebung mitübergeben.
Die Register-Methode wird anschließend nach außen exportiert und zu Beginn des Serverstarts in der Server.js-Datei ausgeführt. %TODO [https://firebase.google.com/docs/admin/setup#initialize-without-parameters]
   
\begin{lstlisting}[caption=Firebase-Service Register, label=lst:firebaseService Register]
  var serviceAccount = require("../sslcert/streamswipe-firebase-adminsdk-uiyci-80bc08a5b2.json");
    firebaseAdmin.initializeApp({
      credential: admin.credential.cert(serviceAccount),
        databaseURL: "https://streamswipe.firebaseio.com"
    });
}
\end{lstlisting}

\subparagraph{UID/TokenID-Dictionary}
Um Zugriffszeiten auf die Firebase-Schnittstelle, werden in einem lokalen Dictionary aus Schlüsselwertpaaren der Zusammenhang zwischen TokenID und den UID samt ihrem Ablaufsdatum zwischen\-gespeichert.

\subparagraph{GetUID}
Die Funktion erwartet einen Firebase Token als Parameter 'uidtoken', welcher an die Funktion 'verifyIdToken' des FirebaseAdmin-Objeekts weitergeleitet wird. Zurück wird ein Objekt gegeben, dass unter anderem die 'uid' des zum Token zugehörigen Nutzers und die Ablaufzeit schickt. Nach erfolgreichem Überprüfen, ob die 'uid' tatsächlich ein Wert übermittelt bekommen hat, wird das Paar aus UidToken und Uid samt Ablaufzeit in der UID/TokenID-Dictionary gespeichert.

\begin{lstlisting}[caption=Firebase-Service Register, label=lst:firebaseService Register]
	verifiedUid = await firebaseAdmin.auth().verifyIdToken(uidToken);
	uid = verifiedUid.uid;
	expireTime = verifiedUid.exp;
	if(uid === undefined || uid == null) { throw {message: "No uid returned!"}; }
    TokenIDDict[uidToken] = {uid,new Date(expireTime*1000)};
    return uid;
    ... //end Try-Catch-Block
\end{lstlisting}
   

\subparagraph{RefreshList}
Diese Funktion wird aufgerufen, um abgelaufene Token in der UID/TokenID-Dictionary zu löschen. Sie wird über das Modul TimedEvents periodisch aufgerufen. Dabei wird zu jedem Paar die aktuelle Uhrzeit und die Ablaufszeit verglichen. Stellt die Ablaufszeit ein größeren Wert dar, wird das Schlüsselwertpaar aus der Dictionary entfernt.


\paragraph{Matching-Algorithmus}

[UML]

\paragraph{Timed Events}