Mit der Verbreitung von Smartphones ist die Benutzung mobiler Apps stark angestiegen und mittlerweile in nahezu jedem Haushalt aufzufinden. Obwohl etwa 9,5\% aller in Deutschland lebenden Menschen einen Schwerbehindertenausweis besitzen (Stand 24.06.2020)\cite{schwerbehindertenausweis} was etwa 7,9 Millionen Menschen entspricht, ist die Implementierung von barrierefreier Bedienung nicht selbstverständlich. Gerade Programmierern/innen aus dem privaten Sektor sind diese Funktionen oft nicht bekannt, es besteht kein Interesse oder sie werden schlichtweg vergessen. Software, die für öffentliche Einrichtungen entwickelt wird, ist durch das Behindertengleichstellungsgesetz von 2002 dazu verpflichtet ihr Softwareangebot bis spätestens dem 23. Juni 2021 barrierefrei zu gestalten (\cite{behindertengleichstellungsgesetz}, §12a Abs.1). Hierzu zählen sämtliche Webseiten sowie mobile Anwendungen. \\
