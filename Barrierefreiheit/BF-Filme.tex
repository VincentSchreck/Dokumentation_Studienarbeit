Auch die Zugänglichkeit von Filmen und Serien für Menschen mit eingeschränkter Wahrnehmung wurde in den letzten Jahren stark verbessert. 
Hierbei lässt sich zwischen optischer und akustischer Einschränkung differenzieren. Für hörgeschädigte Personen werden bereits seit mehreren Jahrzehnten Untertitel eingesetzt. Was früher für vereinzelte Filme durch eine Funktion des Teletextes erreicht wurde, wird heutzutage durch eine integrierte Funktion des Videoplayers verwirklicht. Immer mehr Videos werden mit Untertiteln veröffentlicht. Manche Anbieter wie beispielsweise die Internetplattform YouTube bieten durch Spracherkennung automatisch generierte Untertitel an, was eine flächendeckende Untertitelung ermöglicht.\\
Auch für Menschen mit eingeschränktem Sehvermögen werden Filme und Serien mithilfe von Audiodeskriptionen vermehrt zugänglich gemacht. Hierbei wird die bereits vorhandene Tonspur mit Bildbeschreibungen und Kommentaren versehen. Was bis vor wenigen Jahren noch etwas Besonderes war und nur für ausgewählte Filme bestimmt war, ist heutzutage Standard. Größere Video-On-Demand-Plattformen wie Netflix oder Amazon Prime bieten diese Möglichkeit bei nahezu allen Eigenproduktionen an. Zusätzlich werden bestehende Filme neu mit Audiodeskriptionen versehen.\\


\noindent Hieraus lässt sich leicht erkennen, dass Filme und Serien heutzutage auch von Menschen mit Einschränkungen genutzt werden. Was auf den ersten Blick vielleicht nicht bedacht wird oder als  unwichtig abgestempelt wird, kann einen nicht unerheblichen Vergrößerungsfaktor für den Kundenstamm bewirken. Für die Entwicklung einer mobilen App, bei der Filme und Serien bewertet werden, spielt also die Barrierefreiheit eine wichtige Rolle und darf auf keinen Fall vernachlässigt werden. 
