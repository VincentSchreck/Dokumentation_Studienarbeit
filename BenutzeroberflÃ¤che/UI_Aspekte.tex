Die Vielschichtigkeit einer Benutzeroberfläche kann ausschlaggebend für den Erfolg einer Applikation sein, abhängig davon welche Erfahrungen der User mit der Oberfläche macht und welche Eindrücke sie hinterlässt. Hieraus resultiert wie lange ein User auf der App bleibt und wie oft er zurück kommt. Neben der Nutzungszeit erhöht eine positive User Experience die Weiterempfehlungsrate.\\
Bei erfolgreicher Software besteht ein großer Teil der Entwicklung in der Planung der Oberfläche, da die User Experience nicht zu umgehen ist. Auf die eine oder andere Art erlebt der User immer eine Erfahrung. Neben den offensichtlichen Aus- und Eingabefunktionen werden beispielsweise folgende Kriterien  betrachtet:\\

\noindent
\hangindent1cm
\textbf{Simpel:} Ausgegebene Information kann zum Beispiel durch Icons, Farben oder Symbole vereinfacht werden. Eine Oberfläche sollte weder überladen sein, noch sollten alle Ein- und Ausgaben auf verschiedenen Screens verteilt sein. Bei der Entwicklung wird eine gesunde Mischung aus maximaler Funktionalität und einfacher, übersichtlicher Darstellung angestrebt.\\

\noindent
\hangindent1cm
\textbf{Einheitlich:} Die Bedienung und das Lesen von Applikationen kann erheblich vereinfacht werden wenn einheitliche Bedien- oder Ausgabeelemente verwendet werden. Nicht nur innerhalb einer App ist es sinnvoll konsistente Elemente in der Oberfläche zu verwenden, auch Funktionen von anderen Apps können die Bedienung vereinfachen. Bekannte Funktionen bei Smartphone-Applikationen sind zum Beispiel die Vergrößerung mit zwei Fingern oder das \glqq Daumen nach oben\grqq -Symbol als positive Rückmeldung. Durch das  Einbauen solcher Features wird eine App intuitiv und ohne Einführung bedienbar.\\

\noindent
\hangindent1cm
\textbf{Benutzergesteuert:} Alle ausgeführten Aktionen sollten vom Benutzer ausgehen. Ein gutes Interface unterstützt den User lediglich bei seiner Bedienung, schränkt ihn aber nicht ein. Mit der heutigen Technologie ist die Verführung groß viele Funktionen automatisch ablaufen zu lassen. Was eigentlich der Sinn einer Applikation ist, kann jedoch auch negative Folgen haben. Zu viel Automatisierung verursacht das Gefühl von Kontrollverlust und Unsicherheit, was sich negativ auf das Vertrauen und somit auf die Benutzungszeit von dem User auswirkt. \\

\noindent
\hangindent1cm
\textbf{Klarheit:} Eine mobile App muss ohne Anleitung bedienbar sein. Sobald Unklarheiten beim User entstehen und Funktionen oder Ausgaben nicht erkannt werden können, verliert die Anwendung auf dem freien Markt. \\
Der User sollte zu jeder Zeit wissen welche Optionen ihm zur Verfügung stehen und welche Folgen seine Aktionen haben. Besonders wichtig ist das Feedback infolge einer Aktion. Auch wenn diese Aspekte offensichtlich erscheinen, können sie bei der Entwicklung einer App leicht übersehen werden. Verwendet werden einfache und für den User bekannte Funktionen, wie die Beschriftung aller Buttons oder das haptische, akustische oder optische Feedback beim drücken einem dieser Buttons.\\

\noindent
\hangindent1cm
\textbf{Benutzerfreundlich/Barrierefreiheit:} Die Bedienung der App sollte für Menschen mit Einschränkungen im vollen Umfang möglich sein. In Abschnitt \ref{sec:barrierefreiheit} wird auf dieses Thema tiefer eingegangen. Aber auch Benutzer ohne Einschränkungen erwarten eine einfache und übersichtliche Bedienung, die auch beispielsweise  Eingabefehler mit mehreren Versuchen verzeiht.\\

\noindent
\hangindent1cm
\textbf{Ästhetik:} Das Design spielt bei dieser Betrachtung gleich mehrere wichtige Rollen. Es sollte eine angenehme Arbeitsumgebung für den User erstellen, Ein- und Ausgaben verdeutlichen und gleichzeitig mithilfe eines eigenen Stils ein einzigartiges Image für die App schaffen (sogenanntes Branding) um deren Individualität und Wiedererkennungswert zu steigern. Das Design erschafft ein Erlebnis während der Benutzung und weckt Gefühle im User. \\

\noindent
Gerade weil viele dieser Aspekte unterbewusst wirken, ist eine ausgiebige Betrachtung unumgänglich.\\
Eine Schwierigkeit, die sich bei der Entwicklung ergibt sind die zwei unterschiedlichen Ziele. Einerseits sollten bestehende Design- und Bedienelemente  übernommen werden um die Bedienung intuitiv und übersichtlich zu gestalten, andererseits aber auch neue Ideen und Innovationen eingebracht werden, um sich von anderen Apps abzuheben und bleibenden Wiedererkennungswert aufzubauen.
