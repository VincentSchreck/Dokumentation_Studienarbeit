Die Oberflächen, die dem Benutzer bereitgestellt wird, erfüllt neben den beiden offensichtlichen Eigenschaften wie der Darstellung von Information und der Bedienung von Funktionen noch andere Anforderungen. Es ist sinnvoll diese Eigenschaften in aktive und passive zu unterteilen, da der User mit manchen interagiert und von anderen nur beeinflusst wird.


\noindent - Unterteilung in aktive und passive Eigenschaften \\
- mehrere Anforderungen, wie Funktion, Usability (intuitive Bedienung) und Design \\
$\rightarrow$ User muss ohne Einführung alles bedienen können\\
$\rightarrow$ Am besten Standard-Elemente/Icons/Aufbau/Bedienbewegungen verwenden\\
- Mit dem Design kann eine Aussage/ein Image erstellt werden \\
- Design besitzt auch Funktion (Nachtmodus, Größe von Buttons, ...). Z.B. farbliches Hervorheben bei Erkennung der Swipe-Richtung in Verbindung mit den Buttons.\\
- 

\subsection{Aktive Eigenschaften}
Unter aktiven Eigenschaften versteht man sämtliche Funktionen, die der User bedient, wie zum Beispiel das Klicken eines Buttons oder die Eingabe in ein Textfeld. Hierfür müssen diese Elemente an den User angepasst werden. \\
- gut sichtbar/lesbar/klickbar. Also groß genug und Icons müssen aussagekräftig sein. Farbe von Schrift oder Icon und Hintergrund müssen sich voneinander abheben. \\
- lesbare Schriftart\\
- Standard-Icons, die man bereits kennt\\
- Standard-Funktionen, wie wischen/mit zwei Finger zoomen/Bottom bar/...

\subsection{Passive Eigenschaften}
Passive Eigenschaften sind \\

- Design beeinflusst 