\documentclass[11pt,a4paper]{article}
\usepackage[utf8]{inputenc}
\usepackage[german]{babel}
\usepackage{amsmath}
\usepackage{amsfonts}
\usepackage{hyperref}
\usepackage{booktabs}
\usepackage{setspace}
\usepackage{threeparttable}
\usepackage{amssymb}
\usepackage{graphicx}
\usepackage{fancyhdr}
\usepackage{icomma}
\usepackage{float}
\usepackage{pdfpages}
\usepackage{hyperref}
\usepackage[left=2.5cm,right=2.5cm,top=2cm,bottom=3.5cm]{geometry}
\usepackage{csquotes}
\title{DreamSwipe\\Tinder für Filme\vspace{10px}}
\author{Leon Gieringer, Robin Meckler,Vincent Schreck \\ \\ Studienarbeit \\ \\ \\}
\date{\today}

\begin{document}
\maketitle
\thispagestyle{empty}
\newpage
\pagenumbering{Roman}
\tableofcontents
\newpage
\pagenumbering{arabic}

\pagestyle{fancy}
\fancyhf{}
\setlength{\headheight}{35pt}
\lhead{DreamSwipe}
\rhead{Studienarbeit}
\cfoot{\thepage}
\newpage


\section{Einleitung}


\section{Motivation}


\section{Theoretische Grundlagen}


\subsection{Anwendungsentwicklung für mobile Endgeräte}
Mobile Geräte sind heutzutage ein sehr großer Teil unseres Tagesablaufs. Durchschnittlich verbringen wir 3:54 Stunden pro Tag an mobilen Geräten (hier bezogen aus Bürger der USA). Die meiste Zeit hiervon wird in Apps (ca. 90\%). \footnote{\url{https://www.emarketer.com/content/us-time-spent-with-mobile-2019}, zuletzt aufgerufen: 26.02.2021} 
Laut Cisco wird dieser Markt sich jedoch nicht nur auf Industrieländer beruhen, sondern bis 2023 sollen weltweit 71\% der Bevölkerung mobile Konnektivität haben. \cite{cisco2020}
Diese Entwicklung forcierte viele Firmen immer mehr ihre Anwendungen auch \textit{mobile ready} zu gestalten. Dies kann man bspw. deutlich bei der Anpassung vieler Webseiten an Mobile Seiten- und Größenverhältnisse oder auch dem Anbieten von \textit{Apps}, welche bereits für Desktop o.ä. verfügbar waren, erkennen. \\

Daher ist es für die Wirtschaft und Entwicklung gleichermaßen wichtig sich ständig weiterzuentwickeln und sich nicht auf (Kosten-) ineffiziente Entwicklungsprozesse auszuruhen. Dabei bieten jährliche, wenn nicht sogar halbjährliche Design- und Performanceänderungen von den Geräten selbst oder der Betriebssysteme Herausforderungen an die mobilen Anwendungen und gleichzeitig an deren Programmierumgebung. Trotz einer riesigen Auswahl an \textit{Apps} lassen sich diese allgemein in drei Kategorien eingliedern: Plattformspezifische Native Anwendungen, Adaptive Webanwendungen und Plattformübergreifende Native Anwendungen.

\subsubsection{Plattformspezifische native Anwendungen}
Plattformspezifische oder auch native Anwendungen sind Programme, welche auf eine gewisse Plattform abzielen und in einer der davon unterstützen Programmiersprachen geschrieben wurden. Da diese Art der (mobilen) Anwendung mit plattformspezifischen \textit{Software Development Kits (SDK)} und \textit{Frameworks} entwickelt wird, ist diese Anwendung an eine Plattform gebunden. \\
Dies bringt zum einen natürlich Vorteile wie allgemein best mögliche Performance auf der jeweiligen Plattform und direkt vom Hersteller unterstützte Entwicklungsumgebungen/SKDs.
Zudem lassen sich plattformspezifische Fähigkeiten oder Einstellungen nutzen - beispielsweise mehrere Kameras oder \textit{Global Positioning System (GPS)}.

Gleichzeitig beschränkt man sich aber logischerweise auf eine Plattform und deckt mit einer Anwendung nur einen Teil des gesamten Marktes. Dies bringt im Vergleich zu den anderen Möglichkeiten einen deutlich erhöhten Entwicklungs- und Wartungsaufwand mit sich, da für andere Plattformen Programmcode nicht übernommen werden kann. Zusätzlich benötigen Entwickler spezifische Kompetenzen für beide Plattform und Entwicklungsumgebungen. \\

Zwei der am weitesten verbreiteten Plattformen sind Android von Google und iOS von Apple. Anwendungen für Android können in Kotlin oder Java als Programmiersprache beispielsweise in dem \textit{integrated development environment (IDE)} von Google Android Studio entwickelt werden. Für iOS wird hingegen mit Objective-C und Swift als Programmiersprache primär in der IDE XCode entwickelt.

Beide bieten jeweils Plattform eigene Services an, beispielsweise das direkte Veröffentlichen in den jeweiligen Appstore \cite{fentaw2020}
\subsubsection{Adaptive Webanwendungen}
test
\subsubsection{Plattformübergreifende Anwendungen}
test


\subsection{Language}
Hier steht mein Language Text.

\subsubsection{JavaScript}
In den nächsten Unterkapiteln soll ein zunächst ein historischer Überblick über die Programmiersprache JavaScript gegeben werden. Im Anschluss wird auf die Bedeutung und Nutzung von JavaScript eingegangen. 

\subsubsection{Historie TODO }
Ihren Ursprung findet JavaScript im Jahr 1995, als Brendan Eich, ein damaliger Ingenieur des US-amerikanischen Software-Unternehmens „Netscape Communications Corporation“, innerhalb von zehn Tagen die Sprache für den Browser „Netscape Navigator“ entwickelt hat. [1] Das Ziel dabei war es, eine Skriptsprache zu entwickeln, die es Entwicklern möglich machen sollte, auf ihren Webseiten Skripte umzusetzen. Zunächst noch unter dem Namen Mocha und LiveScript änderte sich der Name aufgrund der Kooperation von Netscape und Sun, der Firma hinter der Programmiersprache Java, und der damaligen Popularität von Java zu JavaScript. [1.05] 
Netscape’s Veröffentlichung des Netscape Navigator 2.0, der erste Browser der JavaScript unterstütze, brachte Microsoft dazu, Netscape als ernstzunehmenden Konkurrenten zu sehen. 
Microsoft antwortete im August 1995 mit der Veröffentlichung des ersten Internet Explorer zusammen mit der Skriptsprache JScript, die einen Dialekt der Sprache JavaScript darstellt. Dies ist ferner als Beginn der „Browserkriege“ bekannt. [1.06]
 Im Jahre 1997 reichte Netscape JavaScript an die European Computer Manufacturers Association, einer privaten, internationalen Normungsorganisation zur Normung von Informations- und Kommunikationssystemen und Unterhaltungselektronik (kurz ECMA[ABK]) ein. Das Ziel war es, von der ECMA einen einheitlichen Standard für die Sprache schaffen zu lassen, die fortan weiterentwickelt und von weiteren Browserherstellern genutzt werden soll. Das resultierende Standard nennt sich ECMAScript, wobei JavaScript die bisher bekannteste Implementierung dieses Standards ist. [1.07] Andere Implementierungen sind zum Beispiel ActionScript von MacroMedia, JScript von Microsoft und ExtendScript von Adobe.
Jährlich wird der Standard seit Juni 2015 erweitert. ECMAScript Version 11 beziehungsweise ECMAScript 2020 bildet zum Zeitraum dieser Dokumentation [??] den aktuellen Standard. [1.08] 
Im Juni 2021 soll ECMAScript 2021 veröffentlicht werden.  [1.09]

\subsubsection{Wesentliche Programmiereigenschaften TODO}
„JavaScript is Not Java“ [1.091 ??]. Die Programmiersprache JavaScript wird aufgrund ihrer Namensgebung oft in falsche Zusammenhänge zu Java gebracht. Das häufigste Missverständnis sei, JavaScript wäre eine vereinfachte Version von Java. [1.091]
JavaScript ist eine interpretierte Programmiersprache mit objektorientierten Umsetzungs-möglichkeiten. Interpretation ist in diesem Zusammenhang so zu verstehen, dass der Quellcode zur Laufzeit eines Programms gelesen, übersetzt und ausgeführt wird.
Syntaktisch ähnelt JavaScript Programmiersprachen wie C, C++ und Java durch gleiche Umsetzung der Programmierkonstrukte wie den Bedingungen, Schleifen oder den booleschen Operatoren. [1.1] Wesentliche Unterschiede sind dagegen, dass JavaScript zum einen eine schwach-typisierte Sprache ist. Durch die schwache Typisierung haben Variablen keinen festen Dateityp und können diesen dynamisch zur Laufzeit ändern. Des Weiteren findet bei JavaScript die Objektorientierung prototypenbasiert statt. Diese Form der Programmierung wird auch klassenlose Objektorientierung bezeichnet. Anders als bei der klassenbasierten Programmierung, bei der Objekte aus vordefinierten Klassen instanziiert werden, werden hier Objekte durch Klonen bereits existierender Objekte erzeugt. Die Objekte, die geklont werden, sind dabei als Prototyp-Objekte zu verstehen. Beim Klonen werden alle Attribute und Methoden des Prototyp-Objekts in das neue Objekt übernommen und können dort überschrieben sowie erweitert werden. Objekte in JavaScript sind eher als Zuordnungslisten, ähnlich wie assoziative Arrays oder Hash-Tabellen, anzusehen, da bei der Eigenschaftszuweisung lediglich ein Mapping eines Namens zu seiner zugehörigen Eigenschaft stattfindet. Ein weiterer Unterschied zu den anderen Programmiersprachen ist, dass alle Funktionen und Variablen außer der primären Datentypen Boolean, Zahl und Zeichenfolge, als Objekte verstanden werden können.

\subsubsection{Anwendungsgebiete TODO}
Ursprünglich fand JavaScript seinen Einsatz hauptsächlich darin, dynamische Webseiten im Web-browser anzuzeigen. Die Verarbeitung erfolgte dabei meist clientseitig durch den Webbrowser (dem sogenannten Frontend). [1.3] 
Heutzutage findet sich die Sprache dagegen in wesentlich größeren Einsatzgebieten wieder. 
Bis vor einigen Jahren war die Serverseite anderen Programmiersprachen wie Java oder PHP vorbehalten. Die Veröffentlichung von Node.js, einer plattformübergreifenden Laufzeitumgebung, die JavaScript außerhalb eines Webbrowsers ausführen kann, führte zu einer immer größeren Verbreitung von serverseitigen Anwendungen (dem Backend), die auf JavaScript basieren. Auf Node.js wird ausführlicher im nächsten Kapitel eingegangen. 
Ferner findet JavaScript heutzutage aber auch seinen Einsatz in mobilen Anwendungen, Desktopanwendungen, Spielen oder 3D-Anwendungen.



\subsection{IDE}
Hier steht mein IDE Text.

\subsection{Database}
Hier steht mein Database Text.

\subsection{Firebase}
Firebase ist eine Backend-as-a-Service (BaaS) Plattform von Google für mobile oder Web-Anwendungen. 
Sie soll es dem Entwickler ermöglichen, einfacher und effizienter Funktionen auf verschiedenen Plattformen bereitzustellen stellt Tools und Infrastruktur zur Verfügung.
Mit dem Firebase SDK bietet die Plattform API Schnittstellen zu den jeweiligen Tools, welche direkt in die Anwendung integriert werden können, ohne dass serverseitiger Code dafür notwendig ist.
Die Firebase Inc. wurde 2011 von James Tamplin und Andrew Lee gegründet und letztendlich 2014 von Google übernommen.\footnote{\href{https://firebase.googleblog.com/2014/10/firebase-is-joining-google.html}{firebase.googleblog.com}, zuletzt aufgerufen am 03.05.2021}
Teile der SDK stehen seit der Google I/O 2017 unter der Apache 2.0 Lizenz, sind somit also Open-Source.\footnote{\href{https://opensource.googleblog.com/2017/05/open-sourcing-firebase-sdks.html}{opensource.googleblog.com}, zuletzt aufgerufen am 03.05.2021}\\
\\
Ein Firebase Projekt ist die oberste Ebene in Firebase. 
Ein Projekt ist letztendlich ein \textit{Google Cloud Projekt}, welches mit speziellen Konfigurationsmöglichkeiten und Services ausgestattet ist. 
Es beinhaltet die Verknüpfung zu den einzelnen Anwendungen (also bspw. Android-, iOS- oder Webanwendung). Nun können variabel Tools, sog. Firebase products hinzugefügt werden. Diese Produkte lassen sich grundlegend in drei Kategorien einteilen. Die hier relevantesten werden im Folgenden besprochen.\cite{firebase2021}

\subsubsection{Firebase Authentifizierung}
Die Authentifizierung gehört zu den \glqq Build\grqq Produkten und bietet eine Token-basierte Nutzerauthentifizierung. 
Hierbei kann zwischen verschiedenen Anmeldeoptionen gewählt werden: klassisch mit E-Mail und Passwort, mit OAuth2.0 Integration für Social Media (Google, Facebook, Twitter, Github, ...) oder per Telefonnummer.
Jeder Nutzer erhält eine einzigartige ID und ein zugehöriges Nutzerobjekt in einer NoSQL Datenbank. Grundlegende Werte wie E-Mail Adresse oder Name können hier abgespeichert werden; zusätzliche Informationen müssen über einen weiteren Datenbank Service abgespeichert werden.
Für die Verwaltung eines Accounts bietet dieses Tool auch eingebaute E-Mail Aktionen an - bspw. Passwort zurücksetzen oder E-Mail Adresse bestätigen.\\
\\
Ein Firebase Nutzer Objekt repräsentiert den Account eines Nutzers, welcher sich von einer Anwendung aus beim zentralen Firebase Projekt angemeldet hat.
Die Instanz eines Firebase Nutzers ist somit unabhängig von der Authentifizierungsinstanz der Anwendung, also kann eine Anwendung mehrere Nutzer anmelden, jedoch kann sich auch ein Nutzer auf mehreren Anwendungen anmelden.
Ist ein Nutzer authentifiziert, erhält die Anwendung eine Referenz des Nutzers, welche so lange existiert, bis er wieder abgemeldet ist.

\subsubsection{Firestore}
\label{sssec:firestore}
Als Datenbank Lösung bietet Firebase zwei unterschiedliche Produkte an: Firestore und Realtime Database.
Firestore ist hier neuer, jedoch ersetzt es Realtime Database nicht. \\
Firestore ist eine flexible und auf Skalierung ausgesetzte NoSQL Cloud Datenbank, welche unter anderem die Echtzeitsynchronisierung der Daten zwischen Anwendung und Server ermöglicht.
Zusätzlich zu REST und RPC APIs in iOS, Android und web SDKs ist Firestore auch in nativen Node.js, Java, Python und Go SDKs verfügbar.\\
\\

\begin{wrapfigure}{R}{0.4\textwidth}
	\begin{center}
		\includegraphics[width=0.35\textwidth]{images/firestore_datastucture.png}
	\end{center}
	\caption{Datenmodell in Firebase \protect \footnotemark}
	\label{fig:firestore_data_structure}
\end{wrapfigure}
\footnotetext{Quelle: \cite{firebase2021}}

Das Datenmodell ist hierarchisch aufgebaut, wobei Daten in Dokumenten (documents) und Dokumente in Sammlungen (collections) gespeichert sind. 
Mithilfe von Sammlungen werden die Daten voneinander abgetrennt und hierüber können Abfragen erstellt werden.
Grundlegende Datentypen sind String, Integer und Boolean, jedoch können auch komplexe Datentypen wie Maps, Arrays oder Geopoints. Unter-Sammlungen und darin verstaute Dokumente sind ebenfalls möglich.\\
\\
Abfragen werden auf Dokumentenebene erstellt, damit nicht eine gesamte Sammlung aufgerufen werden muss.
Dies kann über direkte Sortierung, Filter und/oder Limitierung bzw. genaue Auswahl eines Dokumentes bewerkstelligt werden.
Bei einer Abfrage erhält man einen \textit{Data Snapshot}, wodurch über Änderungen in Echtzeit informiert und diese angezeigt werden können.
Damit es jedoch zu keinen fehlerhaften Daten führt, gelten hier atomare Eigenschaften für Transaktionen.
Eine Transaktion ist eine Folge von Datenbankanweisungen, welche entweder alle gemeinsam oder gar nicht ausgeführt werden. 
Eine Transaktion ist nur dann erfolgreich, wenn alle Anweisungen auf eine Datenbank vollständig geschlossen sind. 
Ist dies nicht der Fall, werden alle Anweisungen bis zum Stand vor der Transaktion rückgängig gemacht. Das nennt man Rollback.\\
\\
Die Sicherheit der Daten stellt Cloud Firestore für Mobil- und Webclient-Bibliotheken über die Firestore-Sicherheitsregeln her. Diese bieten sowohl Zugriffsverwaltung und -authentifizierung, jedoch könne auch Daten hiermit für die Konsistenz der Datenbank validiert werden. 
\medskip
\begin{lstlisting}[caption=Beschränkung des Zugriffs auf Dokumente der Sammlung \texttt{cities}, label=lst:firestorerules_basic]
	service cloud.firestore {
		match /databases/{database}/documents {
			match /cities/{city} {
				allow read, write: if request.auth != null;
			}
		}
	}
\end{lstlisting}
\medskip
Im Beispiel \ref{lst:firestorerules_basic} wird der Lese- und Schreibzugriff auf ein Dokument der Sammlung \texttt{cities} beschränkt. 
Nur falls der anfragende Nutzer eine valide Authentifizierung besitzt, erhält er Zugriff auf das angefragte Dokument. 
Diese simple Darstellung ist jedoch für den wirklichen Produktionseinsatz mit Vorsicht zu nutzen. 
Oftmals müssen \texttt{read} und \texttt{write} in detailliertere Vorgänge aufgeteilt werden. Ein \texttt{read} wird spezialisiert in \texttt{get} und \texttt{list}, wobei ein \texttt{write} in \texttt{create}, \texttt{update} und \texttt{delete} unterteilt werden kann.
Ein \texttt{list} ermöglicht es hierbei auf Sammlungen, also die einzelnen Dokumenten IDs lesend zuzugreifen, jedoch nicht auf die Daten einzelner Dokumente. Hierfür wird dann ein \texttt{get} benötigt. 
Mittels \texttt{create} erhält man Schreibzugriff auf nicht existierende Dokumente, durch \texttt{update} auf bereits vorhandene und Löschrechte ganzer Dokumente erhält man über den \texttt{delete} Operator.\\
\\
Sicherheitsregeln werden gleich dem Datenmodell hierarchisch aufgebaut und ermöglichen differenzierte Zugriffsbeschränkungen auf jeder Ebene.
In Codebeispiel \ref{lst:firestorerules_hierarchy} beinhaltet jedes Dokument (Stadt) der Sammlung \texttt{cities} eine Unter-Sammlung \texttt{landmarks}. Nun lässt sich der Zugriff auf beide separat regeln.
Bei der Sammlung \texttt{villages} hingegen wurde der rekursive Platzhalter verwendet. Hiermit sind Zugriffsregeln auf allen tieferen Ebenen gleich.
Beim Verschachteln von \texttt{match} ist der innere Pfad immer relativ zum äußeren.

Wichtig zu wissen ist hierzu noch, dass falls mehrere \texttt{allow} Ausdrücke auf eine Anfrage zutreffen, wird der Zugriff erlaubt sobald \textbf{eine} Bedingung wahr, also erfüllt ist.

\medskip
\begin{lstlisting}[caption=Hierarchische Zugriffsbeschränkung, label=lst:firestorerules_hierarchy]
	service cloud.firestore {
		match /databases/{database}/documents {
			match /cities/{city} {
				allow read, write: if <condition>;
				
				// Explicitly define rules for the 'landmarks' subcollection
				match /landmarks/{landmark} {
					allow read, write: if <condition>;
				}
			}
			match /villages/{document=**} {
				allow read, write: if <condition>;
			}
		}
	}
\end{lstlisting}
\medskip

Wie bereits oben besprochen können diese Regeln auch zur Validierung von Daten genutzt werden, damit die atomare Eigenschaft von Transaktionen bestehen bleibt.
Hierzu kann die \texttt{getAfter()} Funktion genutzt werden. 
Mit dieser kann man auf Zustand eines Dokumentes zugreifen und diesen validieren, nachdem einer Folge von Anweisungen ausgeführt, jedoch diese noch nicht auf der Firestore Datenbank abgeschlossen wurde.
Im Beispiel \ref{lst:firestorerules_validation} existieren zwei Sammlungen: \texttt{cities} und \texttt{countries}. 
Jedes \texttt{country} Dokument beinhaltet das Feld \texttt{last\_updated} um zu wissen, welche Stadt innerhalb eines Landes zuletzt aktualisiert wurde.
Hierzu wird in den Sicherheitsregeln nach jedem Schreibzugriff auf ein \texttt{city} Dokument gleichzeitig auch das Feld des zugehörigen Landes aktualisiert.\cite{firebase2021}
\medskip
\begin{lstlisting}[caption=Datenvalidierung für atomare Operationen, label=lst:firestorerules_validation]
	service cloud.firestore {
		match /databases/{database}/documents {
			// If you update a city doc, you must also
			// update the related country's last_updated field.
			match /cities/{city} {
				allow write: if request.auth != null &&
				getAfter(
				/databases/$(database)/documents/countries/$(request.resource.data.country)
				).data.last_updated == request.time;
			}
			
			match /countries/{country} {
				allow write: if request.auth != null;
			}
		}
	}
\end{lstlisting}
\medskip

\subsubsection{Storage}
Um Filme, Videos oder andere Nutzer-generierte Inhalte abspeichern zu können, bietet Firebase Cloud Storage an. 
Durch das Firebase SDK für Cloud Storage können Dateien direkt von Client-Anwendungen hoch- bzw. heruntergeladen werden.
Aufgrund von möglicher schlechter Verbindung kann mithilfe von robusten Operationen der Prozess des Hoch- bzw. Herunterladens bei besserer Verbindung an der Stelle weiter geladen werden, an welcher dieser unterbrochen wurde.
Ähnlich wie bei Cloud Firestore in Kapitel \ref{sssec:firestore} bestimmen auch hier Sicherheitsregeln den Zugriff auf bestimmte Dokumente.\\
Zusätzlich hierzu sind weitere Metadaten verfügbar: \texttt{contentType} und \texttt{size}. 
Mit ihnen lassen sich die Dateien beispielsweise validieren.
Im Code \ref{lst:storagerules_validation} können Dateien nur hochgeladen werden, falls sie eine Größe kleiner 5 MB besitzen.
\medskip
\begin{lstlisting}[caption=Validierung nach Dateigröße, label=lst:storagerules_validation]
	service firebase.storage {
		match /b/{bucket}/o {
			match /images/{imageId} {
				allow write: if request.resource.size < 5 * 1024 * 1024
				&& request.resource.contentType.matches('image/.*');
			}
		}
		
\end{lstlisting}
\medskip
Außerdem lassen sich die 
\subsubsection{Cloud Functions}
\subsubsection{Analytics}
\subsubsection{Google AdMob, Google Ads}

\subsection{Recommender System}
\url{http://www.microlinkcolleges.net/elib/files/undergraduate/Photography/504703.pdf}	
		
\section{Konzept?}

		

\section{Funktionen/Komponenten}

\subsection{Swipe/Aussuchen/Voting}		
\subsection{Matches/Chat}		
\subsection{Film-/Serienvorschläge}		
\subsection{Gruppenorgien}		
\subsection{Gespeicherte Filme/Filmliste}		
\subsection{Zugänglichkeit/Behindertenfreundlichkeit}		


\section{Benutzeroberflächen}
\subsection{Home-Screen}
\subsection{Gruppen}		
\subsection{Chat}		
\subsection{Filmliste}

\section{CodeBeispiele}


\section{Probleme}


\section{Fazit}
\section*{Literaturverzeichnis}
\begin{itemize}
	\bibitem[1]{aggarwal2016} Aggarwal, C. C. (2016). Recommender systems (Vol. 1). Cham: Springer International Publishing.
	\bibitem[2]{fentaw2020} Fentaw A. E. (2020). Cross platform mobile application development: a comparison study
	of React Native Vs Flutter.
	\bibitem[3]{flutter2021} Flutter Flutter architectural overview [Online] Verfügbar: \url{https://flutter.dev/docs/resources/architectural-overview}, Aufgerufen am: 04.03.2021
	\bibitem[4]{cisco2020} Cisco (2020) Cisco Annual Internet Report (2018–2023) White Paper [Online] Verfügbar: \url{https://www.cisco.com/c/en/us/solutions/collateral/executive-perspectives/annual-internet-report/white-paper-c11-741490.html}
	
\end{itemize}

\end{document}




%************************************************************%
%*********************End of document************************%
%************************************************************%

\footnote{\url{https://de.wikipedia.org/wiki/Alufolie}} 


Abbildung \ref{fig:allg_kennlinie}

\begin{figure}[tbt]
\begin{center}
\includegraphics[scale=0.45]{Grafiken/allg_kennlinie.png}
\end{center}
\caption{Kennlinie einer Halbleiterdiode \protect \footnotemark}
\label{fig:allg_kennlinie}
\end{figure}
\footnotetext{FUSSNOTE}


Tabelle \ref{tab:cobalt}

\begin{table}[tbt]
\caption{•}
\begin{threeparttable}	%Scheme for footnotes in tables
\begin{center}
\begin{tabular}{c c c c}
\toprule
& keV & keV & keV \\
\midrule
a	& b	& c	& d \\
a	& b	& c	& d \\
a	& b	& c	& d \\
a	& b	& c	& d \\ %direkt hinter jeweiligen Wert /tnote{1}
\bottomrule
\end{tabular}
\end{center}
\begin{tablenotes}\footnotesize 
\item[1]{Quelle: http://www.thinksrs.com/downloads/PDFs/ApplicationNotes/IG1BAgasapp.pdf}
\end{tablenotes}
\end{threeparttable}
\label{tab:cobalt}
\end{table}