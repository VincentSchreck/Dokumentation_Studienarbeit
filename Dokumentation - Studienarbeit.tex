\documentclass[11pt,a4paper]{article}
\usepackage[utf8]{inputenc}
\usepackage[german]{babel}
\usepackage{amsmath}
\usepackage{amsfonts}
\usepackage{hyperref}
\usepackage{subcaption}
\usepackage{booktabs}
\usepackage{setspace}
\usepackage{threeparttable}
\usepackage{amssymb}
\usepackage{graphicx}
\usepackage{fancyhdr}
\usepackage{listings}
\renewcommand*{\lstlistlistingname}{Quellcodeverzeichnis}
\usepackage{icomma}
\usepackage{float}
\usepackage{pdfpages}
\usepackage{hyperref}
\usepackage[left=2.5cm,right=2.5cm,top=2cm,bottom=3.5cm]{geometry}
\title{DreamSwipe\\Tinder für Filme\vspace{10px}}
\author{Leon Gieringer, Robin Meckler,Vincent Schreck \\ \\ Studienarbeit \\ \\ \\}
\date{\today}




\begin{document}
\maketitle
\thispagestyle{empty}
\newpage
\pagenumbering{Roman}
\tableofcontents
\newpage
\pagenumbering{arabic}



\pagestyle{fancy}
\fancyhf{}
\setlength{\headheight}{35pt}

\renewcommand\headrulewidth{0.4pt}

\fancyhead[LE,RO]{\rightmark}% <- changed
\fancyhead[LO,RE]{StreamSwipe}
\renewcommand{\sectionmark}[1]{\markright{#1}}
\renewcommand{\subsectionmark}[1]{\markright{#1}}
\renewcommand{\subsubsectionmark}[1]{\markright{#1}}

\cfoot{\thepage}
\newpage


\section[Einleitung]{Einleitung \hfill \normalfont \small{Vincent Schreck}}
Die Zahl der Scheidungen in Deutschland hat sich während den Einschränkungen durch die COVID-19 Pandemie 2020 verfünffacht \cite{scheidungen}. Viele Menschen suchen sich Partner aus, die sie zwar attraktiv finden, mit denen sie jedoch kaum gemeinsame Interessen und Ansichten teilen. Sobald diese Personen dazu gezwungen sind mehr Zeit miteinander zu verbringen zeigt sich, dass dies keine solide Basis für eine Beziehung ist. \\
Wir wollen diesen gravierenden Fehler in seinem Keim ersticken und revolutionieren das Dating-Game mit einem Verfahren, bei dem persönliche Vorlieben im Vordergrund stehen und das Aussehen zweitrangig ist.

\subsection{Motivation}

Bereits vor tausenden von Jahren haben sich die Menschen Partner gesucht und mit der ersten Monogamie kam auch die erste Beziehung und wahrscheinlich auch die ersten Beziehungsprobleme.\\
Eine der wichtigsten Grundlagen einer Beziehung sind gleiche Ansichten, Interesse und Vorlieben, anstatt Aussehen und Geld, denn Schönheit vergeht und Charakter besteht. Jedoch ist das Problem dabei, dass man erst weiß wie gut man zueinander passt, nachdem man sich kennengelernt hat. Viele möglicherweise sehr glückliche Beziehungen finden gar nicht statt, da die Person durch ein eigentlich weniger wichtiges Kriterium herausgefiltert wurde. Sucht man die Ursache dieses Problems, ist man schnell bei der Art des Kennenlernens. Der erste Eindruck ist gewöhnlicherweise die optischer Natur. Dementsprechend ist Aussehen in der Realität das erste Filterverfahren, was jedoch durch StreamSwipe an eine spätere  Position tritt.\\
Wir bieten die Lösung zu einem jahrtausendealten Problem der Menschheit.



\subsection{Methode}
Gerade in den letzten Jahren genießt das Medium Film und Serie einen immer höheren Stellenwert in der Gesellschaft. Durch Video-on-Demand Plattformen wie Netflix, Disney+ und Amazon Prime Video  sind Filme und Serien omnipräsent geworden und das Angebot scheint endlos zu sein. Der Zugriff auf diese Medien wurde dadurch stark vereinfacht und der Nutzer kann einerseits einer Serie oder Filmreihe treu bleiben, da er keine Folge mehr verpassen kann, und andererseits ganze Staffeln an einem einzigen Tag anschauen. So kann dieses Medium bereits bei vielen Menschen eine Charaktereigenschaft werden und Charaktereigenschaften vieler Zuschauer passen sich an Filmcharaktere an.\\
Bereits 2017 haben die 18- bis 39-Jährigen an durchschnittlich 4 Tagen pro Woche eine Serie angeschaut \cite{serienkonsum}. Aus dem Film- und Serienkonsum können somit individualisierte Daten gesammelt und analysiert werden. Bei StreamSwipe wird auf diesem Wege über die Film- und Serienauswahl des Nutzers ein Geschmack berechnet werden, der über einen Algorithmus mit anderen ähnlichen Geschmäcken gematch wird. Sobald ein Match entstanden ist, öffnet sich ein privater Chat und die beiden Personen können sich austauschen und verabreden.


\section[Theoretische Grundlagen]{Theoretische Grundlagen \hfill \normalfont \small{Autor-Name}}


\subsection{Framework}
%Hier steht mein Framework Text

\newpage
\subsubsection{Node.JS}
Im Jahr 2009 veröffentlichte Ryan Dahl das Framework Node.js, das auf Googles V8-Engine, welche auch als JavaScript-Engine in Googles Browser Chrome zum Einsatz kommt, basiert und sich hervorragend für hochperformante, skalierbare und schnelle Webanwendungen eignet. Zudem ermöglicht es Webentwicklern die Entwicklung von serverseitigem JavaScript-Code\footnote{\url{https://v8.dev/}, letzter Zugriff: 03.April 2021}.


\subsubsection{Architektur}
Eine wesentliche Eigenschaft von Node.js ist die hohe Performance. Im Folgenden soll der Unterschied der Node.js-Architektur zu traditionellen Webservern und der damit verbundenen höheren Performance dargestellt werden.
\newline

\noindent
Herkömmliche Webserver erstellten zunächst für jede ankommende Anfrage einen neuen Thread. Dieses Vorgehen ist eng mit steigendem Speicher- und Rechenaufwand verbunden. Um sich Rechenzeit, die durch die Erstellung und Zerstörung von Threads entstanden, zu sparen, wurden Threadpools eingerichtet. Dieser Threadpool enthält mehrere Threads, denen Aufgaben zugewiesen werden können. Nach erfolgreicher Abarbeitung einer Operation kann einem Thread eine weitere Aufgabe zugeordnet werden.
\newline

\noindent
Es bleibt aber ein weiteres Problem: Bei der Anfragenabarbeitung kann es zu einer Form von blockierender Ein- und Ausgabe (Blocking Input/Output kurz Blocking I/O) kommen: zum Beispiel beim Suchen in einer Datenbank oder dem Laden einer Datei im Dateisystem.
 Während der Abarbeitung wartet der Thread solange, bis die Operation ein Ergebnis zurückwirft und belegt dabei weiterhin Speicherplatz. 
 Bei hohem Aufkommen von Anfragen kommt es dadurch zu einer hohen Speicherauslastung des Servers. Zudem kosten die Kontextwechsel zwischen den Threads im Betriebssystem weitere Rechenzeit \cite{Node1.05}. Man spricht bei diesem Architekturkonzept auch vom Multi-Threaded Server.
\newline

\begin{figure}[h]
\centering
\includegraphics[width=10cm, height = 5.5cm]{images/nodejs_otherthreading.png}
\caption{Multithreaded / Blocking I/O \cite{Node1.1}}
\end{figure}
 
\newpage

\noindent
Node.js verfolgt einen anderen Ansatz: Anfragen werden nur in einem einzigen Thread, dem Hauptthread, abgearbeitet und in einer Warteschlange verwaltet. Dadurch bleiben Kontextwechsel zwischen Threads erspart. Hierbei handelt es sich also um einen Single-Threaded Server. Der Hauptthread verwaltet eine Schleife, die sogenannte Event Loop, die permanent Anfragen aus der Event-Warteschlange überprüft und Ereignisse, die von Ein- und Ausgangsoperationen ausgerufen werden, verarbeitet.
\newline

\noindent
Bei Ankommen einer Nutzeranfrage an einen Node.js Server wird zunächst in der Event Loop geprüft, ob diese Anfrage Blocking I/O benötigt. Falls nicht, kann die Anfrage direkt bearbeitet werden und die Antwort an den Nutzer zurückgesendet werden. 
\newline

\noindent
Im anderen Fall wird einer von Node.js interner Workern, welche prinzipiell auch Threads sind, aufgerufen, um die jeweilige Operation auszuführen. Dabei wird eine Callback-Funktion mitgegeben, die vom Worker aufgerufen wird, sobald die Operation ausgeführt wurde. Diese Callback-Funktion kann anschließend als Ereignis von der Event Loop registriert werden. Man spricht hierbei auch von ereignisgesteuerter Architektur. [1.4]
\newline
 
\begin{figure}[h]
\centering
\includegraphics[width=10cm, height = 5.5cm]{images/nodejs_nodethreading.png}
\caption{Single Threaded / Non Blocking I/0 \cite{Node1.1}}
\end{figure}
 

\noindent
Der große Vorteil hierbei ist, dass der Hauptthread trotz der blockierenden Ein- und Aus\-gabeoperationen nicht anhält, und weitere Anfragen bearbeiten kann. (Non Blocking I/O - Prinzip) 
\newline

\newpage
\subsubsection{Module}

\noindent
Module stellen in Node.js Software-Komponenten dar, die Objekte und Funktionen nach außen hin bereitstellen sollen.
Sie können aus einer Skriptdatei oder einem Verzeichnis von Dateien bestehen. Module können als einzelne Default-Komponenten, die den Hauptteil des Moduls repräsentiert, exportiert werden. 
Bei der anderen Möglichkeit, des sogenannten ‚benannten Exports‘ werden die zu exportierenden Komponenten dagegen explizit angegeben. Letzteres ist in nachfolgender Abbildung dargestellt. 
\newline
  
    
\begin{lstlisting}[caption=Benannter Export von Modulen,label=lst:ModuleExport]
function foo(){}
function bar(){}

//Obige Funktionen exportieren:
module.exports.foo = foo;
module.exports.bar = bar;
\end{lstlisting}


\noindent
Für den Import stehen verschiedene Möglichkeiten zur Verfügung.
Im folgender Abbildung ist ein Import über die require()-Funktion dargestellt. 
Mit mitgeliefertem Modul-Pfad als Parameter gibt diese Funktion ein Objekt des Moduls wieder, das die exportierten Objekte (und Funktionen) enthält.
\newline
  
\begin{lstlisting}[caption=Import von Modulen,label=lst:ModuleImport]
//Importieren der Funktion einer anderen Datei:
const foo = require('./module/path');
const bar = require('./module/path');
\end{lstlisting}

\noindent
Eine wichtige Besonderheit ist, dass importierte Module  beim ersten Aufruf gecached werden. 
Das bedeutet, dass jeder require()-Aufruf auf ein Modul dasselbe Objekt zurückliefert \cite{Node1.21}.


\paragraph{npm}
Ehemals als Node Package Manager bekannt, ist npm ein Paketmanager für Node.js, entwickelt 2010 von Isaac Z. Schlueter \cite{Node1.3}. Es verwaltet ein öffentliches Repository (ein digitales Software-Verzeichnis im Internet) unter dem Name npm Registry. In dem Verzeichnis werden weit über 1 Millionen Pakete (Module) angeboten \cite{Node1.4}. Der Großteil kann unter freier Lizenz verwendet werden. Mit npm können Module installiert, aktualisiert, entfernt und gesucht werden. Node.js liefert seit seiner Version 0.6.3 npm standardmäßig bei der Installation mit \cite{Node1.5}.

\paragraph{Express}
„Express ist ein einfaches und flexibles Node.js-Framework von Webanwendungen, das zahlreiche leistungsfähige Features und Funktionen für Webanwendungen und mobile Anwendungen bereitstellt“\cite{Node1.6}.  Es wurde im November 2010 von Douglas Christopher Wilson und weiteren Entwicklern veröffentlicht und erweitert Node.js um das Abarbeiten verschiedener HTTP-Methoden, das separate Abarbeiten von Anfragen mit verschiedenen URL-Pfaden sowie weiterer nützlicher Möglichkeiten. Im Grunde handelt es sich bei Express um ein Modul, dass durch den npm Package Manager heruntergeladen werden kann. Die aktuelle Version zum Zeitpunkt der Dokumentation [??TODO] ist 4.17.1 \footnote{\url{https://www.npmjs.com/package/express}, letzter Zugriff 04.04.2021}
\newline
\newline
\textbf{Beispiel}
 \newline

\noindent
Das Erstellen einer einfachen Express-Applikation wird im folgenden Beispiel dargestellt:\newline

\begin{lstlisting}[caption=Einfacher Webserver [nodejs 1.8],label=lst:Middleware]
const express = require('express');
const app = express();
const port = 3000;

app.get('/', (req,res)=> {
	res.send('Hello World')
});

app.listen(port, () => {
	console.log("Example app listening on port ${port}!")
});
\end{lstlisting}

\noindent
Die require()-Funktion importiert das Express-Modul und gibt ein Express-Objekt zurück. 
Dieses Objekt als Funktion aufgerufen gibt wiederum ein Objekt der Express-Applikation zurück, welche traditionell „app“ genannt wird, das Kernstück des Express-Frameworks ist und sämtliche Methoden wie das Weiterleiten von HTTP Anfragen, das Konfigurieren von Middleware oder das Modifizieren des Webserver-Verhaltens beinhaltet \cite{Node1.8}.
\newline
\noindent
Im mittleren Block befindet sich eine Routendefinition. Die app.get() Funktion spezifiziert eine Callback-Funktion, die ein „request“- und „response“-Objekt als Parameter erhält und aufgerufen wird, sobald eine HTTP Anfrage der Methode GET mit dem Pfad ‚/‘ empfangen wird. Das Request-Objekt enthält sämtliche Informationen über die HTTP-Anfrage. Das Response-Objekt kann dagegen in der Callback-Funktion mit Informationen gefüllt werden und über die send()-Funktion als HTTP-Antwort an den Sender zurückgesendet werden.
\newline
\noindent
Der unterste Block startet den Webserver auf dem mitgegebenen Port über die Funktion app.listen(). Ihr kann auch eine Callback-Funktion mitgegeben werden, die aufgerufen wird, sobald der Server erfolgreich gestartet ist.
\newpage
\noindent
\subparagraph{Middleware}
Express arbeitet nach dem Middleware-Konzept. Darunter versteht man Funktionen, die für die Verarbeitung von Anfragen hintereinandergeschaltet werden können. Jede Middleware hat Zugriff auf das Anfrageobjekt, das Antwortobjekt und die jeweils nächste Middleware-Funktion \cite{Node1.9}.
Dabei kann die HTTP-Request direkt terminiert oder an die nächste Middleware gesendet werden. Die Verkettung der Middleware-Funktionen wird in folgender Abbildung illustriert:
\newline

\begin{figure}[h]
\centering
\includegraphics[width=12cm]{images/nodejs_middleware.png}
\caption{Middleware \cite{Node1.2}}
\end{figure}

%
%       Middleware
%			express.json
%

\noindent
\subparagraph{Middleware: express.json}

\noindent
Hierbei handelt es sich um eine in express eingebaute Middleware, die die in JSON formatierten Daten im Nachrichtenrumpf aus einer eingehenden HTTP-Anfrage grammatisch analysiert.  Dabei ist zu beachten, dass der Nachrichtenrumpf nur dann analysiert wird, wenn bei der Anfrage eine Header-Informationen namens „Content-Type“ mit dem entsprechenden JSON-Typ als Wert übergeben wird. Nach erfolgreicher Analyse erstellt die Middleware aus den JSON-Informationen eine neues body-Objekt innerhalb des übergebenen request-Objekts. [nodejs 2.1]
\newline

\begin{lstlisting}[caption=Express.json Middleware benutzen,label=lst:ExpressNutzen]
const express = require('express');
const app = express();
app.use(express.json());
\end{lstlisting}

%
%       Middleware
%			Router
%

\newpage
\noindent
\subparagraph{Middleware: Router}

\noindent
Unter dem Begriff Routing (Weiterleitung) versteht man im Kontext von Express „[...] die Definition von Anwendungsendpunkten (URIs) und deren Antworten auf Clientanforderungen.“ [nodejs 2.15]
\newline

\noindent
Die in express eingebaute Middleware express.Router ermöglicht es, modular einbindbare Routenhandler (Weiterleitungsroutinen) zu erstellen. Eine Router-Instanz ist als vollständiges Middleware- und Routingsystem zu sehen und wird deshalb auch als „Mini-App“ angesehen. Der sich durch die Modularität herausziehende Vorteil ist, dass folglich unterschiedliche Anwendungsendpunkte auf entsprechende Dateien ausgelagert werden können.
\newline



\begin{lstlisting}[caption=Routinghandler erstellen \protect \footnotemark,label=lst:RoutingHandlerCreate]
var express = require('express');
var router = express.router();

// Middleware explizit fuer diesen Router
router.use(function timeLog(req,res,next) {
	console.log('Time: ', Date.now());
	next();
});

// Homepage Route - Abhandlung
router.get('/', function(req,req){
	res.send('Birds home page');
});

// About Route - Abhandlung
router.get('/about', function(req,req){
	res.send('About birds');
});
module.exports = router;
\end{lstlisting}
\footnotetext{Express, API-Dokumentation Router. \url{https://expressjs.com/en/api.html\#router}, letzter Zugriff: 05.April 2021}

\noindent
In oberem Beispiel wird ein Routerhandler für das Verzeichnis ‚/birds‘ mit eigen implementierter Middleware und zwei Anwendungsendpunkte ‚/‘ (bezieht sich auf das Stammverzeichnis) und ‚/about‘ erstellt. Der Code wird unter der Datei birds.js abgespeichert. 
Abschließend kann das Routermodul in die Anwendung geladen werden: 
\newline

\begin{lstlisting}[caption=Routinghandler benutzen,label=lst:RoutingHandlerUsage]
var birds = require('./birds');
..
app.use('birds', birds);
\end{lstlisting}

%
%       Mongoose
%
%

\newpage
\paragraph{Mongoose}
Mongoose ist ein öffentliches Modul, das zum Zeitpunkt der Dokumentation[TODO??] im npm Package Manager in der Version 5.12.3 zur Verfügung steht \footnote{npm mongoose. \url{https://www.npmjs.com/package/mongoose}, letzter Zugriff: 05.April 2021}. Bei diesem Modul handelt es sich um ein Object-Document Mapper (ODM), der es ermöglicht, asynchron mit einer NoSql-Datenbank zu kommunizieren. Mongoose ist der populärste und am weitest von MongoDB unterstützte ODM \cite{Node2.55}. Es unterstützt neben transparenter Persistenz auch die Datenvalidierung, das Erstellen von Abfragen (Queries), das Schreiben von logischem Business Code und die Übertragung zwischen Objektem im Code und der Repräsentierung dieser Objekte in der Datenbank.
\newline

%
%       ODM
%
%

\noindent
\subparagraph{Object Document Mapping (ODM)}
Object-Relational Mappers (ORM) finden haupt\-sächlich Einsatz in objektorientieren Anwendungen, dessen Daten in relationalen Datenbanken sind. Dabei werden die Tabellen in persistente Objekte gemappt.
Das Mappen ist aber auch für NoSQL-Datenbanken nützlich \cite{Node2.56}. Die meistverbreiteten NoSQL-Datenbanken basieren auf Dokument-Systemen. Dementsprechend werden für diese Datenbanken Object-Document Mapper für das Mappen zwischen Dokumenten und Objekten genutzt. Einige ODM’s sind Mongoose\footnote{Mongoose Webpage, \url{http://mongoosejs.com}, letzter Zugriff 04.04.2021}, Morphia\footnote{Morphia Webpage, \url{https://github.com/mongodb/morphia}, letzter Zugriff 04.04.2021}, Doctrine \footnote{Doctrine Project Webpage, \url{http://www.doctrine-project.org/}, letzter Zugriff 04.04.2021} und Mandango\footnote{Mandango Webpage, \url{https://mandango.readthedocs.io/en/latest/}, letzter Zugriff 04.04.2021.}
NoSQL Mapper nutzen vom Entwickler definierte Datenschemata, die das Objekt beschreiben. Ein daraus abgeleitetes Model-Objekt ermöglicht dann die Kommunikation zwischen dem im Schema beschriebenen Objekt und der entsprechenden Datenbank-Collection.
\newline

%
%       Schema
%
%

\noindent
\subparagraph{Schema}
Mongoose-Schemata definieren die Struktur der gespeicherten Daten einer Mongo"-DB-"-Collection in der Anwendungsschicht und werden in der JSON-Notation beschrieben. Dokumentenbasierte Datenbanken wie MongoDB enthalten für jede Wurzelentität eine Collection. Mongoose Schemata werden für jede Collection definiert. Innerhalb der JSON-notierten Schemabeschreibung können den einzelnen Eigenschaften bestimmtes Verhalten zugeordnet werden. Zum Beispiel lässt sich explizit der Datentyp angeben (type), eine Eigenschaft verpflichtend (required) oder in Kleinbuchstaben einstellen (lowercase).
\newline


\begin{lstlisting}[caption=Mongoose Schema - Beispiel,label=lst:MongooseSchema]
const schema = new Schema({
 attributeX: {
 	type: String,  // Datentyp
 	required: true,  // Verpflichtendes Attribut?
 	lowercase: true; // Kleinbuchstaben?
});
\end{lstlisting}

%
%       Model
%
%

\newpage
\noindent
\subparagraph{Model}
Ein Model in Mongoose ist ein aus einer Schemadefinition erstellter Konstruktor, aus denen Objekte instanziiert werden können. Diese Instanzen werden auch ‚documents‘ genannt. Sie stehen in direkter Verbindung zu den jeweiligen Collections der verbundenen Datenbank und enthalten Methoden für die persistente Speicherung, Bearbeitung oder Löschung. Beispielsweise wird beim Abspeichern einer Mongoose Instanz eines Models die entsprechende Collection in der Datenbank erzeugt, sofern sie noch nicht vorhanden ist. Eine Konvention in Mongoose sieht vor, dass der Name eines Models dem Singular eines Nomens entspricht, während die Collections nach dem Plural dieses Namens beschrieben werden \cite{Node3.2}. Im folgenden Beispiel wird ein Model über die mongoose.model()-Funktion erstellt unter Angabe des Modelnamens und dem zu verwendenden Schema. Dieses Model wird über module.exports nach außen zur Verfügung gestellt.

\begin{lstlisting}[caption=Model erstellen und exportierenn,label=lst:MongooseObjectExport]
const mongoose = require('mongoose');
const testSchema = new mongoose.Schema({
	attributeX: {
		type:String,
		required:true,
		lowercase: true
	}
});
module.exports = mongoose.model('test',testSchema);
\end{lstlisting}

\noindent
An anderer Stelle kann das Model nun importiert werden. Aus dem Model kann ein Objekt instanziiert werden, welches über die save()-Funktion in der Datenbank gespeichert werden kann.

\begin{lstlisting}[caption=Model importieren - Objekt instanziieren und persistent speichern,label=lst:MongooseObjectInstance]
const testModel = require(test);

var testInstanz = new testModel();
await testInstanz.save();
\end{lstlisting}

\noindent
Mongoose Models enthalten ohne Instanziierung des Weiteren auch Schnittstellen, um Daten der zugehörigen Collection zu kreieren, abfragen, bearbeiten oder löschen. (Create, Receive, Update, Delete oder auch kurz CRUD).
\newline


\begin{lstlisting}[caption=CRUD-Beispielfunktionen eines Mongoose-Models,label=lst:MongooseCrud]
const testModel = require(test);

//Create
testModel.Insert({attributeX: "abc"})
//Receive
var testObjects = await testModel.find();
var testObject = await testModel.findOne({attributeX: "abc"})
//Update
await testModel.updateONe({X:"abc"},{X: "cba"});
//Delete
await testModel.deleteMany({X:"abc"})
\end{lstlisting}

\newpage
\noindent
\subparagraph{Verbindung}
Verbindung zur Datenbank kann über die connect()-Funktion mit Angabe der genutzten Datenbank und des Datenbankpfads hergestellt werden. 
Über das mongoose.connection-Objekt können auf Verbindungsereignisse reagiert werden. 
\newline

\begin{lstlisting}[caption=Mongoose: Verbindung zur Datenbank aufbauen,
label=lst:MongooseConnect]
const mongoose = require('mongoose');
await mongoose.connect("mongodb://127.0.0.1:27017/TestDB");
mongooose.connection.on('error',(error) => console.log(error));
mongooose.connection.on('open',() => console.log('Connected'));
\end{lstlisting}

\noindent
Für den Verbindungsaufbau können weitere Option übergeben werden. Dafür kann ein Objekt wie in folgendem Beispiel erstellt werden, dass die zugehörigen Optionen als Attribute beinhaltet. 
\newline

\begin{lstlisting}[caption=Mongoose Verbindungsoptionen \protect \footnotemark  ,label=lst:MongooseConnect]
const options = {
	useNewUrlParser: true,
	useUnifiedTopology: true,
	useCreateIndex: true,
	useFindAndModify: false,
	autoIndex: false,
	poolSize: 10, // Anzahl der max. Socket Connections
	serverSelectionTimeoutMS: 5000, // TimeOut bis verbunden
	socketTimeoutMS: 45000, // Schliesse Socket bei 45s                 
	                        // Inaktivitaet
	family:4 // Use IPv4
}
\end{lstlisting}
\footnotetext{Mongoose Connections, \url{https://mongoosejs.com/docs/connections.html}, letzter Zugriff: 05.April 2021}


\paragraph{Weitere Module}

\begin{table}[tbt]
\caption{Module}
\begin{center}
    \begin{tabular}{| l | p{8cm} |}
    \hline
    Express-Modul & Beschreibung \\ \hline
    fs & Erlaubt die Interaktion mit dem Dateisystem.\newline
	Zum Beispiel Schreiben/Lesen von Dateien.\\
    
    \hline
    http & Ermöglicht Datentransfer über das Protokol HTTP und das Abhören eines Ports.  \\
    
    \hline
	https & Gesicherte Variante zu HTTP mit SSL.\newline
	Benötigt Private Key und Zertifikat.  \\
	
    \hline    
    firebase-admin & Ermöglicht die Verbindung zu Google Firebase 			Cloud. \\ 
    
    \hline    
    node-cron & Ermöglicht das Einstellen von sich wiederholenden 			Aufgaben zu bestimmten Zeitintervallen.  \\
    \hline
    \end{tabular}
\end{center}
\end{table}


\newpage

\subsection{Language}
%Hier steht mein Language Text.

\subsubsection{JavaScript}
In den nächsten Unterkapiteln soll ein zunächst ein historischer Überblick über die Programmiersprache JavaScript gegeben werden. Im Anschluss wird auf die Bedeutung und Nutzung von JavaScript eingegangen. 

\subsubsection{Historie TODO }
Ihren Ursprung findet JavaScript im Jahr 1995, als Brendan Eich, ein damaliger Ingenieur des US-amerikanischen Software-Unternehmens „Netscape Communications Corporation“, innerhalb von zehn Tagen die Sprache für den Browser „Netscape Navigator“ entwickelt hat. [1] Das Ziel dabei war es, eine Skriptsprache zu entwickeln, die es Entwicklern möglich machen sollte, auf ihren Webseiten Skripte umzusetzen. Zunächst noch unter dem Namen Mocha und LiveScript änderte sich der Name aufgrund der Kooperation von Netscape und Sun, der Firma hinter der Programmiersprache Java, und der damaligen Popularität von Java zu JavaScript. [1.05] 
Netscape’s Veröffentlichung des Netscape Navigator 2.0, der erste Browser der JavaScript unterstütze, brachte Microsoft dazu, Netscape als ernstzunehmenden Konkurrenten zu sehen. 
Microsoft antwortete im August 1995 mit der Veröffentlichung des ersten Internet Explorer zusammen mit der Skriptsprache JScript, die einen Dialekt der Sprache JavaScript darstellt. Dies ist ferner als Beginn der „Browserkriege“ bekannt. [1.06]
 Im Jahre 1997 reichte Netscape JavaScript an die European Computer Manufacturers Association, einer privaten, internationalen Normungsorganisation zur Normung von Informations- und Kommunikationssystemen und Unterhaltungselektronik (kurz ECMA[ABK]) ein. Das Ziel war es, von der ECMA einen einheitlichen Standard für die Sprache schaffen zu lassen, die fortan weiterentwickelt und von weiteren Browserherstellern genutzt werden soll. Das resultierende Standard nennt sich ECMAScript, wobei JavaScript die bisher bekannteste Implementierung dieses Standards ist. [1.07] Andere Implementierungen sind zum Beispiel ActionScript von MacroMedia, JScript von Microsoft und ExtendScript von Adobe.
Jährlich wird der Standard seit Juni 2015 erweitert. ECMAScript Version 11 beziehungsweise ECMAScript 2020 bildet zum Zeitraum dieser Dokumentation [??] den aktuellen Standard. [1.08] 
Im Juni 2021 soll ECMAScript 2021 veröffentlicht werden.  [1.09]

\subsubsection{Wesentliche Programmiereigenschaften TODO}
„JavaScript is Not Java“ [1.091 ??]. Die Programmiersprache JavaScript wird aufgrund ihrer Namensgebung oft in falsche Zusammenhänge zu Java gebracht. Das häufigste Missverständnis sei, JavaScript wäre eine vereinfachte Version von Java. [1.091]
JavaScript ist eine interpretierte Programmiersprache mit objektorientierten Umsetzungs-möglichkeiten. Interpretation ist in diesem Zusammenhang so zu verstehen, dass der Quellcode zur Laufzeit eines Programms gelesen, übersetzt und ausgeführt wird.
Syntaktisch ähnelt JavaScript Programmiersprachen wie C, C++ und Java durch gleiche Umsetzung der Programmierkonstrukte wie den Bedingungen, Schleifen oder den booleschen Operatoren. [1.1] Wesentliche Unterschiede sind dagegen, dass JavaScript zum einen eine schwach-typisierte Sprache ist. Durch die schwache Typisierung haben Variablen keinen festen Dateityp und können diesen dynamisch zur Laufzeit ändern. Des Weiteren findet bei JavaScript die Objektorientierung prototypenbasiert statt. Diese Form der Programmierung wird auch klassenlose Objektorientierung bezeichnet. Anders als bei der klassenbasierten Programmierung, bei der Objekte aus vordefinierten Klassen instanziiert werden, werden hier Objekte durch Klonen bereits existierender Objekte erzeugt. Die Objekte, die geklont werden, sind dabei als Prototyp-Objekte zu verstehen. Beim Klonen werden alle Attribute und Methoden des Prototyp-Objekts in das neue Objekt übernommen und können dort überschrieben sowie erweitert werden. Objekte in JavaScript sind eher als Zuordnungslisten, ähnlich wie assoziative Arrays oder Hash-Tabellen, anzusehen, da bei der Eigenschaftszuweisung lediglich ein Mapping eines Namens zu seiner zugehörigen Eigenschaft stattfindet. Ein weiterer Unterschied zu den anderen Programmiersprachen ist, dass alle Funktionen und Variablen außer der primären Datentypen Boolean, Zahl und Zeichenfolge, als Objekte verstanden werden können.

\subsubsection{Anwendungsgebiete TODO}
Ursprünglich fand JavaScript seinen Einsatz hauptsächlich darin, dynamische Webseiten im Web-browser anzuzeigen. Die Verarbeitung erfolgte dabei meist clientseitig durch den Webbrowser (dem sogenannten Frontend). [1.3] 
Heutzutage findet sich die Sprache dagegen in wesentlich größeren Einsatzgebieten wieder. 
Bis vor einigen Jahren war die Serverseite anderen Programmiersprachen wie Java oder PHP vorbehalten. Die Veröffentlichung von Node.js, einer plattformübergreifenden Laufzeitumgebung, die JavaScript außerhalb eines Webbrowsers ausführen kann, führte zu einer immer größeren Verbreitung von serverseitigen Anwendungen (dem Backend), die auf JavaScript basieren. Auf Node.js wird ausführlicher im nächsten Kapitel eingegangen. 
Ferner findet JavaScript heutzutage aber auch seinen Einsatz in mobilen Anwendungen, Desktopanwendungen, Spielen oder 3D-Anwendungen.




\subsection{IDE}
%Hier steht mein IDE Text.


\subsection{Database}
%Hier steht mein Database Text.


\subsection{Firebase}
%Firebase ist eine Backend-as-a-Service (BaaS) Plattform von Google für mobile oder Web-Anwendungen. 
Sie soll es dem Entwickler ermöglichen, einfacher und effizienter Funktionen auf verschiedenen Plattformen bereitzustellen stellt Tools und Infrastruktur zur Verfügung.
Mit dem Firebase SDK bietet die Plattform API Schnittstellen zu den jeweiligen Tools, welche direkt in die Anwendung integriert werden können, ohne dass serverseitiger Code dafür notwendig ist.
Die Firebase Inc. wurde 2011 von James Tamplin und Andrew Lee gegründet und letztendlich 2014 von Google übernommen.\footnote{\href{https://firebase.googleblog.com/2014/10/firebase-is-joining-google.html}{firebase.googleblog.com}, zuletzt aufgerufen am 03.05.2021}
Teile der SDK stehen seit der Google I/O 2017 unter der Apache 2.0 Lizenz, sind somit also Open-Source.\footnote{\href{https://opensource.googleblog.com/2017/05/open-sourcing-firebase-sdks.html}{opensource.googleblog.com}, zuletzt aufgerufen am 03.05.2021}\\
\\
Ein Firebase Projekt ist die oberste Ebene in Firebase. 
Ein Projekt ist letztendlich ein \textit{Google Cloud Projekt}, welches mit speziellen Konfigurationsmöglichkeiten und Services ausgestattet ist. 
Es beinhaltet die Verknüpfung zu den einzelnen Anwendungen (also bspw. Android-, iOS- oder Webanwendung). Nun können variabel Tools, sog. Firebase products hinzugefügt werden. Diese Produkte lassen sich grundlegend in drei Kategorien einteilen. Die hier relevantesten werden im Folgenden besprochen.\cite{firebase2021}

\subsubsection{Firebase Authentifizierung}
Die Authentifizierung gehört zu den \glqq Build\grqq Produkten und bietet eine Token-basierte Nutzerauthentifizierung. 
Hierbei kann zwischen verschiedenen Anmeldeoptionen gewählt werden: klassisch mit E-Mail und Passwort, mit OAuth2.0 Integration für Social Media (Google, Facebook, Twitter, Github, ...) oder per Telefonnummer.
Jeder Nutzer erhält eine einzigartige ID und ein zugehöriges Nutzerobjekt in einer NoSQL Datenbank. Grundlegende Werte wie E-Mail Adresse oder Name können hier abgespeichert werden; zusätzliche Informationen müssen über einen weiteren Datenbank Service abgespeichert werden.
Für die Verwaltung eines Accounts bietet dieses Tool auch eingebaute E-Mail Aktionen an - bspw. Passwort zurücksetzen oder E-Mail Adresse bestätigen.\\
\\
Ein Firebase Nutzer Objekt repräsentiert den Account eines Nutzers, welcher sich von einer Anwendung aus beim zentralen Firebase Projekt angemeldet hat.
Die Instanz eines Firebase Nutzers ist somit unabhängig von der Authentifizierungsinstanz der Anwendung, also kann eine Anwendung mehrere Nutzer anmelden, jedoch kann sich auch ein Nutzer auf mehreren Anwendungen anmelden.
Ist ein Nutzer authentifiziert, erhält die Anwendung eine Referenz des Nutzers, welche so lange existiert, bis er wieder abgemeldet ist.

\subsubsection{Firestore}
\label{sssec:firestore}
Als Datenbank Lösung bietet Firebase zwei unterschiedliche Produkte an: Firestore und Realtime Database.
Firestore ist hier neuer, jedoch ersetzt es Realtime Database nicht. \\
Firestore ist eine flexible und auf Skalierung ausgesetzte NoSQL Cloud Datenbank, welche unter anderem die Echtzeitsynchronisierung der Daten zwischen Anwendung und Server ermöglicht.
Zusätzlich zu REST und RPC APIs in iOS, Android und web SDKs ist Firestore auch in nativen Node.js, Java, Python und Go SDKs verfügbar.\\
\\

\begin{wrapfigure}{R}{0.4\textwidth}
	\begin{center}
		\includegraphics[width=0.35\textwidth]{images/firestore_datastucture.png}
	\end{center}
	\caption{Datenmodell in Firebase \protect \footnotemark}
	\label{fig:firestore_data_structure}
\end{wrapfigure}
\footnotetext{Quelle: \cite{firebase2021}}

Das Datenmodell ist hierarchisch aufgebaut, wobei Daten in Dokumenten (documents) und Dokumente in Sammlungen (collections) gespeichert sind. 
Mithilfe von Sammlungen werden die Daten voneinander abgetrennt und hierüber können Abfragen erstellt werden.
Grundlegende Datentypen sind String, Integer und Boolean, jedoch können auch komplexe Datentypen wie Maps, Arrays oder Geopoints. Unter-Sammlungen und darin verstaute Dokumente sind ebenfalls möglich.\\
\\
Abfragen werden auf Dokumentenebene erstellt, damit nicht eine gesamte Sammlung aufgerufen werden muss.
Dies kann über direkte Sortierung, Filter und/oder Limitierung bzw. genaue Auswahl eines Dokumentes bewerkstelligt werden.
Bei einer Abfrage erhält man einen \textit{Data Snapshot}, wodurch über Änderungen in Echtzeit informiert und diese angezeigt werden können.
Damit es jedoch zu keinen fehlerhaften Daten führt, gelten hier atomare Eigenschaften für Transaktionen.
Eine Transaktion ist eine Folge von Datenbankanweisungen, welche entweder alle gemeinsam oder gar nicht ausgeführt werden. 
Eine Transaktion ist nur dann erfolgreich, wenn alle Anweisungen auf eine Datenbank vollständig geschlossen sind. 
Ist dies nicht der Fall, werden alle Anweisungen bis zum Stand vor der Transaktion rückgängig gemacht. Das nennt man Rollback.\\
\\
Die Sicherheit der Daten stellt Cloud Firestore für Mobil- und Webclient-Bibliotheken über die Firestore-Sicherheitsregeln her. Diese bieten sowohl Zugriffsverwaltung und -authentifizierung, jedoch könne auch Daten hiermit für die Konsistenz der Datenbank validiert werden. 
\medskip
\begin{lstlisting}[caption=Beschränkung des Zugriffs auf Dokumente der Sammlung \texttt{cities}, label=lst:firestorerules_basic]
	service cloud.firestore {
		match /databases/{database}/documents {
			match /cities/{city} {
				allow read, write: if request.auth != null;
			}
		}
	}
\end{lstlisting}
\medskip
Im Beispiel \ref{lst:firestorerules_basic} wird der Lese- und Schreibzugriff auf ein Dokument der Sammlung \texttt{cities} beschränkt. 
Nur falls der anfragende Nutzer eine valide Authentifizierung besitzt, erhält er Zugriff auf das angefragte Dokument. 
Diese simple Darstellung ist jedoch für den wirklichen Produktionseinsatz mit Vorsicht zu nutzen. 
Oftmals müssen \texttt{read} und \texttt{write} in detailliertere Vorgänge aufgeteilt werden. Ein \texttt{read} wird spezialisiert in \texttt{get} und \texttt{list}, wobei ein \texttt{write} in \texttt{create}, \texttt{update} und \texttt{delete} unterteilt werden kann.
Ein \texttt{list} ermöglicht es hierbei auf Sammlungen, also die einzelnen Dokumenten IDs lesend zuzugreifen, jedoch nicht auf die Daten einzelner Dokumente. Hierfür wird dann ein \texttt{get} benötigt. 
Mittels \texttt{create} erhält man Schreibzugriff auf nicht existierende Dokumente, durch \texttt{update} auf bereits vorhandene und Löschrechte ganzer Dokumente erhält man über den \texttt{delete} Operator.\\
\\
Sicherheitsregeln werden gleich dem Datenmodell hierarchisch aufgebaut und ermöglichen differenzierte Zugriffsbeschränkungen auf jeder Ebene.
In Codebeispiel \ref{lst:firestorerules_hierarchy} beinhaltet jedes Dokument (Stadt) der Sammlung \texttt{cities} eine Unter-Sammlung \texttt{landmarks}. Nun lässt sich der Zugriff auf beide separat regeln.
Bei der Sammlung \texttt{villages} hingegen wurde der rekursive Platzhalter verwendet. Hiermit sind Zugriffsregeln auf allen tieferen Ebenen gleich.
Beim Verschachteln von \texttt{match} ist der innere Pfad immer relativ zum äußeren.

Wichtig zu wissen ist hierzu noch, dass falls mehrere \texttt{allow} Ausdrücke auf eine Anfrage zutreffen, wird der Zugriff erlaubt sobald \textbf{eine} Bedingung wahr, also erfüllt ist.

\medskip
\begin{lstlisting}[caption=Hierarchische Zugriffsbeschränkung, label=lst:firestorerules_hierarchy]
	service cloud.firestore {
		match /databases/{database}/documents {
			match /cities/{city} {
				allow read, write: if <condition>;
				
				// Explicitly define rules for the 'landmarks' subcollection
				match /landmarks/{landmark} {
					allow read, write: if <condition>;
				}
			}
			match /villages/{document=**} {
				allow read, write: if <condition>;
			}
		}
	}
\end{lstlisting}
\medskip

Wie bereits oben besprochen können diese Regeln auch zur Validierung von Daten genutzt werden, damit die atomare Eigenschaft von Transaktionen bestehen bleibt.
Hierzu kann die \texttt{getAfter()} Funktion genutzt werden. 
Mit dieser kann man auf Zustand eines Dokumentes zugreifen und diesen validieren, nachdem einer Folge von Anweisungen ausgeführt, jedoch diese noch nicht auf der Firestore Datenbank abgeschlossen wurde.
Im Beispiel \ref{lst:firestorerules_validation} existieren zwei Sammlungen: \texttt{cities} und \texttt{countries}. 
Jedes \texttt{country} Dokument beinhaltet das Feld \texttt{last\_updated} um zu wissen, welche Stadt innerhalb eines Landes zuletzt aktualisiert wurde.
Hierzu wird in den Sicherheitsregeln nach jedem Schreibzugriff auf ein \texttt{city} Dokument gleichzeitig auch das Feld des zugehörigen Landes aktualisiert.\cite{firebase2021}
\medskip
\begin{lstlisting}[caption=Datenvalidierung für atomare Operationen, label=lst:firestorerules_validation]
	service cloud.firestore {
		match /databases/{database}/documents {
			// If you update a city doc, you must also
			// update the related country's last_updated field.
			match /cities/{city} {
				allow write: if request.auth != null &&
				getAfter(
				/databases/$(database)/documents/countries/$(request.resource.data.country)
				).data.last_updated == request.time;
			}
			
			match /countries/{country} {
				allow write: if request.auth != null;
			}
		}
	}
\end{lstlisting}
\medskip

\subsubsection{Storage}
Um Filme, Videos oder andere Nutzer-generierte Inhalte abspeichern zu können, bietet Firebase Cloud Storage an. 
Durch das Firebase SDK für Cloud Storage können Dateien direkt von Client-Anwendungen hoch- bzw. heruntergeladen werden.
Aufgrund von möglicher schlechter Verbindung kann mithilfe von robusten Operationen der Prozess des Hoch- bzw. Herunterladens bei besserer Verbindung an der Stelle weiter geladen werden, an welcher dieser unterbrochen wurde.
Ähnlich wie bei Cloud Firestore in Kapitel \ref{sssec:firestore} bestimmen auch hier Sicherheitsregeln den Zugriff auf bestimmte Dokumente.\\
Zusätzlich hierzu sind weitere Metadaten verfügbar: \texttt{contentType} und \texttt{size}. 
Mit ihnen lassen sich die Dateien beispielsweise validieren.
Im Code \ref{lst:storagerules_validation} können Dateien nur hochgeladen werden, falls sie eine Größe kleiner 5 MB besitzen.
\medskip
\begin{lstlisting}[caption=Validierung nach Dateigröße, label=lst:storagerules_validation]
	service firebase.storage {
		match /b/{bucket}/o {
			match /images/{imageId} {
				allow write: if request.resource.size < 5 * 1024 * 1024
				&& request.resource.contentType.matches('image/.*');
			}
		}
		
\end{lstlisting}
\medskip
Außerdem lassen sich die 
\subsubsection{Cloud Functions}
\subsubsection{Analytics}
\subsubsection{Google AdMob, Google Ads}

\subsection{Recommendationssystem}
%\url{http://www.microlinkcolleges.net/elib/files/undergraduate/Photography/504703.pdf}	
		

\section[Konzept]{Konzept? \hfill \normalfont \small{Autor-Name}}
\section[Funktionen/Komponenten]{Funktionen/Komponenten \hfill \normalfont \small{Autor-Name}}

\subsection{Swipe/Aussuchen/Voting}		

\subsection{Matches/Chat}		

\subsection{Film-/Serienvorschläge}		

\subsection{Gruppenorgien}		

\subsection{Gespeicherte Filme/Filmliste}		

\subsection{Barrierefreiheit}
\label{sec:barrierefreiheit}

Barrierefreiheit im Allgemeinen bedeutet, dass ein Gegenstand, eine Einrichtung oder Informationsquelle für Menschen mit Behinderung ohne Unzulänglichkeiten nutzbar, zugänglich oder auffindbar ist (\cite{behindertengleichstellungsgesetz}, §4). In der Softwareentwicklung versteht man darunter Applikationen für Menschen mit Einschränkungen zugänglich und bedienbar zu machen. Bezogen auf die Entwicklung von  mobilen Apps gilt es dabei den akustischen, optischen oder motorischen Einschränkungen der Benutzer entgegenzuwirken. \\


\subsubsection{Barrierefreiheit in mobilen Anwendungen}
Mit der Verbreitung von Smartphones ist die Benutzung mobiler Apps stark angestiegen und mittlerweile in nahezu jedem Haushalt aufzufinden. Obwohl etwa 9,5\% aller in Deutschland lebenden Menschen einen Schwerbehindertenausweis besitzen (Stand 24.06.2020)\cite{schwerbehindertenausweis} was etwa 7,9 Millionen Menschen entspricht, ist die Implementierung von barrierefreier Bedienung nicht selbstverständlich. Gerade Programmierern/innen aus dem privaten Sektor sind diese Funktionen oft nicht bekannt, es besteht kein Interesse oder sie werden schlichtweg vergessen. Software, die für öffentliche Einrichtungen entwickelt wird, ist durch das Behindertengleichstellungsgesetz von 2002 dazu verpflichtet ihr Softwareangebot bis spätestens dem 23. Juni 2021 barrierefrei zu gestalten (\cite{behindertengleichstellungsgesetz}, §12a Abs.1). Hierzu zählen sämtliche Webseiten sowie mobile Anwendungen. \\


\subsubsection{Barrierefreiheit in Filmen und Serien}
Auch die Zugänglichkeit von Filmen und Serien für Menschen mit eingeschränkter Wahrnehmung wurde in den letzten Jahren stark verbessert. 
Hierbei lässt sich zwischen optischer und akustischer Einschränkung differenzieren. Für hörgeschädigte Personen werden bereits seit mehreren Jahrzehnten Untertitel eingesetzt. Was früher für vereinzelte Filme durch eine Funktion des Teletextes erreicht wurde, wird heutzutage durch eine integrierte Funktion des Videoplayers verwirklicht. Immer mehr Videos werden mit Untertiteln veröffentlicht. Manche Anbieter wie beispielsweise die Internetplattform YouTube bieten durch Spracherkennung automatisch generierte Untertitel an, was eine flächendeckende Untertitelung ermöglicht.\\
Auch für Menschen mit eingeschränktem Sehvermögen werden Filme und Serien mithilfe von Audiodeskriptionen vermehrt zugänglich gemacht. Hierbei wird die bereits vorhandene Tonspur mit Bildbeschreibungen und Kommentaren versehen. Was bis vor wenigen Jahren noch etwas Besonderes war und nur für ausgewählte Filme bestimmt war, ist heutzutage Standard. Größere Video-On-Demand-Plattformen wie Netflix oder Amazon Prime bieten diese Möglichkeit bei nahezu allen Eigenproduktionen an. Zusätzlich werden bestehende Filme neu mit Audiodeskriptionen versehen.\\


\noindent Hieraus lässt sich leicht erkennen, dass Filme und Serien heutzutage auch von Menschen mit Einschränkungen genutzt werden. Was auf den ersten Blick vielleicht nicht bedacht wird oder als  unwichtig abgestempelt wird, kann einen nicht unerheblichen Vergrößerungsfaktor für den Kundenstamm bewirken. Für die Entwicklung einer mobilen App, bei der Filme und Serien bewertet werden, spielt also die Barrierefreiheit eine wichtige Rolle und darf auf keinen Fall vernachlässigt werden. 


\subsubsection{Barrierefreiheit bei StreamSwipe}
\label{sec:bf-streamswipe}

Bei der Entwicklung von StreamSwipe werden mehrere mögliche Einschränkungen der User betrachtet und entsprechend reagiert. Ziel ist es, dass sowohl der Kunde sowie der Anbieter maximal davon profitieren. Hierfür soll die App für ein möglichst großes Publikum zugänglich gemacht werden, jedoch auch sogenanntes Over-Engineering vermieden werden, da zu viele Funktionen eine App unübersichtlich, teuer und langsamer werden lassen.\\

\noindent
Allgemein wird Leserlichkeit durch große Schriftgrößen, hohe Farbkontraste, große Schaltflächen oder universelles Design erreicht. Alleine in Deutschland tragen 44,5 Millionen Menschen regelmäßig eine Brille oder Kontaktlinsen und benötigen somit Sehhilfen \cite{sehhilfen}. Unterstützung auf Seiten der App kann hierfür durch vergrößerbaren Text geschehen. Da aber davon ausgegangen werden kann, dass Personen, die sich auf Sehhilfen verlassen, bereits eine Brille oder Kontaktlinsen besitzen, wird die Textgröße vorerst nicht variabel gehalten. Außerdem gibt es bei Android- und Apple-Smartphones bereits eingebaute Vergrößerungsfeatures, die Bildausschnitte vergrößert darstellen können. Aus diesem Grund wird in diesem Projekt kein Fokus auf dieses Feature gelegt. \\
Farbblindheit kann jedoch in vielen Formen auftreten. Um der bekannten Farbfehlsicht entgegenzuwirken, werden Farben aus Problembereichen wie Rot und Grün nicht nebeneinander benutzt. Allgemein wird ein schlichtes Design gewählt und Farben nur zu Akzentuierung und als Stilmittel benutzt, statt als Informationsträger wie beispielsweise in den Abbildungen \ref{fig:bf-beispiel_a} erkennbar ist.  Geringe Sehschärfe durch Achromatopsie kann wie weiter oben beschrieben umgangen werden.\\


\begin{figure}[tbt]
	\begin{subfigure}{0.5\textwidth}
	\centering
	\includegraphics[scale=0.15]{Barrierefreiheit/images/bsp-swipe.png}
	\caption{}
	\label{fig:bf-beispiel_a}
	\end{subfigure}
	\begin{subfigure}{0.5\textwidth}
	\centering
	\includegraphics[scale=0.15]{Barrierefreiheit/images/bsp-profil.png}
	\caption{}
	\label{fig:bf-beispiel_b}
	\end{subfigure}
\caption{Screenshots aus der App StreamSwipe als Beispiele zu (a) schlichtem Design, bei dem farbige Akzente nicht der Informationenübertragung dienen um die Zugänglichkeit für farbblinde Menschen zu verbessern und für einen Icon in (b), welcher sonst durch sehgeschädigte Menschen nicht wahrnehmbar ist, wird exemplarisch eine Semantik implementiert.}
\label{fig:BF-Beispiele}
\end{figure}


\noindent
Ist die Sehkraft noch weiter eingeschränkt oder gar nicht mehr vorhanden, werden Semantiken eingesetzt. Hierbei erhält jedes Element auf dem Bildschirm eine Beschreibung, die vorgelesen werden kann. Bei Zahlen und Texten werden diese vorgelesen, sofern keine weitere Information hinterlegt ist. Besonders hilfreich ist dies jedoch bei Abbildungen. Ausgeführt wird das Auslesen von einem Screenreader. Mobile Geräte haben diese Funktion bereits standardmäßig eingebaut (VoiceOver bei Apple und TalkBack bei Android) und wandeln die Semantiken mittels Sprachsynthese in akustische Signale um. Bei Desktopanwendungen wie z.B. JAWS für Windows können diese Informationen zusätzlich auch durch eine Braillezeile wiedergegeben werden.\\
Bei Flutter ist das Hinzufügen von Semantiken bereits eingebaut. Hierfür kann ein String dem jeweiligen Bereich zugeordnet werden. In Beispiel \ref{lst:semantics} ist hierfür der Code des Buttons, der zu den Einstellungen führt. In Abbildung \ref{fig:bf-beispiel_b} ist dieser Button ganz rechts oben im Eck zu sehen.\\
Der GestureDetector erkennt Interaktionen mit dem Touchscreen, wobei hier nur auf Antippen reagieren soll, deshalb die Funktion onTap:()$\{\}$, die auf den Einstellungsbildschirm leitet. Diese Implementierung ist hier aber nicht von Relevanz und wird übersprungen. In dem GestureDetector ist ein Icon eingebettet, von der Form \textit{Settings}, was einem Zahnrad entspricht. Dieses Icon erhält eine Farbe und anschließend eine Semantik aus allem was in den Anführungszeichen steht. Ein Screenreader kann AE erkennen und ihn als den Umlaut Ä aussprechen. \\
So wird im kompletten Programm für jedes relevante Element vorgegangen. Teilweise  müssen den Semantiken Variablen übergeben werden, da sich die vorzulesende Information ändert wie beispielsweise bei den Filmtiteln.
    
\begin{lstlisting}[caption=Codeausschnitt in Dart von einem Button mit Semantiken.,label=lst:semantics]
GestureDetector(
  onTap: () {
     ...
  },
  child: Icon(
    Icons.settings,
    color: Provider.of(context).colors.textSmall,
    semanticLabel: "Einstellungen. Zum Auswaehlen doppeltippen.",
  )
),
\end{lstlisting}

\noindent
Bei einer sauberen Implementierung wird auf diese Weise vorgegangen und eine bereits vorhandene Funktion verwendet. Dies vereinfacht nicht nur die Leserlichkeit des Codes, sondern bietet auch die höchste Modularität, da hierbei normalerweise standardisierte Schnittstellen für Betriebssysteme oder andere Anwendungen verwendet werden. In diesem Fall  müssen die Screenreader von Android und Apple damit arbeiten können.\\

\noindent 
Um für Personen mit eingeschränktem Hörvermögen oder vollständiger Gehörlosigkeit die App zugänglich zu machen, wird auf akustisches Feedback als notwendige Infor"-mations"-über"-tragung verzichtet. Innerhalb der App werden keine Geräusche erzeugt, außer der oben beschriebenen Funktion der Semantiken. Beim Erhalten einer neuen Nachricht oder eines neuen Matches kann weiterhin optional eine akustische Benachrichtigung erhalten werden. Hierbei wird die betriebssystemeigene Funktion übernommen, sodass in der App keine neuen Einstellungen vorgenommen werden müssen.\\

\noindent
Auch feinmotorische Einschränkungen werden versucht zu umgehen. Die Navigation und die Filmbewertung in StreamSwipe können durch großflächige Wischbewegungen ausgeführt werden. Wo diese Lösung nicht möglich ist, werden verhältnismäßig große Buttons eingesetzt. Lediglich beim Registrieren und Einloggen werden feine Bewegungen erfordert. Hierbei öffnet sich allerdings die als Standard eingestellte digitale Tastatur, die in vielen Fällen eine Spracheingabe besitzt, sodass die sehr kleinen Tasten nicht benutzt werden müssen.\\

%TODO Diesen Satz evtl. ganz ans Ende ins Fazit
\noindent
Sollte sich in Zukunft jedoch Kritik in Form von negativen Nutzerbewertungen herauskristallisieren, kann eines der noch nicht implementierten Features über ein Update nachgerüstet werden.


\section[Benutzeroberflächen]{Benutzeroberflächen \hfill \normalfont \small{Vincent Schreck}}
Die Benutzeroberfläche einer Software muss im Grunde genommen nur einen Informationsfluss in zwei Richtungen erzeugen. Die eine Richtung liefert Informationen an den Nutzer und über die andere kann der Nutzer Informationen an das System weitergeben. Um auf dem heutigen Markt Fuß fassen zu können, sollte eine Oberfläche jedoch wesentlich mehr Aspekte erfüllen. 




\subsection{Aspekte von Benutzeroberflächen}
\label{sec:UI-Aspekte}
Die Vielschichtigkeit einer Benutzeroberfläche kann ausschlaggebend für den Erfolg einer Applikation sein, abhängig davon welche Erfahrungen der Nutzer mit der Oberfläche macht und welche Eindrücke sie hinterlässt. Hieraus resultiert wie lange ein Nutzer auf der App bleibt und wie oft er zurück kommt. Neben der Nutzungszeit erhöht eine positive User Experience die Weiterempfehlungsrate.\\
Bei erfolgreicher Software besteht ein großer Teil der Entwicklung in der Planung der Oberfläche, da der Nutzer immer eine User Experience erlebt und mit einem Gefühl verbindet. Neben den offensichtlichen Aus- und Eingabefunktionen werden beispielsweise folgende Kriterien  betrachtet:\\

\noindent
\hangindent1cm
\textbf{Simpel:} Ausgegebene Information kann zum Beispiel durch Icons, Farben oder Symbole vereinfacht werden. Eine Oberfläche sollte weder überladen sein, noch sollten alle Ein- und Ausgaben auf verschiedenen Seiten verteilt sein. Bei der Entwicklung wird eine gesunde Mischung aus maximaler Funktionalität und einfacher, übersichtlicher Darstellung angestrebt.\\

\noindent
\hangindent1cm
\textbf{Einheitlich:} Die Bedienung und das Lesen von Apps kann erheblich vereinfacht werden wenn einheitliche Bedien- oder Ausgabeelemente verwendet werden. Nicht nur innerhalb einer App ist es sinnvoll konsistente Elemente in der Oberfläche zu verwenden, auch Funktionen von anderen Apps können die Bedienung vereinfachen. Bekannte Funktionen bei Smartphone-Applikationen sind zum Beispiel die Vergrößerung mit zwei Fingern oder das \glqq Daumen nach oben\grqq -Symbol als positive Rückmeldung. Durch das  Einbauen solcher Features wird eine App intuitiv und ohne Einführung bedienbar.\\

\noindent
\hangindent1cm
\textbf{Benutzergesteuert:} Alle ausgeführten Aktionen sollten vom Benutzer ausgehen. Eine gute Benutzeroberfläche unterstützt den Nutzer  lediglich bei seiner Bedienung, schränkt ihn aber nicht ein. Mit der heutigen Technologie ist die Verführung groß, viele Funktionen automatisch ablaufen zu lassen. Was eigentlich der Sinn einer App ist, kann jedoch auch negative Folgen haben. Zu viel Automatisierung verursacht das Gefühl von Kontrollverlust und Unsicherheit, was sich negativ auf das Vertrauen und somit auf die Benutzungszeit der Nutzer auswirkt. \\

\noindent
\hangindent1cm
\textbf{Klarheit:} Eine mobile App muss ohne Anleitung bedienbar sein. Sobald Unklarheiten beim Nutzer  entstehen und Funktionen oder Ausgaben nicht erkannt werden können, verliert die Anwendung auf dem freien Markt. \\
Der Nutzer sollte zu jeder Zeit wissen welche Optionen ihm zur Verfügung stehen und welche Folgen seine Aktionen haben. Besonders wichtig ist das Feedback infolge einer Aktion, wie beispielsweise ein Ladebalken nachdem etwas angeklickt wurde und noch nicht vollständig heruntergeladen ist. Auch wenn diese Aspekte offensichtlich erscheinen, können sie bei der Entwicklung einer App leicht übersehen werden. Verwendet werden einfache und für den Nutzer bekannte Funktionen, wie die Beschriftung aller Buttons oder das haptische, akustische oder optische Feedback beim drücken einem dieser Buttons.\\

\noindent
\hangindent1cm
\textbf{Benutzerfreundlich/Barrierefreiheit:} Die Bedienung der App sollte für Menschen mit Einschränkungen im vollen Umfang möglich sein. In Abschnitt \ref{sec:barrierefreiheit} wird auf dieses Thema tiefer eingegangen. Aber auch Benutzer ohne Einschränkungen erwarten eine einfache und übersichtliche Bedienung, die auch beispielsweise  Eingabefehler mit mehreren Versuchen verzeiht.\\

\noindent
\hangindent1cm
\textbf{Ästhetik:} Das Design spielt bei dieser Betrachtung gleich mehrere wichtige Rollen. Es sollte eine angenehme Arbeitsumgebung für den Nutzer erstellen, Ein- und Ausgaben verdeutlichen und gleichzeitig mithilfe eines eigenen Stils ein einzigartiges Image für die App schaffen (sogenanntes Branding) um deren Individualität und Wiedererkennungswert zu steigern. Das Design erschafft ein Erlebnis während der Benutzung und weckt unterbewusst Gefühle im Nutzer. \\

\noindent
Gerade weil viele dieser Aspekte unterbewusst wirken, ist eine ausgiebige Betrachtung unumgänglich.\\
Eine Schwierigkeit, die sich bei der Entwicklung ergibt sind die zwei unterschiedlichen Ziele. Einerseits sollten bestehende Design- und Bedienelemente  übernommen werden um die Bedienung intuitiv und übersichtlich zu gestalten, andererseits aber auch neue Ideen und Innovationen eingebracht werden, um sich von anderen Apps abzuheben und bleibenden Wiedererkennungswert aufzubauen.


\subsection{Oberflächen von StreamSwipe}
Die Smartphone-App lässt sich in mehrere Bereiche aufteilen, die sich in ihren Funktionen unterscheiden. Auf Basis der oben beschriebenen Grundlagen wurden diese Bereiche entworfen und werden in diesem Kapitel analysiert. Auch wenn manches davon als gewöhnlich oder naheliegend erscheint, so ist jedes Element mit Bedacht gewählt, erstellt und angepasst worden.

\subsubsection{Login-Screen}
\label{sec:loginscreen}
Bei erstmaliger Benutzung der App öffnet sich der Login-Screen. An diesem Punkt wird der erste Eindruck für den Benutzer gesetzt, wobei bei StreamSwipe ein schlichtes Design gewählt wurde. Man sieht helle Grautöne mit einem Akzentfarbton, welche sich durch alle Bildschirme der App ziehen werden. Abhängig davon, ob der User in den Systemeinstellungen den dunklen Modus gewählt hat, werden anstatt den hellen Grautönen, dunkle bis schwarze Farben dargestellt, siehe auch Abbildungen \ref{fig:homescreen_c} und \ref{fig:swipescreen_e}.\\
Auf dem Login-Screen (siehe Abbildung \ref{fig:login_a}) sind neben einer Überschrift mehrere beschriftete Textfelder und Buttons zu sehen, welche allesamt mit Semantiken versehen wurden, um durch einen Screenreader erkannt und identifiziert werden zu können. Die gewählte Anordnung wird universell bei Apps, Programmen und Webseiten benutzt, sodass die Felder auch ohne die eingetragenen Hinweistexte korrekt ausgefüllt werden könnten. Beim Antippen der Textfelder, öffnet sich die Standardtastatur des Betriebssystems. Sind alle Felder korrekt ausgefüllt, wird der User in die eigentliche App weitergeleitet, ansonsten wird durch individualisierte Fehlermeldung auf eventuelle Falscheingaben hingewiesen. Nach Erstellen eines neuen Accounts, durchläuft der User einen ähnlich aufgebauten Bildschirm (siehe Abbildung \ref{fig:login_b}) und wird danach aufgefordert weitere Informationen zur Profilvervollständigung einzugeben (siehe Abbildung \ref{fig:login_c} und \ref{fig:login_d}). Auch hierbei werden bekannte Bedienelemente wie Textfelder, Dropdownmenüs und Checkboxen verwendet, wie in der Abbildung \ref{fig:login_e} beispielhaft dargestellt ist. Falls der User in den Systemeinstellungen des Smartphones den Nachtmodus aktiviert hat, wird das Appdesign angepasst, siehe Abbildung \ref{fig:login_f}.


\begin{figure}[H]
	\begin{subfigure}{0.33\textwidth}
	\centering
	\includegraphics[scale=0.13]{Benutzeroberfläche/images/screenshot_login_1.png}
	\caption{}
	\label{fig:login_a}
	\end{subfigure}
	\begin{subfigure}{0.33\textwidth}
	\centering
	\includegraphics[scale=0.13]{Benutzeroberfläche/images/screenshot_login_2.png}
	\caption{}
	\label{fig:login_b}
	\end{subfigure}
	\begin{subfigure}{0.33\textwidth}
	\centering
	\includegraphics[scale=0.13]{Benutzeroberfläche/images/screenshot_login_3.png}
	\caption{}
	\label{fig:login_c}
	\end{subfigure}\\ \vspace{1cm}	
	
	\begin{subfigure}{0.33\textwidth}
	\centering
	\includegraphics[scale=0.13]{Benutzeroberfläche/images/screenshot_login_4.png}
	\caption{}
	\label{fig:login_d}
	\end{subfigure}
	\begin{subfigure}{0.33\textwidth}
	\centering
	\includegraphics[scale=0.13]{Benutzeroberfläche/images/screenshot_login_5.png}
	\caption{}
	\label{fig:login_e}
	\end{subfigure}
	\begin{subfigure}{0.33\textwidth}
	\centering
	\includegraphics[scale=0.13]{Benutzeroberfläche/images/screenshot_darkmode_3.png}
	\caption{}
	\label{fig:login_f}
	\end{subfigure}
\caption[Screenshots der Anmeldeseiten]{Die Anmeldeseiten von StreamSwipe und alle damit zusammenhängenden Screens. Man sieht (a) das Einloggen bei bestehendem Account, (b) das Erstellen eines Accounts, (c) und (d) das Formular für die benötigten Profildaten, (e) ein Texteingabefeld mit Autovervollständigung als Dropdownmenü und (f) das Farbschema der Anmeldeseiten im Darkmode am Beispiel des Login-Screens.}
\label{fig:login_alle}
\end{figure}



\subsubsection{Home-Screen}
\label{sec:homescreen}

Da davon ausgegangen wird, dass der User sich nicht nach jeder Nutzung ab- und wieder anmeldet, erscheint im alltäglichen Gebrauch der in Abbildung \ref{fig:homescreen_alle} dargestellte Bildschirm zuerst. Demnach bietet es sich an Ereignisse wie neue Matches und neue Nachrichten hier anzuzeigen. Diese werden wie Abbildungen \ref{fig:homescreen_a} und \ref{fig:homescreen_b} zeigen klar strukturiert in Abschnitte eingegliedert, welche mit Überschriften kenntlich gemacht sind. Die einzelnen Matches befinden sich mit allen dazugehörenden Funktionen und Informationen jeweils auf einer Karte. Durch diese Karten kann mit einer von anderen \mbox{Apps} bekannten horizontalen Swipemechanik navigiert werden. Weiterführende Funktionen wie das Starten eines Chats oder das Löschen des Matches werden durch  Antippen von allgemein verständlichen Icons ausgeführt. \\
Auch auf diesem Screen findet sich einerseits das bereits eingeführte Farbschema wieder und es werden andererseits ebenfalls Semantiken verwendet. Bei den neuen Nachrichten wird jeweils der Benutzername vorgelesen und bei den neuen Matches je nach ausgewähltem Bereich der Filmname, die Icons oder der Text dazwischen.\\
Am unteren Bildschirmrand ist eine sogenannte Bottom-Navigation-Bar zu sehen. Sie ermöglicht eine kompakte und anschauliche Navigation durch die relevanten Bildschirme. Außerdem zeigt sie an welcher Bildschirm aktuell ausgewählt ist, wobei diese Information wie in Abschnitt \ref{sec:bf-streamswipe} erarbeitet nicht ausschließlich auf einer Farbänderung basieren sollte und deshalb das ausgewählte Icon durch Hinzufügen von Text hervorgehoben wird. Liest der Screenreader die Semantik hiervon, gibt er die Bezeichnung des aktuellen Bildschirms sowie die Anzahl der weiteren Möglichkeiten an. \\
%TODO Motorische Herausforderungen!!! Anklicken des Filmposters aktiviert auch Chat?

\begin{figure}[H]
	\begin{subfigure}{0.33\textwidth}
	\centering
	\includegraphics[scale=0.13]{Benutzeroberfläche/images/screenshot_homescreen_1.png}
	\caption{}
	\label{fig:homescreen_a}
	\end{subfigure}
	\begin{subfigure}{0.33\textwidth}
	\centering
	\includegraphics[scale=0.13]{Benutzeroberfläche/images/screenshot_homescreen_2.png}
	\caption{}
	\label{fig:homescreen_b}
	\end{subfigure}
	\begin{subfigure}{0.33\textwidth}
	\centering
	\includegraphics[scale=0.13]{Benutzeroberfläche/images/screenshot_darkmode_1.png}
	\caption{}
	\label{fig:homescreen_c}
	\end{subfigure}
\caption[Screenshots des Home-Screens]{Der Home-Screen, der beim Öffnen der App zuerst gezeigt wird und Neuigkeiten wie neue Nachrichten und Matches zusammenfasst. Um den gesamten Inhalt dieser Seite sehen zu können, wird in (a) der obere Abschnitt und in (b) der untere Abschnitt gezeigt. Hat der User in den Systemeinstellungen den dunklen Modus aktiviert, so wird (c) der Home-Screen wie alle anderen Screens angepasst.}
\label{fig:homescreen_alle}
\end{figure}




\subsubsection{Swipe-Screen}
\label{sec:swipescreen}
Auf dem Swipe-Seiten (Abbildungen \ref{fig:swipescreen_alle}) findet die Bewertung der Filme statt. Durch das hier verwendete Matchingsystem mithilfe des Filmgeschmacks unterscheidet sich StreamSwipe von anderen Apps und erhält so einen innovativen, individuellen Charakter, womit diese Seite das Herzstück der App bildet.\\
Das zuvor eingeführte Farbschema bleibt auch hier erhalten, wie Abbildung \ref{fig:swipescreen_a} zeigt. Eine Überschrift im selben Stil wie bereits aus Abschnitt \ref{sec:homescreen} bekannt, verdeutlicht durch eine Frage nach welcher Motivation die Filmauswahl getroffen werden soll. Zentral im Bild ist eine Liste von Postern der zu beurteilenden Filme. Wie bereits durch die Dating-App Tinder verbreitet, werden die Antwortmöglichkeiten durch eine Swipe-Bewegung in eine Richtungen ausgewählt. In diesem Fall werden vier Entscheidungsmöglichkeiten auf vier Richtungen verteilt. Abhängig von der Position des Fingers auf dem Touchscreen bewegt sich das Filmposter innerhalb des Bildschirms, was den Effekt einer frei beweglichen Karte hervorruft. Um klarzustellen welche Swipe-Richtung für welche Entscheidung steht, verfärbt sich der jeweilige Indikator in der unteren Reihe bei Verschiebung des Filmposter. Beide Animationen sind in Abbildung \ref{fig:swipescreen_d} zu sehen. Die Indikatoren sind mit Icons versehen, zeigen aber durch Drücken welche Entscheidung sie repräsentieren und in welche Richtung der Nutzer dafür wischen  muss, wie Abbildung \ref{fig:swipescreen_c} am Beispiel des rechten Indikators zeigt. \\
Durch Antippen des Filmposters werden weitere Informationen zu dem jeweiligen Film dargestellt, wie in Abbildung \ref{fig:swipescreen_b} zu sehen. Gleichfalls wird durch ein einfaches Antippen wieder zurück  zu den Postern gewechselt. Eine Rotations-Animation verdeutlicht die Illusion der Karten.\\
Alle diese für die Bedienung der App grundlegenden Steuerungen verlangen keine feinmotorischen Eingaben und können problemlos von Personen mit motorischen Einschränkungen genutzt werden. Auch dieser Bildschirm ist vollkommen mit Semantiken ausgestattet. Anstelle des Filmposters wird der Name des Films ausgelesen und für die vier Indikatoren am unteren Rand werden jeweils deren Funktion und durch welche Swipe-Richtung sie erreicht werden vorgelesen. Sämtliche Textfelder können ebenfalls problemlos von einem Screenreader gelesen werden.



\begin{figure}[H]
	\begin{subfigure}{0.33\textwidth}
	\centering
	\includegraphics[scale=0.13]{Benutzeroberfläche/images/screenshot_swipescreen1.png}
	\caption{}
	\label{fig:swipescreen_a}
	\end{subfigure}
	\begin{subfigure}{0.33\textwidth}
	\centering
	\includegraphics[scale=0.13]{Benutzeroberfläche/images/screenshot_swipescreen2.png}
	\caption{}
	\label{fig:swipescreen_b}
	\end{subfigure}
	\begin{subfigure}{0.33\textwidth}
	\centering
	\includegraphics[scale=0.1742]{Benutzeroberfläche/images/screenshot_swipescreen3.png}
	\caption{}
	\label{fig:swipescreen_c}
	\end{subfigure}\\ \vspace{1cm}	
	
	\begin{subfigure}{0.33\textwidth}
	\centering
	\includegraphics[scale=0.13]{Benutzeroberfläche/images/screenshot_swipescreen4.png}
	\caption{}
	\label{fig:swipescreen_d}
	\end{subfigure}
	\begin{subfigure}{0.33\textwidth}
	\centering
	\includegraphics[scale=0.13]{Benutzeroberfläche/images/screenshot_darkmode_2.png}
	\caption{}
	\label{fig:swipescreen_e}
	\end{subfigure}
\caption[Screenshots der Swipe-Seiten]{Darstellungen und Funktionen der Swipe-Seiten mit (a) der Standarddarstellung, (b) weiteren Filminformationen, (c) einer Animation beim Drücken einer der Indikatoren und (d) der Swipe-Animation.  Hat der Nutzer in den Systemeinstellungen den dunklen Modus aktiviert, so wird (e) die Swipe-Seite wie alle anderen Seiten angepasst.}
\label{fig:swipescreen_alle}
\end{figure}



\subsubsection{Chat}
\label{sec:UI-Chat}
Die Chatseite ist in eine Liste aus aktiven Chats und eine Liste mit Chatanfragen aufgeteilt, siehe \ref{fig:chat_a} und \ref{fig:chat_b}. Um zwischen diesen beiden Listen zu wechseln werden Tabs eingesetzt, wie sie aus Windowsanwendungen bekannt sind. Zwischen diesen Tabs kann entweder gewechselt werden, indem ein anderes Tabfenster angetippt wird, oder der gesamte Bildschirm mit einer Geste zur Seite gewischt wird. Bei jedem Element der Chatliste ist der jeweilige Benutzername und die neueste Nachricht zu sehen, dazu wird falls vorhanden entweder ein Profilbild oder eine einfarbige Fläche mit dem Anfangsbuchstaben des Namens angezeigt. Chatanfragen können jeweils durch das Schieben nach links angenommen oder nach rechts ablehnt werden, wie in den Abbildungen \ref{fig:chat_c}, bzw. \ref{fig:chat_d} zu sehen ist. Diese Mechanik wird häufig in Email-Apps zum Löschen oder Verschieben der Mails benutzt.\\ 
Durch Antippen eines Matches, öffnet sich der Chatverlauf, welcher in Abbildung \ref{fig:chat_e} zu sehen ist. Die Anordnung der Nachrichten innerhalb des Chatverlaufs ist wie aus anderen Messenger bereits bekannt, aber in den Stilfarben von StreamSwipe. Am oberen Bildschirmrand wird der Profilname des Matches angezeigt und rechts davon befindet sich der Button zu dessen Profilseite, auf der genauere Details über diese Person zu finden sind. Die Profilseiten werden in Kapitel \ref{sec:benutzerprofil} genauer vorgestellt. \\
Dem Nutzer wird durch das ihm bereits vorgestellte Design und der ausschließlichen Nutzung von bekannter Mechanik ein vertrautes Umfeld geboten. Wie auf jeder Seite passt sich auch hier das Farbschema automatisch an, falls in den Systemeinstellungen des Smartphones das dunkle Design gewählt wurde, wie beispielsweise in Abbildung \ref{fig:chat_f} dargestellt. Neben der Benutzerfreundlichkeit wird auch die Barrierefreiheit beachtet, indem alle Elemente, die nicht bereits aus einem Text bestehen, mit Semantiken ausgestattet werden. Zusätzlich werden keine feinmotorischen Bewegungen zur Navigation durch die Bildschirme benötigt. Bis auf die Eingabe über die Tastatur kann alles über große Flächen oder Wischmechaniken bedient werden. Im Chatverlauf wir die Standardtastatur des Systems verwendet, mit der der Benutzer bereits vertraut ist. 


\begin{figure}[H]
	\begin{subfigure}{0.33\textwidth}
	\centering
	\includegraphics[scale=0.1742]{Benutzeroberfläche/images/screenshot_chat_1.png}
	\caption{}
	\label{fig:chat_a}
	\end{subfigure}
	\begin{subfigure}{0.33\textwidth}
	\centering
	\includegraphics[scale=0.1742]{Benutzeroberfläche/images/screenshot_chat_2.png}
	\caption{}
	\label{fig:chat_b}
	\end{subfigure}
	\begin{subfigure}{0.33\textwidth}
	\centering
	\includegraphics[scale=0.1742]{Benutzeroberfläche/images/screenshot_chat_3.png}
	\caption{}
	\label{fig:chat_c}
	\end{subfigure}\\ \vspace{1cm}	
	
	\begin{subfigure}{0.33\textwidth}
	\centering
	\includegraphics[scale=0.1741]{Benutzeroberfläche/images/screenshot_chat_4.png}
	\caption{}
	\label{fig:chat_d}
	\end{subfigure}
	\begin{subfigure}{0.33\textwidth}
	\centering
	\includegraphics[scale=0.13]{Benutzeroberfläche/images/screenshot_chat_5.png}
	\caption{}
	\label{fig:chat_e}
	\end{subfigure}
	\begin{subfigure}{0.33\textwidth}
	\centering
	\includegraphics[scale=0.13]{Benutzeroberfläche/images/screenshot_darkmode_4.png}
	\caption{}
	\label{fig:chat_f}
	\end{subfigure}
\caption[Screenshots der Chat-Seiten]{Darstellungen und Funktionen der Chat-Seiten mit (a) den aktiven Chats, (b) den Chats auf der Warteliste, (c) und (d) angenommene, bzw. abgelehnten Chats auf der Warteliste, sowie (e) einem Chatverlauf im hellen und (f) in dunklen Modus.}
\label{fig:chat_alle}
\end{figure}



\subsubsection{Benutzerprofil}
\label{sec:benutzerprofil}

Auf der Profilseite werden ein Profilbild, ein Hintergrundbild und für das Matching relevante persönliche Informationen dargestellt. Es gibt eine Version, die nur von anderen Nutzern sichtbar ist, mit denen ein Match stattgefunden hat, und eine Version, die über die Bottom-Navigation-Bar erreichbar werden kann. Die Letztere wird in Abbildung \ref{fig:profilseite_alle} dargestellt und unterscheidet sich von der Version für andere Nutzer darin, dass Profil- und Hintergrundbild bearbeitet werden können.\\
Das Farbschema und das Design wurden an die bisherigen Seiten angepasst. Um die Oberfläche simpel und selbsterklärend zu halten, wird jede dargestellte Information mit einem passenden Icon und einem Hinweis versehen (siehe Abbildung \ref{fig:profilseite_a}). Die Icons zum Bearbeiten der Bilder sind, wie auch in vielen anderen Apps, platziert und designt. Sie öffnen die systemeigene Bildergalerie des Smartphones um den Nutzer aus einem bekannten Umfeld Bilder auswählen lassen zu können.\\
Beim initialen Öffnen einer Profilseite sollen Namen, Profilbild und ein Hintergrundbild ins Auge springen. Sie stellen die ersten Informationen dar, die dem Betrachter wichtig sind, weshalb sie wie in Abbildung \ref{fig:profilseite_a} deutlich sichtbar ist  beim Öffnen mehr als die Hälfte des Bildschirms einnehmen. Anschließend wird der Fokus auf detailliertere Informationen gerichtet. Auf der Profilseite von StreamSwipe wird hierfür heruntergescrollt um den Block mit den Profildaten sehen zu können. Bei dieser Aktion blendet eine Animation das Profilbild aus und verschmälert das Hintergrundbild. Der Benutzername wird ebenfalls aus dem Fokus gezogen, bleibt aber wie in Abbildung \ref{fig:profilseite_b} zu sehen mit dem verbleibenden Hintergrundbildausschnitt erhalten. Dies hilft dem Betrachter unterbewusst bei dem Fokuswechsel und schafft ein modernes, responsives Feedback bei der User Experience.\\
Um das durchgängig schlichte Design der App zu erhalten ist der Zugang zu den Einstellungen ausschließlich auf der Profilseite zu finden. Hierfür ist im rechten oberen Bildschirmbereich das repräsentative Icon. Der hierdurch erreichbare Bildschirm (Abbildung \ref{fig:profilseite_c}) ist gleich aufgebaut wie die Informationeneingabe nachdem ein neuer Account erstellt wurde (Abbildungen \ref{fig:login_c} und \ref{fig:login_d}). Die dort angegebenen Informationen können hier wieder angepasst werden. %TODO Referenz auf account erstellen bild


\begin{figure}[tbt]
	\begin{subfigure}{0.33\textwidth}
	\centering
	\includegraphics[scale=0.13]{Benutzeroberfläche/images/screenshot_profilseite_1.png}
	\caption{}
	\label{fig:profilseite_a}
	\end{subfigure}
	\begin{subfigure}{0.33\textwidth}
	\centering
	\includegraphics[scale=0.13]{Benutzeroberfläche/images/screenshot_profilseite_2.png}
	\caption{}
	\label{fig:profilseite_b}
	\end{subfigure}
	\begin{subfigure}{0.33\textwidth}
	\centering
	\includegraphics[scale=0.13]{Benutzeroberfläche/images/screenshot_profilseite_3}
	\caption{}
	\label{fig:profilseite_c}
	\end{subfigure}
\caption[Screenshots der Profilseite]{Profilseite wie sie für den Nutzer selbst angezeigt wird (a) im normalen Zustand und (b) nach vollständigem Einklappen des Profilkopfes durch eine Animation während dem Herunterscrollen. Mit den von hier aus erreichbaren Einstellungen (c) können die anfänglich gegebenen Profilangaben abgepasst werden.}
\label{fig:profilseite_alle}
\end{figure}



\subsection{Filmliste}


\section[CodeBeispiele]{CodeBeispiele \hfill \normalfont \small{Autor-Name}}


\section[Probleme]{Probleme \hfill \normalfont \small{Autor-Name}}


\section[Fazit]{Fazit \hfill \normalfont \small{Autor-Name}}


\section{Literaturverzeichnis}
\begin{itemize}
\bibitem[1]{scheidungen} Meinungsforschungsinstituts Civey (2020):  \url{https://www.presseportal.de/pm/145489/4627304}, letzter Zugriff: 13. Mai 2021
\bibitem[1]{serienkonsum} Splendid Research (2017): \url{https://www.springerprofessional.de/konsumforschung/marketingstrategie/konsumenten-auf-der-serien-welle/15146374}, letzter Zugriff: 14. Mai 2021
\bibitem[1]{schwerbehindertenausweis} Statistisches Bundesamt (2020): \url{https://www.destatis.de/DE/Themen/Gesellschaft-Umwelt/Gesundheit/Behinderte-Menschen/Tabellen/schwerbehinderte-alter-geschlecht-quote.html;jsessionid=885260788D4FFC7F670576B72E5089F4.live741}, letzter Zugriff: 17. April 2021
\bibitem[2]{behindertengleichstellungsgesetz} Behindertengleichstellungsgesetz (2002): \url{https://www.gesetze-im-internet.de/bgg/BGG.pdf}, letzter Zugriff: 19. April 2021
\bibitem[3]{sehhilfen} Institut für Demoskopie Allensbach (2019): \textit{Untersuchung zum Sehbewusstsein der Deutschen},  \url{file:///C:/Users/Vincent/AppData/Local/Temp/ZVA_Brillenstudie_2019-1.pdf}, letzter Zugriff: 11. Mai 2021


\bibitem[99]{muster} Mustermann, Max (2020): Methode und Nutzung der Literatur-Zitierweise, 2. Aufl., Boston: Harvard’s Eleven Publications.
\bibitem[99]{asdf} Autor (2008): Name des Buches, 22. Aufl., Berlin: Foxtrott.
\end{itemize}


\end{document}



\footnote{\url{https://de.wikipedia.org/wiki/Alufolie}} 


Abbildung \ref{fig:allg_kennlinie}

\begin{figure}[tbt]
\begin{center}
\includegraphics[scale=0.45]{Grafiken/allg_kennlinie.png}
\end{center}
\caption{Kennlinie einer Halbleiterdiode \protect \footnotemark}
\label{fig:allg_kennlinie}
\end{figure}
\footnotetext{FUSSNOTE}


Tabelle \ref{tab:cobalt}

\begin{table}[tbt]
\caption{•}
\begin{threeparttable}	%Scheme for footnotes in tables
\begin{center}
\begin{tabular}{c c c c}
\toprule
& keV & keV & keV \\
\midrule
a	& b	& c	& d \\
a	& b	& c	& d \\
a	& b	& c	& d \\
a	& b	& c	& d \\ %direkt hinter jeweiligen Wert /tnote{1}
\bottomrule
\end{tabular}
\end{center}
\begin{tablenotes}\footnotesize 
\item[1]{Quelle: http://www.thinksrs.com/downloads/PDFs/ApplicationNotes/IG1BAgasapp.pdf}
\end{tablenotes}
\end{threeparttable}
\label{tab:cobalt}
\end{table}