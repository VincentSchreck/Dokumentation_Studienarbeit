\documentclass[11pt,a4paper]{article}
\usepackage[utf8]{inputenc}
\usepackage[german]{babel}
\usepackage{amsmath}
\usepackage{amsfonts}
\usepackage{hyperref}
\usepackage{subcaption}
\usepackage{booktabs}
\usepackage{setspace}
\usepackage{threeparttable}
\usepackage{amssymb}
\usepackage{graphicx}
\usepackage{fancyhdr}
\usepackage{listings}
\renewcommand*{\lstlistlistingname}{Quellcodeverzeichnis}
\usepackage{icomma}
\usepackage{float}
\usepackage{pdfpages}
\usepackage{hyperref}
\usepackage{csquotes}
\usepackage[left=2.5cm,right=2.5cm,top=2cm,bottom=3.5cm]{geometry}

\title{StreamSwipe\\Tinder für Filme\vspace{10px}}
\author{Leon Gieringer, Robin Meckler,Vincent Schreck \\ \\ Studienarbeit \\ \\ \\}
\date{\today}
\usepackage{wrapfig}
\usepackage{colortbl}
\usepackage{array}
\usepackage{xspace}



%%%%%%%%%%%% listing & color for Code %%%%%%%%%%%%
\usepackage{color}
\definecolor{lightgray}{rgb}{.9,.9,.9}
\definecolor{darkgray}{rgb}{0.5,0.5,0.5}
\definecolor{ballblue}{rgb}{0.13, 0.67, 0.8}
\definecolor{purple}{rgb}{0.65, 0.12, 0.82}
\definecolor{darkpastelgreen}{rgb}{0.01, 0.75, 0.24}

\usepackage{listings}
\renewcommand*{\lstlistlistingname}{Quellcodeverzeichnis}


\lstdefinelanguage{JavaScript}{
	keywords={typeof, new, true, false, catch, function, return, null, catch, switch, var, if, in, while, do, else, case, break, const, var, from, export, import, defaultValue, class, boolean, throw, implements,this, style},
	keywordstyle=\color{orange}\bfseries,
	ndkeywords={height, borderColor, borderWidth, isHungry, setIsHungry},
	ndkeywordstyle=\color{ballblue}\bfseries,
	identifierstyle=\color{black},
	sensitive=false,
	comment=[l]{//},
	morecomment=[s]{/*}{*/},
	commentstyle=\color{darkgray}\ttfamily,
	stringstyle=\color{darkpastelgreen}\ttfamily,
	morestring=[b]',
	morestring=[b]"
}

\lstset{
	language=JavaScript,
	backgroundcolor=\color{lightgray},
	extendedchars=true,
	basicstyle=\ttfamily,
	showstringspaces=false,
	showspaces=false,
	numbers=left,
	numberstyle=\footnotesize,
	numbersep=9pt,
	tabsize=2,
	breaklines=true,
	showtabs=false,
	captionpos=b
}

%%%%%%%%%%%% titlesec for paragraphs %%%%%%%%%%%%
\usepackage{titlesec}

\setcounter{secnumdepth}{4}
\titleformat{\paragraph}
{\normalfont\normalsize\bfseries}{\theparagraph}{1em}{}
\titlespacing*{\paragraph}{0pt}{3.25ex plus 1ex minus .2ex}{1.5ex plus .2ex}



%%%%%%%%%%%% begin document %%%%%%%%%%%%

\begin{document}
\maketitle
\thispagestyle{empty}
\newpage
\pagenumbering{Roman}
\tableofcontents
\newpage
\listoffigures
\newpage
\lstlistoflistings
\newpage
\pagenumbering{arabic}



\pagestyle{fancy}
\fancyhf{}
\setlength{\headheight}{35pt}

\renewcommand\headrulewidth{0.4pt}

\fancyhead[LE,RO]{\rightmark}% <- changed
\fancyhead[LO,RE]{StreamSwipe}
\renewcommand{\sectionmark}[1]{\markright{#1}}
\renewcommand{\subsectionmark}[1]{\markright{#1}}
\renewcommand{\subsubsectionmark}[1]{\markright{#1}}

\cfoot{\thepage}
\newpage


\section[Einleitung]{Einleitung \hfill \normalfont \small{Vincent Schreck}}
Die Zahl der Scheidungen in Deutschland hat sich während den Einschränkungen durch die COVID-19 Pandemie 2020 verfünffacht \cite{scheidungen}. Viele Menschen suchen sich Partner aus, die sie zwar attraktiv finden, mit denen sie jedoch kaum gemeinsame Interessen und Ansichten teilen. Sobald diese Personen dazu gezwungen sind mehr Zeit miteinander zu verbringen zeigt sich, dass dies keine solide Basis für eine Beziehung ist. \\
Wir wollen diesen gravierenden Fehler in seinem Keim ersticken und revolutionieren das Dating-Game mit einem Verfahren, bei dem persönliche Vorlieben im Vordergrund stehen und das Aussehen zweitrangig ist.
\subsection{Motivation}

Bereits vor tausenden von Jahren haben sich die Menschen Partner gesucht und mit der ersten Monogamie kam auch die erste Beziehung und wahrscheinlich auch die ersten Beziehungsprobleme.\\
Eine der wichtigsten Grundlagen einer Beziehung sind gleiche Ansichten, Interesse und Vorlieben, anstatt Aussehen und Geld, denn Schönheit vergeht und Charakter besteht. Jedoch ist das Problem dabei, dass man erst weiß wie gut man zueinander passt, nachdem man sich kennengelernt hat. Viele möglicherweise sehr glückliche Beziehungen finden gar nicht statt, da die Person durch ein eigentlich weniger wichtiges Kriterium herausgefiltert wurde. Sucht man die Ursache dieses Problems, ist man schnell bei der Art des Kennenlernens. Der erste Eindruck ist gewöhnlicherweise die optischer Natur. Dementsprechend ist Aussehen in der Realität das erste Filterverfahren, was jedoch durch StreamSwipe an eine spätere  Position tritt.\\
Wir bieten die Lösung zu einem jahrtausendealten Problem der Menschheit.


\subsection{Methode}
Gerade in den letzten Jahren genießt das Medium Film und Serie einen immer höheren Stellenwert in der Gesellschaft. Durch Video-on-Demand Plattformen wie Netflix, Disney+ und Amazon Prime Video  sind Filme und Serien omnipräsent geworden und das Angebot scheint endlos zu sein. Der Zugriff auf diese Medien wurde dadurch stark vereinfacht und der Nutzer kann einerseits einer Serie oder Filmreihe treu bleiben, da er keine Folge mehr verpassen kann, und andererseits ganze Staffeln an einem einzigen Tag anschauen. So kann dieses Medium bereits bei vielen Menschen eine Charaktereigenschaft werden und Charaktereigenschaften vieler Zuschauer passen sich an Filmcharaktere an.\\
Bereits 2017 haben die 18- bis 39-Jährigen an durchschnittlich 4 Tagen pro Woche eine Serie angeschaut \cite{serienkonsum}. Aus dem Film- und Serienkonsum können somit individualisierte Daten gesammelt und analysiert werden. Bei StreamSwipe wird auf diesem Wege über die Film- und Serienauswahl des Nutzers ein Geschmack berechnet werden, der über einen Algorithmus mit anderen ähnlichen Geschmäcken gematch wird. Sobald ein Match entstanden ist, öffnet sich ein privater Chat und die beiden Personen können sich austauschen und verabreden.
\clearpage

\section{Theoretische Grundlagen}
\subsection{Netzwerkprotokolle}
\subsubsection{Schichtenmodell}

Eine der gängigsten Arten der Kommunikation findet heutzutage über das Internet statt. 
Dabei handelt es sich um ein weltweit verbundenes Netz von Rechnern. Zur Gewährleistung einer effizienten und geregelten Datenübertragung der heterogenen Computer im Internet wurden Regelwerke, die sogenannten Netzwerkprotokolle, benötigt.
Um das Jahr 1980 wurden daraufhin von verschiedenen Computerherstellern modularisierte Protokolle entwickelt, die fortan als Standard für die digitale Übertragung innerhalb von Rechnernetzen gelten sollen.[1.0] 
Es musste eine Vielzahl von Aufgaben bewältigt und Anforderung bezüglich Zuverlässigkeit, Sicherheit, Effizienz etc. erfüllt werden. Die Aufgaben reichten dabei von der elektronischen Übertragung der Signale bis zur geregelten Reihenfolge der in der Kommunikation abstrakteren Aufgaben.[1.05] 
Aus den zu lösenden Problemen und Anforderung kristallisierten sich sieben Schichten bzw. Ebenen heraus. 
Jede einzelne Schicht setzt dabei separat eine Anforderung um und kann dabei durch verschiedene Protokolle realisiert werden. In dem sich etablierten OSI-Schichtenmodell bauen die einzelnen Schichten aufeinander auf, wobei die unterste Schicht das Fundament ist. 
Die Open Systems Interconnection (OSI) wurde von der International Organization for Standardization (ISO), der Internationalen Organisation für Normung, als Grundlage für die Bildung von offenen Kommunikationsstandards entworfen. 
\newline

Zusätzlich zum OSI-Modell existiert das in den 1960er-Jahren entwickelte TCP/IP-Referenz\-modell. Entwickler dieses Schichtmodells war das Verteidigungsministerium der Vereinigten Staaten, auch bekannt als das Departments of Defense (DoD). Dementsprechend trägt das TCP/IP-Referenzmodell auch den Namen DoD-Schichtenmodell. [1.06]

\begin{figure}[h]
\centering
\includegraphics[width=\textwidth]{images/Netzwerkprotokolle_OSI-Schicht.PNG}
\caption{OSILAYER [1.1]}
\end{figure}

\newpage

\begin{table}[]
\begin{center}
    \begin{tabular}{| l | p{8cm} |}
    \hline
     OSI-Schicht & Aufgabe \\ 
     \hline
        
       Anwendungen & Funktionen für Anwendungen, sowie die Dateneingabe und -ausgabe. \\
    \hline    
       Darstellung & Umwandlung der systemabhängigen Daten in ein unabhängiges Format.  \\
    \hline   
       Sitzung & Steuerung der Verbindungen und des Datenaustauschs.  \\
    \hline  
        Transport & Zuordnung der Datenpakete zu einer Anwendung. \\ 
    
    \hline     
    	Vermittlung & Routing der Datenpakete zum nächsten Knoten. \\
	
    \hline       
      Sicherung & Fehlererkennungsmechanismen / Segmentierung der Pakete in Frames und Hinzufügen von Prüfsummen.  \\   
    \hline
    
    Bitübertragung & Umwandlung der Bits in ein zum Medium passendes Signal und physikalische Übertragung.\\ 
    \hline    
    \end{tabular}
\end{center}
\caption{Kurzbeschreibung der OSI-Schichten [1.2]}
\end{table}

\subparagraph{IP}
In der Vermittlungsschicht des OSI-Schichtenmodells findet, unabhängig des Über\-tra\-gungs\-mediums und der genutzten Topologie, die logische Adressierung der Endgeräte statt. Das geläufigste Protokoll dafür ist das Internet Procotol (IP). Jedem am Netz verbundenen Teilnehmer wird eine IP-Adresse zugewiesen. Die bekannteste Notation ist die 32 Bit lange IPv4-Adressen und die IPv6-Adressen mit einer Größe von 128 Bit. 
\newline

\subparagraph{TCP/UDP}
In der Transportschicht wird eine Ende-zu-Ende-Kommunikation ermöglicht. Sie ist das Bindeglied zwischen den anwendungsorientierten und den transportorientierten Schichten. Die geläufigsten Protokolle sind das verbindungslose, unzuverlässige, aber weniger Overhead belastete User Datagram Protocol (UDP) und das verbindungsorientierte und datentransferzuverlässige Transmission Control Protocol (TCP). Jedes netzwerkfähige Gerät enthält eine Vielzahl von Ports, die primär zur
Unterscheidung zwischen Datenströmen aus Anwendungen bei Netzwerkverbindungen
genutzt werden. Anhand des genutzten Ports bei Netzwerkanfragen
wissen Webserver, welches Protokollverfahren genutzt werden soll.

\subsubsection{HTTP}
Das Hypertext Transfer Protocol, kurz HTTP, ist ein zustandloses Protokoll zur Übertragung von Daten auf der Anwendungsschicht.

\subparagraph{Kommunikation}
Unter einer Nachricht versteht man in HTTP die Kommunikationseinheiten zwischen dem Zentralrechner (Server) und dem, der einen Dienst vom Server abruft (Client). 
Man unterscheidet dabei zwischen der Anfrage (Request) vom Client an den Server und der Antwort (Response) als Reaktion vom Server zum Client. 
\newline

Eine Nachricht besteht aus dem Nachrichtenkopf (Message Header, kurz Header) und dem Nachrichtenrumpf (Message Body, kurz Body). 
Der Header enthält generelle Informationen über die Nachricht wie zum Beispiel den Methodentyp, das Datenformat, den genutzten Kompressionsalgorithmus, die Länge der Nachricht oder die verwendete Kodierung im Body. 
\newline

Die erste Zeile des Nachrichtenkopfs ist dreiteilig und besteht bei der Anfrage aus dem Namen der Anfragemethode, dem Pfad zur angeforderten Ressource (Uniform Resource Locator, kurz URL) und der verwendet HTTP-Version. Die Anfangszeile einer HTTP-Antwort dagegen besteht zunächst aus der verwendeten HTTP-Version, gefolgt von dem zweiteiligem Status-Code. 
Der Anfangszeile beider Nachrichtentypen folgt eine Reihe von Headerzeilen, wobei jede Zeile aus einem Schlüsselwort/Wert-Paar besteht und die für die Datenübertragung wichtigen Informationen übergibt. 
Der Nachrichtenrumpf, der mit den Nachrichtenkopf über einen Zeilenumbruch syntaktisch voneinander getrennt wird, enthält schließlich die Nutzdaten.
\newline

\begin{figure}[h]
\centering
\includegraphics[width=\textwidth]{images/netzwerkprotokolle_http.PNG}
\caption{HTTP-Nachrichtenaufbau}
\end{figure}

\subparagraph{Methoden}
HTTP bietet fest definierte Standard-Methoden für Anfragen, die für verschiedene Aufgaben gedacht sind. Im Folgenden werden die wichtigsten Methoden beschrieben:

\begin{enumerate}
		\item \textit{GET} ist die gebräuchlichste Methode. Sie fordert vom Server eine Ressource, die bei Erfolg in der Antwort im Body zurückgegeben wird. 
		
		\item \textit{POST} ist für die Änderung oder Erzeugung einer Ressource vorgesehen. Dafür werden bei der Anfrage zusätzlich Daten im Body der Nachricht übertragen.
		
		\item \textit{PUT} dient dazu, eine Ressource zu verändern, oder bei Nichtexistenz zu erstellen.
	
		\item \textit{PATCH} ändert eine bestehende Ressource ohne diese wie bei PUT vollständig zu ersetzen. 
		
		\item \textit{DELETE} löscht die angegebene Ressource auf dem Server.
		
		\item \textit{OPTIONS} liefert eine Liste von Methoden und Merkmale, die vom Server unterstützt werden.
		
\end{enumerate}
\newpage

\subparagraph{Statuscodes}
HTTP-Antworten senden in der Anfangszeile ihrer Nachricht Statuscodes. Die Angabe ist zweiteilig und besteht aus einer standardisierten Statuskennzahl sowie einer kurzen textuellen Beschreibung, die zusammen Auskunft über den Bearbeitungszustand der zugehörigen Anfrage geben.\newline

\begin{table}[h]
\begin{center}
    \begin{tabular}{| l | l |  p{8cm} |}
    \hline
    Typ & Status-Code & Beispiele \\ \hline
    
    Informational & 1xx & 100 Continue, 101 Switching\\
    
    \hline
    Success & 2xx & 200 OK, 201 Created, 202 Accepted  \\
    
    \hline
	Redirection & 3xx & 300 Multiple Choice, 301 Moved Permanently  \\
	
    \hline    
    Client Error & 4xx & 400 Bad Request, 403 Forbidden\\ 
    
    \hline    
    Server Error & 5xx & 500 Internal Server Error \\
    \hline
    \end{tabular}
\end{center}
\caption{Module}
\end{table}

%\begin{figure}[h]
%\centering
%\includegraphics[width=\textwidth]{images/netzwerkprotokolle_Httpstatuscodes.png}
%\caption{HTTP-Statuscodes}
%\end{figure}

\subsubsection{HTTPS}
Das HTTP-Protokoll hat den großen Nachteil, dass die Nachrichten unverschlüsselt und ungesichert übertragen werden. 
Die Daten können bei der Übertragung von Dritten empfangen, gelesen und verändert werden. Hypertext Transfer Protocol Secure, kurz HTTPS, soll dem entgegenwirken und die Sicherheit bei der Kommunikation gewährleisten. 
Dafür dienen zwei Konzepte:
\newline

Ersteres ist das Verschlüsseln der Kommunikation von Sender und Empfänger. Die zugrundeliegende Technik nennt sich Transport Layer Security (TLS), ist aber auch als Secure Sockets Layer (SSL) bekannt. Die Idee dahinter ist, dass jeder Teilnehmer der Kommunikation einen öffentlich bekannten Schlüssel (Public Key) und einen geheimen, nicht-öffentlichen Schlüssel (Private Key) besitzen. Über den Public-Key des Empfängers verschlüsselt der Sender seine Nachricht. Diese kann nur über den Private-Key des Empfängers entschlüsselt werden, der vom Empfänger nicht weitergegeben werden sollte.
\newline

Das zweite Konzept von HTTPS ist die Webserver-Authentifizierung. Ein Zertifikat, dass zu Beginn der Kommunikation an den Webclient gesendet wird, bescheinigt die Vertrauenswürdigkeit des Senders. Dafür vertrauen Browser- und Betriebssystemhersteller bestimmten Zertifizierungsstellen, deren Zertifikate sie in ihrem Browser bzw. Betriebssystem hinterlegen. Die Kommunikation zwischen Webserver und Webclient findet folglich erst nach vollständiger Authentifizierung statt.

\subsection{JavaScript}
In den nächsten Unterkapiteln soll zunächst ein historischer Überblick über die Programmiersprache JavaScript gegeben werden. Im Anschluss wird auf die Bedeutung und Nutzung von JavaScript eingegangen. 
\newline

\subsubsection{Historie}
Ihren Ursprung findet die Programmiersprache JavaScript im Jahr 1995, als Brendan Eich, ein damaliger Ingenieur des US-amerikanischen Software-Unternehmens „Netscape Communications Corporation“, innerhalb von zehn Tagen diese Sprache für den Browser „Netscape Navigator“ entwickelt hat \cite{JS1}. Das Ziel dabei war es, eine Skriptsprache zu entwickeln, die es Entwicklern möglich machen sollte, auf ihren Webseiten Skripte auszuführen. Zunächst noch unter dem Namen Mocha und LiveScript änderte sich der Name zu JavaScript aufgrund der Kooperation von Netscape und Sun, der Firma hinter der Programmiersprache Java, und damit verbundenen Marketinggründen. Netscape wollte von der damaligen Popularität von Java profitieren \cite{JS1.05}.
\newline
\noindent
Netscape’s Veröffentlichung des Netscape Navigator 2.0, der erste Browser der JavaScript unterstützte, brachte Microsoft dazu, Netscape als ernstzunehmenden Konkurrenten zu sehen. 
Microsoft antwortete im August 1995 mit der Veröffentlichung des ersten Internet Explorer zusammen mit der Skriptsprache JScript, die einen Dialekt der Sprache JavaScript darstellt. Dies ist ferner als der Beginn der „Browserkriege“ bekannt \cite{JS1.06}.
\newline
\noindent
Im Jahre 1997 reichte Netscape JavaScript bei der European Computer Manufacturers Association (kurz ECMA), einer privaten, internationalen Normungsorganisation zur Normung von Informations- und Kommunikationssystemen und Unterhaltungselektronik, ein. Das Ziel war es, von der ECMA einen einheitlichen Standard für die Sprache schaffen zu lassen, die fortan weiterentwickelt werden und von weiteren Browserherstellern genutzt werden soll. Das resultierende Standard nennt sich ECMAScript, wobei JavaScript die bisher bekannteste Implementierung dieses Standards ist \cite{JS1.07}.
Andere Implementierungen sind zum Beispiel ActionScript von Macro\-Media, JScript von Microsoft und ExtendScript von Adobe.
\newline
\noindent
Jährlich wird dieser Standard seit Juni 2015 erweitert. ECMAScript Version 11 beziehungs\-weise ECMAScript 2020 bildet zum Zeitraum dieser Dokumentation den aktuellen Standard \cite{JS1.08}. 
Im Juni 2021 soll die neueste Version ECMAScript 2021 veröffentlicht werden \cite{JS1.09}. 
\newline

\subsubsection{Wesentliche Programmiereigenschaften}
„JavaScript is Not Java“  \cite{JS1.091}. Die Programmiersprache JavaScript wird aufgrund ihrer Namensgebung oft in falsche Zusammenhänge zu Java gebracht. Das häufigste Missverständnis ist, JavaScript wäre eine vereinfachte Version von Java \cite{JS1.091}.
\newline
\noindent
JavaScript ist eine interpretierte Programmiersprache mit objektorientierten Um"-setz"-ungs"-mög"-lich"-kei"-ten. Interpretation ist in diesem Zusammenhang so zu verstehen, dass der Quellcode zur Laufzeit eines Programms gelesen, übersetzt und ausgeführt wird.
Syntaktisch ähnelt JavaScript kompilierten Programmiersprachen wie C, C++ und Java durch gleiche Umsetzung der Kontrollstrukturen wie den Bedingungen, Schleifen oder den booleschen Operatoren \cite{JS1.1}.
Wesentliche Unterschiede sind dagegen, dass JavaScript zum einen eine schwach-typisierte Sprache ist.
Durch die schwache Typisierung haben Variablen keinen festen Datentyp und können diesen dynamisch zur Laufzeit ändern.
Des Weiteren findet bei JavaScript die Objektorientierung prototypenbasiert statt. Diese Form der Programmierung wird auch klassenlose Objektorientierung bezeichnet.
Im Gegensatz zur klassenbasierten Programmierung, werden hier Objekte nicht aus vordefinierten Klassen instanziiert, sondern durch das Klonen bereits existierender Objekte erzeugt.
Die Objekte, die geklont werden, sind dabei als Prototyp-Objekte zu ver\-steh\-en.
Beim Klonen werden alle Attribute und Methoden des Prototyp-Objekts in das neue Objekt übernommen und können dort überschrieben sowie erweitert werden.
Objekte in JavaScript sind eher als Zuordnungslisten, ähnlich wie assoziative Arrays oder Hash-Tabellen, anzusehen, da bei der Eigenschaftszuweisung lediglich ein Mapping eines Schlüsselworts (Key) zu seiner zugehörigen Eigenschaft ( Value) stattfindet, wie es in Abbildung \ref{fig:JavascriptObjekt} zu sehen ist.
\newline
\noindent
Ein weiterer Unterschied zu den anderen Programmiersprachen ist, dass alle Funktionen und Variablen außer der primären Datentypen Boolean, Zahl und Zeichenfolge, als Objekte verstanden werden können.\\

\begin{figure}[tbt]
\centering
\includegraphics[width=\textwidth]{images/JavaScript_Object.PNG}
\caption[JavaScript Objekt]{JavaScript Objekt \cite{JS1.29}}
\label{fig:JavascriptObjekt}
\end{figure}

\noindent
Ein weiterer Unterschied zu den anderen Programmiersprachen ist, dass alle Funktionen und Variablen außer der primären Datentypen Boolean, Zahl und Zeichenfolge, als Objekte verstanden werden können.

\subsubsection{Anwendungsgebiete}
Ursprünglich fand JavaScript seinen Einsatz hauptsächlich darin, dynamische Webseiten im Web\-browser anzuzeigen. Die Verarbeitung erfolgte dabei meist clientseitig durch den Webbrowser (dem sogenannten Frontend) \cite{JS1.3}.
\newline
\noindent
Heutzutage findet sich die Sprache dagegen in wesentlich größeren Einsatzgebieten wieder. 
Bis vor einigen Jahren waren für die serverseitigen Implementierungen unter anderem Programmiersprachen wie Java oder PHP vorbehalten. Die Veröffentlichung von Node.js, einer plattformübergreifenden Laufzeitumgebung, die JavaScript außerhalb eines Webbrowsers ausführen kann, führte zu einer immer größeren Verbreitung von serverseitigen Anwendungen (dem Backend), die auf JavaScript basieren. Auf Node.js wird ausführlicher im nächsten Kapitel eingegangen. 
Ferner findet JavaScript heutzutage aber auch seinen Einsatz in mobilen Anwendungen, Desktopanwendungen, Spielen oder 3D-Anwendungen \cite{JS1.4}.


\subsection{Node.JS}
Im Jahr 2009 veröffentlichte Ryan Dahl das Framework Node.js, das auf Googles V8-Engine, welche auch als JavaScript-Engine in Googles Browser Chrome zum Einsatz kommt, basiert und sich hervorragend für hochperformante, skalierbare und schnelle Webanwendungen eignet. Zudem ermöglicht es Webentwicklern die Entwicklung von serverseitigem JavaScript-Code\footnote{\url{https://v8.dev/}, letzter Zugriff: 03.April 2021}.


\subsubsection{Architektur}
Eine wesentliche Eigenschaft von Node.js ist die hohe Performance. Im Folgenden soll der Unterschied der Node.js-Architektur zu traditionellen Webservern und der damit verbundenen höheren Performance dargestellt werden.
\newline

\noindent
Herkömmliche Webserver erstellten zunächst für jede ankommende Anfrage einen neuen Thread. Dieses Vorgehen ist eng mit steigendem Speicher- und Rechenaufwand verbunden. Um sich Rechenzeit, die durch die Erstellung und Zerstörung von Threads entstanden, zu sparen, wurden Threadpools eingerichtet. Dieser Threadpool enthält mehrere Threads, denen Aufgaben zugewiesen werden können. Nach erfolgreicher Abarbeitung einer Operation kann einem Thread eine weitere Aufgabe zugeordnet werden.
\newline

\noindent
Es bleibt aber ein weiteres Problem: Bei der Anfragenabarbeitung kann es zu einer Form von blockierender Ein- und Ausgabe (Blocking Input/Output kurz Blocking I/O) kommen: zum Beispiel beim Suchen in einer Datenbank oder dem Laden einer Datei im Dateisystem.
 Während der Abarbeitung wartet der Thread solange, bis die Operation ein Ergebnis zurückwirft und belegt dabei weiterhin Speicherplatz. 
 Bei hohem Aufkommen von Anfragen kommt es dadurch zu einer hohen Speicherauslastung des Servers. Zudem kosten die Kontextwechsel zwischen den Threads im Betriebssystem weitere Rechenzeit \cite{Node1.05}. Man spricht bei diesem Architekturkonzept auch vom Multi-Threaded Server.
\newline

\begin{figure}[h]
\centering
\includegraphics[width=10cm, height = 5.5cm]{images/nodejs_otherthreading.png}
\caption{Multithreaded / Blocking I/O \cite{Node1.1}}
\end{figure}
 
\newpage

\noindent
Node.js verfolgt einen anderen Ansatz: Anfragen werden nur in einem einzigen Thread, dem Hauptthread, abgearbeitet und in einer Warteschlange verwaltet. Dadurch bleiben Kontextwechsel zwischen Threads erspart. Hierbei handelt es sich also um einen Single-Threaded Server. Der Hauptthread verwaltet eine Schleife, die sogenannte Event Loop, die permanent Anfragen aus der Event-Warteschlange überprüft und Ereignisse, die von Ein- und Ausgangsoperationen ausgerufen werden, verarbeitet.
\newline

\noindent
Bei Ankommen einer Nutzeranfrage an einen Node.js Server wird zunächst in der Event Loop geprüft, ob diese Anfrage Blocking I/O benötigt. Falls nicht, kann die Anfrage direkt bearbeitet werden und die Antwort an den Nutzer zurückgesendet werden. 
\newline

\noindent
Im anderen Fall wird einer von Node.js interner Workern, welche prinzipiell auch Threads sind, aufgerufen, um die jeweilige Operation auszuführen. Dabei wird eine Callback-Funktion mitgegeben, die vom Worker aufgerufen wird, sobald die Operation ausgeführt wurde. Diese Callback-Funktion kann anschließend als Ereignis von der Event Loop registriert werden. Man spricht hierbei auch von ereignisgesteuerter Architektur. [1.4]
\newline
 
\begin{figure}[h]
\centering
\includegraphics[width=10cm, height = 5.5cm]{images/nodejs_nodethreading.png}
\caption{Single Threaded / Non Blocking I/0 \cite{Node1.1}}
\end{figure}
 

\noindent
Der große Vorteil hierbei ist, dass der Hauptthread trotz der blockierenden Ein- und Aus\-gabeoperationen nicht anhält, und weitere Anfragen bearbeiten kann. (Non Blocking I/O - Prinzip) 
\newline

\newpage
\subsubsection{Module}

\noindent
Module stellen in Node.js Software-Komponenten dar, die Objekte und Funktionen nach außen hin bereitstellen sollen.
Sie können aus einer Skriptdatei oder einem Verzeichnis von Dateien bestehen. Module können als einzelne Default-Komponenten, die den Hauptteil des Moduls repräsentiert, exportiert werden. 
Bei der anderen Möglichkeit, des sogenannten ‚benannten Exports‘ werden die zu exportierenden Komponenten dagegen explizit angegeben. Letzteres ist in nachfolgender Abbildung dargestellt. 
\newline
  
    
\begin{lstlisting}[caption=Benannter Export von Modulen,label=lst:ModuleExport]
function foo(){}
function bar(){}

//Obige Funktionen exportieren:
module.exports.foo = foo;
module.exports.bar = bar;
\end{lstlisting}


\noindent
Für den Import stehen verschiedene Möglichkeiten zur Verfügung.
Im folgender Abbildung ist ein Import über die require()-Funktion dargestellt. 
Mit mitgeliefertem Modul-Pfad als Parameter gibt diese Funktion ein Objekt des Moduls wieder, das die exportierten Objekte (und Funktionen) enthält.
\newline
  
\begin{lstlisting}[caption=Import von Modulen,label=lst:ModuleImport]
//Importieren der Funktion einer anderen Datei:
const foo = require('./module/path');
const bar = require('./module/path');
\end{lstlisting}

\noindent
Eine wichtige Besonderheit ist, dass importierte Module  beim ersten Aufruf gecached werden. 
Das bedeutet, dass jeder require()-Aufruf auf ein Modul dasselbe Objekt zurückliefert \cite{Node1.21}.


\paragraph{npm}
Ehemals als Node Package Manager bekannt, ist npm ein Paketmanager für Node.js, entwickelt 2010 von Isaac Z. Schlueter \cite{Node1.3}. Es verwaltet ein öffentliches Repository (ein digitales Software-Verzeichnis im Internet) unter dem Name npm Registry. In dem Verzeichnis werden weit über 1 Millionen Pakete (Module) angeboten \cite{Node1.4}. Der Großteil kann unter freier Lizenz verwendet werden. Mit npm können Module installiert, aktualisiert, entfernt und gesucht werden. Node.js liefert seit seiner Version 0.6.3 npm standardmäßig bei der Installation mit \cite{Node1.5}.

\paragraph{Express}
„Express ist ein einfaches und flexibles Node.js-Framework von Webanwendungen, das zahlreiche leistungsfähige Features und Funktionen für Webanwendungen und mobile Anwendungen bereitstellt“\cite{Node1.6}.  Es wurde im November 2010 von Douglas Christopher Wilson und weiteren Entwicklern veröffentlicht und erweitert Node.js um das Abarbeiten verschiedener HTTP-Methoden, das separate Abarbeiten von Anfragen mit verschiedenen URL-Pfaden sowie weiterer nützlicher Möglichkeiten. Im Grunde handelt es sich bei Express um ein Modul, dass durch den npm Package Manager heruntergeladen werden kann. Die aktuelle Version zum Zeitpunkt der Dokumentation [??TODO] ist 4.17.1 \footnote{\url{https://www.npmjs.com/package/express}, letzter Zugriff 04.04.2021}
\newline
\newline
\textbf{Beispiel}
 \newline

\noindent
Das Erstellen einer einfachen Express-Applikation wird im folgenden Beispiel dargestellt:\newline

\begin{lstlisting}[caption=Einfacher Webserver [nodejs 1.8],label=lst:Middleware]
const express = require('express');
const app = express();
const port = 3000;

app.get('/', (req,res)=> {
	res.send('Hello World')
});

app.listen(port, () => {
	console.log("Example app listening on port ${port}!")
});
\end{lstlisting}

\noindent
Die require()-Funktion importiert das Express-Modul und gibt ein Express-Objekt zurück. 
Dieses Objekt als Funktion aufgerufen gibt wiederum ein Objekt der Express-Applikation zurück, welche traditionell „app“ genannt wird, das Kernstück des Express-Frameworks ist und sämtliche Methoden wie das Weiterleiten von HTTP Anfragen, das Konfigurieren von Middleware oder das Modifizieren des Webserver-Verhaltens beinhaltet \cite{Node1.8}.
\newline
\noindent
Im mittleren Block befindet sich eine Routendefinition. Die app.get() Funktion spezifiziert eine Callback-Funktion, die ein „request“- und „response“-Objekt als Parameter erhält und aufgerufen wird, sobald eine HTTP Anfrage der Methode GET mit dem Pfad ‚/‘ empfangen wird. Das Request-Objekt enthält sämtliche Informationen über die HTTP-Anfrage. Das Response-Objekt kann dagegen in der Callback-Funktion mit Informationen gefüllt werden und über die send()-Funktion als HTTP-Antwort an den Sender zurückgesendet werden.
\newline
\noindent
Der unterste Block startet den Webserver auf dem mitgegebenen Port über die Funktion app.listen(). Ihr kann auch eine Callback-Funktion mitgegeben werden, die aufgerufen wird, sobald der Server erfolgreich gestartet ist.
\newpage
\noindent
\subparagraph{Middleware}
Express arbeitet nach dem Middleware-Konzept. Darunter versteht man Funktionen, die für die Verarbeitung von Anfragen hintereinandergeschaltet werden können. Jede Middleware hat Zugriff auf das Anfrageobjekt, das Antwortobjekt und die jeweils nächste Middleware-Funktion \cite{Node1.9}.
Dabei kann die HTTP-Request direkt terminiert oder an die nächste Middleware gesendet werden. Die Verkettung der Middleware-Funktionen wird in folgender Abbildung illustriert:
\newline

\begin{figure}[h]
\centering
\includegraphics[width=12cm]{images/nodejs_middleware.png}
\caption{Middleware \cite{Node1.2}}
\end{figure}

%
%       Middleware
%			express.json
%

\noindent
\subparagraph{Middleware: express.json}

\noindent
Hierbei handelt es sich um eine in express eingebaute Middleware, die die in JSON formatierten Daten im Nachrichtenrumpf aus einer eingehenden HTTP-Anfrage grammatisch analysiert.  Dabei ist zu beachten, dass der Nachrichtenrumpf nur dann analysiert wird, wenn bei der Anfrage eine Header-Informationen namens „Content-Type“ mit dem entsprechenden JSON-Typ als Wert übergeben wird. Nach erfolgreicher Analyse erstellt die Middleware aus den JSON-Informationen eine neues body-Objekt innerhalb des übergebenen request-Objekts. [nodejs 2.1]
\newline

\begin{lstlisting}[caption=Express.json Middleware benutzen,label=lst:ExpressNutzen]
const express = require('express');
const app = express();
app.use(express.json());
\end{lstlisting}

%
%       Middleware
%			Router
%

\newpage
\noindent
\subparagraph{Middleware: Router}

\noindent
Unter dem Begriff Routing (Weiterleitung) versteht man im Kontext von Express „[...] die Definition von Anwendungsendpunkten (URIs) und deren Antworten auf Clientanforderungen.“ [nodejs 2.15]
\newline

\noindent
Die in express eingebaute Middleware express.Router ermöglicht es, modular einbindbare Routenhandler (Weiterleitungsroutinen) zu erstellen. Eine Router-Instanz ist als vollständiges Middleware- und Routingsystem zu sehen und wird deshalb auch als „Mini-App“ angesehen. Der sich durch die Modularität herausziehende Vorteil ist, dass folglich unterschiedliche Anwendungsendpunkte auf entsprechende Dateien ausgelagert werden können.
\newline



\begin{lstlisting}[caption=Routinghandler erstellen \protect \footnotemark,label=lst:RoutingHandlerCreate]
var express = require('express');
var router = express.router();

// Middleware explizit fuer diesen Router
router.use(function timeLog(req,res,next) {
	console.log('Time: ', Date.now());
	next();
});

// Homepage Route - Abhandlung
router.get('/', function(req,req){
	res.send('Birds home page');
});

// About Route - Abhandlung
router.get('/about', function(req,req){
	res.send('About birds');
});
module.exports = router;
\end{lstlisting}
\footnotetext{Express, API-Dokumentation Router. \url{https://expressjs.com/en/api.html\#router}, letzter Zugriff: 05.April 2021}

\noindent
In oberem Beispiel wird ein Routerhandler für das Verzeichnis ‚/birds‘ mit eigen implementierter Middleware und zwei Anwendungsendpunkte ‚/‘ (bezieht sich auf das Stammverzeichnis) und ‚/about‘ erstellt. Der Code wird unter der Datei birds.js abgespeichert. 
Abschließend kann das Routermodul in die Anwendung geladen werden: 
\newline

\begin{lstlisting}[caption=Routinghandler benutzen,label=lst:RoutingHandlerUsage]
var birds = require('./birds');
..
app.use('birds', birds);
\end{lstlisting}

%
%       Mongoose
%
%

\newpage
\paragraph{Mongoose}
Mongoose ist ein öffentliches Modul, das zum Zeitpunkt der Dokumentation[TODO??] im npm Package Manager in der Version 5.12.3 zur Verfügung steht \footnote{npm mongoose. \url{https://www.npmjs.com/package/mongoose}, letzter Zugriff: 05.April 2021}. Bei diesem Modul handelt es sich um ein Object-Document Mapper (ODM), der es ermöglicht, asynchron mit einer NoSql-Datenbank zu kommunizieren. Mongoose ist der populärste und am weitest von MongoDB unterstützte ODM \cite{Node2.55}. Es unterstützt neben transparenter Persistenz auch die Datenvalidierung, das Erstellen von Abfragen (Queries), das Schreiben von logischem Business Code und die Übertragung zwischen Objektem im Code und der Repräsentierung dieser Objekte in der Datenbank.
\newline

%
%       ODM
%
%

\noindent
\subparagraph{Object Document Mapping (ODM)}
Object-Relational Mappers (ORM) finden haupt\-sächlich Einsatz in objektorientieren Anwendungen, dessen Daten in relationalen Datenbanken sind. Dabei werden die Tabellen in persistente Objekte gemappt.
Das Mappen ist aber auch für NoSQL-Datenbanken nützlich \cite{Node2.56}. Die meistverbreiteten NoSQL-Datenbanken basieren auf Dokument-Systemen. Dementsprechend werden für diese Datenbanken Object-Document Mapper für das Mappen zwischen Dokumenten und Objekten genutzt. Einige ODM’s sind Mongoose\footnote{Mongoose Webpage, \url{http://mongoosejs.com}, letzter Zugriff 04.04.2021}, Morphia\footnote{Morphia Webpage, \url{https://github.com/mongodb/morphia}, letzter Zugriff 04.04.2021}, Doctrine \footnote{Doctrine Project Webpage, \url{http://www.doctrine-project.org/}, letzter Zugriff 04.04.2021} und Mandango\footnote{Mandango Webpage, \url{https://mandango.readthedocs.io/en/latest/}, letzter Zugriff 04.04.2021.}
NoSQL Mapper nutzen vom Entwickler definierte Datenschemata, die das Objekt beschreiben. Ein daraus abgeleitetes Model-Objekt ermöglicht dann die Kommunikation zwischen dem im Schema beschriebenen Objekt und der entsprechenden Datenbank-Collection.
\newline

%
%       Schema
%
%

\noindent
\subparagraph{Schema}
Mongoose-Schemata definieren die Struktur der gespeicherten Daten einer Mongo"-DB-"-Collection in der Anwendungsschicht und werden in der JSON-Notation beschrieben. Dokumentenbasierte Datenbanken wie MongoDB enthalten für jede Wurzelentität eine Collection. Mongoose Schemata werden für jede Collection definiert. Innerhalb der JSON-notierten Schemabeschreibung können den einzelnen Eigenschaften bestimmtes Verhalten zugeordnet werden. Zum Beispiel lässt sich explizit der Datentyp angeben (type), eine Eigenschaft verpflichtend (required) oder in Kleinbuchstaben einstellen (lowercase).
\newline


\begin{lstlisting}[caption=Mongoose Schema - Beispiel,label=lst:MongooseSchema]
const schema = new Schema({
 attributeX: {
 	type: String,  // Datentyp
 	required: true,  // Verpflichtendes Attribut?
 	lowercase: true; // Kleinbuchstaben?
});
\end{lstlisting}

%
%       Model
%
%

\newpage
\noindent
\subparagraph{Model}
Ein Model in Mongoose ist ein aus einer Schemadefinition erstellter Konstruktor, aus denen Objekte instanziiert werden können. Diese Instanzen werden auch ‚documents‘ genannt. Sie stehen in direkter Verbindung zu den jeweiligen Collections der verbundenen Datenbank und enthalten Methoden für die persistente Speicherung, Bearbeitung oder Löschung. Beispielsweise wird beim Abspeichern einer Mongoose Instanz eines Models die entsprechende Collection in der Datenbank erzeugt, sofern sie noch nicht vorhanden ist. Eine Konvention in Mongoose sieht vor, dass der Name eines Models dem Singular eines Nomens entspricht, während die Collections nach dem Plural dieses Namens beschrieben werden \cite{Node3.2}. Im folgenden Beispiel wird ein Model über die mongoose.model()-Funktion erstellt unter Angabe des Modelnamens und dem zu verwendenden Schema. Dieses Model wird über module.exports nach außen zur Verfügung gestellt.

\begin{lstlisting}[caption=Model erstellen und exportierenn,label=lst:MongooseObjectExport]
const mongoose = require('mongoose');
const testSchema = new mongoose.Schema({
	attributeX: {
		type:String,
		required:true,
		lowercase: true
	}
});
module.exports = mongoose.model('test',testSchema);
\end{lstlisting}

\noindent
An anderer Stelle kann das Model nun importiert werden. Aus dem Model kann ein Objekt instanziiert werden, welches über die save()-Funktion in der Datenbank gespeichert werden kann.

\begin{lstlisting}[caption=Model importieren - Objekt instanziieren und persistent speichern,label=lst:MongooseObjectInstance]
const testModel = require(test);

var testInstanz = new testModel();
await testInstanz.save();
\end{lstlisting}

\noindent
Mongoose Models enthalten ohne Instanziierung des Weiteren auch Schnittstellen, um Daten der zugehörigen Collection zu kreieren, abfragen, bearbeiten oder löschen. (Create, Receive, Update, Delete oder auch kurz CRUD).
\newline


\begin{lstlisting}[caption=CRUD-Beispielfunktionen eines Mongoose-Models,label=lst:MongooseCrud]
const testModel = require(test);

//Create
testModel.Insert({attributeX: "abc"})
//Receive
var testObjects = await testModel.find();
var testObject = await testModel.findOne({attributeX: "abc"})
//Update
await testModel.updateONe({X:"abc"},{X: "cba"});
//Delete
await testModel.deleteMany({X:"abc"})
\end{lstlisting}

\newpage
\noindent
\subparagraph{Verbindung}
Verbindung zur Datenbank kann über die connect()-Funktion mit Angabe der genutzten Datenbank und des Datenbankpfads hergestellt werden. 
Über das mongoose.connection-Objekt können auf Verbindungsereignisse reagiert werden. 
\newline

\begin{lstlisting}[caption=Mongoose: Verbindung zur Datenbank aufbauen,
label=lst:MongooseConnect]
const mongoose = require('mongoose');
await mongoose.connect("mongodb://127.0.0.1:27017/TestDB");
mongooose.connection.on('error',(error) => console.log(error));
mongooose.connection.on('open',() => console.log('Connected'));
\end{lstlisting}

\noindent
Für den Verbindungsaufbau können weitere Option übergeben werden. Dafür kann ein Objekt wie in folgendem Beispiel erstellt werden, dass die zugehörigen Optionen als Attribute beinhaltet. 
\newline

\begin{lstlisting}[caption=Mongoose Verbindungsoptionen \protect \footnotemark  ,label=lst:MongooseConnect]
const options = {
	useNewUrlParser: true,
	useUnifiedTopology: true,
	useCreateIndex: true,
	useFindAndModify: false,
	autoIndex: false,
	poolSize: 10, // Anzahl der max. Socket Connections
	serverSelectionTimeoutMS: 5000, // TimeOut bis verbunden
	socketTimeoutMS: 45000, // Schliesse Socket bei 45s                 
	                        // Inaktivitaet
	family:4 // Use IPv4
}
\end{lstlisting}
\footnotetext{Mongoose Connections, \url{https://mongoosejs.com/docs/connections.html}, letzter Zugriff: 05.April 2021}


\paragraph{Weitere Module}

\begin{table}[tbt]
\caption{Module}
\begin{center}
    \begin{tabular}{| l | p{8cm} |}
    \hline
    Express-Modul & Beschreibung \\ \hline
    fs & Erlaubt die Interaktion mit dem Dateisystem.\newline
	Zum Beispiel Schreiben/Lesen von Dateien.\\
    
    \hline
    http & Ermöglicht Datentransfer über das Protokol HTTP und das Abhören eines Ports.  \\
    
    \hline
	https & Gesicherte Variante zu HTTP mit SSL.\newline
	Benötigt Private Key und Zertifikat.  \\
	
    \hline    
    firebase-admin & Ermöglicht die Verbindung zu Google Firebase 			Cloud. \\ 
    
    \hline    
    node-cron & Ermöglicht das Einstellen von sich wiederholenden 			Aufgaben zu bestimmten Zeitintervallen.  \\
    \hline
    \end{tabular}
\end{center}
\end{table}

\subsection{Representational State Transfer - Application Programming Interface}
\input{Theoretische_Grundlagen/Grundlagen_RestAPI}
\subsection{NoSQL-Datenbank}
Unter NoSQL („Not only SQL“) werden Datenbanksysteme bezeichnet, die einen nicht-relationalen Ansatz verfolgen. Im Vergleich zu relationalen Datenbanken, welche die Daten in tabellenförmigen Strukturen mit Spalten und Zeilen speichern, nutzt eine NoSQL-Datenbank andere Strukturkonzepte für die Speicherung der Daten wie zum Beispiel Wertpaare, Dokumente, Objekte oder Listen und Reihen. Da NoSQL einige der bekannten Schwächen von relationalen Datenbank, wie Performance-Schwierigkeiten bei hohem Lastaufkommen oder bei dem Umgang mit großen Datenmengen, vermeidet, erfreut sich diese Technologie in der heutigen Zeit des großen Datenaufkommens immer größerer Beliebtheit (\cite{DB1}, S. 149).
Zu den bekanntesten NoSQL-Datenbanken gehören beispielsweise Apache Cassandra, MongoDB und CouchDB.
\newline

\subsubsection{NoSQL-Datenbanktypen}
NoSQL-Datenbanken werden hauptsächlich in vier verschiedene Kategorien unterteilt, die unterschiedliche Konzepte verfolgen.
\newline

\noindent
\hangindent1cm
\textbf{Graphendatenbanken:}
Dieses Datenbankenkonzept speichert die Informationen in Netzstrukturen, den sogenannten Graphen ab. Die einzelnen Informationselemente werden durch Knoten mit Eigenschaften repräsentiert. Um die Beziehungen zwischen den Knoten darzustellen, werden Kanten genutzt, die gerichtet und benannt sein können und ebenfalls Eigenschaften besitzen. Das Konzept wird in Abbildung \ref{Graphendatenbank} illustriert.\\

\begin{figure}[tbt]
\centering
\includegraphics[]{images/graphikdatabase.jpg}
\caption[Graphendatenbank Beispiel]{Graphendatenbank Beispiel \protect \footnotemark}
\label{Graphendatenbank}
\end{figure}
\footnotetext{\url{https://entwickler.de/online/datenbanken/wunderbare-welt-der-graphen-114728.html}, letzter Zugriff: 9. Februar 2021}

\noindent
\hangindent1cm
\textbf{Dokumentenorientierte Datenbanken:}
Im Kontext der dokumentenorientierte Datenbanken sind Dokumente Objekte mit Eigenschaften, die in einer Sammlung gespeichert werden. 
Während eine Sammlung (Collection) eine Tabelle im relationalen Datenbank widerspiegelt, ist ein Dokument als ein Eintrag beziehungsweise einer Zeile dieser Tabelle gleichzusetzen, mit dem großen Unterschied, dass dokumentenorientierte Datenbanken schemafrei sind und kein bestimmtes Datenschema voraussetzen.
Dokumente können weitere Dokumente als Attribut oder in einer Liste einbetten und bieten so die Möglichkeit, komplexe Datenstrukturen zu speichern.  
In aktuellen Datenbanksystemen wie CouchDB und MongoDB nutzen diese Dokumente Datenformate wie JSON oder XML.\\

\begin{lstlisting}
{
	"Vorname": "Max",
	"Nachname": "Mustermann",
	"Telefon-Nr": "0124567",
	"Alter": 33,
	"Adresse": "Musterstrasse 34, MusterStadt",
	"Kinder":  ["Junior","Elenor"]
}
\caption{Inhalt eines dokumentenorientieren Datensatzes}
\end{lstlisting}

\noindent
\hangindent1cm
\textbf{Key-Value-Datenbanken:}
Key-Value-Datenbanksysteme bilden mit einer Abbildung der Daten in Schlüssel- und Wertpaare die einfachste NoSQL-Datenbankumsetzung ab. Bei diesem Konzept werden eindeutigen Schlüsselattributen (Key) jeweils ein beliebiger Wert (Value) zugeordnet.\\ %(siehe Tabelle \ref{tab:key_value_db}).

%\begin{table}[tbt]
%\caption{Key-Value Beispiel}
%\begin{center}
%    \begin{tabular}{ l  l }
%	\toprule
%    \textbf{Key} & \textbf{Value} \\
%    \midrule
%    K1 & AAA,BBB,CCC\\
%    
%
%    K2 &  AAA,BBB\\
%
%
%	K3 & AAA,DDD\\
%
%
%    K4 & BBB,2,01/01/2015 \\
%	\bottomrule
%    \end{tabular}
%\end{center}
%\label{tab:key_value_db}
%\end{table}

\noindent
\hangindent1cm
\textbf{Spaltenorientierte Datenbanken:}
Spaltenorientierte Datenbanken, oder auch „Wide\--Column“-Datenbanken genannt, speichern ihre Datensätze in Form von Tabellen.  Sie wirken zunächst den Tabellen der relationalen Datenbanksysteme sehr ähnlich, unterscheiden sich grundlegend aber in der Speicherung der Daten, die nicht zeilenorientiert, sondern spaltenorientiert abgelegt werden. 
Die zeilen- und die spaltenorientierte Speicherung ist in Abbildung \ref{fig:Spaltenorientiert} dargestellt. Zeilenorientierung bei der Speicherung bietet vorallem Vorteile bei Abfragen, bei denen Informationen aus mehreren Spalten benötigt werden.
Spaltenorientierung dagegen eignet sich sehr gut für die Auswertung von Aggregaten.\\

\begin{figure}[h]
	\begin{subfigure}{\textwidth}
		\centering
		\includegraphics[width=8cm, height = 3cm]{images/SpaltenorientiereDatenbank.png}
		\caption{}
		\label{fig:spalten_zeilen_speicher}
	\end{subfigure}
	\newline
	\begin{subfigure}{\textwidth}
		\centering
		\includegraphics[width=15cm]{images/zeilenspaltenorientiert.png}
		\caption{ }
		\label{fig:spalten_zeilen}
	\end{subfigure}
	\caption[Zeilen- und spaltenorientierte Speicherung]{(a) Spaltenorientierte Datenbank Beispiel (b) Zeilen- und spaltenorientierte Speicherung \protect \footnotemark}
\label{fig:Spaltenorientiert}
\end{figure}
\footnotetext{\url{https://www.axiom-net.de/die-datenbank-hana-ist-spaltenorientiert}, letzter Zugriff: 9. Februar 2021}

%\subsubsection{BASE}
%TODO
%\newline

\subsubsection{CAP-Theorem}
Der Informatiker Eric Brewer von der Universität Berkeley stellte Anfang des Jahres 2000 die Annahme auf, dass ein System nicht gleichzeitig die drei Kerneigenschaften Consistency (Konsistenz), Availability (Verfügbarkeit) und Partition Tolerance (Ausfalltoleranz) abdecken kann (\cite{DB1.6}, S. 34). Diese Annahme wird in Abbildung \ref{fig:CAP} dargestellt.
\newline

\begin{figure}[tbt]
\centering
\includegraphics[width=8cm, height=6.5cm]{images/Visualization-of-CAP-theorem.png}
\caption[Visualisierung des CAP-Theorems]{Visualization-of-CAP-theorem \protect \footnotemark}
\label{fig:CAP}
\end{figure}
\footnotetext{\url{https://www.researchgate.net/profile/Hamzeh-Khazaei/publication/282679529/figure/fig2/AS:614316814372880@1523475950595/Visualization-of-CAP-theorem.png}, letzter Zugriff: 10. Februar 2021}

\noindent
Die \textbf{Konsistenz (C)} beschreibt, dass Datenzustände, die in einem verteilten System geändert  werden, in jedem zusammenhängenden System gleich sein müssen. Der Zustand der Daten soll somit im gesamten System übereinstimmen. Ein System, das einen ununterbrochenen Betrieb und eine akzeptable Antwortzeit aufweisen kann, besitzt eine hohe \textbf{Verfügbarkeit (A)}. Die \textbf{Ausfalltoleranz (P)} steht für ein Verhalten, bei dem es bei Ausfallen eines Bestandteiles innerhalb eines Systems  zu keinem Gesamtausfall kommt.

\newpage
\subsubsection{MongoDB}
MongoDB ist ein in C++ geschriebenes, dokumentenorientiertes NoSQL-Datenbanksystem, das im Jahre 2009 von den Entwicklern Horowitz und Merriman als Open-Source Datenbank veröffentlicht wurde und die am weitest-verbreiteste NoSQL-Datenbank (Stand April 2021) \cite{DB1.7}. Die Intention der Gründer war es, eine Datenbank mit höherer Skalierbarkeit, Flexibilät und Performance zu entwerfen, die auf auf einer einfachen Handhabung beruht \cite{DB1.65}.
Gründe der Popularität der Datenbank ist neben den oben erwähnten Eigenschaften die flexible Gestaltungsmöglichkeit der Datenstrukturen sowie die Unterstützung durch zahlreiche Programmiersprachen und Betriebssysteme.
\noindent
Dem Konzept des CAP-Theorems folgend steht MongoDB für Konsistenz und Partitionstoleranz, dafür ordnet sich die Verfügbarkeit den anderen Eigenschaften unter.
\newline

\paragraph{Struktur}

\begin{figure}[htb]
\centering
\includegraphics[width=14cm]{images/MongoDB_Architektur.png}
\caption{Graphikdatenbank Beispiel \protect \footnotemark}
\end{figure}
\footnotetext{\url{https://www.informatik-aktuell.de/betrieb/datenbanken/mongodb-fuer-software-entwickler.html}, letzter Zugriff: 9.02.2021}


\noindent
Letztere Abbildung stellt die grundsätzliche Struktur einer MongoDB-Instanz dar.
Ein \textbf{Server} kann mehrere logische \textbf{Datenbanken} verwalten, die ihrerseits einen oder mehrere logische Namensräume enthalten, die sogenannten \textbf{Collections}. Eine Collection verwaltet die einzelnen Datensätze, die als \textbf{Dokumente} bekannt sind. 
\newline

\noindent
\subparagraph{Schema-Freiheit}
Collections sind schemafrei. Dadurch gibt es für die zugehörigen Dokumente kein vorausgesetztes Schema. Aufgrund der Schema-Freiheut dürfen dennoch Architekturentscheidungen bei der Datenmodellierung nicht ignoriert werden. Stattdessen werden diese Entscheidungen des Schema-Managements auf die Anwendungsentwicklung verlagert.
\newline

\noindent
\subparagraph{BSON}
Während MongoDB für Datenaustausch das JSON-Format nutzt, hält es seine Dokumente im Binary JSON-Format (BSON), einer binärcodiertem Erweiterung des JSON-Formats. Daten im BSON-Format enthalten zusätzlich Informationen zum Typ und zur Länge der Informationen, wodurch schnelleres Parsen von Daten möglich ist. Des Weiteren ist BSON um zusätzliche Datentypen wie 32- und 64-bit Integer oder das Datum erweitert \footnote{\url{https://www.mongodb.com/json-and-bson]}}.
\newline

\begin{lstlisting}[caption=JSON - BSON Vergleich, label=lst:JSONBSON]
/* JSON */
{"hello": "world"}  // "key":"value"

/* BSON */
\x16\x00\x00\x00           // Gesamtgroesse des Dokuments
\x02                       // 0x02 = Typ String
hello\x00                  // Feldname
\x06\x00\x00\x00world\x00  // Feldwert
\x00                       // 0x00 = Typ EOO ('end of object')
\end{lstlisting}


\noindent
\subparagraph{ObjectID und Primärschlüssel}
Für die eindeutige Identifikation eines Dokuments vergibt MongoDB automatisch erstellte '\_id\'-Felder. Dabei wird bei der Generierung des BSON-Dokuments für den Wert das '\_id'-Feld ein Datentyp 'ObjectId\'  hinzugefügt. Dieses Identifikationsfeld ist gleichzusetzen mit den Primärschlüsseln aus relationalen Datenbanken \cite{DB1.85}. Ist eine automatische Generierung des eindeutigen Identifikationsfeldes nicht erwünscht, kann der Anwendungsentwickler dieses Feld auch selbst erzeugen, indem er die Eigenschaft explizit dem zu erstellenden Objekt hinzufügt und mit einem Wert verseht. 
\newline

\paragraph{Datenbankabfrage}
Im Vergleich zu relationalen Datenbanken benutzt MongoDB keine Abfragesprache wie SQL. Stattdessen ermöglicht dieses Datenbanksystem drei Arten von Abfragen, wobei eine Abfrage sich immer auf genau eine Collection bezieht. Ein Bezug einer Abfrage zu mehreren Collections, wie es relationale Datenbanken mit „Join“-Operationen erlauben, ist hier nicht gegeben und muss in der Anwendung realisiert werden \cite{1.9}. Abfragen in MongoDB werden grundsätzlich in Form von Dokumenten formuliert. Dieses Verfahren ermöglicht schnelles Abhandeln komplexer Abfragen von tief geschachtelten Dokumentenstrukturen. Nachfolgend werden die drei Abfragemöglichkeiten erläutert.

\noindent
\subparagraph{Query-By-Example}
Das folgende Beispiel zeigt einen Aufruf aller Einträge, die als Ort Karlsruhe eingespeichert haben. 
\newline
\begin{lstlisting}[caption=MongoDB Read, label=lst:MongoDBRead]
db.pois.find( {"adresse.ort": "Karlsruhe" } )
\end{lstlisting}

\noindent
Dabei kommt das Prinzip Query-by-Example \footnote{Query By Example: \url{https://de.wikipedia.org/wiki/Query_by_Example}, letzter Zugriff 12.02.2021} zum Einsatz, bei dem die als JSON-Dokument beschriebenen Suchkriterien als Filter auf die durchsuchte Collection wirkt. Zusätzlich stehen für diese Art von Abfragemöglichkeit sämtliche logische Verknüpfungen und Vergleichsoperatoren zur Verfügung\footnote{Query-Operatoren\url{https://docs.mongodb.com/manual/reference/operator/query/}, letzter Zugriff 12.02.2021}.
Das Ergebnis der obigen Find-Operationen liefert dabei einen Cursor zurück, über den die aufrufende Anwendung durch die einzelnen zurückgelieferten Dokumente iterieren kann. Außerdem kann der Cursor modifiziert werden, um beispielsweise Beschränkungen der Trefferanzahl oder Sortierung vorzunehmen.
\newline
\begin{lstlisting}[caption=MongoDB Read Modifikation, label=lst:MongoDBReadModifikation]
db.pois.find().limit(5).sort({"adresse.ort": -1})
\end{lstlisting}

\noindent
\subparagraph{MapReduce}
MapReduce ist ein allgmeines Programmiermodell zur verteilten und parallelen Verarbeitung von großen Datenmengen in aggregierte Ergebnisse. Dabei unterteilt sich der Algorithmus im Wesentlichen in zwei Operationen \footnote{MapReduce Dokumentation \url{https://docs.mongodb.com/manual/core/map-reduce/}, letzter Zugriff 10.02.2021}
\newline
-	Map: Emittieren der beliebig vielen Key-Value-Paare für jedes Dokument
\newline
-	Reduce: Zusammenfassen aller Daten, die einem Kriterium entsprechen.

\noindent
\subparagraph{Aggregation-Framework}
Das Aggregation-Framework bietet als Alternative zum MapReduce-Verfahren den Vorteil der besseren Performance\footnote{\url{https://docs.mongodb.com/manual/meta/aggregation-quick-reference/}, letzter Zugriff 14.02.2021}. Dabei wird eine Pipeline genutzt, die das resultierende Dokument einer Operation an die Eingabe der nächsten Operation weiterleitet. Die entsprechende Funktion ist die Aggregate()-Operation, deren Parameter als Array von Dokumenten die Pipeline-Operatoren abbilden . 
\newline 

\begin{lstlisting}[caption=MongoDB Aggregate, label=lst:MongoDBAggregate]
db.pois.aggregate([
    {$group: {_id: "$adresse.ort", n: {$sum:1}}},
    {$sort: {n: -1}}
])
\end{lstlisting}


Es stehen folgende Pipeline-Operatoren zur Verfügung:

 

\begin{table}[h]
\begin{center}
    \begin{tabular}{| l | p{14cm} |}
    \hline
    \textbf{Operator}  & \textbf{Beschreibung} \\ \hline
    \$match &Sucht Dokumente analog zu find(). Sollte idealerweise mindesten 1x zu Beginn der Pipeline ausgeführt werden, um die Ergebnismenge einzuschränken.\\
    
    \hline
    \$project & Schränkt auf eine Teilmenge von Feldern ein und verändert die Feldwerte.  \\
    
    \hline
	\$sort & Sortiert die Dokumente. Analog zu sort() bei find(). .\newline
	Benötigt Private Key und Zertifikat.  \\
	
    \hline    
    \$skip & Überspringt n Dokumente. Analog zu skip() bei find().  \\ 
    
    \hline    
    \$limit & Begrenzt auf n Dokumente. Analog zu limit() bei find().  \\
    \hline 
   \$group   &	Gruppiert nach einem oder mehreren Feldern. \\ 
     \hline 
\$unwind  &	Wird auf ein Array angewendet. Jeder Array-Eintrag generiert dann ein neues Dokument für die nächste Pipeline-Stufe.\\ 
\hline  
\$redact  &	Filtert Felder des Dokuments in Abhängigkeit vom Inhalte anderer Felder.\\ 
\hline  
\$out  &	Leitet das Ergebnis der Aggregation in eine Collection um. Kann nur als letzter Operator verwendet werden.  \\ 
    
    \hline
    \end{tabular}
\end{center}
\caption{Aggregation Framework: Pipeline-Operatoren \protect \footnotemark}
\end{table}
\footnotetext{\url{https://www.informatik-aktuell.de/betrieb/datenbanken/mongodb-fuer-software-entwickler.html}, letzter Zugriff: 9.02.2021}



\paragraph{CRUD} TODO TABLE :(
MongoDB unterstützt eine Vielzahl an Operationen zum Erzeugen, Lesen, Updaten und Löschen von Daten.
Über folgende Kommandos lassen sich die CRUD-Operationen realisieren:
\newline

\begin{table}[htb]
\begin{center}
    \begin{tabular}{| l | p{8cm} | l |}
    \hline
    \textbf{Operation} & \textbf{Beschreibung} & \textbf{MongoDB Methode} \\
    
    \hline
    Create & Datensätze erstellen & .Insert() \\
    
    \hline
	Read & Datensätze lesen & .Find() \\
	
    \hline    
    Update & Daten aktualisieren & .Update() \\ 
   
    \hline    
    Delete & Datensätze entfernen & .Remove()  \\ 
    
    \hline
    \end{tabular}
\end{center}
\caption{CRUD-Operatoren}
\end{table}

\noindent
\subparagraph{Create}
„InsertOne()“ erzeugt ein einzelnes Dokument in der jeweiligen Collection der Datenbank. Dabei wird ein Dokument im JSON-Format als Parameter mitgegeben. ObjectID wird, wenn nicht als Eigenschaft im Dokument angegeben, automatisch von MongoDB erzeugt. Ebenfalls wird die angesprochene Collection automatisch erzeugt, falls sie nicht bereits existiert. Die Funktion „InsertMany()“  erlaubt das Hinzufügen mehrerer Datensätze über ein Array von JSON-Dokumenten als Parameter. Die Methode „Insert“ ist flexibler und bietet beide Parametrisierungsmöglichkeiten\footnote{\url{https://docs.mongodb.com/manual/reference/method/db.collection.insert/\#mongodb-method-db.collection.insert}, letzter Zugriff 14.02.2021}.
\newline
\begin{lstlisting}[caption=MongoDB Create, label=lst:MongoDBCreate]
db.personen.insertOne({"vorname" : "Robin"});
\end{lstlisting}

\noindent
\subparagraph{Read}
Um Datensätze aus der Datenbank zu lesen, wird die find()-Methode zur Verfügung gestellt. Sie gibt alle Dokumente einer Kollektion zurück. Wie im Kapitel Query-By-Example beschrieben, kann die Funktion mit Filtern versehen werden. Die Funktion findOne() bietet die gleiche Funktionsweise, die Dokumentenausgabe ist aber auf ein einzelnes Dokument begrenzt. 
Ein Beispiel ist dargestellt in Listing \ref{lst:MongoDBRead}.
\newline\newline

\noindent
\subparagraph{Update}
Die Update()-Methode ermöglicht es, Änderungen an Datensätzen vorzunehmen.  Die Methode erhält zwei Parameter, einen zur Angabe der gesuchten Dokumente und den anderen mit  der vorzunehmenden Änderung. Simultan zu den Insert-Methoden gibt es die UpdateOne()-Methode für Änderungen an einem Dokument und UpdateMany()-Methode für mehrere Dokumente.
\newline

\begin{lstlisting}[caption=MongoDB Update, label=lst:MongoDBUpdate]
db.personen.updateOne({"vorname" : "Robin"} , currentDate("last_login") );

\end{lstlisting}

\noindent
\subparagraph{Delete}
Für das Löschen von Datensätzen aus einer Collection gibt es die DeleteOne()- und DeleteMany()-Methode. Über mitgegegeben Parameter können Filter eingestellt werden. Eine ganze Collection kann mit der drop()-Methode entfernt werden.
\newline

\begin{lstlisting}[caption=MongoDB Remove, label=lst:MongoDBRemove]
db.personen.deleteOne( {"vorname" : "Robin"} );

\end{lstlisting}

\noindent
\paragraph{Atomare Operationen}
Als atomare Operationen wird ein Verbund von Einzeloperationen bezeichnet, der als logische Einheit betrachtet wird und nur als Ganzes erfolgreich abläuft oder fehlschlägt. In Bezug auf Datenbanken spricht man von Transaktionen, die entweder als Ganzes erfolgreich ablaufen (Commit) oder nach einer fehlerhaften Einzeloperation rückgangig gemacht werden (Rollback). Ohne atomare Operationen können Probleme aufkommen, die zu einer Inkonsistenz der Datenzustände führt. Beispielsweise würde ein Abbruch inmitten mehrerer zusammenhängender Operationsabläufe dazu führen,  dass nur ein Teil der Operationen ausgeführt wurde, während die restlichen Operationen und somit die verbleibenden Datenänderungen verworfen werden.
\newline
Bis vor dem Jahre 2018 unterstützte MongoDB keine Transaktionen. Mit der Veröffentlichung der Version 4.0 wurde der Funktionsumfang von MongoDB stark erweitert. Eines der neuen Erweiterungen war die Möglichkeit, Transaktionen durchführen zu können. Dafür werden sogenannte „Sessions“ genutzt. Sämtlichen CRUD-Operationen, die einer Transaktion zugehören sollen, werden eine Session als zusätzlicher Parameter hinzugefügt.  Bei Abschluss der Transaktion kann sie mit der Methode „session.commitTransaction()“ bestätigt werden. Andernfalls, sollte ein Fehler aufgetreten sein, kann die Methode „session.abortTransaction()“ die Operationen rückgängig machen. Eine wichtige Vorraussetzung, um Transaktionen in MongoDB nutzen zu können, ist ein Replica Set aufzusetzen. Darauf wird im weiteren Kapitel eingegangen. 
\newline


\paragraph{Architektur}
Bei der Inbetriebnahme einer MongoDB Datenbank spielen zwei Prozesse eine wichtige  Rolle. \newline Der  „mongod“-Prozess ist der primäre Hintergrundprozess für das Datenbanksystem. Er behandelt Datenabfragen, Datenzugriffe und führt benötigte Hintergrundoperationen durch\footnote{Mongod \url{https://docs.mongodb.com/manual/reference/program/mongod/}}. Beim Starten des mongod-Prozesses können über Flags Konfigurationen vorgenommen werden, beispielsweise ändert „--port 5000“ den Port. Nach erfolgreichem Start des Prozesses kann über den Standard Port der MongoDB (27017) bzw. über den konfigurierten Port eine Verbindung hergestellt werden.\newline
Der Prozess, der als Controller für sich auf mehreren Datenbanken befindenden verteilte Daten dient, nennt sich „mongos“\footnote{Mongos \url{https://docs.mongodb.com/manual/reference/program/mongos/}}. Dieser Prozess findet seinen Einsatz hauptsächlich in Kombination mit Sharding, auf die im weiteren Unterkapitel eingegangen wird. \newline

\noindent
\subparagraph{Engine}
Als Storage Engine, oder auch Datenbank-Engine, wird die zugrundeliegende Softwarekomponente eines Datenbanksystems zum Verwalten der Daten bezeichnet. Die gebräuchlichsten Engines in MongoDB sind die MMAPv1-Engine und die WiredTiger-Engine.\newline
Bis vor Version 3.2 war MMAPv1-Engine die Standard-Storage Engine von MongoDB. Eine Kompression der Daten wurde nicht unterstützt und Transaktionen waren nur auf einem Dokument ausführbar. Ab MongoDB 3.2 wird standardmäßig die Wired Tiger-Engine eingesetzt. Neben der Kompression der Daten und der miteingehenden Verringerung des benötigten Speicherplatzes bietet diese Engine auch eine Verschlüsselung der Daten an. Abhängig des benutzten Kompressionsalgorithmus kann der benötigte Verbrauch um 70\% für Daten und um 50\% für Indizes reduziert werden \cite{DB3.5}. Die WiredTiger-Engine skaliert im Vergleich zur MMAPv1 mit der Anzahl der CPU-Kerne und ermöglicht Transaktionen über mehrere Dokumente hinweg. Die genutzte Engine kann in den Konfigurationen eingestellt werden.
\newline

\noindent
\subparagraph{Replica Sets}
Im standardmäßgen Standalone-Modus besteht bei Ausfall des Servers die potenzielle Gefahr des Datenverlusts. Das Problem lässt sich durch Replikation beheben. Hierbei spricht man von der bloßen Herstellung von Mehrexemplaren(Kopien) derselben Daten, die meistens regelmäßig abgeglichen werden \cite{DB3.6}. MongoDB realisiert die Replikation der Daten über Replica Sets\cite{DB3.7}. Dabei handelt es sich um eine Gruppe von mongod-Prozessen, die dieselben Daten enthalten. Nach dem CAP-Theorem ist MongoDB nicht auf Verfügbarkeit ausgelegt. Dennoch umgeht die Datenbank diesen Nachteil über die Anwendung von Replica Sets. Das Replica Set besteht aus einem primären Knoten und daraus replizierenden Sekundären Knoten. 
Nur der primäre Knoten führt datensatzändernde Operationen aus. Veränderungen werden in sogenannten 'Oplogs' (Operations logs) gespeichert, die zum Austausch der Daten innerhalb des Replica Sets genutzt wird \cite{DB3.8}. Bei den Oplog-Dateien handelt es sich um Collections mit festgelegten Speichergrößen, den sogenannten „capped collections“. Während die Oplog-Dateien ähnlich wie ein Logbuch mit allen Veränderungen in den Datensätzen der Datenbank beschrieben werden, werden bei erreichter Maximalgröße der Collection die ältesten Daten von den neuen Daten überschrieben. Dabei enthält jeder Knoten des Replica Sets seine eigene Kopie des Oplogs und gleicht ihn mit dem des primären Knotens ab.

\begin{figure}[htb]
\centering
\includegraphics[width=7cm]{images/replicaset1.png}
\caption{MongoDB Replica Set}
\end{figure}
\footnotetext{Bild aus \cite{DB3.8}}


\noindent
\subparagraph{Sharding}
Hierbei ist die Rede von der Methodik zur Aufteilung der Daten auf mehrere Datenbanken\cite{DB4.1}.
Der Datenbestand wird nach logischen Kriterien in  Teilstücke, die sogenannten „Shards“, zerlegt und über mehrere Replica Sets verteilt.
In Hinblick auf das hinzufügen weiterer Server gewährt das Sharding eine horizontale Skalierbarkeit.
Des Weiteren werden durch die Tatsache, dass jeder Shard für seinen Bestandteil zuständig ist,  Suchanfragen schneller durchgeführt. 
Sharding ist besonders bei großen Datenmengen in Bezug auf Performance von Vorteil.
Der mongos-Prozess leitet Anfragen an die jeweiligen Shards weiter und dient als Schnittstelle zwischen den Clientanwendungen.

\paragraph{Verwaltungswerkzeuge}
\noindent
\subparagraph{Mongo Shell}
Der „mongo“-Prozess ist eine interaktive JavaScript Schnittstelle auf der Kommandozeile. Er bietet umfassende Funktionalitäten für die Systemadministration des Datenbankensystems als auch Zugriff auf die Datensätze\footnote{Mongo \url{https://docs.mongodb.com/manual/reference/program/mongo/}}. Dazu erhält man eine Eingabeaufforderung, auf dem Befehle in der Sprache JavaScript ausgeführt werden können.
\noindent
\subparagraph{Treiber}
MongoDB bietet für viele Programmiersprachen bzw. Frameworks Softwarebibliotheken zum Zugriff auf die Datenbank an.\footnote{MongoDB Treiber \url{https://docs.mongodb.com/drivers/}, letzter Zugriff 01.05.2021}  Eine offiziell unterstützte ist Mongoose für das Node.js-Framework\footnote{Mongoose - Offizielle Webpage \url{https://mongoosejs.com/docs/}, letzter Zugriff 01.05.2021}
\newpage
\noindent
\subparagraph{Grafische Oberflächen}
Es gibt einige Anwendungen zur visuellen Darstellung und Bearbeitung der Datenbanken in MongoDB, die eine grafische Benutzeroberfläche bieten. Zum Beispiel MongoDB Compass, Studio3T oder Fang of Mongo. 


\begin{figure}[h]
\centering
\includegraphics[width=15cm]{images/mongodb_compass.png}
\caption{CRUD-Besipielfunktionen eines Mongoose-Models}
\end{figure}
\subsection{Firebase}
Firebase ist eine Backend as a Service-Plattform (BaaS)  von Google für mobile und Web-Anwendungen. 
Sie soll es dem Entwickler ermöglichen, einfacher und effizienter Funktionen auf verschiedenen Plattformen bereitzustellen. 
Es stellt die nötigen Tools und Infrastruktur zur Verfügung.
Mit dem Firebase SDK bietet die Plattform API Schnittstellen zu den jeweiligen Tools, welche direkt in die Anwendung integriert werden können, ohne dass serverseitiger Code dafür notwendig ist.
Die Firebase Inc. wurde 2011 von James Tamplin und Andrew Lee gegründet und letztendlich 2014 von Google übernommen\footnote{\href{https://firebase.googleblog.com/2014/10/firebase-is-joining-google.html}{firebase.googleblog.com}, letzter Zugriff: 03. Mai 2021}.
Teile der SDK stehen seit der Google I/O 2017 unter der Apache 2.0 Lizenz, sind somit also Open-Source\footnote{\href{https://opensource.googleblog.com/2017/05/open-sourcing-firebase-sdks.html}{opensource.googleblog.com}, letzter Zugriff: 03. Mai 2021}.\\
\\
Es existieren zwei Kostenmodelle für die Nutzung von Firebase: Ein kostenloses Modell \glqq Spark Plan\grqq und ein pay-as-you-go \glqq Blaze Plan\grqq . Das kostenlose Modell beinhaltet die wichtigsten Tools, viele dieser Tools sind jedoch begrenzt durch beispielsweise Bandbreite oder Speicherplatz.
Der Pay-as-you-go Plan ist eine Erweiterung des kostenlosen Plans. 
Er bietet daher das Nutzen von Tools bis zu einem gewissen Limit kostenfrei an; darüber hinaus kostet es jedoch dann pro Nutzung.\\
\\
Ein Firebase Projekt ist die oberste Ebene in Firebase. 
Ein Projekt ist letztendlich ein \textit{Google Cloud Projekt}, welches mit speziellen Konfigurationsmöglichkeiten und Services ausgestattet ist. 
Es beinhaltet die Verknüpfung zu den einzelnen Anwendungen (also bspw. Android-, iOS- oder Webanwendung). Nun können variabel Tools, sog. Firebase products hinzugefügt werden. Diese Produkte lassen sich grundlegend in drei Kategorien einteilen. Die hier relevantesten werden im Folgenden besprochen \cite{firebase2021}.

\subsubsection{Firebase Authentifizierung}
Die Authentifizierung gehört zu den \glqq Build\grqq -Produkten und bietet eine Token-basierte Nutzerauthentifizierung. 
Hierbei kann zwischen verschiedenen Anmeldeoptionen gewählt werden: klassisch mit E-Mail und Passwort, mit OAuth2.0 Integration für Social Media (Google, Facebook, Twitter, Github, ...) oder per Telefonnummer.
Jeder Nutzer erhält eine einzigartige ID und ein zugehöriges Nutzerobjekt in einer NoSQL Datenbank. Grundlegende Werte wie E-Mail Adresse oder Name können hier abgespeichert werden; zusätzliche Informationen müssen über einen weiteren Datenbank Service abgespeichert werden.
Für die Verwaltung eines Accounts bietet dieses Tool auch eingebaute E-Mail Aktionen an - bspw. Passwort zurücksetzen oder E-Mail Adresse bestätigen.\\
\\
Ein Firebase Nutzer Objekt repräsentiert den Account eines Nutzers, welcher sich von einer Anwendung aus beim zentralen Firebase Projekt angemeldet hat.
Die Instanz eines Firebase Nutzers ist somit unabhängig von der Authentifizierungsinstanz der Anwendung, also kann eine Anwendung mehrere Nutzer anmelden, jedoch kann sich auch ein Nutzer auf mehreren Anwendungen anmelden.
Ist ein Nutzer authentifiziert, erhält die Anwendung eine Referenz des Nutzers, welche so lange existiert, bis er wieder abgemeldet ist \cite{firebase2021}.

\subsubsection{Cloud Firestore}
\label{sec:firestore}
Als Datenbank Lösung bietet Firebase zwei unterschiedliche Produkte an: Cloud Firestore und Realtime Database.
Firestore ist hier neuer, jedoch ersetzt es Realtime Database nicht. \\
Cloud Firestore ist eine flexible und auf Skalierung ausgesetzte NoSQL Cloud Datenbank, welche unter anderem die Echtzeitsynchronisierung der Daten zwischen Anwendung und Server ermöglicht.
Zusätzlich zu REST und RPC APIs in iOS, Android und web SDKs ist Firestore auch in nativen Node.js, Java, Python und Go SDKs verfügbar.\\
\\

\begin{figure}
	\centering
	\includegraphics[width=0.35\textwidth]{./Theoretische_Grundlagen/images/firestore_datastucture.png}
	\caption[Datenmodell in Firebase]{Datenmodell in Firebase \cite{firebase2021}}
	\label{fig:firestore_data_structure}
\end{figure}

\noindent
Das Datenmodell ist hierarchisch aufgebaut, wobei Daten in Dokumenten (documents) und Dokumente in Sammlungen (collections) unter einem einzigartigen Namen (Dokumenten IDs) gespeichert sind. 
Mithilfe von Sammlungen werden die Daten voneinander abgetrennt und hierüber können Abfragen erstellt werden.
Grundlegende Datentypen sind String, Integer und Boolean, jedoch können auch komplexe Datentypen wie Maps, Arrays oder Geopoints abgespeichert werden. Unter-Sammlungen und darin verstaute Dokumente sind ebenfalls möglich.
Das Datenmodell ist in Abbildung \ref{fig:firestore_data_structure}.\\
\\
Abfragen werden auf Dokumentenebene erstellt, damit nicht eine gesamte Sammlung aufgerufen werden muss.
Dies kann über direkte Sortierung, Filter und/oder Limitierung bzw. genaue Auswahl eines Dokumentes bewerkstelligt werden.
Bei einer Abfrage erhält man einen \textit{Data Snapshot}, wodurch über Änderungen in Echtzeit informiert und diese angezeigt werden können.
Damit es jedoch zu keinen fehlerhaften Daten führt, gelten hier atomare Eigenschaften für Transaktionen.\\
Die Sicherheit der Daten stellt Cloud Firestore für Mobil- und Webclient-Bibliotheken über die Firestore-Sicherheitsregeln her. Diese bieten sowohl Zugriffsverwaltung und -authentifizierung, jedoch können auch Daten hiermit für die Konsistenz der Datenbank validiert werden. 
\medskip
\begin{lstlisting}[caption=Beschränkung des Zugriffs auf Dokumente der Sammlung \texttt{cities}, label=lst:firestorerules_basic]
	service cloud.firestore {
		match /databases/{database}/documents {
			match /cities/{city} {
				allow read, write: if request.auth != null;
			}
		}
	}
\end{lstlisting}
\medskip
Im Beispiel \ref{lst:firestorerules_basic} wird der Lese- und Schreibzugriff auf ein Dokument der Sammlung \texttt{cities} beschränkt. 
Nur falls der anfragende Nutzer eine valide Authentifizierung besitzt, erhält er Zugriff auf das angefragte Dokument. 
Diese simple Darstellung ist jedoch für den wirklichen Produktionseinsatz mit Vorsicht zu nutzen. 
Oftmals müssen \texttt{read} und \texttt{write} in detailliertere Vorgänge aufgeteilt werden. Ein \texttt{read} wird spezialisiert in \texttt{get} und \texttt{list}, wobei ein \texttt{write} in \texttt{create}, \texttt{update} und \texttt{delete} unterteilt werden kann.
Ein \texttt{list} ermöglicht es hierbei auf Sammlungen, also die einzelnen Dokumenten IDs, lesend zuzugreifen, jedoch nicht auf die Daten einzelner Dokumente. Hierfür wird dann ein \texttt{get} benötigt. 
Mittels \texttt{create} erhält man Schreibzugriff auf nicht existierende Dokumente, durch \texttt{update} auf bereits vorhandene und Löschrechte ganzer Dokumente erhält man über den \texttt{delete} Operator.\\
\\
Sicherheitsregeln werden gleich dem Datenmodell hierarchisch aufgebaut und ermöglichen differenzierte Zugriffsbeschränkungen auf jeder Ebene.
In Codebeispiel \ref{lst:firestorerules_hierarchy} beinhaltet jedes Dokument (Stadt) der Sammlung \texttt{cities} eine Unter-Sammlung \texttt{landmarks}. Nun lässt sich der Zugriff auf beide separat regeln.
Bei der Sammlung \texttt{villages} hingegen wurde der rekursive Platzhalter verwendet. Hiermit sind Zugriffsregeln auf allen tieferen Ebenen gleich.
Beim Verschachteln von \texttt{match} ist der innere Pfad immer relativ zum äußeren.

Wichtig zu wissen ist hierzu noch, dass falls mehrere \texttt{allow} Ausdrücke auf eine Anfrage zutreffen, wird der Zugriff erlaubt sobald \textbf{eine} Bedingung wahr, also erfüllt ist.\\

\begin{lstlisting}[caption=Hierarchische Zugriffsbeschränkung, label=lst:firestorerules_hierarchy]
	service cloud.firestore {
		match /databases/{database}/documents {
			match /cities/{city} {
				allow read, write: if <condition>;
				
				// Explicitly define rules for the 'landmarks' subcollection
				match /landmarks/{landmark} {
					allow read, write: if <condition>;
				}
			}
			match /villages/{document=**} {
				allow read, write: if <condition>;
			}
		}
	}
\end{lstlisting}

\noindent
Wie bereits oben besprochen können diese Regeln auch zur Validierung von Daten genutzt werden, damit die atomare Eigenschaft von Transaktionen bestehen bleibt.
Hierzu kann die \texttt{getAfter()} Funktion genutzt werden. 
Mit dieser kann man auf den Zustand eines Dokumentes zugreifen und diesen validieren, nachdem einer Folge von Anweisungen ausgeführt, jedoch diese noch nicht auf der Firestore Datenbank abgeschlossen wurde.
Im Beispiel \ref{lst:firestorerules_validation} existieren zwei Sammlungen: \texttt{cities} und \texttt{countries}. 
Jedes \texttt{country} Dokument beinhaltet das Feld \texttt{last\_updated} um zu wissen, welche Stadt innerhalb eines Landes zuletzt aktualisiert wurde.
Hierzu wird in den Sicherheitsregeln nach jedem Schreibzugriff auf ein \texttt{city} Dokument gleichzeitig auch das Feld des zugehörigen Landes aktualisiert \cite{firebase2021}.\\

\begin{lstlisting}[caption=Datenvalidierung für atomare Operationen, label=lst:firestorerules_validation]
	service cloud.firestore {
		match /databases/{database}/documents {
			// If you update a city doc, you must also
			// update the related country's last_updated field.
			match /cities/{city} {
				allow write: if request.auth != null &&
				getAfter(
				/databases/$(database)/documents/countries/$(request.resource.data.country)
				).data.last_updated == request.time;
			}
			
			match /countries/{country} {
				allow write: if request.auth != null;
			}
		}
	}
\end{lstlisting}

\subsubsection{Cloud Storage}
\label{sec:firebase_storage}
Um Filme, Videos oder andere Nutzer-generierte Inhalte abspeichern zu können, bietet Firebase Cloud Storage an. 
Durch das Firebase SDK für Cloud Storage können Dateien direkt von Client-Anwendungen hoch- bzw. heruntergeladen werden.
Aufgrund von möglicher schlechter Verbindung kann mithilfe von robusten Operationen der Prozess des Hoch- bzw. Herunterladens bei besserer Verbindung an der Stelle weiter geladen werden, an welcher dieser unterbrochen wurde.
Ähnlich wie bei Cloud Firestore in Kapitel \ref{sec:firestore} bestimmen auch hier Sicherheitsregeln den Zugriff auf bestimmte Dokumente.\\
Zusätzlich hierzu sind weitere Metadaten verfügbar: \texttt{contentType} und \texttt{size}. 
Mit ihnen lassen sich die Dateien beispielsweise validieren.
Im Code \ref{lst:storagerules_validation} können Dateien nur hochgeladen werden, falls sie eine Größe kleiner 5\,MB (Mega Byte) besitzen.\\

\begin{lstlisting}[caption=Validierung nach Dateigröße, label=lst:storagerules_validation]
	service firebase.storage {
		match /b/{bucket}/o {
			match /images/{imageId} {
				allow read, write: if request.resource.size < 5 * 1024 * 1024
				&& request.auth != null;
			}
		}
		
\end{lstlisting}

Außerdem lassen sich durch Cloud Functions aus dem nächsten Kapitel Prozesse automatisieren. Beispielsweise lässt sich beim Upload eines Bildes direkt ein individuelles Thumbnail erstellen lassen \cite{firebase2021}.
\subsubsection{Cloud Functions}
\label{sec:cloudfunctions}
Da Firebase - bis auf vereinfachte Sicherheitsregeln - eigentlich keinen Backend Code benötigt, jedoch manche Features eben genau diesen brauchen, um beispielsweise Benachrichtigungen an Nutzer zu senden oder Bilder zu komprimieren, existieren Cloud Functions.\\
Diese ermöglichen es, als Antwort auf ein Event automatisch oder durch HTTPS Anfrage manuell Backend Code auszuführen.
Der gesamte Code ist hierbei in der Google Cloud gespeichert und wird in einer verwalteten Umgebung ausgeführt.
Als Programmiersprache kann sowohl JavaScript als auch Typescript verwendet werden.\\
\\
\glqq Google Cloud Functions ist die serverlose Computerlösung von Google zum Erstellen ereignisgesteuerter Anwendungen\grqq \cite{firebase2021}.
Es kann sowohl auf der Google Cloud Platform (GCP), als auch für Firebase genutzt werden. 
Es ist bei beiden ein Verbindungsglied zwischen Logik und entsprechenden Diensten, welche dadurch mit serverseitigen Code erweitert und kombiniert werden.
\begin{figure}[tbt]
	\begin{center}
		\includegraphics[scale=0.2]{Theoretische_Grundlagen/images/firebase_functions_notify.png}
	\end{center}
	\caption{Cloud Functions Anwendungsfall Benachrichtigung}
	\label{fig:functions_notifications}
\end{figure}
In Abbildung \ref{fig:functions_notifications} ist ein typischer Anwendungsfall beschrieben. 
Ein Event auf der Datenbank wird ausgelöst, hier: ein neuer Nutzer folgt einem weiteren Nutzer.
Es wird also ein Dokument in der Unter-Sammlung \texttt{followers} erzeugt. Diese Unter-Sammlung befindet sich innerhalb des Dokumentes \texttt{uid} der Sammlung \texttt{users}.
Im zweiten Schritt erstellt die Funktion eine Nachricht, welche über Firebase Cloud Messaging (FCM) versendet werden soll.
Über abgespeicherte Tokens sendet FCM die Benachrichtigung an das Gerät des Nuters \texttt{uid} \cite{firebase2021}.
\subsubsection{Cloud Messaging}
Firebase Cloud Messaging ist eine plattformübergreifende Messaging-Lösung zum zuverlässigen Versenden von Nachrichten an Nutzergeräte.
In Abbildung \ref{fig:cloudmessaging_architecture} ist die Architektur dieses Tools dargestellt.
Hierbei wird es grundlegend in das Erstellen, Transportieren und Empfangen der Nachrichten unterteilt \cite{firebase2021}.\\

\noindent
\hangindent1cm
\textbf{Erstellen:} Die zu versendenden Nachrichten können, wie in Kapitel \ref{sec:cloudfunctions} beschrieben, manuell oder automatisiert erzeugt werden. 
Bei der Automatisierung ist es wichtig, dass die Nachrichten in einer vertrauenswürdigen Serverumgebung erstellt werden, damit alle Nachrichtentypen unterstützt werden (Schritt 1). 
Das FCM Backend akzeptiert dann in Schritt 2 Nachrichtenanfragen, ordnet die Nachrichten verschiedenen Themen zu und erzeugt unter anderem Metadaten für Nachrichten, wie bspw. die Nachricht ID.

\noindent
\hangindent1cm
\textbf{Transportieren:} Die Nachrichten werden hierbei an die entsprechenden Geräte weitergeleitet.
Da verschiedene Geräte auf unterschiedlichen Plattformen basieren, muss die Transportschicht auf Plattformebene arbeiten.
Hierfür werden folgende Ebenen genutzt:
	\begin{itemize}
		\item Android Transport Layer (ATL) für Android-Geräte mit Google Play-Diensten
		\item Apple Push Notification Service (APNs) für iOS-Geräte
		\item Web-Push-Protokoll für Web-Apps
	\end{itemize}

\noindent
\hangindent1cm
\textbf{Empfangen:} Das FCM SDK behandelt die Benachrichtigung oder Nachricht. Dies ist abhängig vom Vorder-/ Hintergrundstatus der Anwendung und der jeweiligen Anwendungslogik.


\begin{figure}[tbt]
	\begin{center}
		\includegraphics[scale=0.23]{Theoretische_Grundlagen/images/firebase_cloudmessaging_architecture.png}
	\end{center}
	\caption{Firebase Cloud Messaging Architektur}
	\label{fig:cloudmessaging_architecture}
\end{figure}

\subsubsection{Google AdMob}
Google AdMob bietet eine einfache Art, gezielte Werbung innerhalb der Anwendung zu schalten und somit die Anwendung zu monetarisieren.
Zusätzlich bietet das Tool in Kombination mit Google Analytics\footnote{Ein freies Analysetool, welches über alle Tools hinweg Ereignisse sammelt und diese Werte direkt graphisch darstellt. Da es für die Implementierung nicht weiter relevant ist, wird es nicht detaillierter besprochen. Zusätzliche Informationen unter \url{https://firebase.google.com/docs/analytics}, letzter Zugriff: 26. Februar 2021} zusätzliche Anwendungsdaten und Analysefähigkeiten.\\
Werbung lässt sich in unterschiedlicher Weise anzeigen (siehe Abbildung \ref{fig:firebase_admob}) und lässt sich reibungslos in UI Komponenten integrieren. 
Verschiedene Features sind hier jedoch plattformabhängig. 
Auf der Android Plattform ist es für Nutzer möglich, beworbene Produkte direkt aus der Anwendung heraus zu kaufen.\\
Ein weiteres Werbetool \textit{Google Mobile Ads SDK} ist eine alleinstehende SDK.\cite{firebase2021}

\begin{figure}[tbt]
	\begin{center}
		\includegraphics[scale=0.55]{Theoretische_Grundlagen/images/firebase_admob_ads.PNG}
	\end{center}
	\caption{Google AdMob Anzeigemöglichkeiten}
	\label{fig:firebase_admob}
\end{figure}





































\subsection{Anwendungsentwicklung für mobile Endgeräte}
Mobile Geräte sind heutzutage ein sehr großer Teil unseres Tagesablaufs. Durchschnittlich verbringen wir 3:54 Stunden pro Tag an mobilen Geräten (hier bezogen aus Bürger der USA). Die meiste Zeit hiervon wird in Apps (ca. 90\%). \footnote{\url{https://www.emarketer.com/content/us-time-spent-with-mobile-2019}, zuletzt aufgerufen: 26.02.2021} 
Laut Cisco wird dieser Markt sich jedoch nicht nur auf Industrieländer beruhen, sondern bis 2023 sollen weltweit 71\% der Bevölkerung mobile Konnektivität haben. \cite{cisco2020}
Diese Entwicklung forcierte viele Firmen immer mehr ihre Anwendungen auch \textit{mobile ready} zu gestalten. Dies kann man bspw. deutlich bei der Anpassung vieler Webseiten an Mobile Seiten- und Größenverhältnisse oder auch dem Anbieten von \textit{Apps}, welche bereits für Desktop o.ä. verfügbar waren, erkennen. \\

Daher ist es für die Wirtschaft und Entwicklung gleichermaßen wichtig sich ständig weiterzuentwickeln und sich nicht auf (Kosten-) ineffiziente Entwicklungsprozesse auszuruhen. Dabei bieten jährliche, wenn nicht sogar halbjährliche Design- und Performanceänderungen von den Geräten selbst oder der Betriebssysteme Herausforderungen an die mobilen Anwendungen und gleichzeitig an deren Programmierumgebung. Trotz einer riesigen Auswahl an \textit{Apps} lassen sich diese allgemein in drei Kategorien eingliedern: Plattformspezifische Native Anwendungen, Adaptive Webanwendungen und Plattformübergreifende Native Anwendungen.

\subsubsection{Plattformspezifische native Anwendungen}
Plattformspezifische oder auch native Anwendungen sind Programme, welche auf eine gewisse Plattform abzielen und in einer der davon unterstützen Programmiersprachen geschrieben wurden. Da diese Art der (mobilen) Anwendung mit plattformspezifischen \textit{Software Development Kits (SDK)} und \textit{Frameworks} entwickelt wird, ist diese Anwendung an eine Plattform gebunden. \\
Dies bringt zum einen natürlich Vorteile wie allgemein best mögliche Performance auf der jeweiligen Plattform und direkt vom Hersteller unterstützte Entwicklungsumgebungen/SKDs.
Zudem lassen sich plattformspezifische Fähigkeiten oder Einstellungen nutzen - beispielsweise mehrere Kameras oder \textit{Global Positioning System (GPS)}.

Gleichzeitig beschränkt man sich aber logischerweise auf eine Plattform und deckt mit einer Anwendung nur einen Teil des gesamten Marktes. Dies bringt im Vergleich zu den anderen Möglichkeiten einen deutlich erhöhten Entwicklungs- und Wartungsaufwand mit sich, da für andere Plattformen Programmcode nicht übernommen werden kann. Zusätzlich benötigen Entwickler spezifische Kompetenzen für beide Plattform und Entwicklungsumgebungen. \\

Zwei der am weitesten verbreiteten Plattformen sind Android von Google und iOS von Apple. Anwendungen für Android können in Kotlin oder Java als Programmiersprache beispielsweise in dem \textit{integrated development environment (IDE)} von Google Android Studio entwickelt werden. Für iOS wird hingegen mit Objective-C und Swift als Programmiersprache primär in der IDE XCode entwickelt.

Beide bieten jeweils Plattform eigene Services an, beispielsweise das direkte Veröffentlichen in den jeweiligen Appstore \cite{fentaw2020}
\subsubsection{Adaptive Webanwendungen}
test
\subsubsection{Plattformübergreifende Anwendungen}
test


\subsection{Recommender System}
Auf der Webseite Youtube allein werden minütlich mehr als 500 Stunden Videomaterial hochgeladen. (\url{https://blog.youtube/press/}, 10.02.2021)
Um bei einer solch unvorstellbaren Menge an Daten (allein auf einer Webseite) den Überblick als Endnutzer behalten zu können, ist ein personalisiertes Filtersystems unausweichlich.

\noindent
Solche Filtersysteme, auch Recommendation System genannt, nutzen bisher gesammelte Daten um Nutzern potentiell interessante Objekte jeweils individuell vorzuschlagen.
Ein sogenannter \textit{Candidate Generator} ist hierbei ein Recommendation System, welches die Menge $M$ als Eingabe erhält und für jeden Nutzer eine Menge $N$ ausgibt. Hierbei umfasst $M$ alle Objekte und gleichzeitig gilt $N \subset M$. 

\noindent
Die Bestimmung einer solchen Menge $N$ beruht grundlegend auf zwei Informationsarten. Erstens die sogenannten Nutzer-Objekt Interaktionen, also beispielsweise Bewertungen oder auch Verhaltensmuster; Und zweitens die Attributwerte von jeweils Nutzer oder Item, also beispielsweise Vorlieben von Nutzern oder Eigenschaften von Items \cite{aggarwal2016}.
Systeme, welche zum Bewerten ersteres benutzen, werden \textit{collaborative filtering} Modelle genannt. Andere, welche zweiteres verwenden, werden \textit{content-based filtering} Modelle genannt. Wichtig hierbei ist jedoch, dass \textit{content-based filtering} Modelle ebenfalls Nutzer-Objekt Interaktionen (v.a. Bewertungen) verwenden können, allerdings bezieht sich dieses Modell nur auf einzelne Nutzer - \textit{collaborative filtering} basiert auf Verhaltensmustern von allen Nutzern bzw. allen Objekten.

\noindent
Ein solches Recommendation System kann im einfachsten Fall wie in \ref{Recommendation Matrix} als Matrix dargestellt werden.

\begin{table}[tbt]
	\centering
	\caption[Nutzer-Item Matrix mit Bewertungen]{Nutzer-Item Matrix mit Bewertungen. Jede Zelle $r_{u;i}$ steht hierbei für die Bewertung des Nutzers $u$ an der Stelle $i$}
	\label{Recommendation Matrix}
	\begin{tabular}{lcllllll}
		& \multicolumn{7}{c}{Items}                                                                                                                                                        \\
		& \multicolumn{1}{l}{}     & \multicolumn{1}{c}{1}  & \multicolumn{1}{c}{2}  & \multicolumn{1}{c}{...} & \multicolumn{1}{c}{i}  & \multicolumn{1}{c}{...} & \multicolumn{1}{c}{m}  \\ \cline{3-8} 
		& \multicolumn{1}{c|}{1}   & \multicolumn{1}{l|}{2} & \multicolumn{1}{l|}{}  & \multicolumn{1}{l|}{1}  & \multicolumn{1}{l|}{}  & \multicolumn{1}{l|}{}   & \multicolumn{1}{l|}{3} \\ \cline{3-8} 
		Users & \multicolumn{1}{c|}{2}   & \multicolumn{1}{l|}{4} & \multicolumn{1}{l|}{}  & \multicolumn{1}{l|}{}   & \multicolumn{1}{l|}{5} & \multicolumn{1}{l|}{}   & \multicolumn{1}{l|}{}  \\ \cline{3-8} 
		& \multicolumn{1}{c|}{...} & \multicolumn{1}{l|}{}  & \multicolumn{1}{l|}{}  & \multicolumn{1}{l|}{1}  & \multicolumn{1}{l|}{}  & \multicolumn{1}{l|}{}   & \multicolumn{1}{l|}{4} \\ \cline{3-8} 
		& \multicolumn{1}{c|}{u}   & \multicolumn{1}{l|}{}  & \multicolumn{1}{l|}{4} & \multicolumn{1}{l|}{}   & \multicolumn{1}{l|}{5} & \multicolumn{1}{l|}{}   & \multicolumn{1}{l|}{1} \\ \cline{3-8} 
		& \multicolumn{1}{l|}{}    & \multicolumn{1}{l|}{2} & \multicolumn{1}{l|}{}  & \multicolumn{1}{l|}{}   & \multicolumn{1}{l|}{}  & \multicolumn{1}{l|}{3}  & \multicolumn{1}{l|}{}  \\ \cline{3-8} 
		& \multicolumn{1}{l|}{n}   & \multicolumn{1}{l|}{}  & \multicolumn{1}{l|}{4} & \multicolumn{1}{l|}{}   & \multicolumn{1}{l|}{3} & \multicolumn{1}{l|}{}   & \multicolumn{1}{l|}{}  \\ \cline{3-8} 
	\end{tabular}
\end{table}

\subsubsection{Nutzerinformation}
Damit ein \textit{Recommender System} einem Nutzer Vorschläge bereitstellen kann, benötigt es Nutzerinformationen. Das Design des jeweiligen Systems hängt auch, wie oben beschrieben, von der Art der Information und von der Art der Beschaffung dieser ab.

\paragraph{Explizite Nutzerinformation}
Bei der expliziten Methode muss der Nutzer individuelle Informationen aktiv über sich preisgeben. Dies kann über konkrete Fragestellungen zu beispielsweise Geburtsdatum, Geschlecht oder Interessen geschehen. Diese Art der Information beschreiben einen Nutzer konkret. 

\noindent
Eine andere Art der Information sind Bewertungen von Objekten. Diese lassen sich beispielsweise Intervall basiert darstellen. Hierbei werden geordnete Zahlen in einem Intervall als Indikator genutzt, ob ein Objekt gut oder schlecht war - zum Beispiel eine Bewertung eines Produktes von 0 bis 5 Sternen bei Amazon. Diese Information beschreiben die Vorlieben eines Nutzers konkret.

\noindent
Je größer diese Skala ist, desto differenzierter ist auch das Meinungsbild, da jeder Nutzer sich genau ausdrücken kann. Jedoch desto komplizierter und unübersichtlicher wird auch das Bewertungsverfahren an sich, da man einen zu großen Entscheidungsraum für den Nutzer darbietet.

\paragraph{Implizite Nutzerinformation}
Um implizit Nutzerinformationen zu erfassen, muss ein System die Verhaltensmuster seiner Kunden als Daten abspeichern. Beispielsweise könnte das System von YouTube erfassen, ob Videos frühzeitig abgebrochen oder ganz angeschaut werden. Anklicken von Webseiten und die darauf verbrachte Zeit könnte ebenfalls als Bewertung gespeichert und zur Generierung von Vorschlägen genutzt werden.

\subsubsection{Content-based filtering}
Unter \textit{content-based filtering} versteht man das Betrachten von Ähnlichkeiten zwischen Objekten anhand von Schlüsselwörtern (Eigenschaften) und daraus dann das Vorhersagen der Nutzer-Objekt Kombination für ein bestimmtes Objekt. 
Nimmt man an, Film 1 und Film 2 haben ähnliche Eigenschaften (gleiches Genre, gleiche Schauspieler, ...) und Nutzer A mag Film 1, so wird das System Film 2 vorschlagen.

\noindent
Das System ist also unabhängig von anderen Nutzerdaten, da die Vorschläge nur auf Präferenzen eines einzelnen Nutzers basieren. Dies bietet im Hinblick auf eine App auch gute Skalierungs"-möglich"-keiten. Zudem kann auf Nischen-Präferenzen gut eingegangen werden, da nicht mit anderen Nutzerdaten verglichen wird, sondern nur ein Nutzer für sich betrachtet wird.

\noindent
Gleichzeitig schlagen \textit{content-based filtering} Systeme aber eher offensichtliche Objekte vor, da Nutzer oft unzureichend genaue "Beschreibungen", also Vorlieben mit sich bringen. Dadurch, dass nur basierend auf Schlüsselwörter neue Objekte vorgeschlagen und andere Nutzerwertungen nicht miteinbezogen werden, sind die Vorschläge sehr wahrscheinlich oftmals ähnlich bis gleich - man \enquote{verfängt} sich quasi in eine Richtung \cite{aggarwal2016}.

\subsubsection{Collaborative Filtering}
Unter \textit{collaborative filtering} versteht man das Betrachten von Ähnlichkeiten im Verhalten von Nutzern anhand von Bewertungen und Prä"-fer"-enzen, bzw. anhand der Ähn"-lich"-keiten von Objekten.

\noindent
Generell wird bei \cite{aggarwal2016} in zwei Typen unterschieden:

\begin{enumerate}		
	\item \textit{Memory-based Methoden}: Es wird, wie oben beschrieben, aus gesammelten Daten Ähn"-lich"-keit herausgearbeitet und Nutzer-Objekt Kombinationen durch eben diese vorhergesagt. Daher wird dieser Typ auch \textit{neighborhood-based collaborative filtering} genannt. Man unterscheidet weiter in:
	\begin{enumerate}
		\item \textit{User-based}: Ausgehend von einem Nutzer A werden andere Nutzer mit ähnlichen Nutzer-Objekt Kombinationen gesucht, um Vorhersagen für Bewertungen von A zu treffen. Ähnlichkeitsbeziehungen werden also über die Reihen der Bewertungsmatrix berechnet.
		\item \textit{Item-based}: Hierbei werden ähnliche Objekte gesucht und diese genutzt, um die Bewertung eines Nutzers für ein Objekt vorherzusagen. Es werden somit Spalten für die Berechnung der Ähnlichkeitsbeziehungen verwendet.
	\end{enumerate}
	\item \textit{Model-based Methoden}: Machine Learning und Data Mining Methoden werden verwendet um Vorhersagen über Nutzer-Objekt Kombinationen zu treffen. Hierbei sind auch gute Vorhersagen bei niedriger Bewertungsdichte in der Matrix möglich.
\end{enumerate}

\noindent
Vereinfacht gesagt: Wenn Nutzer A ähnliche Bewertungen verteilt wie Nutzer B, und B den Film 1 positiv bewertet hat, wird das System Film 1 auch dem Nutzer A vorschlagen. Dasselbe gilt auch umgekehrt (\textit{Item-based}).

\noindent
Diese Art leidet sehr unter dem \textit{sparsity} Problem. 
Es werden also zu wenige Bewertungen von den Nutzern ausgeübt. Daher sind Vorhersagen über Ähnlichkeit von Nutzern aufgrund unzureichender Datensätze nicht sinnvoll möglich. Dieses Problem wird \textit{Cold-Start Problem} genannt.

\subsubsection{Ähnlichkeit von Objekten und Nutzern}
Sowohl bei \textit{collaborative filtering}, als auch bei \textit{content-based filtering} wird jedes Objekt und jeder Nutzer als ein Vektor im Vektorraum-Modell $E = \mathbb{R}^d$ (englisch \textit{embedding space}) erfasst. Sind Objekte beispielsweise ähnlich, haben sie eine geringe Distanz voneinander. 

\noindent
Ähn"-lich"-keits"-funk"-tion"-en sind Funktionen $s : E \times E  \rightarrow \mathbb{R}$ welche aus zwei Vektoren beispielsweise von einem Objekt $q \in E$ und einem Nutzer $x \in E$ ein Skalar berechnen, welches die Ähnlichkeit dieser zwei beschreibt $s(q,x)$.

\noindent
Hierfür werden mindestens eine der folgenden Funktionen verwendet:\\

\noindent
\hangindent1cm
\textbf{Cosinus-Funktion:}
Hier wird einfach der Winkel zwischen beiden Vektoren berechnet: $s(q,x) = \cos(q,x)$ \\

\noindent
\hangindent1cm
\textbf{Skalarprodukt:}
Je größer das Skalarprodukt, desto ähnlicher sind sich die Vektoren. $s(q,x) = q \circ x = \sum_{i=1}^{d}q_i x_i$ \\

\noindent
\hangindent1cm
\textbf{Euklidischer Abstand:}
$s(q,x) = ||q-x|| = [\sum_{i=1}^{d}(q_i - x_i)^2]^\frac{1}{2}$\\

\noindent
Die Ähnlichkeit der Berechnung mithilfe des Skalarproduktes steigt deutlich mit der Größe der Norm eines Vektors.
Da populäre Objekte, beispielsweise beliebte YouTube Videos, öfter in einem Satz von Trainingsdaten auftauchen, besitzen diese eine höhere Norm.
Somit würden über das Skalarprodukt auch eher populäre Objekte vorgeschlagen werden.
Je nach Anwendungsfall muss also die Ähn"-lich"-keits"-funk"-tion gewählt sein.\footnote{Quelle: \url{https://developers.google.com/machine-learning/recommendation/}, letzter Zugriff: 07. Januar 2021}
\clearpage

\section{Konzept}
Test..

Anforderungen / Konzept?

\begin{figure}[h]
\centering
\includegraphics[width=14cm]{images/Konzept.PNG}
\caption{Konzept}
\end{figure}

-------------------------------
\newline
Thinclient Architecture
https://www.forcepoint.com/de/cyber-edu/thin-client

=> Daten+Funktionalität im Backend
   Visualisierung in Frontend
   
   
  
----------------------------
\newline
Backend Struktur

\begin{figure}[h]
\centering
\includegraphics[width=\textwidth]{images/backendstruktur.PNG}
\caption{Node.JS Installation}
\end{figure}

= Warum Module/Komponenten.. Weil bessere Übersichtlichkeit, Wartung etc..
\clearpage

\section[Funktionen/Komponenten]{Funktionen/Komponenten \hfill \normalfont \small{Autor-Name}}
\clearpage

\section{Auswahl geeigneter Technologie}
Den Anforderungen der Entwicklung entsprechend wurde zunächst entschieden, welche Technologien zum Einsatz kommen. 

\subsection{Anwendungsframework}
Bei der Auswahl des Framework zur Programmierung der eigentlichen Anwendung gibt es, wie in Kapitel \ref{sec:mobile_development} erläutert, eine Vielzahl von möglichen Herangehensweisen.
Die Anwendung ist letztendlich der Teil unseres Systems, welches direkt mit dem Nutzer in Berührung kommt.
Daher sind vor allem die Anforderungen an Performance, Aussehen und Benutzbarkeit der Oberfläche essentiell wichtig.
Gleichzeitig ist bei der Entwicklung auf die verwendete Programmiersprache, die Entwicklungsumgebung, welche das Framework mit sich bringt und die Kenntnisse aller Entwickler zu achten.\\
\\
Wie bereits in Abbildung \ref{fig:crossplattform_popularity} gezeigt, können sich Marktanteile verschiedener Frameworks sehr schnell ändern. 
Dies hängt auch deutlich mit den unterliegenden Plattformen und deren Update-Zyklen zusammen.
Google beispielsweise veröffentlicht beinahe jährlich eine neue Version von Android\footnote{\href{https://engineerbabu.com/blog/evolution-of-android-versions/}{engineerbabu.com}}, weshalb sich eben Programmierumgebungen immer mitentwickeln.
Die Popularität eines Frameworks spiegelt somit unter anderem wieder, wie gut es mit den neuesten Features und Programmierkonzepten umgeht. 
Daher wurde sich bei den plattformübergreifenden Frameworks \ref{sec:framework_introduction} auf React Native und Flutter beschränkt. 
Beide bieten vordefinierte, nativ implementierte Benutzeroberflächenkomponenten - welche im Falle von Flutter sogar auf dem Design System \href{https://material.io/}{\textit{Material Design}} beruht.\\
Das Konzept aus Kapitel \ref{sec:app_concept} schlägt vor, dem Nutzer die Bewertung einzelner Filme über eine Wisch-Funktion bereitzustellen.
Allein dieser Bildschirm muss zunächst einmal eine Reihe von Filmen inklusive Titelbild und Informationen über HTTP Anfragen von unserer Datenbank laden, die Animationen und Logik hinter dem Bewertungssystem abarbeiten und zusätzlich die Bewertung wieder zurück an unseren Server senden.
Aufgrund dieses enorm hohen Performance-Anspruchs sind nativ implementierte Komponente unabdingbar.
Da sowohl Flutter als auch React Native es ermöglichen, die UI Komponenten in der plattformspezifischen Sprache zu implementieren, erweist sich der Leistungsvergleich beider Frameworks als schwierig.
Grundlegend kann jedoch angenommen werden, dass Flutter eher weniger CPU-Nutzung beansprucht, jedoch dazu tendiert, mehr Speicher vor der eigentlichen Anwendung anzufordern\cite{bjorn-hansen2020}

Somit kristallisierten sich drei Entwicklungsmöglichkeiten per Ausschlussverfahren heraus:
\begin{itemize}
	\item Native Anwendung für jeweils Android und iOS
	\item Plattformübergreifend mit Flutter
	\item Plattformübergreifend mit React Native
\end{itemize}

Mit einer nativen Anwendung für jede Plattform erhält man nicht nur doppelten Entwicklungsaufwand, sondern gleichzeitig auch doppelten Wartungsaufwand. 
Zusätzlich gegen die native Möglichkeit spricht, dass für die Entwicklung einer iOS Anwendung die Entwicklungsumgebung XCode benötigt wird, welche eine Desktopanwendung ausschließlich für das Betriebssystem macOS ist.
Daher ist die Entwicklung nicht nur in Sachen Codezeilen ein deutlicher Mehraufwand, welcher sich aber bei der Größe des Projektes nicht bezahlt macht.
Bei den plattformübergreifenden Lösungen entfällt dieser Aspekt, jedoch ist das veröffentlichen einer iOS Anwendung im Vergleich zu Android weiterhin deutlich komplizierter - hierauf wird im weiteren Verlauf der Arbeit noch eingegangen.\\
\\
Beim direkten Vergleich zwischen React Native und Flutter ist der deutlichste Unterschied, dass React Native die Entwicklung einer Laufzeit-basierten Anwendung bietet und Flutter die einer kompilierten Anwendung.
Hierbei werden unterschiedliche Programmiersprachen verwendet: Flutter beruht auf der Google eigenen Sprache Dart. 
Da diese sich bisher noch nicht etablieren konnte, ist diese Sprache unbekannt und muss neu erlernt werden.
JavaScript hingegen ist weit verbreitet, jedoch bringt die spezielle Erweiterung JSX bei React Native ebenfalls zusätzlichen Lernaufwand mit sich.\\
Während der Entwicklung bieten beide Frameworks eine \glqq Hot Reload\grqq Funktion an, welche vor allem bei der Erstellung von Benutzeroberflächen deutliche Zeitersparnisse ermöglicht.
Flutter punktet jedoch vor allem hierbei durch die direkte Unterstützung von UI Bibliotheken, welche React Native ausschließlich über externe Bibliotheken bezieht. 
Gleichzeitig bringt Flutter allgemein mehr \glqq out of the box\grqq mit sich, weshalb bei React Native eher Probleme durch Abhängigkeiten von Drittanbieter entstehen können. \\
\\
Der ausschlaggebende Punkt ist jedoch tatsächlich die BaaS-Plattform (Backend-as-a-Service).
Es existiert zwar ein großer Konkurrent zu Firebase, nämlich \href{https://docs.amplify.aws/}{AWS Amplify}.
Hierbei handelt es sich ebenfalls um ein Backend-Service für Mobil- und Webanwendungen. 
Aufgrund der auf Skalierung ausgesetzten Tools mit besseren Echtzeit-Features ist Firebase jedoch speziell für Nachrichtenaustausch besser geeignet.
Gleichzeitig spielt die Vertrautheit mit dieser BaaS Plattform und die breit gefächerten Tools eine große Rolle bei der Auswahl des Services.
Letztendlich wurde die Entscheidung getroffen, das allgemeines Nutzermanagement und Chat-Funktion der Anwendung mit Firebase zu realisieren.\\
Da AWS Amplify erst seit Beginn 2021 eine offizielle Unterstützung für Flutter anbieten\footnote{\url{https://aws.amazon.com/de/about-aws/whats-new/2021/02/announcing-general-availability-amplify-flutter-data-authentication-support/}}, wurde Flutter als Frontend-Framework und Firebase als BaaS-Plattform ausgewählt.

\subsection{Server}
Der Webserver ist jener Dienst,  der die zugrundeliegenden Funktionalitäten bezüglich der Filmabfragen, der Matching-Logik und der Filmempfehlung bietet. 
Anforderungen an die genutzte Webservertechnologie sind zum einen eine grundlegend hohe Performance, um mit einer hohen Anzahl an Serveranfragen umzugehen. Des Weiteren sollte die Technologie Skalierbarkeit in Bezug auf die Performance und wachsenden Ressourcen wie Hardware aufweisen und eine Unterstützung, Dokumentation und umfassende Funktionalitäten des Frameworks bieten. 
Ein Paketmanager, der umfassende Kern-Funktionalitäten zur Verfügung stellt, sollte vorhanden sein, damit diese nicht neu implementiert werden müssen. Um offen für das genutzte Zielbetriebssystem zu bleiben, sollten mehrere Betriebssysteme unterstützt werden. 
Im folgenden werden keine proprietären Webserver-technologien betrachtet.
\newline
Nach genauerer Recherche kamen drei Webservertechnologien in die engere Betrachtung:

\begin{itemize}
	\item PHP
	\item Django
	\item Node.js
\end{itemize} 

\noindent
Im Hinblick auf die Performance sticht Node.js aufgrund seiner ereignisgesteuerter Architektur  und dem Non-Blocking I/O-Mechanismus heraus und verspricht eine bessere Ressourcennutzung. 
\newline
Große Firmen wie Uber \footnote{siehe \url{https://eng.uber.com/uber-tech-stack-part-two/ }, letzter Zugriff 18.03.2021}, Ebay und Netflix \footnote{siehe \url{https://entwickler.de/online/javascript/7-gruende-node-js-579924149.html}, letzter Zugriff 18.03.2021 } haben ihre Systeme bereits auf Node.js umgestellt. Die Wahl als Webservertechnologie fällt auf Node.js, da es breite Unterstützung erfährt, die auch durch den mächtigen Paketmanager npm ergänzt wird und eine hohe Performance errichtet.


\subsection{Datenbank}
Im Hinblick auf die Speicherung der potenziell hohen Anzahl an Nutzern, deren Swipe-Ent\-schei\-dungen und deren Matches untereinander sowie der Anzahl von über 500.000 Filmen\footnote{siehe \url{https://www.themoviedb.org/faq/general}, letzter Zugriff 18.03.2021} wird ein performanter Umgang der Datenbank mit vielen Datensätzen notwendig sein.  
Um massive Daten speichern zu können sind relationale Datenbanken nicht die passende Wahl. 
Es hat sich gezeigt, dass je größer die Menge an Daten ist und je mehr Tabellen in einer Anfrage enthalten sind, desto größer ist der Performanceverlust durch SQL. \cite{4.5}
\newline
Die Verwendung von dokumentenbasierten Datenbanken führt dagegen zu einer strukturlosen Zusammensetzung an Daten, bei denen ein Dokument ein einzelnes Objekt repräsentieren kann. 
Somit muss für die Wiedergabe eines Objekts nur ein Dokument angefragt werden. Die fehlenden Möglichkeiten zur Normalisierung können jedoch zu Redundanzen in den Daten führen, wodurch die Entwicklung der aufrufenden Anwendung komplexer werden kann. 
Die Redundanz wird jedoch in Kauf genommen, um schnelle Abfragen zu ermöglichen und eine hohe Performance zu erhalten. Da Datensätze in dokumentenbasierten NoSQL-Datenbanken schemalos als JSON-Objekte abgelegt werden, begünstigt dies den generischen Dokumentenaufbau in der Entwicklung.
\newline
Einige NoSQL-Datenbanken wie MongoDB verfolgen einen nicht-relationalen Ansatz und werden mit JavaScript-fähigen Schnittstellen bereitgestellt. Durch die Kommunikation im JSON-Format eignen ist ein optimaler Einsatz mit Node.js gegeben. MongoDB ist über seine horizontale Skalierbarkeit darauf ausgelegt, in einem kurzen Zeitraum sehr viele Daten zu verarbeiten \cite{Tech6}. Während relationale Systeme vertikal im Sinne von neuen Tabelleneinträgen skalieren, werden Dokumente hingegen werden in einer Kollektion, die horizontal erweitert werden können, indem Datenmengen im Sinne des Shardings auf mehreren Systemen verteilt werden, anstatt ein einzelnes System zu verwenden.
\newline
Als Datenbank für die Backend-Implementierung wurde aufgrund von Performance, der guten Einbindung an Node.js und der Skalierbarkeit MongoDB ausgewählt. Um die Vorteile der Da\-ten\-kompression sowie der Transaktionen über mehrere Dokumente hinweg zu nutzen, ist die WiredTiger-Engine die Wahl für das genutzte Storage-Engine.  


\subsection{Kommunikationsschnittstelle}
Als Kommunikationsschnittstelle wird eine WebApi entwickelt, dessen Kommunikation auf HTTP-Nachrichten basiert, deren Informationen im JSON-Format übergeben werden. Diese Technologie bietet eine einfache und dennoch effiziente Form der Kommunikation zwischen Server und Client.

\subsection{Film-Datenbank}
=>TMDb API, https://developers.themoviedb.org/4/getting-started/authorization\newline
OMDb API, http://www.omdbapi.com/\newline
Verschiedene kleinere Anbieter, https://rapidapi.com/search/movie\newline
    vince?
\newline

Github.
\clearpage

\section{Backend-Implementierung}
Unter dem \enquote{Server} versteht man die Komponenten eines digitalen verteilten Systems, welche in einem Netzwerk bestimmte Aufgaben für weitere verbundene Systeme übernimmt.
\newline
Die mobile Anwendung kann direkt mit dem Server interagieren und dessen Benutzerdienste nutzen sowie die Schnittstellen mit allen erforderlichen Ressourcen ansprechen.

\subsection{Datenbank}
\subsubsection{Bereitstellung der Datenbank}
Über die offizielle Seite der MongoDB-Hersteller wird das Community-Installationspaket heruntergeladen und ausgeführt. \footnote{MongoDB Download: \url{ https://www.mongodb.com/try/download/community}}
Ausserdem wird die Kommandozeilenanwendung MongDBShell\footnote{MongDBShell Download: \url{ https://www.mongodb.com/try/download/shell}} und die
graphische Benutzeroberfläche MongoDBCompass installiert\footnote{MongDBCompass  Download: \url{https://www.mongodb.com/try/download/compass}}. 

Nach erfolgreichem Initiieren eines Replica-Sets\footnote{Vergleiche Replica Set: \url{https://docs.mongodb.com/manual/tutorial/deploy-replica-set/}}
kann der mongod-Prozess unter Angabe des zu verwendenden Replica-Sets, des Ports und des Datenspeicherpfads gestartet werden. \newline

%\begin{lstlisting}[caption=Mongod-Aufruf, %label=lst:mongodcall]
%mongod --port 27017 --replSet rs0 --dbpath=”..\data\db0
%\end{lstlisting} %TODO

\noindent
Standardmäßig wird bei Installation von MongoDB eine Konfigurationsdatei erstellt, dessen Name und Verzeichnis anhängig vom benutzten Betriebssystem sind. Auf diese Datei wird bei fehlender Angabe im mongod-ausführendem Kommando zugegriffen\footnote{MongDB Konfiguration: \url{https://docs.mongodb.com/manual/reference/configuration-options/}}.

Sie ermöglicht das Konfigurieren der zu nutzen Storage-Engine. Dementsprechend wird als Storage Engine die WiredTiger-Engine eingestellt. 
Über die graphische Benutzeroberfläche MongoDBCompass wird nach erfolgreicher Verbindung mit der MongoDB-Datenbanksystem eine neue Datenbank hinzugefügt namens 'StreamSwipeDatabase'. Ihr werden die benötigten Collections hinzugefügt:

\begin{itemize}
\item movies
\item users
\item matches
\item swipes
\end{itemize}

\subsubsection{Importieren der Film- und Städtedaten}
Um Nutzer innerhalb einer Stadt miteinander matchen zu können, werden Städtenamendaten auf der Datenbank benötigt. Auf der Webseite "datenb"orse.net" wird eine Liste deutscher Städtenamen zur Verfügung gestellt\footnote{Städteverzeichnis: \url{https://www.datenborse.net/item/Liste\_von\_deutschen\_Staedtenamen\_.csv}}.
Die von uns genutzte Filmdatenbank TMDB bietet eine Liste der datenbankseitig vorhandenen Filme. Sie kann über die Filmdatenbankanbieter-API heruntergeladen werden\footnote{TMDB-API Daily File Exports: \url{https://developers.themoviedb.org/3/getting-started/daily-file-exports}}. 
Über MongoDBCompass werden die heruntergeladenen Städtenamen in die Datenbank als Collection 'cities' und die Filminformationen in die 'movies'-Collection importiert.

\subsection{StreamSwipe-Webserver}
Die Serverzuständigkeiten des StreamSwipe-Webservers lassen sich in folgende Aspekte zusammenfassen:
\begin{itemize}
\item Empfangen und Beantworten der Clientanfragen
\item Routing der Anfragen zu den entsprechenden Abhandlungsroutinen
\item Überprüfen der Identität des Clients
\item Kommunikation zur Datenbank für die persistente Speicherung der Zustände der Nutzer und ihrer Präferenzen, Swipes und Matches.
\item Matching-Algorithmus
\end{itemize} 

\noindent
In den folgenden Unterkapiteln wird auf die Einrichtung des Node.js-Webservers und der MongoDB-Datenbank, sowie auf die Erstellung der sicheren Kommunikationsschnittstelle und auf die Implementierung der serverzuständigen Funktionalitäten eingegangen. Der StreamSwipe-Webserver wird im weiteren Text in der Kurzform als Webserver bezeichnet. 

\subsubsection{Bereitstellung des Webservers}
Zunächst wird das Node.js-Installationspaket aus der offiziellen Seite der Hersteller heruntergeladen und ausgeführt, wie in Abbildung \ref{fig:installation_nodejs} dargestellt. Hierbei werden sowohl die Laufzeitumgebung für Node.JS, als auch der npm package manager installiert (siehe nächste Abbildung). 
Zusätzlich wird bei der Installation ausgewählt, dass Node.js sowie npm und dessen Module zu den Umgebungsvariablen hinzugefügt werden. Dabei werden Variablen unter ihrem Applikationsnamen gespeichert und ihre entsprechenden Datei-Pfade hinterlegt.
Über die Benutzung dieser Umgebungsvariablen ist ein schneller Zugriff über ein Terminal beziehungsweise einer anderen Applikation gewährleistet.


\begin{figure}[tbt]
\centering
\includegraphics[width=8cm]{images/nodejs_install.png}
\caption{Node.JS Installation}
\label{fig:installation_nodejs}
\end{figure}

\noindent
Nach der Installation von Node.js (siehe Abbildung \ref{fig:installation_nodejs}) kann das Projekt mithilfe des Befehls \glqq npm init\grqq \, im Terminal initialisiert werden. 
Hier werden nacheinander Input für relevante Projektaspekte wie dem Projektnamen, der Initialversion, der Startprogrammdatei oder dem GIT-Repository abgefragt.  
Im Anschluss wird im aktuellen Verzeichnis eine Datei \glqq package.json\grqq \, erstellt,  bei der es sich um eine Manifest-Datei im JSON-Format handelt, die unter anderem die benötigten Pakete sowie dessen Version, als auch projektspezifische Meta-Informationen wie den Projektnamen, die Projektversion, die Projektbeschreibung und den Autor enthält.
\newline
Im Anschluss an die Initialisierung werden die benötigten Pakete installiert. Dafür wird der Befehl \glqq npm install\grqq \, in Kombination mit dem angeforderten Modul genutzt. 
Nach der ersten Installation eines Moduls wird im Hauptverzeichnis des Projekts automatisch ein Ordner \glqq node\_modules\grqq \, erzeugt. Dieser enthält die Quelldaten der Node.js-Module. 
\newline
Da die Funktionalität, die nodemon bietet, nur in der Entwicklung benötigt wird, wird in der Datei \glqq package.json\grqq \, ein Entwicklungsskript \glqq devStart\grqq \, definiert. 
Skripte erlauben das automatische Starten von anderen Applikationen. Über \glqq npm run\grqq \, in Kombination mit dem auszuführenden Skript wird die Hauptapplikation über die Datei, die im package.json unter \glqq main \grqq \, hinterlegt ist, zusammen mit den Applikationen, welche im package.json unter dem entsprechenden Skript aufgezählt sind, gestartet.
\newline
Als Applikationsstartpunkt wird die Datei \glqq server.js\grqq \, erzeugt und im package.json unter main hinterlegt.\\

\begin{lstlisting}[caption=Datei package.json, label=lst:packagejson]
{
  "name": "StreamSwipeServer",
  "version": "1.0.0",
  "description": "Our Backend-Server for the StreamSwipe Mobile Application",
  "main": "server.js",
  "scripts": {
    "start": "",
    "devStart": "nodemon"
  },
  "author": "Robin Meckler, Vincent Schreck, Leon Gieringer",
  "license": "-",
  "dependencies": {
    "express": "^4.17.1",
    "firebase-admin": "^9.5.0",
    "mongoose": "^5.11.17",
    "node-cron": "^2.0.3",
    "dotenv": "^8.2.0"
  },
  "devDependencies":{
    "nodemon": "^2.0.7"
  }
}
\end{lstlisting}

\subsubsection{Geplante Architektur}
Die Software für den StreamSwipe-Server wird in Komponenten/Module aufgeteilt. Vorteile dieser Modularisierung sind, dass die einzelnen Module schneller verstanden und dementsprechend leichter überarbeitet werden können. Außerdem können Codeduplizierungen vermieden werden und Softwarekomponenten an verschiedenen Punkten im Code wiederverwendet werden. Abbildung \ref{fig:WebserverArchitektur} stellt die Architektur des Webservers dar. Die Route-Komponenten dienen als Empfangsschnittstelle für HTTPS-Anfragen. Die tatsächlichen Abhandlungsroutinen finden in den Controller-Komponenten statt, welche zum Zugriff auf die Datenbank auf die Service-Komponenten zugreifen und die HTTPS-Antworten zurückschicken.
\begin{figure}[tbt]
\centering

\includegraphics[width=13cm]{images/backendstruktur.PNG}
\caption{Webserver Architektur}
\label{fig:WebserverArchitektur}
\end{figure}

\subsubsection{Sichere Kommunikation}
\label{sec:SichereKommunikation}
Das http-Modul ermöglicht eine Kommunikation über das http-Protokoll.\\
 
\begin{lstlisting}[caption=Einfache Verbindung, label=lst:nodejs_easyconnection]
{
 const app = express();
 app.use(express.json()); 
 var httpServer = http.createServer(app);
 httpServer.listen(process.env.HTTP_PORT, () => 
 console.log("HTTP-Server started on " + process.env.HTTP_PORT));
}
\end{lstlisting}

\noindent
Dabei werden jedoch die Daten unverschlüsselt versendet. Um ausreichend Datenschutz zu gewährleisten, wird stattdessen das https-modul genutzt\footnote{Siehe Dokumentation: \url{https://nodejs.org/api/https.html}, letzter Zugriff: 24. April 2021}.
\newline
Benötigt für einen HTTPS-Server werden ein Sicherheitszertifikat und ein privater Schlüssel, die zunächst mithilfe des Tools OpenSSL erzeugt werden.  
Dabei ist zu beachten, dass während der Entwicklungsphase das Zertifikat nicht von einer zuständigen Zertfikatsstelle signiert wurde und somit von anderen Gegenstellen nicht akzeptiert wird.
\newline
In der Anwendung wird zunächst ein Objekt 'httpsOptions' erzeugt, das unter dem Attribut 'cert' das generierte Sicherheitszertifikat und unter dem Attribut 'key' den privaten Schlüssel enthält. Anschließend wird über die Funktion 'createServer' des https-Objekts der https-Server gestartet, woraufhin ein Objekt vom Typ https.Server zurückgegeben wird\footnote{Siehe Dokumentation:  \url{[https://nodejs.org/api/https.html\#https_class_https_server]}, letzter Zugriff: 24. April 2021}. 
Diesem Serverobjekt wird über seine Methode 'listen' aufgefordet, auf eingehende Nachrichten in dem als Parameter übergebenen Port einzugehen.\\

\begin{lstlisting}[caption=Gesicherte Verbindung, label=lst:nodejs_safeconnection]
{
 ...
 const https = require("https");
 const httpsOptions = {
  cert: fs.readFileSync('sslcert/server.crt', 'utf8'),
  key: fs.readFileSync('sslcert/server.key', 'utf8')
 }
 var httpsServer = https.createServer(httpsOptions, app);
 httpsServer.listen(process.env.HTTPS_PORT, () => {console.log("HTTPS - 	Server started on " + process.env.HTTPS_PORT)});
}
\end{lstlisting}


\subsubsection{Datenbankverbindung}
Wie bereits erwähnt, wird das Modul „mongoose“ für die Verbindung mit der MongoDB-Datenbank verwendet, wie im Beispiel \ref{lst:mongodbconnection} dargestellt.
Da der Quellcode in anderen Dateien hinterlegt ist, muss für den Zugriff auf dessen Funktionalitäten das entsprechende Modul zunächst inkludiert werden. Dazu wird die require-Methode aufgerufen, die ein Objekt zurückgibt, dass die aus dem Modul exportierten Methoden enthält und im Folgenden als Variabel mit dem Namen „mongoose“ gespeichert wird. 
\newline
Über die connect-Methode des zurückgelieferten Objekts wird nun bei Parameterübergabe der Datenbank-URL versucht, eine Verbindung aufzubauen.  
Dabei wird unter der Objekt-Membervariable  „connection“ ein Objekt vom Typ „Connection“ hinterlegt, über das bei erfolgreicher Verbindung mit der Datenbank kommuniziert werden kann und das nachfolgend unter der Variable „database“ abgespeichert ist.\\

\begin{lstlisting}[caption=Verbindung zur MongoDB-Datenbank, label=lst:mongodbconnection]
{
 const mongoose = require('mongoose');
 let database = null;

 async function startDatabase() {
  await mongoose.connect(process.env.DATABASE_URL, 
   {useNewUrlParser: true,
   useUnifiedTopology: true}); 
  database = mongoose.connection;
  database.on('error',(error) => console.log(error));
  	database.on('open',(error) => console.log('Connected to DB'))}

 async function getDatabase() {
 if (!mongoose.connection) await startDatabase();
  return database; }

 module.exports = {
  getDatabase,
  startDatabase }
}
\end{lstlisting}

%\subsection{Datenbank}
\subsubsection{Datenbankmodelle und Schemata}
Ein Model in Mongoose ist ein aus einer Schemadefinition erstellter Konstruktor, aus denen Objekte instanziiert werden können. Diese Instanzen stehen in direkter Verbindung zu den jeweiligen Collections der verbundenen Datenbank und enthalten Methoden für die persistente Speicherung, Bearbeitung oder Löschung.

%TODO Konzeptioneller Aufbau
\begin{figure}[tbt]
\centering
\includegraphics[width=\textwidth]{images/databasemodells.PNG}
\caption{MongoDB - Aufbau der Collections und Beziehungen}
\label{fig:databasemodells}
\end{figure}

\noindent
Folgender Code zeigt den Aufbau des Schemas für die Swipe-Collection. \\

\begin{lstlisting}[caption=Swipe Schema und Model, label=lst:modelswipe]
const mongoose = require('mongoose')

const swipeSchema = new mongoose.Schema({
 uid: {
  type: String,
  required: true
 },
 swipes :
 [{ movieid: { type: String },
    swipeaction: {type: Number}}]
 })

module.exports = mongoose.model('Swipe',swipeSchema)
\end{lstlisting}

\noindent
Die einzelnen Schemata wurden nach dem im Diagramm der Abbildung \ref{fig:databasemodells} dargestellten Aufbau für jede Collection in separaten Dateien unter dem Verzeichnis '/database/models' erstellt (siehe Abbildung \ref{fig:node_structure}). Jede Datei exportiert dabei das aus dem zugehörigen Schema erzeugten Model.

\begin{figure}[tbt]
\centering
\includegraphics[width=12cm]{images/modelsstruktur.PNG}
\caption{Node.js Server - Models Struktur}
\label{fig:node_structure}
\end{figure}


\subsubsection{Datenbankzugriff}                   
Für den Datenzugriff auf die Datenbank wurden zu jeder Collection Service-Module unter dem Verzeichnis '/services' erstellt, die entsprechenden Zugriff gewähren. Dafür wurden innerhalb der Service-Module die benötigten Zugriffsfunktionen implementiert (siehe Abbildung \ref{fig:node_service_structure}).

\begin{figure}[tbt]
\centering
\includegraphics[width=12cm]{images/serviceStruktur.PNG}
\caption{Node.js Server - Services Struktur}
\label{fig:node_service_structure}
\end{figure}


%
%              Movie
%                   Service
%

\paragraph{Movie Service}
Die Funktionen des Moduls movieService.js bieten innerhalb der Projektumgebung den Zugriff auf bestimmte Operationen, die auf die Collection movies angewandt werden. Das Funktionsspektrum begrenzt sich für diesen Service auf die Funktion 'FindMovieExcept', die in Listing \ref{lst:findmoviesexcept} zu sehen ist.\\

\noindent
\textbf{FindMoviesExcept:}
Diese Funktion erhält als Parameter 'excludedMovies' eine Liste von MovieID's und als Parameter 'amount' einen Integerwert.
Über die Find-Funktion des importierten Movie-Models wird eine über den Wert von 'amount' begrenzte Anzahl an Movie-Dokumenten ausgelesen, deren IDs nicht in der übergebenen 'excludedMovies'-Liste vorhanden sind. Für das Filtern wird die 'nin'-Operation verwendet\footnote{Siehe Dokumentation: \url{https://docs.mongodb.com/manual/reference/operator/query/nin/}, letzter Zugriff: 26. April 2021}.\\
Sollte die Datenbank bei der Datenauslese einen Fehler zurückgeben, wird dieser über den try-catch-Block gefangen und an im Aufrufstack liegende Funktion über das Schlüsselwort throw weitergeleitet.\\

\begin{lstlisting}[caption=movieService.js - FindMoviesExcept, label=lst:findmoviesexcept]
const Movie = require('../database/models/movie')

async function FindMoviesExcept(excludedMovies, amount) {
    var movies;

    try{ 
    movies = await Movie.find({ id: { $nin: excludedMovies }}).limit(amount); }
    }
    catch(err){throw err;}

    return movies;
}

module.exports.FindMoviesExcept = FindMoviesExcept;
\end{lstlisting}


%
%              User
%                   Service
%

\paragraph{User Service}
Die Funktionen des Moduls userService.js bieten innerhalb der Projektumgebung den Zugriff auf bestimmte Operationen, die auf die Collection User angewandt werden. Dieser Service import das User-Model.\\

\noindent
\textbf{CreateUser:}
Die in Listing \ref{lst:userservicecreateuser} dargestellte Funktion erhält sämtliche Eigenschaften, die im User-Schema beschrieben sind, als Parameter. Diese Funktion wird in der Projektumgebung in Zusammenhang mit mindestens einer weiteren datenbankzugreifenden Funktion aufgerufen. Im Sinne einer Transaktion müssen sie als atomare Operationen ausgeführt werden, um den Datenbestand konsistent zu halten. Daher wird ein Session-Objekt als Parameter mitgeliefert. Innerhalb der Funktion wird über die Create-Funktion des importierten User-Models ein neuer Eintrag in der User-Collection der Datenbank erstellt.\\

\begin{lstlisting}[caption=User Service - CreateUser, label=lst:userservicecreateuser]
async function CreateUser(uid, swipeid, matchid, city, malewanted, femalewanted, diversewanted, mygender, session) {
    try {
        return (await User.create([{
            _id: mongoose.Types.ObjectId(),
            uid: uid,
            _swipeid: swipeid,
            _matchid: matchid,

            city: city,
            malewanted: malewanted,
            femalewanted: femalewanted,
            diversewanted: diversewanted,
            mygender: mygender
        }
        ], { session: session }))[0];
    }
    catch (Exception) {
        throw Exception;
    }
}
\end{lstlisting}

\noindent
\textbf{CheckExistence:}
Wie in Listing \ref{lst:userservicecheckexistence} zu sehen prüft diese Funktion, ob innerhalb der User-Collection ein User mit entsprechendem Wert für die Eigenschaft 'uid', die als Parameter übergeben wird, existiert und gibt entsprechend den boolschen Wert 'true' bei Vorhandensein beziehungsweise 'false' bei Nicht-Vorhandensein zurück.\\

\begin{lstlisting}[caption=User Service - CheckExistence, label=lst:userservicecheckexistence]
async function CheckExistence(uid) {
    return await User.exists({ uid: uid });
}
\end{lstlisting}

\noindent
\textbf{GetCityFromUser:}
Diese Funktion gibt den Eintrag der Eigenschaft 'city' eines Dokuments zurück, dessen 'uid'-Attribut mit dem übergebenen 'uid'-Parameter übereinstimmt, wie in Listing \ref{lst:userservicecheckexistence} zu sehen ist.\\

\begin{lstlisting}[caption=User Service - CheckExistence, label=lst:userservicecheckexistence]
async function GetCityFromUser(uid) {
    try {
        var user = await User.findOne({ 'uid': uid });
        if (user) {
            return user.city;
        }
        else throw { message: "No user Found" + uid };
    }
    catch (err) { console.log(err); throw err; }
\end{lstlisting}

\noindent
\textbf{ChangeCityFromUser:}
Die in Listing \ref{lst:userservicechangecityfromuser} dargestellte Funktion führt ein Update auf einem User-Dokument aus, dessen 'uid'-Eigenschaft mit dem gleichnamigen übergebenem Parameter übereinstimmt. Dabei wird die 'city'-Eigenschaft innerhalb des User-Dokuments auf den Wert des gleichnamigen Parameters aktualisiert.\\

\begin{lstlisting}[caption=User Service - ChangeCityFromUser, label=lst:userservicechangecityfromuser]
async function ChangeCityFromUser(uid, city, session) {  
   try {
        if (CheckExistence(uid)) {
            await user.UpdateOne(
                { 'uid': uid },
                { city: city },
                { session: session });
        }
        else throw { message: "No User Found" + uid }
    }
    catch (err) { console.log(err); throw err; }
\end{lstlisting}

\noindent
\textbf{ChangeGenderWantedFromUser:}
Diese Funktion erfüllt die gleiche Funktionalität wie die ChangeCityFromUser-Funktion  mit dem Unterschied, dass statt der 'city'-Eigenschaft die Eigenschaften 'malewanted','femalewanted' und 'diverswanted' aktualisiert werden.\\

\noindent
\textbf{ChangeGenderFromUser:}
Die Funktion ChangeGenderFromUser erfüllt die gleiche Funktionalität wie die ChangeCityFromUser-Funktionalität mit dem Unterschied, dass statt der 'city'-Eigenschaft die 'mygender'-Eigenschaft aktualisiert wird.\\


%
%              Match
%                   Service
%


\paragraph{Match Service}
Die Funktionen des Moduls matchService.js bieten innerhalb der Projektumgebung den Zugriff auf bestimmte Operationen, die auf die Collection matches angewandt werden. Dieser Service import das Match-Model.

\noindent
\textbf{CreateUserMatchDocument:}
Die in Listing \ref{lst:matchserviceCreateUserMatchDocument} dargestellte Funktion überprüft zunächst, ob ein Match-Dokument mit übergebener 'uid'-Eigenschaft bereits existiert. Ist dies nicht der Fall, wird ein neues Match-Dokument in der matches-Collection erzeugt.\\

\begin{lstlisting}[caption=Match Service - CreateUserMatchDocument, label=lst:matchserviceCreateUserMatchDocument]
async function CreateUserMatchDocument(uid, session) {
    if (!(await Match.exists({ uid: uid }))) {
        try {
            return (await Match.create([{
                _id: mongoose.Types.ObjectId(),
                uid: uid,
                swipes: []}
            ], { session: session }))[0];
        }
        catch (Exception) {
            throw Exception;
        }
    }
    else 
      throw { message: "Match already exists for " + uid 		};
    
}
\end{lstlisting}

\noindent
\textbf{CheckMatchExists:}
Wie in Listing \ref{lst:matchserviceCreateUserMatchDocument} zu sehen, überprüft diese  Funktion ob ein Match-Dokument existiert, welches den übergebenen 'uid'-Eigenschaftswert hat sowie innerhalb seiner 'supermatches'- oder 'normalmatches'-Liste die übergebene 'matchedUid' enthält. Zurück wird ein entsprechender boolescher Wert geschickt. \\

\begin{lstlisting}[caption=Match Service - CreateUserMatchDocument, label=lst:matchserviceCreateUserMatchDocument]
async function CheckMatchExists(uid, matchedUid, session) {
    try {
        var match = await Match.findOne({ uid: uid, 'supermatches.uid': matchedUid }).session(session)

        if (match) { return true; }
        else {
            var match = await Match.findOne({ uid: uid, 'normalmatches.uid': matchedUid }).session(session)

            if (match) return true;
            else return false;
        }
    }
    catch (err) { console.log(err); throw err; }
}
\end{lstlisting}

\noindent
\textbf{AddNormalMatchToUser:}
Die Funktion AddNormalMatchToUser, welche in Listing \ref{lst:matchserviceAddNormalMatchToUser} zu sehen ist,  fügt der 'normalmatches'-Liste eines Match-Dokuments ein neues Objekt mit den Eigenschaften 'uid','matchUid' und 'movieid', die ihren Wert über die übergebenen Funktionsparameter erhalten. Des Weiteren wird das Attribut 'startedChat' und 'removed' jeweils mit dem booleschen Standardwert 'false' hinzugefügt. Außerdem wird die Eigenschaft 'newChanges' des Match-Dokuments auf true gesetzt.\\

\begin{lstlisting}[caption=Match Service - AddNormalMatchToUser, label=lst:matchserviceAddNormalMatchToUser]
async function AddNormalMatchToUser(uid, matchedUid, movieid, session) {
 try {
  if (Match.exists{ 'uid': uid }) {
   await Match.findOneAndUpdate(
    { 'uid': uid },
    { newChanges: true,
    $push: { normalmatches: { uid: matchedUid, 
                              movieid: movieid, 
                              startedChat: false, 
                              removed: false } }},
    { session: session });
    } else throw { message: "No Match Found" }; }
    catch (err) { console.log(err); throw err; }
}
\end{lstlisting}

\noindent
\textbf{AddSuperMatchToUser:}
Die Funktion unterscheidet sich von 'AddNormalMatchToUser' nur in dem Aspekt, dass ein neuer Eintrag in die 'supermatches'- statt der 'normalmatches'-Liste hinzugefügt wird.\\

\noindent
\textbf{GetMatches:}
Die Funktion empfängt eine 'uid' als Parameter und gibt ein entsprechendes Match-Dokument zurück, sofern es existiert.\\

\noindent
\textbf{SuperMatchMarkAsRemoved:}
Innerhalb der Funktion wird, wie in Listing \ref{lst:matchserviceSuperMatchMarkAsRemoved} dargestellt, anhand des übergebenen 'uid' und 'matchesUid' der Eintrag in der 'supermatches'-Liste angepasst. Dabei wird der Wert für 'removed' auf true gesetzt. Ausserdem wird 'newChanges' des betroffenen Match-Dokuments auf true gesetzt.\\

\begin{lstlisting}[caption=Match Service - SuperMatchMarkAsRemoved, label=lst:matchserviceSuperMatchMarkAsRemoved]
async function SuperMatchMarkAsRemoved(uid, matchUid, session) {
 try {
  var match = await Match.findOneAndUpdate({ uid: uid, "supermatches.uid": matchUid },
  { "$set": { newChanges: true,
             "supermatches.$.removed": true }},
            { session: session });
  if (!match) {
   throw { message: "No Match Found:" + 
           uid + "matching: " + matchUid };}
 } catch (err) { console.log(err); throw err; }
}
\end{lstlisting}

\noindent
\textbf{NormalMatchMarkAsRemoved:}
Die Funktion unterscheidet sich von 'SuperMatchMarkAsRemoved' nur in dem Aspekt, dass der  Eintrag in der 'supermatches'- statt der 'normalmatches'-Liste angepasst wird.\\

\noindent
\textbf{MatchesReceived:}
Anhand einer übergebenen 'uid' und 'matchUid' wird die Eigenschaft 'new"-Changes' eines entsprechenden Match-Dokuments in der Match-Collection auf den booleschen Wert false gesetzt. 



%
%              Swipe
%                   Service
%

\paragraph{Swipe Service}
Die Funktionen des Moduls swipeService.js bieten innerhalb der Projektumgebung den Zugriff auf bestimmte Operationen, die auf die Collection swipes angewandt werden. Dieser Service importiert das Swipe-Model.\\

\noindent
\textbf{CreateUserSwipeDocument:}
Die Funktion erfüllt die gleiche Funktionalität wie die 'Create\-UserMatchDocument'-Funktionalität des Match-Services mit dem Unterschied, dass anstelle eines Match-Dokuments ein Swipe-Dokument in der swipes-Collection erstellt wird.\\

\noindent
\textbf{AddSwipe:}
Die in Listing \ref{lst:swipeserviceaddswipe} dargestellte Funktion erstellt zunächst ein 'swipe'-Objekt mit den Eigenschaften 'movieid' und 'swipeaction' und entnimmt die Werte dafür aus den gleichnamigen übergebenen Parametern. 
Wenn ein Swipe-Dokument mit übergebener 'uid' existiert, wird überprüft ob das Dokument in der 'swipes'-Liste bereits ein Eintrag mit entsprechender movieid enthält. Ist dies der Fall, wird die 'swipeaction'-Eigenschaft auf den Wert des übergebenen gleichnamigen Parameters aktualisiert. Ansonsten wird der Liste das zu Beginn erstellte 'swipe'-Objekt hinzugefügt. Letzlich wird das Swipe-Dokument über die Save-Funktion des Save-Models gespeichert.\\

\begin{lstlisting}[caption=Swipe Service - AddSwipe, label=lst:swipeserviceaddswipe]
async function AddSwipe(uid, movieid, swipeaction) {
 var swipe = { movieid: movieid, swipeaction: swipeaction };
 var dbSwipe;
 if (Swipe.exists({ 'uid': uid })) {
  try {
   dbSwipe = await Swipe.findOne({ 'uid': uid });

   //Ueberpruefe Vorhandensein des Swipes
   var index = dbSwipe.swipes.findIndex(x => x.movieid === movieid);

   if (index >= 0) {
    if (dbSwipe.swipes[index].swipeaction != swipeaction)
      dbSwipe.swipes[index].swipeaction = swipeaction; }
    else { dbSwipe.swipes.push(swipe); }
    dbSwipe.save(); 
    } catch (err) { throw err; }
  } else { throw "No Swipe available for this uid " + uid;}
  dbSwipe.save();
  return swipe;
}
\end{lstlisting}

\noindent
\textbf{RequestSwipes:}
Die Funktion empfängt eine 'uid' als Parameter und gibt ein entsprechendes Match-Dokument zurück, sofern es existiert.\\

\noindent
\textbf{RequestSuperlikeSwipes:}
Innerhalb dieser in Listing \ref{lst:swipeserviceRequestSuperlikeSwipes} dargestellten Funktion kommt es zum Einsatz einer Aggregation. 
Dabei kommt es zum Einsatz mehrerer Pipeline-Operatoren.
% %TODO Link auf Tabelle 5 Aggregation Framework
Über den 'unwind'-Operator wird gesetzt, dass für jeden Listeneintrag innerhalb der 'swipes'-Liste neue Dokumente erzeugt werden, die in die nachfolgende Pipeline-Stufen weitergeleitet werden. Anhand des 'match'-Operators werden dann die neuen Dokumente gefiltert. Letztlich wird eine Liste von Swipes zurückgeschickt, die eine 'swipeaction' von 2 (repräsentativ für Superlike) aufweisen.\\

\begin{lstlisting}[caption=Swipe Service - RequestSuperlikeSwipes, label=lst:swipeserviceRequestSuperlikeSwipes]
async function RequestSuperlikeSwipes(uid) {
 var dbSwipe;
 if (Swipe.exists({ 'uid': uid })) {
  try {
   dbSwipe = await (await Swipe.aggregate([
    { $unwind: '$swipes' },
    { $match: { uid: uid,
                'swipes.swipeaction': 2 }},
    { $group: { _id: '$_id',
                swipes: { $push: { movieid: "$swipes.movieid",
                                  swipeaction: "$swipes.swipeaction" 
     }}}}]));
   if (dbSwipe.length > 0 && dbSwipe[0]) {
    return dbSwipe[0].swipes; }
   } catch (err) { throw err; } }
  else {throw { message: "Swipe uid not existing " + uid }; }
}
\end{lstlisting}

\noindent
\textbf{FindAllSwipedMoviesByUserID:}
Innerhalb dieser in Listing \ref{lst:swipeserviceFindAllSwipedMoviesByUserID} dargestellten Funktion werden für eine übergebene 'uid' sämtlich 'movieid's zurückgeben, die in der 'swipes'-Liste des entsprechenden Swipe-Dokuments vorhanden sind.\\

\begin{lstlisting}[caption=Swipe Service - FindAllSwipedMoviesByUserID, label=lst:swipeserviceFindAllSwipedMoviesByUserID]
var swipedMovieIDs = [];
    
if (Swipe.exists({"uid": uid)) {
 try { var dbSwipe = await FindOne(uid);
  await dbSwipe.swipes.forEach(x => swipedMovieIDs.push(x.movieid));
 } catch (err) { throw err; }}
return swipedMovieIDs;
\end{lstlisting}


%
%              City
%                   Service
%
\paragraph{City Service}
Die Funktionen des Moduls cityService.js bieten innerhalb der Projektumgebung den Zugriff auf bestimmte Operationen, die auf die Collection cities angewandt werden. Dieser Service import das Swipe-Model.\\

\noindent
\textbf{GetAllInhabitedCities:}
Diese Funktion wird im Matching-Algorithmus aufgerufen. Wie in Listing \ref{lst:cityServiceGetAllInhabitedCities} zu sehen liefert er sämtliche Städte, mit mindestens zwei Nutzern. \\

\begin{lstlisting}[caption=City Service - GetAllInhabitedCities, label=lst:cityServiceGetAllInhabitedCities]
async function GetAllInhabitedCities() {
    var cities;
    try{ cities = await City.find({ user : {$exists:true},$where:'this.user.length>1'}) }
    catch (err) { throw err; }
    return cities;
}
\end{lstlisting}

\noindent
\textbf{AddUserToCity:}
Diese Funktion fügt einem 'City'-Dokument eine 'uid' hinzu.\\

\noindent
\textbf{RemoveUserFromCity:}
Diese Funktion entfernt eine 'uid' aus einem 'City'-Dokument.


\subsubsection{Controller} 
Die Controller enthalten die Abhandlungsroutinen für die HTTPS-Anfragen (siehe Abbildung \ref{ControllerStruktur}). Die einzelnen Funktionen empfangen jeweils das Request- und das Response-Objekt der Anfrage. Das Füllen und Zurückschicken des Response-Objekts ist ebenfalls Aufgabe der Controller. Zum Zugriff auf die Datenbank greifen sie auf die Services zu.\\



\begin{figure}[tbt]
\centering
\includegraphics[width=12cm]{images/controllerStruktur.PNG}
\caption{Node.js Server - Controller Struktur}
\label{ControllerStruktur}
\end{figure}

\paragraph{Movie Controller}

Der Movie-Controller  aus Listing \ref{lst:movieController_js} nutzt den Movie-Service zum Zugriff auf die Datenbank. \\

\begin{lstlisting}[caption=movieController.js Imports und Funktionen, label=lst:movieController_js]
const SwipeService = require('../services/swipeService')
const MovieService = require('../services/movieService')
const FirebaseService = require('../services/firebaseService')

exports.RequestMovies = async function(req, res){
   ...
}
\end{lstlisting}

\noindent
Er enthält die Funktion 'RequestMovies', welche in engem Kontakt zum SwipeManager des Frontends steht und es Nutzern ermöglichen sollen, neue Filminformationen, die vom Nutzer noch nicht empfangen wurden, abzufragen.
\newline
Wie in Listing \ref{lst:controllerfirebaseauth} zu erkennen ist, erwartet die Funktion im 'body'-Objekt des als Parameter übergebenen Request-Objekts eine Eigenschaft 'uidtoken', dessen Wert eine aus Firebase generierte Token-Referenz zur eindeutigen Authentifizierung des Nutzers ist. Der Token wird an die 'GetUID'-Funktion des Firebase-Services weitergeleitet, die bei erfolgreicher Authentifizierung die entsprechende 'uid' zurückschickt. Bei einem aufgetretenen Fehler, wie beispielsweise einem ungültigen Token, wird das Response-Objekt mit dem Statuscode 401 sowie der aufgetretenen Fehlernachricht zurückgeschickt und die Funktion beendet. Der folgende Code kommt in weiteren Funktionen anderer Controller ebenfalls zum Einsatz, wenn eine Authentifizierung benötigen.

\begin{lstlisting}[caption=Controller Firebase-Authentifizierung, label=lst:controllerfirebaseauth]
    var uid; 
    const uidToken = req.body.uidtoken;
    try{ uid = await FirebaseService.GetUID(uidToken); }
    catch(Exception)
    { res.status(401).json({title: "TOKEN ERROR", message: Exception}); return; }
\end{lstlisting}

\noindent
Nach erfolgreicher Authentifizerung des Firebase-Tokens werden weitere Eigenschaftswerte aus dem Request-Body als Variabeln gespeichert. Erwartet wird ein Zahlenwert 'amount', der die Anzahl der abgefragten Filme darstellt. Über die lokale Funktion 'RestrictAmount' wird geprüft, dass die begrenzende Zahl den Wert 10 nicht übersteigt. Damit soll sichergestellt werden, dass die Datenbankabfrage mit den weit über 500.000 Filmen nicht ausgelastet wird. Des Weiteren werden in der Variable 'alreadyRequestedMovieIDs' eine Liste von   eindeutigen Identifizierern aus der Movie-Collection erwartet. Die Werte sollen jene Film-ID's wiederspiegeln, die bereits vom Nutzer abgefragt, aber noch nicht über einen Swipe-Request in der Datenbank hinterlegt wurden. Die vom Nutzer bereits getätigten Swipes werden über die 'FindAllSwipesByUserID'-Funktion des SwipeServices abgefragt. Zurück\-gegeben wird eine Liste von Film-ID's, die zusammen mit der Liste der 'alreadyRequestedMovieIDs' in die Variabel 'excludedMovieIDs' gespeichert werden.\\

\begin{lstlisting}[caption=MovieController - RequestMovie - Excluded Movies, label=lst:MovieControllerExcludedMovies]
    var amount = RestrictAmount(req.body.amount);
    var alreadyRequestedMovieIDs = req.body.alreadyRequestedMovieIDs;
    var excludedMovieIDs = [];
    var newMovies;
    
    // Frage bereits geswipete Filme ab
    try{ var swipedMovieIDs = await SwipeService.FindAllSwipesByUserID(uid) }
    catch(err){ res.status(400).json({message: err.message}); return; }
        excludedMovieIDs.push(...swipedMovieIDs); }

    // Speichere bereits abgefragte Filme ab
    if(alreadyRequestedMovieIDs !== undefined && alreadyRequestedMovieIDs != null)
    { await alreadyRequestedMovieIDs.forEach(element => excludedMovieIDs.push(element)); }
\end{lstlisting}

\noindent
Letzlich wird bei vorhandenen zu exkludierenden Filmen die Funktion 'FindMoviesExcept' des Movie-Services aus Listing \ref{lst:MovieControllerExcludedMovies}  aufgerufen.  Ist die Liste 'excludedMovieIDs' dagegen leer, so wird die 'Find\-ExactAmount'-Funktion aufgerufen. Bei Erfolg wird das Response-Objekt mit dem Statuscode 200 und den Movie-Dokumenten als JSON-Objekt im Body der Antwort zurückgeschickt.\\

\begin{lstlisting}[caption=MovieController - RequestMovie - Excluded Movies, label=lst:MovieControllerExcludedMovies]
    if(excludedMovieIDs.length > 0)
    {
        try{ newMovies = await MovieService.FindMoviesExcept(excludedMovieIDs,amount) } 
        catch(err){ res.status(400).json({message: err.message}); return; }
    } else {
        try{ newMovies = await MovieService.FindExactAmount(amount); } 
        catch(err){ res.status(400).json({message: err.message}); return; }
    }

    res.status(200).json(newMovies);
\end{lstlisting}




%
%			User	
%				Controller
%



\paragraph{User Controller}
Der User-Controller bietet Funktionen, die die users-Collection der Datenbank betreffen. Sie greift dafür auf das User-Service zu. Die folgenden Funktionen greifen teils auch auf andere Collections zu. Daher werden auch die entsprechenden weiteren Services importiert.\\

\noindent
\textbf{CreateUser:}
Die Funktion erstellt ein neues User-Dokument in der users-Collection. Dafür werden gleichnamige Eigenschaften des User-Models im Request-Body der eingehenden Anfrage erwartet. Nach erfolgreicher Authentifizierung des Firebase-Tokens und Überprüfung über die 'CheckExistence'-Funktion des User-Services, ob ein User-Dokument mit der gleichen 'uid' bereits existiert, wird für die weiteren Datenbankabfragen eine Transaktion gestartet.\\

\begin{lstlisting}[caption=UserController - Create User - Transaktionsstart, label=lst:UserControllertransaction]
    // 1. Starte Transaktion!
    const session = await mongoose.startSession();
    await session.startTransaction();
\end{lstlisting}

\noindent
Das dafür genutzte 'session'-Objekt wird in den weiteren Service-Funktionen übergeben. In Listing \ref{lst:UserControllerdocumentscreation} wird:
\begin{itemize}
\item ein Swipe-Dokument über die 'CreateUserSwipeDocument'-Funktion des Swipe-Services erstellt.
\item ein Match-Dokument über die 'CreateUserMatchDocument'-Funktion des Match-Services erstellt. 
\item ein User-Dokument über die 'CreateUser'-Funktion des User-Services mit entsprechender Parametrisierung erstellt.
\item die 'uid' der entsprechenen Stadt über die 'AddUserToCity'-Funktion des City-Services hinzugefügt.
 \end{itemize}
Nur wenn alle Operationen erfolgreich ausgeführt wurden, wird die Transaktion über die 'commit\-Transaction'-Methode ausgeführt. Damit soll sichergestellt sein, dass einzelne, zusammenhängende Dokumente und Informationen nur im Ganzen erstellt werden. Bei Misserfolg einer Operation wird die komplette Transaktion über die 'abortTransaction'-Methode abgebrochen.\\

\begin{lstlisting}[caption=UserController - Create User - Dokumente erstellen, label=lst:UserControllerdocumentscreation]
try {
        // 2. Erstelle SWIPE-Dokument
        var createdSwipe = await SwipeService.CreateUserSwipeDocument(uid, session);

        // 3. Erstelle MATCH-Dokument
        var createdMatch = await MatchService.CreateUserMatchDocument(uid, session);

        // 4. Erstelle USER-Dokument
        var createdUser = await UserService.CreateUser(uid, createdSwipe._id, createdMatch._id, city, malewanted, femalewanted, diversewanted, mygender, session);

        // 5. Fuege User zu City hinzu
        if (createdUser._id)
            await CityService.AddUserToCity(uid,city,session);
            
        // Transaktion erfolgreich abschliessen
        await session.commitTransaction();
        res.status(201).json();
    }
catch (Exception) {
        res.status(501).json({
         title: "Server-User Creation Error", message: Exception });
        // Fehler => Transaktion abbrechen
        await session.abortTransaction(); }
        
session.endSession();
\end{lstlisting}
   
\noindent
\textbf{ChangeUser:}
Diese Funktion erlaubt einem Nutzer, seine Eigenschaften innerhalb der Datenbank zu aktualisieren. Die Schritte sind in Listing \ref{lst:UserControlleruserchange} zu sehen. Dafür wird nach erfolgreicher Authentifizierung eine Transaktion gestartet. Im folgendem Try-Block werden die einzelnen Operationen dargestellt, die für eine erfolgreiche Transaktion ausgeführt werden. Über das User-Service werden die Methoden 'ChangeGenderWantedFromUser' und 'ChangeGenderFromUser' aufgerufen. Anschließend muss der Stadteintrag angepasst werden, welcher an  mehreren Stellen in der Datenbank geändert werden muss. 
So muss zunächst die 'uid' aus dem alten 'city'-Dokument entfernt (Schritt 3 und 4) und dem  Dokument hinzugefügt werden, dass dem aktualisierten Stadtwert entspricht (Schritt 5). Letzlich wird der Wert der 'city'-Eigenschaft über die 'ChangeCityFromUser' des User-Services angepasst.\\

\begin{lstlisting}[caption=UserController - Change User, label=lst:UserControlleruserchange]
//1. Aenderung an GenderWanted
await UserService.ChangeGenderWantedFromUser(uid, malewanted,femalewanted,diversewanted,session);

//2. Aenderung an Gender
await UserService.ChangeGenderFromUser(uid, mygender, session);

//3. Frage vorherige Stadt ab
var oldCity = await UserService.GetCityFromUser(uid,session);

//4. Loesche Nutzer aus vorheriger Stadt
await CityService.RemoveUserFromCity(uid, oldCity, session);

//5. Fuege Nutzer zu neuer Stadt hinzu
await CityService.AddUserToCity(uid, newCity,session);

//6. Aktualisiere den Stadteintrag beim Nutzer
await UserService.ChangeCityFromUser(uid, newCity,session);

await session.commitTransaction();
res.status(200).json();
\end{lstlisting}
  
\noindent
\textbf{InfoUser:}     
Diese Funktion dient dazu, nach erfolgreicher Authentifizierung die gespeicherten Eigenschaft und ihre Werte des abgefragten User-Dokuments zu erhalten. Dafür wird die 'GetInfoFromUser'-Methode des User-Services aufgerufen.
        

%
%			Match	
%				Controller
%


\paragraph{Match Controller}
Der Match-Controller bietet Funktionen, die die matches-Collection der Datenbank betreffen. Sie greift dafür vorrangig auf den Match-Service zu.\\

\noindent
\textbf{RequestMatches:} 
Das Frontend erlaubt es, Matches anzeigen zu lassen. Dafür bietet die in Listing \ref{lst:matchcontrollerrequestmatches} dargestellte Funktion Informationen über das Match-Dokument des jeweiligen Nutzers. Nach erfolgreicher Authentifizierung wird das zugehörige Match-Dokument über die 'GetMatches'-Methode abgefragt. Das Dokument enthält zwei Listen: supermatches und normalmatches. Beide enthalten jeweils eine Eigenschaft 'removed'. Ist diese auf true gesetzt, so soll impliziert werden, dass der Nutzer das Match entfernt hat. Folglich soll das gelöschte Match nicht mehr angezeigt werden. Daher werden die beiden Listen über die 'filter'-Funktion nach den nicht entfernten Matches gefiltert. Bei Erfolg wird ein JSON-Objekt mit beiden Listen und der Eigenschaft 'newChanges' aus dem Match-Dokument zurückgesendet.\\
 
\begin{lstlisting}[caption=MatchController - RequestMatches, label=lst:matchcontrollerrequestmatches]
var match = await MatchService.GetMatches(uid);
var filteredSupermatches =  match.supermatches.filter(match => match.removed == false)
var filteredNormalmatches =  match.normalmatches.filter(match => match.removed == false)
res.status(200).json({ newChanges: match.newChanges,
            supermatches: filteredSupermatches, 
            normalmatches: filteredNormalmatches} );
\end{lstlisting}

\noindent
\textbf{DeleteSupermatch:} 
Hier wird die 'SuperMatchMarkAsRemoved'-Methode des Match-Services aufgerufen, um die 'removed'-Eigenschaft des entsprechenden Supermatches auf true zu setzen. Dieser Supermatch wird folglich nicht mehr über die 'RequestMatches'-Funktion zurückgegeben.\\

\noindent
\textbf{DeleteNormalmatch:} 
Gleiches Prinzip wie 'DeleteSupermatch' mit Normalmatches.\\

\noindent
\textbf{Received:} 
Hier wird die 'newChanges'-Eigenschaft auf false gesetzt. Es wird impliziert, dass der Nutzer den aktuellsten Stand der Matches hat.\\

\noindent
\textbf{Trigger:} 
Diese Funktion ruft MatchManager.startMatching() auf. Sie ist vorerst nur für die Entwicklung gedacht, und soll es ermöglichen, über das Frontend den Matching-Algorithmus im Backend zu starten.\\

%
%			Swipe	
%				Controller
%


\paragraph{Swipe Controller}
Der Swipe-Controller bietet eine Funktion, die die swipes-Collection der Datenbank betrifft. Hierfür greift sie auf den Swipe-Service zu.
Sie bietet lediglich die Funktion \textbf{CreateSwipe} an. Sie ruft die SwipeService.AddSwipeToDB-Methode auf.



\subsubsection{Routing} 
%\begin{figure}[h]
%\centering
%\includegraphics[width=8cm]{images/routestruktur.PNG}
%\caption{Node.js Server - Controller Struktur}
%\end{figure}
In der Server.js werden dem Express-Objekt 'app' die einzelnen Routen für die Weiterleitung der HTTPS-Anfragen an die entsprechenden Controller hinzugefügt. Dabei werden die zu den Anfragen gehörenden Request- und Response-Objekte als Parameter an die Controller übergeben.

\begin{lstlisting}[caption=Routing in server.js, label=lst:routingserver]
//Movies
const moviesRouter = require('./routes/movies')
app.use('/movies', moviesRouter)

//Users
const usersRouter = require('./routes/users')
app.use('/users', usersRouter)

//Matches
const matchesRouter = require('./routes/matches')
app.use('/matches', matchesRouter)

//Swipes
const swipesRouter = require('./routes/swipes')
app.use('/swipes', swipesRouter)

\end{lstlisting}

\paragraph{Movie Router}
Innerhalb des Movie Routers wird die '/movies/request' an die Funktion RequestMovie des Movie-Controller weitergeleitet.
\begin{lstlisting}[caption=Routing in movieRouter.js, label=lst:routingmovie]
// Importiere benoetigtes Controllermodul.
const MovieController = require('../controllers/movieController')

// Leite Anfragen weiter an MovieController
router.post('/request', MovieController.RequestMovies)
\end{lstlisting}

\paragraph{User Router}
Der User Router leitet '/users/create', '/users/change' und '/users/info' an die entsprechenden Funktionen des User-Controllers weiter.
\begin{lstlisting}[caption=Routing in userRouter.js, label=lst:routinguser]
// Importiere benoetigtes Controllermodul.
const UserController = require('../controllers/userController')

// Leite Anfragen weiter an UserController
router.post('/create', UserController.CreateUser)
router.post('/change', UserController.ChangeUser)
router.post('/info', UserController.InfoUser)
\end{lstlisting}

\paragraph{Match Router}
Innerhalb des Match Routers werden die unten dargestellten URL's an die Funktion des Match-Controller weitergeleitet.

\begin{lstlisting}[caption=Routing in matchRouter.js, label=lst:routingmatch]
// Importiere benoetigtes Controllermodul.
const MatchController = require('../controllers/matchController')

// Leite Anfragen weiter an MatchController
router.post('/request', MatchController.RequestMatches)
router.post('/deleteSupermatch',MatchController.DeleteSupermatch)
router.post('/deleteNormalmatch',MatchController.DeleteNormalmatch)
router.post('/received', MatchController.Received)
router.post('/trigger', MatchController.Trigger)
\end{lstlisting}


\paragraph{Swipe Router}
Der Swipe Router leitet '/swipes/create' an die entsprechende Funktionen des Swipe-Controllers weiter.

\begin{lstlisting}[caption=Routing in swipeRouter.js, label=lst:routingswipe]
// Importiere benoetigtes Controllermodul.
const SwipeController = require('../controllers/swipeController')

// Leite Anfragen weiter an SwipeController
router.post('/create',  SwipeController.CreateSwipe )
\end{lstlisting}










\subsubsection{Weitere Backendfunktionalit"aten} 
Nachfolgend werden weitere Systemfunktionalitäten des Backends dargestellt.

\paragraph{Firebase-Service}
Um unberechtigte Zugriffe zu vermeiden, findet für die nutzerbezogenenen Anfragen eine Authentifizierung statt. Die erste Authentifizierung des Nutzers findet über den Login des Frontends in Firebase statt. Nachträglich muss bei Anfragen an den Webserver sichergestellt werden, dass der Nutzer weiterhin authentifiziert ist. Ohne diesen Vorgang könnte man sich über das Schicken einer willkürlich übermittelten Nutzer-Uid fälschlicherweise als anderer Nutzer ausgeben. Für den Zugriff auf die Firebase-Authentifizierungsfunktionen wird in die firebaseService.js-Datei das Modul 'firebase-Admin' importiert \footnote{Siehe Dokumentation: \url{https://firebase.google.com/docs/admin/setup}, letzter Zugriff: 3. April 2021}.\\

\noindent
%\hangindent1cm
\textbf{Register:}
Um die Anwendung bei Firebase zu registrieren, wird die Funktion 'initializeApp' des Firebase-Moduls ausgeführt. 
Ein aus Firebase generierter Authentifizierungsschlüssel wird dabei für den Zugriff auf die StreamSwipe-Umgebung mitübergeben.
Die Register-Methode wird anschließend nach außen exportiert und zu Beginn des Serverstarts in der Server.js-Datei ausgeführt 
\footnote{Siehe Dokumentation: \url{https://firebase.google.com/docs/admin/setup\#initialize-without-parameters}, letzter Zugriff: 3. April 2021}.\\
   
\begin{lstlisting}[caption=Firebase-Service Register, label=lst:firebaseService Register]
var serviceAccount = require("../sslcert/streamswipe-firebase-adminsdk-uiyci-80bc08a5b2.json");
firebaseAdmin.initializeApp({
      credential: admin.credential.cert(serviceAccount),
        databaseURL: "https://streamswipe.firebaseio.com"
    });
\end{lstlisting}

\noindent
\hangindent1cm
\textbf{UID/TokenID-Dictionary:}
Um Zugriffszeiten auf die Firebase-Schnittstelle, werden in einem lokalen Dictionary aus Schlüsselwertpaaren der Zusammenhang zwischen TokenID und den UID samt ihrem Ablaufsdatum zwischen\-gespeichert.\\

\noindent
%\hangindent1cm
\textbf{GetUID:}
Die Funktion erwartet einen Firebase Token als Parameter 'uidtoken', welcher an die Funktion 'verifyIdToken' des FirebaseAdmin-Objeekts weitergeleitet wird. Zurück wird ein Objekt gegeben, dass unter anderem die 'uid' des zum Token zugehörigen Nutzers und die Ablaufzeit schickt. Nach erfolgreichem Überprüfen, ob die 'uid' tatsächlich ein Wert übermittelt bekommen hat, wird das Paar aus UidToken und Uid samt Ablaufzeit in der UID/TokenID-Dictionary gespeichert.\\

\begin{lstlisting}[caption=Firebase-Service Register, label=lst:firebaseServiceRegister]
verifiedUid = await firebaseAdmin.auth().verifyIdToken(uidToken);
uid = verifiedUid.uid;
expireTime = verifiedUid.exp;
if(uid === undefined || uid == null) { throw {message: "No uid returned!"}; }
TokenIDDict[uidToken] = {uid,new Date(expireTime*1000)};
return uid;
... //Ende Try-Catch-Block
\end{lstlisting}
   
\noindent
%\hangindent1cm
\textbf{RefreshList:}
Diese Funktion wird aufgerufen, um abgelaufene Token in der UID/TokenID-Dictionary zu löschen. Sie wird über das Modul TimedEvents periodisch aufgerufen. Dabei wird zu jedem Paar die aktuelle Uhrzeit und die Ablaufszeit verglichen. Stellt die Ablaufszeit ein größeren Wert dar, wird das Schlüsselwertpaar aus der Dictionary entfernt.\\

\paragraph{Timed Events}
Über das 'node-cron'-Modul \footnote{Siehe Dokumentation: \url{//https://www.npmjs.com/package/node-cron}, letzter Zugriff: 26. April 2021}
können zeitlich definierte und periodische Funktionen ausgeführt werden. Dafür wird das 'node-cron'-Modul in die TimedEvents.js-Datei importiert. Die Funktionalität des Moduls wird beispielsweise für das periodische Aktualisieren der movies-Collection, das periodische Ausführen des Matching-Algorithmus und das Aufrufen der 'firebaseService.RefreshList'-Funktion zum Aktualisieren der UID/TokenID-Dictionary verwendet.\\

\paragraph{MatchManager}
\noindent
%\hangindent1cm
\textbf{StartMatching - Teil 1:}
Die aktuelle Implementierung des Matching-Algorithmus sucht für jede Stadt Nutzerpaare, die einen gleichen Film mit einem Superlike versehen haben. Dafür werden zunächst über die 'GetAllInhabitedCities'-Funktion des City-Services die Städte in einer Liste gespeichert, die mindestens zwei Nutzer aufweisen. Folglich finden eine Verschachtelung von Iterationsabläufen zum Ausführen von Programmcode auf jedem Element einer Liste statt.
In der ersten Iterationsstufe wird durch die Städte iteriert.
Die zweite Iterationsstufe vom ersten bis zum vorletzten Nutzereintrag ('uid') innerhalb der aktuell iterierten Stadt. Innerhalb des zugehörigen Codeblocks wird die 'RequestSuperlikeSwipes'-Funktion des SwipeServices aufgerufen mit der 'uid' des aktuell iterierten Nutzers als Parameterübergabe. Die zurückerhaltene Liste enthält sämtliche Film-ID's, die vom Nutzer mit einem Superlike versehen wurden. Die Liste wird samt dem Index des nächsten Users in der Liste, der aktuell iterierten Stadt und dem aktuell iterierten User. Ausserdem wird eine Referenz auf das 'foundSupermatches'-Objekt, dass später mit Informationen zu den errechneten Supermatches gefüllt wird, mitgegeben.\\

\begin{lstlisting}[caption=Match Manager - startMatching - Teil 1: Finde Matches, label=lst:findMatches]
var cities = await CityService.GetAllInhabitedCities();

// City - 1. Iterationsstufe
for (let cityIterator = 0; cityIterator < cities.length; cityIterator++) {
	// Fuer jeden Nutzer ausser den letzten
	var prelastSupermatchIndex = cities[cityIterator].user.length - 2;
           
    // User - 2. Iterationsstufe
    for (let user1Iterator = 0; user1Iterator <= prelastSupermatchIndex; user1Iterator++) {
    	var matchingUserNextIndex = user1Iterator + 1;
    	var User1 = cities[cityIterator].user[user1Iterator];
      
    	//SUPERMATCH-CHECK
    	var user1SuperlikedMovies = await SwipeService.RequestSuperlikeSwipes(User1.uid);
    	if (user1Superlikes) {
    	// Ueberpruefe Superlikes mit den nachfolgenden Usern
    	await checkSuperMatches(matchingUserNextIndex, cities[cityIterator], user1SuperlikedMovies, foundSupermatches, User1);
                }

    	//NORMALMATCH-CHECK
    	var user1swipes = await SwipeService.RequestSwipes(User1.uid);
    	if (user1swipes) {
    	await checkNormalMatches(matchingUserNextIndex, cities[cityIterator], user1swipes, normalmatches, User1);
    }}}}
    ... //end Try-Catch-Block
\end{lstlisting}

\noindent
%\hangindent1cm
\textbf{checkSuperMatches:} Die Funktion wird im Matching-Algorithmus von der 'findMatches'-Methode aufgerufen. Es wird ausgehened vom übergebenen 'matchingUserNextIndex', welches der Index des nächsten Users nach dem aktuell iterierten Users der 2. Iterationsstufe ist, durch die restliche Nutzerliste der übergebenen Stadt 'currentCity' iteriert. Dies entspricht der 3. Iterationsstufe.
Hierbei wird zunächst die lokale Funktion 'checkGenderPreference' aufgerufen, die anhand der übergebenen Nutzerobjekte prüft, ob jeweils das Geschlecht des einen Nutzers und das für das Matchen preferierte Geschlecht des anderen Nutzers übereinstimmen. Ist dies der Fall, so werden die Superlikes des zweiten Users angefragt. Abschließend werden die Superlikes-Listen beider Nutzer verglichen. Dabei wird überprüft, ob eines der MovieID's der einen Liste in der anderen vorhanden ist. Bei einem Treffer wird ein Objekt mit den 'uid' der bei Nutzer und die 'movieid' des zugehörigen Films in die 'supermatches'-Liste hinzugefügt. \\

\begin{lstlisting}[caption=Match Manager - checkSuperMatches, label=lst:checkSuperMatches]
async function checkSuperMatches(matchingUserNextIndex, currentCity, user1Superlikes, foundSupermatches, User1) {

 for (let user2Iterator = matchingUserNextIndex; user2Iterator <= currentCity.user.length - 1; user2Iterator++) {
   var User2 = currentCity.user[user2Iterator];
   
  // Ueberpruefe GenderPreferenz
  if (!(await checkGenderPreference(User1, User2))) return;
  
  // Frage Superlikes ab
  var user2Superlikes = await SwipeService.RequestSuperlikeSwipes(User2.uid);
  if (user2Superlikes) {
        
  //Vergleiche beide
   for (let superlikeIterator = 0; superlikeIterator < user1Superlikes.length; superlikeIterator++) {
    for (let superlike2Iterator = 0; superlike2Iterator < user2Superlikes.length; superlike2Iterator++) {
    
      if (user1Superlikes[superlikeIterator].movieid === user2Superlikes[superlike2Iterator].movieid) {
        foundSupermatches.push({ matchid1: User1.uid,
                            matchid2: User2.uid, 
                            movieid: user1Superlikes
                            [superlikeIterator].movieid });
                            continue;
}}}}}}
\end{lstlisting}

\noindent
%\hangindent1cm
\textbf{checkNormalMatches:}
Diese Funktion ist zum Zeitpunkt der Dokumentation nicht implementiert \\

\noindent
%\hangindent1cm
\textbf{StartMatching - Teil 2:}
Die 'startMatching'-Funktion wird beendet, nachdem die gefundenen Matchinformationen in den entsprechenden Match-Dokumenten gespeichert werden. Es wird durch die Liste 'foundSupermatches' durchiteriert. Dafür wird zunächst geprüft, dass ein Match der zwei Nutzer noch nicht in der Datenbank hinterlegt ist. Anschließend werden beiden zugehörigen Match-Dokumenten die gegenseitigen 'uid' in die 'supermatches'-Liste hinzugefügt.
Der Vorgang wird für die 'normalmatches'-Liste wiederholt mit entsprechender Speicherung in die 'normalmatches'-Listen.

\begin{lstlisting}[caption=Match Manager - startMatching - Teil 2: Speichere Matches, label=lst:startMatchingteil2]
await session.startTransaction();

for (let i = 0; i < foundSupermatches.length; i++) {        
 // Ueberpruefe, ob Match vorhanden ist
 if (!(await MatchService.CheckMatch(foundSupermatches[i].matchid1,   foundSupermatches[i].matchid2, session))) {
  // Fuege Match zu User1's Match-Dokument hinzu
  await MatchService.AddSuperMatchToUser(foundSupermatches[i].matchid1, foundSupermatches[i].matchid2, foundSupermatches[i].movieid, session);

  // Fuege Match zu User2's Match-Dokument hinzu
  await MatchService.AddSuperMatchToUser(foundSupermatches[i].matchid2,  foundSupermatches[i].matchid1, foundSupermatches[i].movieid, session);
}}
await session.commitTransaction();
\end{lstlisting}





\subsection{Firebase}
\label{sec:implementierung_firebase}
Firebase bietet viele verschiedene Werkzeuge zur Entwicklung und Überwachung von Mobil- und Webanwendungen.
Nach dem Konzept nach Kapitel \ref{sec:konzept} beschränkt sich der Aufgabenbereich des Firebase Backends auf Nutzerauthentifizierung, Verwaltung der Nutzerdaten und der Chatfunktion.
Die hierfür genutzten Werkzeuge werden im Folgenden besprochen.\\
Grundlegend muss zunächst eine Anwendung erstellt, ein Firebase Projekt aufgesetzt und diese zusammen verknüpft werden.
Sobald dieser Prozess abgeschlossen ist, muss sichergestellt sein, dass Firebase in der Anwendung initialisiert wird.

\subsubsection{Authentifizierung}
Um Nutzern eine sichere Registrierung, bzw. An- und Abmeldung ermöglichen zu können bietet Firebase das Authentifizierungswerkzeug. 
Dieser Backendservice verfügt über unterschiedliche Authentifizierungsmethoden.
Um es zu nutzen, muss nur die jeweilige Methode ausgewählt werden. \\

In unserem Fall wurde die E-Mail und Passwort Authentifizierung gewählt, um unabhängig von Drittanbietern zu sein.
Nun muss das SDK \texttt{firebase\_auth} integriert werden und mithilfe der Funktionen \texttt{createUserWithEmailAndPassword()} bzw. \texttt{signInWithEmailAndPassword()} ein Nutzer registriert und angemeldet werden.
Hierbei können mithilfe von Fehlercodes geeignete Fehlermeldungen erstellt werden. 
Zusätzlich besitzt die Klasse \texttt{User} das Feld \texttt{emailVerified} (Boolean) und die Methode \texttt{sendEmailVerification()}, wodurch eine Verifizierung der E-Mail über Nachrichtenvorlage durchgeführt wird.
Ist ein Nutzer einmal authentifiziert, muss unterschieden werden, welchen Bildschirm er sehen darf.
Hierzu wird darauf geachtet ob ein aktueller Nutzerobjekt existiert.
Ist dies nicht der Fall, wird der Anmeldebildschirm angezeigt; ansonsten der Hauptbildschirm.
\medspace
\begin{lstlisting}[caption= Anzeige abhängig ob ein aktueller Nutzer existiert]
	// Globale Instanz des Authentifizierungsservice
	final AuthService auth = Provider.of(context).auth;
	return FutureBuilder<User>(
		future: auth.getCurrentUser(),
		builder: (BuildContext context, AsyncSnapshot<User> snapshot) {
			// Existiert ein Nutzer, wird der Hauptbildschirm angezeigt
			if (snapshot.hasData) {
				return MessageHandler();
			} 
			// Zeige ansonsten den Login-Bildschirm
			else {
				return SignUpScreen(authFormType: AuthFormType.signIn);
			}
		}
	);
\end{lstlisting}
\medspace

\subsubsection{Nutzerdaten}
Bei der Authentifizierung eines Nutzers wird mittels des Auth-Services ein Nutzerobjekt erstellt.
Dieses kann folgende Felder besitzen:
\begin{itemize}
	\item Einzigartige Identifikation (UID)
	\item E-Mail Adresse
	\item Namen
	\item Bild-URL
\end{itemize}
Es ist jedoch nicht möglich über diesen Service weitere Eigenschaften abzuspeichern.
Diese müssen über zusätzliche Speicherwerkzeuge, wie beispielsweise Cloud Firestore (siehe \ref{sec:firestore}) gesichert werden.\\
\\
Hierzu wurde die Sammlung \enquote{users} erstellt und bei der Registrierung für jeden Nutzer ein Dokument mit der selben ID, wie die Nutzer UID erstellt. 
Dadurch ist sichergestellt, dass es ein einzigartiges Dokument und der Zugriff einfach geregelt ist.
Hier werden nun weitere Eigenschaften gesichert, welche teilweise ausschließlich zu Anzeigezwecken in Firebase doppelt abgespeichert werden.
\begin{itemize}
	\item Anzeigename
	\item E-Mail
	\item Wohnort, welcher aus den wichtigsten Städten Deutschlands bestehen
	\item Geschlechter, nach welchen gesucht wird
	\item Eigenes Geschlecht
	\item Lieblingsfilm
\end{itemize}
Unser Nutzer jedoch speichert seine Profil und Hintergrundbilder weder im Auth-Service Nutzerobjekt noch im Cloud Firestore Dokument.
Bei Auth-Service lässt sich lediglich die URL zu einem einzigen Bild hinterlegen.
Bei Firestore ist es zwar möglich eine theoretisch unbegrenzte Menge an Bild URLs abzuspeichern, jedoch muss der Nutzer die Möglichkeit haben seine eigenen Bilder hochzuladen und nicht nur eine URL auf ein bereits hochgeladenes Bild abspeichern.\\
\\
Hierfür bietet Firestore das Werkzeug Storage. 
Wie in Kapitel \ref{sec:firebase_storage} beschrieben, können hier Nutzerinhalte hoch- und heruntergeladen werden. 
Dazu wird beim ersten Hochladen ein Ordner für jeden Nutzer erstellt.
Darin werden daraufhin die Bilder unter dem Namen \texttt{profile-picture} oder \texttt{background-picture} jeweils abgespeichert und eventuell überschrieben.
Mit einer Funktion kann über einen Boolean-Parameter entschieden werden, welches Bild somit angezeigt wird.
\medspace
\begin{lstlisting}
	Future<String> getPictureFromStorage(String uid, bool isProfilePicture) async {
		try {
			// Die Referenz auf Storage
			Reference storage = FirebaseStorage.instance.ref();
			// Ordner UID mit Datei profile-picture oder background-picture
			Reference ref = storage.child(uid)
				.child(isProfilePicture ? '/profile-picture' : '/background-picture');
			// Gebe die URL zum Download zurueck
			return await ref.getDownloadURL();
		} on Exception catch (e) {
			// Fehlerbehandlung eine Ebene oberhalb
			throw e;
		}
	}
\end{lstlisting}
\medspace
\subsubsection{Chatfunktion}
Damit Nutzer bei einem erfolgreichen Match sich unterhalten und vielleicht auch verabreden können, muss eine Anwendung dieser Art eine Chatfunktion bieten.
Diese Funktion jedoch beschränkt sich auf eine 1-zu-1 Kommunikation, es werden also keine Gruppenchats benötigt.\\

Da Google weltweit über Server verfügt, ist diese Chatanwendung mithilfe von Firebase direkt auch global verfügbar.
Zudem würde der Backend Server, welcher die Filmdaten bereitstellt und Nutzer über Matching Algorithmen zusammenführt, zusätzlich durch Netzwerkverkehr der Chatanwendung belastet.
Hierfür bietet Firebase extra auf Skalierung ausgelegte Werkzeuge, damit sich Entwickler nicht zwingend mit diesen Problematiken auseinandersetzen müssen.

\subsubsection{Sicherheit}
Um unbefugte Zugriffe nicht zuzulassen, müssen die Sicherheitsregeln korrekt gewählt werden.
Diese können mittels der \enquote{Emulator Suite} und Unit Tests (hier das Test Framework \enquote{mocha}) auf ihre Korrektheit getestet werden\footnote{\url{https://firebase.google.com/docs/rules/unit-tests}, zuletzt aufgerufen am 05.05.2021}.\\
\\
Da jeder letztendlich die Profilbilder sehen darf, ist hier nur als Bedingung gegeben, dass der Sender der Anfrage angemeldet sein und die Datei kleiner als 5 MB sein muss (siehe Codebeispiel \ref{lst:storagerules_validation}). 
Dies wurde gewählt, da ab einer Menge von 5 GB verbrauchter Speicherplatz Kosten in Höhe von 0.026\$ pro GB anfallen.\footnote{\url{https://firebase.google.com/pricing}, zuletzt aufgerufen am 05.05.2021}\\
\\
Bei den Firestore Regeln ist dies jedoch etwas komplizierter - diese sind im Anhang als Codebeispiel \ref{lst:appendix_firestore_rules} zu finden.
Die Zeilenabgabe ist bei den folgenden Erklärungen am Ende des Satzes zu finden.
Grundsätzlich ist für Nutzer essentiell wichtig, dass Zugriffe nur gewährt werden, falls man authentifiziert ist.
Ein Nutzer soll logischerweise Schreibrechte auf seinen eigenen Eintrag in der Datenbank haben (5).
Ein Fremder hingegen darf dieses Lesen, falls er einen Eintrag in der Datenbank besitzt, damit keine Fehler auf der Oberfläche entstehen, falls veraltete Accounts (ohne Eintrag in der Datenbank) auf etwas zugreifen wollen (6).
Auf die eigene Untersammlung \texttt{tokens} darf nur der Nutzer selbst zugreifen, damit zum Beispiel Benachrichtigungen nicht auch fälschlicherweise auf anderen Geräten angezeigt werden kann (7).\\
Die eigenen Chaträume darf ebenfalls nur der Nutzer selbst sehen und verändern (11).
Die eingehenden Chatanfragen in der Untersammlung \texttt{pendingChatrooms} darf jeder Lesen, der einen existenten Eintrag in der Datenbank besitzt, da die Funktion \texttt{isPartOfChat()} für unseren Anwendungsfall Fehler liefern würde (14).
Der Ursprung liegt bei der Überprüfung der Chat-Anwendung, ob ein Nutzer das Profil des anderen sehen darf oder nicht.
Es wird also überprüft, ob der Nutzer in der eingehenden Chatanfrage des anderen Nutzers ist.
Es tritt ein Fehler auf, welcher in der Testumgebung bisher nicht reproduziert werden konnte.
Gleichzeitig schreiben darf nur jemand, der Teil des Chats ist oder eben der Nutzer selbst (15).\\
Auf die Sammlung \texttt{unreadMessage} hat jeder angemeldete Nutzer mit Datenbankeintrag Zugriff, damit jeder Benachrichtigungen an Personen versenden kann (18).\\
Ein Chatraum darf jeder valide Nutzer erstellen (22) und die generellen Informationen über ihn auch lesen (24).
Das Lesen hatte gleiche Hintergründe, wie bei (14).
Schreibzugriff durch beispielsweise Namensänderungen besitzt jeder, der Teil des Chats ist (23).
Um einzelne Nachrichten im Chat lesen und schreiben zu dürfen, muss der Nutzer auf jeden Fall Teil des Chats und valide sein.\\
\\
Da auf Nutzerdaten, welche im Auth-Service abgespeichert werden sowieso nur der eigentliche Nutzer zugreifen darf, gibt es hier auch keine Regeln.




\subsection{Swipe/Aussuchen/Voting}		
\subsection{Matches/Chat}		
\subsection{Film-/Serienvorschläge}
\subsection{Gespeicherte Filme/Filmliste}		
\subsection{Barrierefreiheit}
\label{sec:barrierefreiheit}

Barrierefreiheit im Allgemeinen bedeutet, dass ein Gegenstand, eine Einrichtung oder Informationsquelle für Menschen mit Behinderung ohne Unzulänglichkeiten nutzbar, zugänglich oder auffindbar ist (\cite{behindertengleichstellungsgesetz}, §4). In der Softwareentwicklung versteht man darunter Applikationen für Menschen mit Einschränkungen zugänglich und bedienbar zu machen. Bezogen auf die Entwicklung von  mobilen Apps gilt es dabei den akustischen, optischen oder motorischen Einschränkungen der Benutzer entgegenzuwirken. \\

\subsubsection{Barrierefreiheit in mobilen Anwendungen}
Mit der Verbreitung von Smartphones ist die Benutzung mobiler Apps stark angestiegen und mittlerweile in nahezu jedem Haushalt aufzufinden. Obwohl etwa 9,5\% aller in Deutschland lebenden Menschen einen Schwerbehindertenausweis besitzen (Stand 24.06.2020)\cite{schwerbehindertenausweis} was etwa 7,9 Millionen Menschen entspricht, ist die Implementierung von barrierefreier Bedienung nicht selbstverständlich. Gerade Programmierern/innen aus dem privaten Sektor sind diese Funktionen oft nicht bekannt, es besteht kein Interesse oder sie werden schlichtweg vergessen. Software, die für öffentliche Einrichtungen entwickelt wird, ist durch das Behindertengleichstellungsgesetz von 2002 dazu verpflichtet ihr Softwareangebot bis spätestens dem 23. Juni 2021 barrierefrei zu gestalten (\cite{behindertengleichstellungsgesetz}, §12a Abs.1). Hierzu zählen sämtliche Webseiten sowie mobile Anwendungen. \\

\subsubsection{Barrierefreiheit in Filmen und Serien}
Auch die Zugänglichkeit von Filmen und Serien für Menschen mit eingeschränkter Wahrnehmung wurde in den letzten Jahren stark verbessert. 
Hierbei lässt sich zwischen optischer und akustischer Einschränkung differenzieren. Für hörgeschädigte Personen werden bereits seit mehreren Jahrzehnten Untertitel eingesetzt. Was früher für vereinzelte Filme durch eine Funktion des Teletextes erreicht wurde, wird heutzutage durch eine integrierte Funktion des Videoplayers verwirklicht. Immer mehr Videos werden mit Untertiteln veröffentlicht. Manche Anbieter wie beispielsweise die Internetplattform YouTube bieten durch Spracherkennung automatisch generierte Untertitel an, was eine flächendeckende Untertitelung ermöglicht.\\
Auch für Menschen mit eingeschränktem Sehvermögen werden Filme und Serien mithilfe von Audiodeskriptionen vermehrt zugänglich gemacht. Hierbei wird die bereits vorhandene Tonspur mit Bildbeschreibungen und Kommentaren versehen. Was bis vor wenigen Jahren noch etwas Besonderes war und nur für ausgewählte Filme bestimmt war, ist heutzutage Standard. Größere Video-On-Demand-Plattformen wie Netflix oder Amazon Prime bieten diese Möglichkeit bei nahezu allen Eigenproduktionen an. Zusätzlich werden bestehende Filme neu mit Audiodeskriptionen versehen.\\


\noindent Hieraus lässt sich leicht erkennen, dass Filme und Serien heutzutage auch von Menschen mit Einschränkungen genutzt werden. Was auf den ersten Blick vielleicht nicht bedacht wird oder als  unwichtig abgestempelt wird, kann einen nicht unerheblichen Vergrößerungsfaktor für den Kundenstamm bewirken. Für die Entwicklung einer mobilen App, bei der Filme und Serien bewertet werden, spielt also die Barrierefreiheit eine wichtige Rolle und darf auf keinen Fall vernachlässigt werden. 

\subsubsection{Barrierefreiheit bei StreamSwipe}
\label{sec:bf-streamswipe}

Bei der Entwicklung von StreamSwipe werden mehrere mögliche Einschränkungen der User betrachtet und entsprechend reagiert. Ziel ist es, dass sowohl der Kunde sowie der Anbieter maximal davon profitieren. Hierfür soll die App für ein möglichst großes Publikum zugänglich gemacht werden, jedoch auch sogenanntes Over-Engineering vermieden werden, da zu viele Funktionen eine App unübersichtlich, teuer und langsamer werden lassen.\\

\noindent
Allgemein wird Leserlichkeit durch große Schriftgrößen, hohe Farbkontraste, große Schaltflächen oder universelles Design erreicht. Alleine in Deutschland tragen 44,5 Millionen Menschen regelmäßig eine Brille oder Kontaktlinsen und benötigen somit Sehhilfen \cite{sehhilfen}. Unterstützung auf Seiten der App kann hierfür durch vergrößerbaren Text geschehen. Da aber davon ausgegangen werden kann, dass Personen, die sich auf Sehhilfen verlassen, bereits eine Brille oder Kontaktlinsen besitzen, wird die Textgröße vorerst nicht variabel gehalten. Außerdem gibt es bei Android- und Apple-Smartphones bereits eingebaute Vergrößerungsfeatures, die Bildausschnitte vergrößert darstellen können. Aus diesem Grund wird in diesem Projekt kein Fokus auf dieses Feature gelegt. \\
Farbblindheit kann jedoch in vielen Formen auftreten. Um der bekannten Farbfehlsicht entgegenzuwirken, werden Farben aus Problembereichen wie Rot und Grün nicht nebeneinander benutzt. Allgemein wird ein schlichtes Design gewählt und Farben nur zu Akzentuierung und als Stilmittel benutzt, statt als Informationsträger wie beispielsweise in den Abbildungen \ref{fig:bf-beispiel_a} erkennbar ist.  Geringe Sehschärfe durch Achromatopsie kann wie weiter oben beschrieben umgangen werden.\\


\begin{figure}[tbt]
	\begin{subfigure}{0.5\textwidth}
	\centering
	\includegraphics[scale=0.15]{Barrierefreiheit/images/bsp-swipe.png}
	\caption{}
	\label{fig:bf-beispiel_a}
	\end{subfigure}
	\begin{subfigure}{0.5\textwidth}
	\centering
	\includegraphics[scale=0.15]{Barrierefreiheit/images/bsp-profil.png}
	\caption{}
	\label{fig:bf-beispiel_b}
	\end{subfigure}
\caption{Screenshots aus der App StreamSwipe als Beispiele zu (a) schlichtem Design, bei dem farbige Akzente nicht der Informationenübertragung dienen um die Zugänglichkeit für farbblinde Menschen zu verbessern und für einen Icon in (b), welcher sonst durch sehgeschädigte Menschen nicht wahrnehmbar ist, wird exemplarisch eine Semantik implementiert.}
\label{fig:BF-Beispiele}
\end{figure}


\noindent
Ist die Sehkraft noch weiter eingeschränkt oder gar nicht mehr vorhanden, werden Semantiken eingesetzt. Hierbei erhält jedes Element auf dem Bildschirm eine Beschreibung, die vorgelesen werden kann. Bei Zahlen und Texten werden diese vorgelesen, sofern keine weitere Information hinterlegt ist. Besonders hilfreich ist dies jedoch bei Abbildungen. Ausgeführt wird das Auslesen von einem Screenreader. Mobile Geräte haben diese Funktion bereits standardmäßig eingebaut (VoiceOver bei Apple und TalkBack bei Android) und wandeln die Semantiken mittels Sprachsynthese in akustische Signale um. Bei Desktopanwendungen wie z.B. JAWS für Windows können diese Informationen zusätzlich auch durch eine Braillezeile wiedergegeben werden.\\
Bei Flutter ist das Hinzufügen von Semantiken bereits eingebaut. Hierfür kann ein String dem jeweiligen Bereich zugeordnet werden. In Beispiel \ref{lst:semantics} ist hierfür der Code des Buttons, der zu den Einstellungen führt. In Abbildung \ref{fig:bf-beispiel_b} ist dieser Button ganz rechts oben im Eck zu sehen.\\
Der GestureDetector erkennt Interaktionen mit dem Touchscreen, wobei hier nur auf Antippen reagieren soll, deshalb die Funktion onTap:()$\{\}$, die auf den Einstellungsbildschirm leitet. Diese Implementierung ist hier aber nicht von Relevanz und wird übersprungen. In dem GestureDetector ist ein Icon eingebettet, von der Form \textit{Settings}, was einem Zahnrad entspricht. Dieses Icon erhält eine Farbe und anschließend eine Semantik aus allem was in den Anführungszeichen steht. Ein Screenreader kann AE erkennen und ihn als den Umlaut Ä aussprechen. \\
So wird im kompletten Programm für jedes relevante Element vorgegangen. Teilweise  müssen den Semantiken Variablen übergeben werden, da sich die vorzulesende Information ändert wie beispielsweise bei den Filmtiteln.
    
\begin{lstlisting}[caption=Codeausschnitt in Dart von einem Button mit Semantiken.,label=lst:semantics]
GestureDetector(
  onTap: () {
     ...
  },
  child: Icon(
    Icons.settings,
    color: Provider.of(context).colors.textSmall,
    semanticLabel: "Einstellungen. Zum Auswaehlen doppeltippen.",
  )
),
\end{lstlisting}

\noindent
Bei einer sauberen Implementierung wird auf diese Weise vorgegangen und eine bereits vorhandene Funktion verwendet. Dies vereinfacht nicht nur die Leserlichkeit des Codes, sondern bietet auch die höchste Modularität, da hierbei normalerweise standardisierte Schnittstellen für Betriebssysteme oder andere Anwendungen verwendet werden. In diesem Fall  müssen die Screenreader von Android und Apple damit arbeiten können.\\

\noindent 
Um für Personen mit eingeschränktem Hörvermögen oder vollständiger Gehörlosigkeit die App zugänglich zu machen, wird auf akustisches Feedback als notwendige Infor"-mations"-über"-tragung verzichtet. Innerhalb der App werden keine Geräusche erzeugt, außer der oben beschriebenen Funktion der Semantiken. Beim Erhalten einer neuen Nachricht oder eines neuen Matches kann weiterhin optional eine akustische Benachrichtigung erhalten werden. Hierbei wird die betriebssystemeigene Funktion übernommen, sodass in der App keine neuen Einstellungen vorgenommen werden müssen.\\

\noindent
Auch feinmotorische Einschränkungen werden versucht zu umgehen. Die Navigation und die Filmbewertung in StreamSwipe können durch großflächige Wischbewegungen ausgeführt werden. Wo diese Lösung nicht möglich ist, werden verhältnismäßig große Buttons eingesetzt. Lediglich beim Registrieren und Einloggen werden feine Bewegungen erfordert. Hierbei öffnet sich allerdings die als Standard eingestellte digitale Tastatur, die in vielen Fällen eine Spracheingabe besitzt, sodass die sehr kleinen Tasten nicht benutzt werden müssen.\\

%TODO Diesen Satz evtl. ganz ans Ende ins Fazit
\noindent
Sollte sich in Zukunft jedoch Kritik in Form von negativen Nutzerbewertungen herauskristallisieren, kann eines der noch nicht implementierten Features über ein Update nachgerüstet werden.
\clearpage

\section[Benutzeroberflächen]{Benutzeroberflächen \hfill \normalfont \small{Vincent Schreck}}
Die Benutzeroberfläche einer Software muss im Grunde genommen nur einen Informationsfluss in zwei Richtungen erzeugen. Die eine Richtung liefert Informationen an den Nutzer und über die andere kann der Nutzer Informationen an das System weitergeben. Um auf dem heutigen Markt Fuß fassen zu können, sollte eine Oberfläche jedoch wesentlich mehr Aspekte erfüllen. 



\subsection{Aspekte von Benutzeroberflächen}
\label{sec:UI-Aspekte}
Die Vielschichtigkeit einer Benutzeroberfläche kann ausschlaggebend für den Erfolg einer Applikation sein, abhängig davon welche Erfahrungen der Nutzer mit der Oberfläche macht und welche Eindrücke sie hinterlässt. Hieraus resultiert wie lange ein Nutzer auf der App bleibt und wie oft er zurück kommt. Neben der Nutzungszeit erhöht eine positive User Experience die Weiterempfehlungsrate.\\
Bei erfolgreicher Software besteht ein großer Teil der Entwicklung in der Planung der Oberfläche, da der Nutzer immer eine User Experience erlebt und mit einem Gefühl verbindet. Neben den offensichtlichen Aus- und Eingabefunktionen werden beispielsweise folgende Kriterien  betrachtet:\\

\noindent
\hangindent1cm
\textbf{Simpel:} Ausgegebene Information kann zum Beispiel durch Icons, Farben oder Symbole vereinfacht werden. Eine Oberfläche sollte weder überladen sein, noch sollten alle Ein- und Ausgaben auf verschiedenen Seiten verteilt sein. Bei der Entwicklung wird eine gesunde Mischung aus maximaler Funktionalität und einfacher, übersichtlicher Darstellung angestrebt.\\

\noindent
\hangindent1cm
\textbf{Einheitlich:} Die Bedienung und das Lesen von Apps kann erheblich vereinfacht werden wenn einheitliche Bedien- oder Ausgabeelemente verwendet werden. Nicht nur innerhalb einer App ist es sinnvoll konsistente Elemente in der Oberfläche zu verwenden, auch Funktionen von anderen Apps können die Bedienung vereinfachen. Bekannte Funktionen bei Smartphone-Applikationen sind zum Beispiel die Vergrößerung mit zwei Fingern oder das \glqq Daumen nach oben\grqq -Symbol als positive Rückmeldung. Durch das  Einbauen solcher Features wird eine App intuitiv und ohne Einführung bedienbar.\\

\noindent
\hangindent1cm
\textbf{Benutzergesteuert:} Alle ausgeführten Aktionen sollten vom Benutzer ausgehen. Eine gute Benutzeroberfläche unterstützt den Nutzer  lediglich bei seiner Bedienung, schränkt ihn aber nicht ein. Mit der heutigen Technologie ist die Verführung groß, viele Funktionen automatisch ablaufen zu lassen. Was eigentlich der Sinn einer App ist, kann jedoch auch negative Folgen haben. Zu viel Automatisierung verursacht das Gefühl von Kontrollverlust und Unsicherheit, was sich negativ auf das Vertrauen und somit auf die Benutzungszeit der Nutzer auswirkt. \\

\noindent
\hangindent1cm
\textbf{Klarheit:} Eine mobile App muss ohne Anleitung bedienbar sein. Sobald Unklarheiten beim Nutzer  entstehen und Funktionen oder Ausgaben nicht erkannt werden können, verliert die Anwendung auf dem freien Markt. \\
Der Nutzer sollte zu jeder Zeit wissen welche Optionen ihm zur Verfügung stehen und welche Folgen seine Aktionen haben. Besonders wichtig ist das Feedback infolge einer Aktion, wie beispielsweise ein Ladebalken nachdem etwas angeklickt wurde und noch nicht vollständig heruntergeladen ist. Auch wenn diese Aspekte offensichtlich erscheinen, können sie bei der Entwicklung einer App leicht übersehen werden. Verwendet werden einfache und für den Nutzer bekannte Funktionen, wie die Beschriftung aller Buttons oder das haptische, akustische oder optische Feedback beim drücken einem dieser Buttons.\\

\noindent
\hangindent1cm
\textbf{Benutzerfreundlich/Barrierefreiheit:} Die Bedienung der App sollte für Menschen mit Einschränkungen im vollen Umfang möglich sein. In Abschnitt \ref{sec:barrierefreiheit} wird auf dieses Thema tiefer eingegangen. Aber auch Benutzer ohne Einschränkungen erwarten eine einfache und übersichtliche Bedienung, die auch beispielsweise  Eingabefehler mit mehreren Versuchen verzeiht.\\

\noindent
\hangindent1cm
\textbf{Ästhetik:} Das Design spielt bei dieser Betrachtung gleich mehrere wichtige Rollen. Es sollte eine angenehme Arbeitsumgebung für den Nutzer erstellen, Ein- und Ausgaben verdeutlichen und gleichzeitig mithilfe eines eigenen Stils ein einzigartiges Image für die App schaffen (sogenanntes Branding) um deren Individualität und Wiedererkennungswert zu steigern. Das Design erschafft ein Erlebnis während der Benutzung und weckt unterbewusst Gefühle im Nutzer. \\

\noindent
Gerade weil viele dieser Aspekte unterbewusst wirken, ist eine ausgiebige Betrachtung unumgänglich.\\
Eine Schwierigkeit, die sich bei der Entwicklung ergibt sind die zwei unterschiedlichen Ziele. Einerseits sollten bestehende Design- und Bedienelemente  übernommen werden um die Bedienung intuitiv und übersichtlich zu gestalten, andererseits aber auch neue Ideen und Innovationen eingebracht werden, um sich von anderen Apps abzuheben und bleibenden Wiedererkennungswert aufzubauen.

\subsection{Oberflächen von StreamSwipe}
Die Smartphone-App lässt sich in mehrere Bereiche aufteilen, die sich in ihren Funktionen unterscheiden. Auf Basis der oben beschriebenen Grundlagen wurden diese Bereiche entworfen und werden in diesem Kapitel analysiert. Auch wenn manches davon als gewöhnlich oder naheliegend erscheint, so ist jedes Element mit Bedacht gewählt, erstellt und angepasst worden.
\subsubsection{Login-Screen}
\label{sec:loginscreen}
Bei erstmaliger Benutzung der App öffnet sich der Login-Screen. An diesem Punkt wird der erste Eindruck für den Benutzer gesetzt, wobei bei StreamSwipe ein schlichtes Design gewählt wurde. Man sieht helle Grautöne mit einem Akzentfarbton, welche sich durch alle Bildschirme der App ziehen werden. Abhängig davon, ob der User in den Systemeinstellungen den dunklen Modus gewählt hat, werden anstatt den hellen Grautönen, dunkle bis schwarze Farben dargestellt, siehe auch Abbildungen \ref{fig:homescreen_c} und \ref{fig:swipescreen_e}.\\
Auf dem Login-Screen (siehe Abbildung \ref{fig:login_a}) sind neben einer Überschrift mehrere beschriftete Textfelder und Buttons zu sehen, welche allesamt mit Semantiken versehen wurden, um durch einen Screenreader erkannt und identifiziert werden zu können. Die gewählte Anordnung wird universell bei Apps, Programmen und Webseiten benutzt, sodass die Felder auch ohne die eingetragenen Hinweistexte korrekt ausgefüllt werden könnten. Beim Antippen der Textfelder, öffnet sich die Standardtastatur des Betriebssystems. Sind alle Felder korrekt ausgefüllt, wird der User in die eigentliche App weitergeleitet, ansonsten wird durch individualisierte Fehlermeldung auf eventuelle Falscheingaben hingewiesen. Nach Erstellen eines neuen Accounts, durchläuft der User einen ähnlich aufgebauten Bildschirm (siehe Abbildung \ref{fig:login_b}) und wird danach aufgefordert weitere Informationen zur Profilvervollständigung einzugeben (siehe Abbildung \ref{fig:login_c} und \ref{fig:login_d}). Auch hierbei werden bekannte Bedienelemente wie Textfelder, Dropdownmenüs und Checkboxen verwendet, wie in der Abbildung \ref{fig:login_e} beispielhaft dargestellt ist. Falls der User in den Systemeinstellungen des Smartphones den Nachtmodus aktiviert hat, wird das Appdesign angepasst, siehe Abbildung \ref{fig:login_f}.


\begin{figure}[H]
	\begin{subfigure}{0.33\textwidth}
	\centering
	\includegraphics[scale=0.13]{Benutzeroberfläche/images/screenshot_login_1.png}
	\caption{}
	\label{fig:login_a}
	\end{subfigure}
	\begin{subfigure}{0.33\textwidth}
	\centering
	\includegraphics[scale=0.13]{Benutzeroberfläche/images/screenshot_login_2.png}
	\caption{}
	\label{fig:login_b}
	\end{subfigure}
	\begin{subfigure}{0.33\textwidth}
	\centering
	\includegraphics[scale=0.13]{Benutzeroberfläche/images/screenshot_login_3.png}
	\caption{}
	\label{fig:login_c}
	\end{subfigure}\\ \vspace{1cm}	
	
	\begin{subfigure}{0.33\textwidth}
	\centering
	\includegraphics[scale=0.13]{Benutzeroberfläche/images/screenshot_login_4.png}
	\caption{}
	\label{fig:login_d}
	\end{subfigure}
	\begin{subfigure}{0.33\textwidth}
	\centering
	\includegraphics[scale=0.13]{Benutzeroberfläche/images/screenshot_login_5.png}
	\caption{}
	\label{fig:login_e}
	\end{subfigure}
	\begin{subfigure}{0.33\textwidth}
	\centering
	\includegraphics[scale=0.13]{Benutzeroberfläche/images/screenshot_darkmode_3.png}
	\caption{}
	\label{fig:login_f}
	\end{subfigure}
\caption[Screenshots der Anmeldeseiten]{Die Anmeldeseiten von StreamSwipe und alle damit zusammenhängenden Screens. Man sieht (a) das Einloggen bei bestehendem Account, (b) das Erstellen eines Accounts, (c) und (d) das Formular für die benötigten Profildaten, (e) ein Texteingabefeld mit Autovervollständigung als Dropdownmenü und (f) das Farbschema der Anmeldeseiten im Darkmode am Beispiel des Login-Screens.}
\label{fig:login_alle}
\end{figure}

\subsubsection{Home-Screen}
\label{sec:homescreen}

Da davon ausgegangen wird, dass der User sich nicht nach jeder Nutzung ab- und wieder anmeldet, erscheint im alltäglichen Gebrauch der in Abbildung \ref{fig:homescreen_alle} dargestellte Bildschirm zuerst. Demnach bietet es sich an Ereignisse wie neue Matches und neue Nachrichten hier anzuzeigen. Diese werden wie Abbildungen \ref{fig:homescreen_a} und \ref{fig:homescreen_b} zeigen klar strukturiert in Abschnitte eingegliedert, welche mit Überschriften kenntlich gemacht sind. Die einzelnen Matches befinden sich mit allen dazugehörenden Funktionen und Informationen jeweils auf einer Karte. Durch diese Karten kann mit einer von anderen \mbox{Apps} bekannten horizontalen Swipemechanik navigiert werden. Weiterführende Funktionen wie das Starten eines Chats oder das Löschen des Matches werden durch  Antippen von allgemein verständlichen Icons ausgeführt. \\
Auch auf diesem Screen findet sich einerseits das bereits eingeführte Farbschema wieder und es werden andererseits ebenfalls Semantiken verwendet. Bei den neuen Nachrichten wird jeweils der Benutzername vorgelesen und bei den neuen Matches je nach ausgewähltem Bereich der Filmname, die Icons oder der Text dazwischen.\\
Am unteren Bildschirmrand ist eine sogenannte Bottom-Navigation-Bar zu sehen. Sie ermöglicht eine kompakte und anschauliche Navigation durch die relevanten Bildschirme. Außerdem zeigt sie an welcher Bildschirm aktuell ausgewählt ist, wobei diese Information wie in Abschnitt \ref{sec:bf-streamswipe} erarbeitet nicht ausschließlich auf einer Farbänderung basieren sollte und deshalb das ausgewählte Icon durch Hinzufügen von Text hervorgehoben wird. Liest der Screenreader die Semantik hiervon, gibt er die Bezeichnung des aktuellen Bildschirms sowie die Anzahl der weiteren Möglichkeiten an. \\
%TODO Motorische Herausforderungen!!! Anklicken des Filmposters aktiviert auch Chat?

\begin{figure}[H]
	\begin{subfigure}{0.33\textwidth}
	\centering
	\includegraphics[scale=0.13]{Benutzeroberfläche/images/screenshot_homescreen_1.png}
	\caption{}
	\label{fig:homescreen_a}
	\end{subfigure}
	\begin{subfigure}{0.33\textwidth}
	\centering
	\includegraphics[scale=0.13]{Benutzeroberfläche/images/screenshot_homescreen_2.png}
	\caption{}
	\label{fig:homescreen_b}
	\end{subfigure}
	\begin{subfigure}{0.33\textwidth}
	\centering
	\includegraphics[scale=0.13]{Benutzeroberfläche/images/screenshot_darkmode_1.png}
	\caption{}
	\label{fig:homescreen_c}
	\end{subfigure}
\caption[Screenshots des Home-Screens]{Der Home-Screen, der beim Öffnen der App zuerst gezeigt wird und Neuigkeiten wie neue Nachrichten und Matches zusammenfasst. Um den gesamten Inhalt dieser Seite sehen zu können, wird in (a) der obere Abschnitt und in (b) der untere Abschnitt gezeigt. Hat der User in den Systemeinstellungen den dunklen Modus aktiviert, so wird (c) der Home-Screen wie alle anderen Screens angepasst.}
\label{fig:homescreen_alle}
\end{figure}


\subsubsection{Swipe-Screen}
\label{sec:swipescreen}
Auf dem Swipe-Seiten (Abbildungen \ref{fig:swipescreen_alle}) findet die Bewertung der Filme statt. Durch das hier verwendete Matchingsystem mithilfe des Filmgeschmacks unterscheidet sich StreamSwipe von anderen Apps und erhält so einen innovativen, individuellen Charakter, womit diese Seite das Herzstück der App bildet.\\
Das zuvor eingeführte Farbschema bleibt auch hier erhalten, wie Abbildung \ref{fig:swipescreen_a} zeigt. Eine Überschrift im selben Stil wie bereits aus Abschnitt \ref{sec:homescreen} bekannt, verdeutlicht durch eine Frage nach welcher Motivation die Filmauswahl getroffen werden soll. Zentral im Bild ist eine Liste von Postern der zu beurteilenden Filme. Wie bereits durch die Dating-App Tinder verbreitet, werden die Antwortmöglichkeiten durch eine Swipe-Bewegung in eine Richtungen ausgewählt. In diesem Fall werden vier Entscheidungsmöglichkeiten auf vier Richtungen verteilt. Abhängig von der Position des Fingers auf dem Touchscreen bewegt sich das Filmposter innerhalb des Bildschirms, was den Effekt einer frei beweglichen Karte hervorruft. Um klarzustellen welche Swipe-Richtung für welche Entscheidung steht, verfärbt sich der jeweilige Indikator in der unteren Reihe bei Verschiebung des Filmposter. Beide Animationen sind in Abbildung \ref{fig:swipescreen_d} zu sehen. Die Indikatoren sind mit Icons versehen, zeigen aber durch Drücken welche Entscheidung sie repräsentieren und in welche Richtung der Nutzer dafür wischen  muss, wie Abbildung \ref{fig:swipescreen_c} am Beispiel des rechten Indikators zeigt. \\
Durch Antippen des Filmposters werden weitere Informationen zu dem jeweiligen Film dargestellt, wie in Abbildung \ref{fig:swipescreen_b} zu sehen. Gleichfalls wird durch ein einfaches Antippen wieder zurück  zu den Postern gewechselt. Eine Rotations-Animation verdeutlicht die Illusion der Karten.\\
Alle diese für die Bedienung der App grundlegenden Steuerungen verlangen keine feinmotorischen Eingaben und können problemlos von Personen mit motorischen Einschränkungen genutzt werden. Auch dieser Bildschirm ist vollkommen mit Semantiken ausgestattet. Anstelle des Filmposters wird der Name des Films ausgelesen und für die vier Indikatoren am unteren Rand werden jeweils deren Funktion und durch welche Swipe-Richtung sie erreicht werden vorgelesen. Sämtliche Textfelder können ebenfalls problemlos von einem Screenreader gelesen werden.



\begin{figure}[H]
	\begin{subfigure}{0.33\textwidth}
	\centering
	\includegraphics[scale=0.13]{Benutzeroberfläche/images/screenshot_swipescreen1.png}
	\caption{}
	\label{fig:swipescreen_a}
	\end{subfigure}
	\begin{subfigure}{0.33\textwidth}
	\centering
	\includegraphics[scale=0.13]{Benutzeroberfläche/images/screenshot_swipescreen2.png}
	\caption{}
	\label{fig:swipescreen_b}
	\end{subfigure}
	\begin{subfigure}{0.33\textwidth}
	\centering
	\includegraphics[scale=0.1742]{Benutzeroberfläche/images/screenshot_swipescreen3.png}
	\caption{}
	\label{fig:swipescreen_c}
	\end{subfigure}\\ \vspace{1cm}	
	
	\begin{subfigure}{0.33\textwidth}
	\centering
	\includegraphics[scale=0.13]{Benutzeroberfläche/images/screenshot_swipescreen4.png}
	\caption{}
	\label{fig:swipescreen_d}
	\end{subfigure}
	\begin{subfigure}{0.33\textwidth}
	\centering
	\includegraphics[scale=0.13]{Benutzeroberfläche/images/screenshot_darkmode_2.png}
	\caption{}
	\label{fig:swipescreen_e}
	\end{subfigure}
\caption[Screenshots der Swipe-Seiten]{Darstellungen und Funktionen der Swipe-Seiten mit (a) der Standarddarstellung, (b) weiteren Filminformationen, (c) einer Animation beim Drücken einer der Indikatoren und (d) der Swipe-Animation.  Hat der Nutzer in den Systemeinstellungen den dunklen Modus aktiviert, so wird (e) die Swipe-Seite wie alle anderen Seiten angepasst.}
\label{fig:swipescreen_alle}
\end{figure}

\subsubsection{Chat}
\label{sec:UI-Chat}
Die Chatseite ist in eine Liste aus aktiven Chats und eine Liste mit Chatanfragen aufgeteilt, siehe \ref{fig:chat_a} und \ref{fig:chat_b}. Um zwischen diesen beiden Listen zu wechseln werden Tabs eingesetzt, wie sie aus Windowsanwendungen bekannt sind. Zwischen diesen Tabs kann entweder gewechselt werden, indem ein anderes Tabfenster angetippt wird, oder der gesamte Bildschirm mit einer Geste zur Seite gewischt wird. Bei jedem Element der Chatliste ist der jeweilige Benutzername und die neueste Nachricht zu sehen, dazu wird falls vorhanden entweder ein Profilbild oder eine einfarbige Fläche mit dem Anfangsbuchstaben des Namens angezeigt. Chatanfragen können jeweils durch das Schieben nach links angenommen oder nach rechts ablehnt werden, wie in den Abbildungen \ref{fig:chat_c}, bzw. \ref{fig:chat_d} zu sehen ist. Diese Mechanik wird häufig in Email-Apps zum Löschen oder Verschieben der Mails benutzt.\\ 
Durch Antippen eines Matches, öffnet sich der Chatverlauf, welcher in Abbildung \ref{fig:chat_e} zu sehen ist. Die Anordnung der Nachrichten innerhalb des Chatverlaufs ist wie aus anderen Messenger bereits bekannt, aber in den Stilfarben von StreamSwipe. Am oberen Bildschirmrand wird der Profilname des Matches angezeigt und rechts davon befindet sich der Button zu dessen Profilseite, auf der genauere Details über diese Person zu finden sind. Die Profilseiten werden in Kapitel \ref{sec:benutzerprofil} genauer vorgestellt. \\
Dem Nutzer wird durch das ihm bereits vorgestellte Design und der ausschließlichen Nutzung von bekannter Mechanik ein vertrautes Umfeld geboten. Wie auf jeder Seite passt sich auch hier das Farbschema automatisch an, falls in den Systemeinstellungen des Smartphones das dunkle Design gewählt wurde, wie beispielsweise in Abbildung \ref{fig:chat_f} dargestellt. Neben der Benutzerfreundlichkeit wird auch die Barrierefreiheit beachtet, indem alle Elemente, die nicht bereits aus einem Text bestehen, mit Semantiken ausgestattet werden. Zusätzlich werden keine feinmotorischen Bewegungen zur Navigation durch die Bildschirme benötigt. Bis auf die Eingabe über die Tastatur kann alles über große Flächen oder Wischmechaniken bedient werden. Im Chatverlauf wir die Standardtastatur des Systems verwendet, mit der der Benutzer bereits vertraut ist. 


\begin{figure}[H]
	\begin{subfigure}{0.33\textwidth}
	\centering
	\includegraphics[scale=0.1742]{Benutzeroberfläche/images/screenshot_chat_1.png}
	\caption{}
	\label{fig:chat_a}
	\end{subfigure}
	\begin{subfigure}{0.33\textwidth}
	\centering
	\includegraphics[scale=0.1742]{Benutzeroberfläche/images/screenshot_chat_2.png}
	\caption{}
	\label{fig:chat_b}
	\end{subfigure}
	\begin{subfigure}{0.33\textwidth}
	\centering
	\includegraphics[scale=0.1742]{Benutzeroberfläche/images/screenshot_chat_3.png}
	\caption{}
	\label{fig:chat_c}
	\end{subfigure}\\ \vspace{1cm}	
	
	\begin{subfigure}{0.33\textwidth}
	\centering
	\includegraphics[scale=0.1741]{Benutzeroberfläche/images/screenshot_chat_4.png}
	\caption{}
	\label{fig:chat_d}
	\end{subfigure}
	\begin{subfigure}{0.33\textwidth}
	\centering
	\includegraphics[scale=0.13]{Benutzeroberfläche/images/screenshot_chat_5.png}
	\caption{}
	\label{fig:chat_e}
	\end{subfigure}
	\begin{subfigure}{0.33\textwidth}
	\centering
	\includegraphics[scale=0.13]{Benutzeroberfläche/images/screenshot_darkmode_4.png}
	\caption{}
	\label{fig:chat_f}
	\end{subfigure}
\caption[Screenshots der Chat-Seiten]{Darstellungen und Funktionen der Chat-Seiten mit (a) den aktiven Chats, (b) den Chats auf der Warteliste, (c) und (d) angenommene, bzw. abgelehnten Chats auf der Warteliste, sowie (e) einem Chatverlauf im hellen und (f) in dunklen Modus.}
\label{fig:chat_alle}
\end{figure}

\subsubsection{Benutzerprofil}
\label{sec:benutzerprofil}

Auf der Profilseite werden ein Profilbild, ein Hintergrundbild und für das Matching relevante persönliche Informationen dargestellt. Es gibt eine Version, die nur von anderen Nutzern sichtbar ist, mit denen ein Match stattgefunden hat, und eine Version, die über die Bottom-Navigation-Bar erreichbar werden kann. Die Letztere wird in Abbildung \ref{fig:profilseite_alle} dargestellt und unterscheidet sich von der Version für andere Nutzer darin, dass Profil- und Hintergrundbild bearbeitet werden können.\\
Das Farbschema und das Design wurden an die bisherigen Seiten angepasst. Um die Oberfläche simpel und selbsterklärend zu halten, wird jede dargestellte Information mit einem passenden Icon und einem Hinweis versehen (siehe Abbildung \ref{fig:profilseite_a}). Die Icons zum Bearbeiten der Bilder sind, wie auch in vielen anderen Apps, platziert und designt. Sie öffnen die systemeigene Bildergalerie des Smartphones um den Nutzer aus einem bekannten Umfeld Bilder auswählen lassen zu können.\\
Beim initialen Öffnen einer Profilseite sollen Namen, Profilbild und ein Hintergrundbild ins Auge springen. Sie stellen die ersten Informationen dar, die dem Betrachter wichtig sind, weshalb sie wie in Abbildung \ref{fig:profilseite_a} deutlich sichtbar ist  beim Öffnen mehr als die Hälfte des Bildschirms einnehmen. Anschließend wird der Fokus auf detailliertere Informationen gerichtet. Auf der Profilseite von StreamSwipe wird hierfür heruntergescrollt um den Block mit den Profildaten sehen zu können. Bei dieser Aktion blendet eine Animation das Profilbild aus und verschmälert das Hintergrundbild. Der Benutzername wird ebenfalls aus dem Fokus gezogen, bleibt aber wie in Abbildung \ref{fig:profilseite_b} zu sehen mit dem verbleibenden Hintergrundbildausschnitt erhalten. Dies hilft dem Betrachter unterbewusst bei dem Fokuswechsel und schafft ein modernes, responsives Feedback bei der User Experience.\\
Um das durchgängig schlichte Design der App zu erhalten ist der Zugang zu den Einstellungen ausschließlich auf der Profilseite zu finden. Hierfür ist im rechten oberen Bildschirmbereich das repräsentative Icon. Der hierdurch erreichbare Bildschirm (Abbildung \ref{fig:profilseite_c}) ist gleich aufgebaut wie die Informationeneingabe nachdem ein neuer Account erstellt wurde (Abbildungen \ref{fig:login_c} und \ref{fig:login_d}). Die dort angegebenen Informationen können hier wieder angepasst werden. %TODO Referenz auf account erstellen bild


\begin{figure}[tbt]
	\begin{subfigure}{0.33\textwidth}
	\centering
	\includegraphics[scale=0.13]{Benutzeroberfläche/images/screenshot_profilseite_1.png}
	\caption{}
	\label{fig:profilseite_a}
	\end{subfigure}
	\begin{subfigure}{0.33\textwidth}
	\centering
	\includegraphics[scale=0.13]{Benutzeroberfläche/images/screenshot_profilseite_2.png}
	\caption{}
	\label{fig:profilseite_b}
	\end{subfigure}
	\begin{subfigure}{0.33\textwidth}
	\centering
	\includegraphics[scale=0.13]{Benutzeroberfläche/images/screenshot_profilseite_3}
	\caption{}
	\label{fig:profilseite_c}
	\end{subfigure}
\caption[Screenshots der Profilseite]{Profilseite wie sie für den Nutzer selbst angezeigt wird (a) im normalen Zustand und (b) nach vollständigem Einklappen des Profilkopfes durch eine Animation während dem Herunterscrollen. Mit den von hier aus erreichbaren Einstellungen (c) können die anfänglich gegebenen Profilangaben abgepasst werden.}
\label{fig:profilseite_alle}
\end{figure}

\subsection{Filmliste}
\clearpage

\section[Probleme]{Probleme \hfill \normalfont \small{Autor-Name}}
\clearpage

\section[Fazit]{Fazit \hfill \normalfont \small{Autor-Name}}
\clearpage

\appendix
\section{Verfasser einzelner Abschnitte}
\newcolumntype{L}[1]{>{\raggedright\let\newline\\\arraybackslash\hspace{0pt}}m{#1}}
\newcolumntype{C}[1]{>{\centering\let\newline\\\arraybackslash\hspace{0pt}}m{#1}}

\begin{table}[H]
	\centering
	\begin{tabular}{L{1cm} L{6.5cm} C{7.5cm}}
		\toprule
		\multicolumn{2}{l}{\textbf{Kapitel / Abschnitt}}                               							& \textbf{Verfasser}\\ 
		\midrule
		\multicolumn{2}{l}{1. Einleitung}                                             							& Vincent Schreck\\ 
		 	& 1.1 Motivation                              														& Vincent Schreck\\
			& 1.2 Methode                                    													& Vincent Schreck\\ 
		\midrule
		\multicolumn{2}{l}{2. Theoretische Grundlagen}                        									& Leon Gieringer \& Robin Meckler\\
		 	& 2.1 Netzwerkprotokolle                                                  							& Robin Meckler\\
			& 2.2 JavaScript                                     												& Robin Meckler\\
			& 3.3 NodeJS                                                              							& Robin Meckler\\
			& 2.4 Representational State Transfer - Application Programming Interface							& Robin Meckler\\
			& 2.5 NoSQL-Datenbank                                                     							& Robin Meckler\\ 
		\midrule
			& 2.6 Firebase                                       												& Leon Gieringer\\
			& 2.7 Anwendungsentwicklung für mobile Endgeräte                          							& Leon Gieringer\\
			& 2.8 Frameworks zur mobile, plattformübergreifenden Entwicklung          							& Leon Gieringer\\
			& 2.9 Recommender Systems                                                 							& Leon Gieringer\\ 
		\midrule
		\multicolumn{2}{l}{3. Konzept}                                      									& Vincent Schreck\\
		\midrule
		\multicolumn{2}{l}{4. Auswahl geeigneter Technologie}                                      		 		& Leon Gieringer \& Robin Meckler\\
			& 4.1 Anwendungsframework																			& Leon Gieringer \\
			& 4.2 Server                                         												& Robin Meckler \\
			& 4.3 Datenbank                                                           							& Robin Meckler \\
			& 4.4 Kommunikationsschnittstelle                    												& Robin Meckler \\
			& 4.5 Film-Datenbank                    															& \\
		\midrule
		\multicolumn{2}{l}{5. Backend-Implementierung}                                              			& Robin Meckler\\
			& 5.1 Server																						& Robin Meckler\\
			& 5.2 Datenbank                                                           							& Robin Meckler\\
			& 5.3 Kommunikationsschnittstelle                    												& Robin Meckler\\
			& 5.4 Filmdatenbank				                    												& \\
		\midrule
		\multicolumn{2}{l}{6. Implementierung der Mobile App}                                               			& \\
			& 6.1 Swipe/Aussuchen/Voting                         												& \\
			& 6.2 Matches/Chat                                                        							& Leon Gieringer\\
			& 6.3 Filmvorschläge                                	 											& \\
			& 6.4 Gespeicherte Filme                                                  							& \\
			& 6.6 Barrierefreiheit			       																& Vincent Schreck\\ 
		\midrule
		\multicolumn{2}{l}{7. Benutzeroberfläche}                                                   			& Vincent Schreck\\
			& 7.1 Aspekte von Benutzeroberflächen                                    							& Vincent Schreck\\
			& 7.2 Oberflächen von StreamSwipe                                                             		& Vincent Schreck\\
		\midrule
		\multicolumn{2}{l}{8. Probleme}                                                            				& \\ 
		\midrule
		\multicolumn{2}{l}{9. Anwendbarkeit}                                  									& \\ 
		\midrule
		\multicolumn{2}{l}{10. Fazit}                                       									& \\
		\bottomrule
	\end{tabular}
\end{table}

\clearpage

\section*{Literaturverzeichnis}
\begin{itemize}
	\bibitem[1]{aggarwal2016} Aggarwal, C. C. (2016). Recommender systems (Vol. 1). Cham: Springer International Publishing.

	\bibitem[2]{fentaw2020} Fentaw, A. E. (2020). Cross platform mobile application development: a comparison study of React Native Vs Flutter.

	\bibitem[3]{reactnative2021} Facebook Inc. (2021) React Native documentation. [Online] Verfügbar: \url{https://reactnative.dev/docs/getting-started}, Zuletzt aufgerufen am: 13.04.2021

	\bibitem[4]{react2021} Facebook Inc. (2021) React documentation. [Online] Verfügbar: \url{https://reactjs.org/docs/getting-started.html}, Zuletzt aufgerufen am: 13.04.2021

	\bibitem[5]{flutter2021} Flutter (2021) Flutter architectural overview [Online] Verfügbar: \url{https://flutter.dev/docs/resources/architectural-overview}, Aufgerufen am: 04.03.2021

	\bibitem[6]{johnson1988} Johnson R. E. \&  Foote B. “Designing Reusable Classes.” Journal of ObjectOriented Programming 1, 2 (June/July 1988). Page 22-35.

	\bibitem[7]{majchrzak2015} Majchrzak TA, Ernsting J, Kuchen H (2015) Achieving business practicability of model-driven crossplatform apps. OJIS 2(2):3–14

	\bibitem[8]{cisco2020} Cisco (2020) Cisco Annual Internet Report (2018–2023) White Paper [Online] Verfügbar: \url{https://www.cisco.com/c/en/us/solutions/collateral/executive-perspectives/annual-internet-report/white-paper-c11-741490.html}

	\bibitem[9]{charland2011} Charland, A. \& Leroux, B. (2011). Mobile application development: Web vs. native. Communications of the ACM, 54(5):49–53.

	\bibitem[10]{lachgar2017} Lachgar, M., \& Abdelmounaim, A. (2017). Decision Framework for Mobile Development Methods. International Journal of Advanced Computer Science and Applications, 8.

	\bibitem[11]{bjorn-hansen2020} Biørn-Hansen, A., Rieger, C., Grønli, TM. et al. (2020) An empirical investigation of performance overhead in cross-platform mobile development frameworks. Empir Software Eng 25, 2997–3040. https://doi.org/10.1007/s10664-020-09827-6

	\bibitem[12]{stahl2006} Stahl T, Volter M (2006) Model-driven software development. Wiley, Chichester

	\bibitem[13]{firebase2021} Firebase Inc. (2021) Firebase documentation. [Online] Verfügbar: \url{https://firebase.google.com/docs}, Aufgerufen am: 25.04.2021

    \bibitem[14]{scheidungen} Meinungsforschungsinstituts Civey (2020):  \url{https://www.presseportal.de/pm/145489/4627304}, letzter Zugriff: 13. Mai 2021

    \bibitem[15]{serienkonsum} Splendid Research (2017): \url{https://www.springerprofessional.de/konsumforschung/marketingstrategie/konsumenten-auf-der-serien-welle/15146374}, letzter Zugriff: 14. Mai 2021

    \bibitem[16]{schwerbehindertenausweis} Statistisches Bundesamt (2020): \url{https://www.destatis.de/DE/Themen/Gesellschaft-Umwelt/Gesundheit/Behinderte-Menschen/Tabellen/schwerbehinderte-alter-geschlecht-quote.html;jsessionid=885260788D4FFC7F670576B72E5089F4.live741}, letzter Zugriff: 17. April 2021

    \bibitem[17]{behindertengleichstellungsgesetz} Behindertengleichstellungsgesetz (2002): \url{https://www.gesetze-im-internet.de/bgg/BGG.pdf}, letzter Zugriff: 19. April 2021

    \bibitem[18]{sehhilfen} Institut für Demoskopie Allensbach (2019): \textit{Untersuchung zum Sehbewusstsein der Deutschen},  \url{file:///C:/Users/Vincent/AppData/Local/Temp/ZVA_Brillenstudie_2019-1.pdf}, letzter Zugriff: 11. Mai 2021
\end{itemize}
\end{document}

\section*{Literaturverzeichnis}
\begin{itemize}
\bibitem[99]{muster} Mustermann, Max (2020): Methode und Nutzung der Literatur-Zitierweise, 2. Aufl., Boston: Harvard’s Eleven Publications.
\bibitem[99]{asdf} Autor (2008): Name des Buches, 22. Aufl., Berlin: Foxtrott.
\end{itemize}

\footnote{\url{https://de.wikipedia.org/wiki/Alufolie}} 


Abbildung \ref{fig:allg_kennlinie}

\begin{figure}[tbt]
\begin{center}
\includegraphics[scale=0.45]{Grafiken/allg_kennlinie.png}
\end{center}
\caption{Kennlinie einer Halbleiterdiode \protect \footnotemark}
\label{fig:allg_kennlinie}
\end{figure}
\footnotetext{FUSSNOTE}


Tabelle \ref{tab:cobalt}

\begin{table}[tbt]
\caption{•}
\begin{threeparttable}	%Scheme for footnotes in tables
\begin{center}
\begin{tabular}{c c c c}
\toprule
& keV & keV & keV \\
\midrule
a	& b	& c	& d \\
a	& b	& c	& d \\
a	& b	& c	& d \\
a	& b	& c	& d \\ %direkt hinter jeweiligen Wert /tnote{1}
\bottomrule
\end{tabular}
\end{center}
\begin{tablenotes}\footnotesize 
\item[1]{Quelle: http://www.thinksrs.com/downloads/PDFs/ApplicationNotes/IG1BAgasapp.pdf}
\end{tablenotes}
\end{threeparttable}
\label{tab:cobalt}
\end{table}