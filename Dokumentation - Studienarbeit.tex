\documentclass[11pt,a4paper]{article}
\usepackage[utf8]{inputenc}
\usepackage[german]{babel}
\usepackage{amsmath}
\usepackage{amsfonts}
\usepackage{hyperref}
\usepackage{booktabs}
\usepackage{setspace}
\usepackage{threeparttable}
\usepackage{amssymb}
\usepackage{graphicx}
\usepackage{fancyhdr}
\usepackage{icomma}
\usepackage{float}
\usepackage{pdfpages}
\usepackage{hyperref}
\usepackage[left=2.5cm,right=2.5cm,top=2cm,bottom=3.5cm]{geometry}
\title{DreamSwipe\\Tinder für Filme\vspace{10px}}
\author{Leon Gieringer, Robin Meckler,Vincent Schreck \\ \\ Studienarbeit \\ \\ \\}
\date{\today}

\begin{document}
\maketitle
\thispagestyle{empty}
\newpage
\pagenumbering{Roman}
\tableofcontents
\newpage
\pagenumbering{arabic}

\pagestyle{fancy}
\fancyhf{}
\setlength{\headheight}{35pt}
\lhead{DreamSwipe}
\rhead{Studienarbeit}
\cfoot{\thepage}
\newpage

\section{Einleitung}


\section{Motivation}

	
\section{Theoretische Grundlagen}

\subsection{Framework}
\subsection{Language}	
\subsection{IDE}		
\subsection{Database}		
\subsection{Firebase}		
		
\section{Konzept?}

		

\section{Funktionen/Komponenten}

\subsection{Swipe/Aussuchen/Voting}		
\subsection{Matches/Chat}		
\subsection{Film-/Serienvorschläge}		
\subsection{Gruppenorgien}		
\subsection{Gespeicherte Filme/Filmliste}		
\subsection{Zugänglichkeit/Behindertenfreundlichkeit}		


\section{Benutzeroberflächen}
\subsection{Home-Screen}
\subsection{Gruppen}		
\subsection{Chat}		
\subsection{Filmliste}

\section{CodeBeispiele}


\section{Probleme}


\section{Fazit}

\end{document}



\footnote{\url{https://de.wikipedia.org/wiki/Alufolie}} 


Abbildung \ref{fig:allg_kennlinie}

\begin{figure}[tbt]
\begin{center}
\includegraphics[scale=0.45]{Grafiken/allg_kennlinie.png}
\end{center}
\caption{Kennlinie einer Halbleiterdiode \protect \footnotemark}
\label{fig:allg_kennlinie}
\end{figure}
\footnotetext{FUSSNOTE}


Tabelle \ref{tab:cobalt}

\begin{table}[tbt]
\caption{•}
\begin{threeparttable}	%Scheme for footnotes in tables
\begin{center}
\begin{tabular}{c c c c}
\toprule
& keV & keV & keV \\
\midrule
a	& b	& c	& d \\
a	& b	& c	& d \\
a	& b	& c	& d \\
a	& b	& c	& d \\ %direkt hinter jeweiligen Wert /tnote{1}
\bottomrule
\end{tabular}
\end{center}
\begin{tablenotes}\footnotesize 
\item[1]{Quelle: http://www.thinksrs.com/downloads/PDFs/ApplicationNotes/IG1BAgasapp.pdf}
\end{tablenotes}
\end{threeparttable}
\label{tab:cobalt}
\end{table}