Die Vision einer Dating-App, die Matches auf Basis des Film- und Seriengeschmacks erstellt, konnte in dem vorgegebenen Zeitrahmen erfolgreich erstellt werden. Alle grundlegenden Funktionen wurden implementiert und das Backend bildet eine saubere Einheit mit dem Frontend wodurch ein  angenehmes Benutzererlebnis erzeugt wird. \\
Die daraus erwartete Revolutionierung des Onlinedatings kann jedoch noch nicht demonstriert werden, da die App noch nicht veröffentlicht wurde, beziehungsweise außer von den beteiligten Entwicklern noch kein Benutzerfeedback zur Auswertung vorliegt. Auch wenn die App alle notwendigen Funktionen bereits erfüllt, sollten wie in Kapitel \ref{sec:ausblick} erarbeitet vor der Veröffentlichung noch ein paar Features hinzugefügt werden. Neben den notwendigen Aktualisierungen wurden auch einige neue Ideen angesprochen, mit welchen die App weiter wachsen kann. Solche Implementierungen  können jedoch nicht als negativer Aspekt gewertet werden, da diese lediglich zum Fortschritt und zur Optimierung einer App beitragen. Dieser Wandel gehört zum natürlichen Entwicklungsprozess und ist auch bei populären und professionell erstellten Apps zu beobachten. Kritiken, neue Ideen und Verbesserungsvorschläge von Benutzern und Entwicklern lassen auf eine regelmäßige und aktive Benutzung einer App schließen.\\
Unter diesen Gesichtspunkten zeigt StreamSwipe ein gelungenes Konzept mit einem enormen Marktpotential. 
