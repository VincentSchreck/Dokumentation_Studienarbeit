MongoDB ist ein in C++ geschriebenes, dokumentenorientiertes NoSQL-Datenbanksystem, das im Jahre 2009 von den Entwicklern Horowitz und Merriman als Open-Source Datenbank veröffentlicht wurde und die am weitest-verbreiteste NoSQL-Datenbank. (Stand April 2021) [MongoDB1.7] Die Intention der Gründer war es, eine Datenbank mit höherer Skalierbarkeit, Flexibilät und Performance zu entwerfen, die auf auf einer einfachen Handhabung beruht. [MongoDB1.65]
\newline

Gründe der Popularität der Datenbank ist neben den oben erwähnten Eigenschaften die flexible Gestaltungsmöglichkeit der Datenstrukturen sowie die Unterstützung durch zahlreiche Programmiersprachen und Betriebssysteme.
\newline

Dem Konzept des CAP-Theorems folgend steht MongoDB für Konsistenz und Partitionstoleranz, dafür ordnet sich die Verfügbarkeit den anderen Eigenschaften unter.
\newline

\textbf{BSON}
\newline


\textbf{CRUD}
\newline



Wired Tiger?
Storage Engine
Replica Sets
Oplog
Sharding
\newline

\textbf{Technische Grundlagen}
/
Replica
Transaktionen
\newline

\textbf{Verwaltungswerkzeuge}
Mongo Shell
Treiber
Grafische Oberflächen
