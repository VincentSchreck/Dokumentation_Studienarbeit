Nachfolgend werden die Klassenmodelle innerhalb der mobilen Anwendung dargestellt.\\
\noindent
Die Filmdaten, die der Nutzer für das Swipen auf der Oberfläche erhält, werden aus dem Webserver bezogen. Der Informationsgehalt eines Films beschränkt sich dabei auf den eindeutigen Identifikator (ID) und den Titel des Films.
Im weiteren Schritt werden anhand der Film-ID's die weiteren benötigten Informationen aus dem TMDB-Server angefragt. \\

\noindent
\textbf{Movie Modell:}
Zum Konvertieren des JSON-Objekts aus dem Webserver und zum Speichern der Daten in die Umgebung der mobilen Anwendung wird die Klasse 'Movie' erstellt\footnote{Zum Konvertieren von JSON Objekten in Dart-Code wurde die Online-Konvertierung \url{https://app.quicktype.io/} genutzt.}. 


\begin{lstlisting}[caption=Movie Modell - JSON Konvertierung, label=lst:moviemodel]
Movie movieFromJson(String str) => Movie.fromJson(json.decode(str));
String movieToJson(Movie data) => json.encode(data.toJson());

class Movie {
  Movie({
    this.original_title,
    this.id
  });

  String original_title;
  String id;

  factory Movie.fromJson(Map<String, dynamic> json) => Movie(
    original\_title: json["original_title"],
    id: json["id"],
  );

  Map<String, dynamic> toJson() => {
    "original\_title": original_title,
    "id": id,
  };
}
\end{lstlisting}

\noindent
\textbf{TMDB-Movie Modell:}
Die Klasse 'TMDBMovie' ist eine Zusammenstellung mehrerer Daten. Sie enthält jeweils ein Objekt der folgenden aufgelisteten Klassen:

\begin{itemize}
\item \textbf{TmdbMovieDetails:} Enthält filmbezogene Informationen wie Titel, Posterpfad, Rating..
\item \textbf{TmdbMovieCredits:} Enthält Informationen zu den mitwirkenden Personen wie Schauspieler und Regisseure.
\item \textbf{TmdbMovieProvider:} Enthält Informationen bezüglich anbietenden Streaming-Providern.
\item \textbf{TmdbMovieTranslations:} Enthält Informationen zu Sprachen zum Film.
\end{itemize}

\noindent
\textbf{Match Modell:}
Nutzer werden ihre Matches vom Webserver abfragen. Hierbei erhalten erhalten sie zwei Listen 'supermatches' und 'normalmatches' sowie die Informationen 'newChanges', die Auskunft darüber gibt, ob es seit der letzten Anfrage zu Änderungen kam. Die Listen sind jeweils mit den entsprechenden Matchinformationen gefüllt. Der Informationsgehalt begrenzt sich auf die 'uid' der Nutzer und die 'id' des Films/der Filme, über das beide Nutzer gematcht sind. Für die Konvertierung und Speicherung dieser Informationen wurde die Klasse 'PartInformationMatches' erstellt. \\
Für die Darstellung eines Matches werden jedoch weitere Informationen wie der Titel oder der Posterpfad des Filmes und dem Namen des gematchten Nutzers. Die Klasse 'FullInformationMatches' enthält Eigenschaft für die Speicherung dieser zusätzlichen Daten.
