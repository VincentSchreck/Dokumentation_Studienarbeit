In den nächsten Unterkapiteln soll ein zunächst ein historischer Überblick über die Programmiersprache JavaScript gegeben werden. Im Anschluss wird auf die Bedeutung und Nutzung von JavaScript eingegangen. 

\subsubsection{Historie TODO }
Ihren Ursprung findet JavaScript im Jahr 1995, als Brendan Eich, ein damaliger Ingenieur des US-amerikanischen Software-Unternehmens „Netscape Communications Corporation“, innerhalb von zehn Tagen die Sprache für den Browser „Netscape Navigator“ entwickelt hat. [1] Das Ziel dabei war es, eine Skriptsprache zu entwickeln, die es Entwicklern möglich machen sollte, auf ihren Webseiten Skripte umzusetzen. Zunächst noch unter dem Namen Mocha und LiveScript änderte sich der Name aufgrund der Kooperation von Netscape und Sun, der Firma hinter der Programmiersprache Java, und der damaligen Popularität von Java zu JavaScript. [1.05] 
Netscape’s Veröffentlichung des Netscape Navigator 2.0, der erste Browser der JavaScript unterstütze, brachte Microsoft dazu, Netscape als ernstzunehmenden Konkurrenten zu sehen. 
Microsoft antwortete im August 1995 mit der Veröffentlichung des ersten Internet Explorer zusammen mit der Skriptsprache JScript, die einen Dialekt der Sprache JavaScript darstellt. Dies ist ferner als Beginn der „Browserkriege“ bekannt. [1.06]
 Im Jahre 1997 reichte Netscape JavaScript an die European Computer Manufacturers Association, einer privaten, internationalen Normungsorganisation zur Normung von Informations- und Kommunikationssystemen und Unterhaltungselektronik (kurz ECMA[ABK]) ein. Das Ziel war es, von der ECMA einen einheitlichen Standard für die Sprache schaffen zu lassen, die fortan weiterentwickelt und von weiteren Browserherstellern genutzt werden soll. Das resultierende Standard nennt sich ECMAScript, wobei JavaScript die bisher bekannteste Implementierung dieses Standards ist. [1.07] Andere Implementierungen sind zum Beispiel ActionScript von MacroMedia, JScript von Microsoft und ExtendScript von Adobe.
Jährlich wird der Standard seit Juni 2015 erweitert. ECMAScript Version 11 beziehungsweise ECMAScript 2020 bildet zum Zeitraum dieser Dokumentation [??] den aktuellen Standard. [1.08] 
Im Juni 2021 soll ECMAScript 2021 veröffentlicht werden.  [1.09]

\subsubsection{Wesentliche Programmiereigenschaften TODO}
„JavaScript is Not Java“ [1.091 ??]. Die Programmiersprache JavaScript wird aufgrund ihrer Namensgebung oft in falsche Zusammenhänge zu Java gebracht. Das häufigste Missverständnis sei, JavaScript wäre eine vereinfachte Version von Java. [1.091]
JavaScript ist eine interpretierte Programmiersprache mit objektorientierten Umsetzungs-möglichkeiten. Interpretation ist in diesem Zusammenhang so zu verstehen, dass der Quellcode zur Laufzeit eines Programms gelesen, übersetzt und ausgeführt wird.
Syntaktisch ähnelt JavaScript Programmiersprachen wie C, C++ und Java durch gleiche Umsetzung der Programmierkonstrukte wie den Bedingungen, Schleifen oder den booleschen Operatoren. [1.1] Wesentliche Unterschiede sind dagegen, dass JavaScript zum einen eine schwach-typisierte Sprache ist. Durch die schwache Typisierung haben Variablen keinen festen Dateityp und können diesen dynamisch zur Laufzeit ändern. Des Weiteren findet bei JavaScript die Objektorientierung prototypenbasiert statt. Diese Form der Programmierung wird auch klassenlose Objektorientierung bezeichnet. Anders als bei der klassenbasierten Programmierung, bei der Objekte aus vordefinierten Klassen instanziiert werden, werden hier Objekte durch Klonen bereits existierender Objekte erzeugt. Die Objekte, die geklont werden, sind dabei als Prototyp-Objekte zu verstehen. Beim Klonen werden alle Attribute und Methoden des Prototyp-Objekts in das neue Objekt übernommen und können dort überschrieben sowie erweitert werden. Objekte in JavaScript sind eher als Zuordnungslisten, ähnlich wie assoziative Arrays oder Hash-Tabellen, anzusehen, da bei der Eigenschaftszuweisung lediglich ein Mapping eines Namens zu seiner zugehörigen Eigenschaft stattfindet. Ein weiterer Unterschied zu den anderen Programmiersprachen ist, dass alle Funktionen und Variablen außer der primären Datentypen Boolean, Zahl und Zeichenfolge, als Objekte verstanden werden können.

\subsubsection{Anwendungsgebiete TODO}
Ursprünglich fand JavaScript seinen Einsatz hauptsächlich darin, dynamische Webseiten im Web-browser anzuzeigen. Die Verarbeitung erfolgte dabei meist clientseitig durch den Webbrowser (dem sogenannten Frontend). [1.3] 
Heutzutage findet sich die Sprache dagegen in wesentlich größeren Einsatzgebieten wieder. 
Bis vor einigen Jahren war die Serverseite anderen Programmiersprachen wie Java oder PHP vorbehalten. Die Veröffentlichung von Node.js, einer plattformübergreifenden Laufzeitumgebung, die JavaScript außerhalb eines Webbrowsers ausführen kann, führte zu einer immer größeren Verbreitung von serverseitigen Anwendungen (dem Backend), die auf JavaScript basieren. Auf Node.js wird ausführlicher im nächsten Kapitel eingegangen. 
Ferner findet JavaScript heutzutage aber auch seinen Einsatz in mobilen Anwendungen, Desktopanwendungen, Spielen oder 3D-Anwendungen.

