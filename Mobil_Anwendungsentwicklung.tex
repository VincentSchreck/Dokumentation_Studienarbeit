Mobile Geräte sind heutzutage ein sehr großer Teil unseres Tagesablaufs. Durchschnittlich verbringen wir 3:54 Stunden pro Tag an mobilen Geräten (hier bezogen aus Bürger der USA). Die meiste Zeit hiervon wird in Apps (ca. 90\%). \footnote{\url{https://www.emarketer.com/content/us-time-spent-with-mobile-2019}, zuletzt aufgerufen: 26.02.2021} 
Laut Cisco wird dieser Markt sich jedoch nicht nur auf Industrieländer beruhen, sondern bis 2023 sollen weltweit 71\% der Bevölkerung mobile Konnektivität haben. \cite{cisco2020}
Diese Entwicklung forcierte viele Firmen immer mehr ihre Anwendungen auch \textit{mobile ready} zu gestalten. Dies kann man bspw. deutlich bei der Anpassung vieler Webseiten an Mobile Seiten- und Größenverhältnisse oder auch dem Anbieten von \textit{Apps}, welche bereits für Desktop o.ä. verfügbar waren, erkennen. \\

Daher ist es für die Wirtschaft und Entwicklung gleichermaßen wichtig sich ständig weiterzuentwickeln und sich nicht auf (Kosten-) ineffiziente Entwicklungsprozesse auszuruhen. Dabei bieten jährliche, wenn nicht sogar halbjährliche Design- und Performanceänderungen von den Geräten selbst oder der Betriebssysteme Herausforderungen an die mobilen Anwendungen und gleichzeitig an deren Programmierumgebung. Trotz einer riesigen Auswahl an \textit{Apps} lassen sich diese allgemein in drei Kategorien eingliedern: Plattformspezifische Native Anwendungen, Adaptive Webanwendungen und Plattformübergreifende Native Anwendungen.

\subsubsection{Plattformspezifische native Anwendungen}
Plattformspezifische oder auch native Anwendungen sind Programme, welche auf eine gewisse Plattform abzielen und in einer der davon unterstützen Programmiersprachen geschrieben wurden. Da diese Art der (mobilen) Anwendung mit plattformspezifischen \textit{Software Development Kits (SDK)} und \textit{Frameworks} entwickelt wird, ist diese Anwendung an eine Plattform gebunden. \\
Dies bringt zum einen natürlich Vorteile wie allgemein best mögliche Performance auf der jeweiligen Plattform und direkt vom Hersteller unterstützte Entwicklungsumgebungen/SKDs.
Zudem lassen sich plattformspezifische Fähigkeiten oder Einstellungen nutzen - beispielsweise mehrere Kameras oder \textit{Global Positioning System (GPS)}.

Gleichzeitig beschränkt man sich aber logischerweise auf eine Plattform und deckt mit einer Anwendung nur einen Teil des gesamten Marktes. Dies bringt im Vergleich zu den anderen Möglichkeiten einen deutlich erhöhten Entwicklungs- und Wartungsaufwand mit sich, da für andere Plattformen Programmcode nicht übernommen werden kann. Zusätzlich benötigen Entwickler spezifische Kompetenzen für beide Plattform und Entwicklungsumgebungen. \\

Zwei der am weitesten verbreiteten Plattformen sind Android von Google und iOS von Apple. Anwendungen für Android können in Kotlin oder Java als Programmiersprache beispielsweise in dem \textit{integrated development environment (IDE)} von Google Android Studio entwickelt werden. Für iOS wird hingegen mit Objective-C und Swift als Programmiersprache primär in der IDE XCode entwickelt.

Beide bieten jeweils Plattform eigene Services an, beispielsweise das direkte Veröffentlichen in den jeweiligen Appstore \cite{fentaw2020}
\subsubsection{Adaptive Webanwendungen}
test
\subsubsection{Plattformübergreifende Anwendungen}
test

