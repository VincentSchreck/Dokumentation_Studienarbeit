In der Planungsphase eines Projektes werden oft große Meilensteine gesetzt. Sobald das Thema bekannt ist, werden durch Brainstorming und andere Methoden der Ideenfindung in kurzer Zeit viele Ziele gesteckt und ein in der Theorie fertiges Projekt ausgearbeitet. Eine Begrenzung wird nur durch die Fantasie der Beteiligten gezogen, die jedoch später in der Umsetzung schnell erreicht wird. Aber auch bei realistischen Zielen wird oft eine sehr spezielle Vorstellung angestrebt, die mit den gegebenen oder gewählten Mittel nur bedingt umsetzbar ist. Neben den Arbeitsmitteln zählen aber auch Zeit und Geld zu den einschränkenden Ressourcen.\\

\noindent
Wie unzählige Projekte vor uns, konnten wir ebenfalls nicht alle zu Beginn gesteckten Ziele zu unserer vollsten Zufriedenheit abschließen. Auch wenn alle grundlegend wichtigen Kriterien erfüllt sind und funktionieren, gibt es auch nicht erreichte Ziele. Diese bestehen aus erkennbaren und nicht erkennbaren Schwächen in der App. Ersteres sind Unsauberkeiten, die der User bei der Benutzung in manchen Situationen bemerken kann. Hierzu zählen beispielsweise Designfehler, die bei der Programmierung nicht erkannt wurden. Nicht erkennbare Schwächen sind Ideen und Ziele, die im vorgegebenen Rahmen nicht mehr eingebaut werden konnten. Diese fallen nur dem Programmierer auf. In unserem Fall ist das Fehlen mancher Funktonen hauptsächlich auf Zeitmangel zurückzuführen. Bei einem zeitlich begrenzten Bearbeitungsrahmen, während dem noch einige andere Projekte, Vorlesungen, Prüfungen und eine Arbeitsstelle in Vollzeit belegt werden, können nicht alle beabsichtigten Features eingebaut werden. Alle geplanten Erweiterungen wurden jedoch bereits durchdacht und werden nach Abgabe der schriftlichen Arbeit implementiert, aber hierzu mehr in Kapitel \ref{sec:fazit}.\\

\noindent
Da StreamSwipe in Deutschland entwickelt und programmiert wurde, liegt es nahe Deutsch als In-App-Sprache zu verwenden. Außer den auf den Oberflächen angezeigten Texten müssen auch Fehlermeldungen, Semantiken und Autovervollständigungen angepasst werden. Wie auf den Screenshots im ganzen Kapitel \ref{sec:UI-alle} zu sehen ist, wird dies in der App konsequent durchgesetzt. Die Filmdatenbank, aus der die dargestellten Informationen erhalten werden, liefert diese jedoch nur auf Englisch, was an der Handlungsangabe in Abbildung  \ref{fig:swipescreen_b} zu sehen ist. Alle Datenbanken, die in Kapitel  \ref{sec:filmdatenbank} betrachtet werden, bieten ihre Informationen nur in einer Sprache an. Eine Lösung dieses Problems wird in Kapitel \ref{sec:ausblick} beschrieben.\\

\noindent
Bei einer sauberen Implementierung werden die Elemente auf den Bildschirmen in Abhängigkeit der Gesamtgröße des Screens angeordnet, sodass sie sich auf unterschiedlichen Geräten entsprechend anordnen und ihre Größe anpassen können. Man spricht hierbei von responsivem Design, auf welches bei StreamSwipe ebenfalls Wert gelegt wurde. Je nach Verhältnis von Breite und Höhe der Bildschirmgrößen unterschiedlicher Geräte können so in Grenzfällen jedoch ungewollte Verzerrungen auftreten. Das Aussehen einer Benutzeroberfläche kann außerdem auch von dem verwendeten Betriebssystem abhängen, da unterschiedliche Generationen unterschiedliche Standards verwenden. Durch die begrenzte Entwicklungszeit  und den limitierten Zugang zu Ressourcen konnte die App nur auf einer kleinen Anzahl von  Endgeräten getestet und angepasst werden. Es kann somit leider keine Garantie für eine saubere Darstellung auf älteren Geräten gegeben werden. Längere Testphasen könnten dieses Problem minimieren, jedoch ist der Markt von Smartphones mittlerweile so unübersichtlich groß, dass es nahezu unmöglich ist jede Variation zu testen.\\

\noindent
Das wahrscheinlich größte Manko ist das Ausbleiben der Veröffentlichung der App am Abgabetermin der schriftlichen Arbeit. Ursprünglich geplant war eine Veröffentlichung in Google Play für Androidgeräte und im App Store für iPhones, weshalb wie in Kapitel \ref{sec:flutter} beschrieben Flutter als Framework verwendet wurde. Bei einer Veröffentlichung in Google Play wird die App vor Release einer umfangreichen Überprüfung unterzogen, was voraussichtlich eine Woche und in manchen Fällen auch länger dauern kann \cite{playstore_release}. Außerdem muss das vom Webserver benutzte Sicherheitszertifikat zur Kommunikation über HTTPS von einer offiziellen Zertifizierungsstelle signiert werden. Wie bereits in Kapitel \ref{sec:SichereKommunikation} angesprochen, benutzen wir ein selbstsigniertes Zertifikat. Zu den Kosten dieser Zertifizierung wird noch eine Mitgliedschaft als Google Play Developer von einmalig \$\,25 benötigt \cite{kostenPlayStoreDeveloper}. Für eine  Veröffentlichung im App Store ist eine Mitgliedschaft im Apple Developer Programm notwendig, die jährlich \$\,99 kostet \cite{appstore_release}. Diese Investitionen werden erst sinnvoll, wenn ein Monetarisierungsplan der App ausgearbeitet wurde.. \\

\noindent
Wie in Kapitel \ref{sec:komponenten} bereits erwähnt, werden die Verteilung der Film-IDs und die Auswertung der Filmbewertungen auf einem Server ausgeführt. Dieser Server besteht zur Zeit aus einem Raspberry Pi. Für den Gebrauch während der Entwicklungsphase ist die somit erreichte Rechenkapazität völlig ausreichend, da nur maximal drei Personen gleichzeitig darauf zugegriffen haben. Sollte die App jedoch veröffentlicht werden, werden die Nutzerzahlen unvorhersehbar ansteigen, sodass nicht vorausgesagt werden kann, ab wann der Server in diesem Aufbau überlastet sein wird. Um den Benutzern eine positive User Experience bieten zu können und eventuelle Hardwareschäden an dem Raspberry Pi zu vermeiden, muss mit der Veröffentlichung der App gewartet werden bis eine leistungsstärkere Lösung gefunden wurde.