Auf der Webseite Youtube allein werden minütlich mehr als 500 Stunden Videomaterial hochgeladen. (\url{https://blog.youtube/press/}, 10.02.2021)
Um bei einer solch unvorstellbaren Menge an Daten (allein auf einer Webseite) den Überblick als Endnutzer behalten zu können, ist ein personalisierten Filtersystems unausweichlich.

\noindent
Solche Filtersysteme, auch Recommendation System genannt, nutzt bisher gesammelte Daten um Nutzern potentiell interessante Objekte jeweils individuell vorzuschlagen.
Ein sogenannter \textit{Candidate Generator} ist hierbei ein Recommendation System, welches die Menge $M$ als Eingabe erhält und für jeden Nutzer eine Menge $N$ ausgibt. Hierbei umfasst $M$ alle Objekte und gleichzeitig gilt $N \subset M$. 

\noindent
Die Bestimmung einer solchen Menge $N$ beruht grundlegend auf zwei Informationsarten. Erstens die sogenannten Nutzer-Objekt Interaktionen, also beispielsweise Bewertungen oder auch Verhaltensmuster; Und zweitens die Attributwerte von jeweils Nutzer oder Item, also beispielsweise Vorlieben von Nutzern oder Eigenschaften von Items \cite{aggarwal2016}.
Systeme, welche zum Bewerten ersteres benutzen, werden \textit{collaborative filtering} Modelle genannt. Andere, welche zweiteres verwenden, werden \textit{content-based filtering} Modelle genannt. Wichtig hierbei ist jedoch, dass \textit{content-based filtering} Modelle ebenfalls Nutzer-Objekt Interaktionen (v.a. Bewertungen) verwenden können, jedoch bezieht sich dieses Modell nur auf einzelne Nutzer - \textit{collaborative filtering} basiert auf Verhaltensmustern von allen Nutzern bzw. allen Objekten.

\noindent
Ein solches Recommendation System kann im einfachsten Fall wie in \ref{Recommendation Matrix} als Matrix dargestellt werden.

\begin{table}[tbt]
	\centering
	\caption{Nutzer-Item Matrix mit Bewertungen. Jede Zelle $r_{u;i}$ steht hierbei für die Bewertung des Nutzers $u$ an der Stelle $i$}
	\label{Recommendation Matrix}
	\begin{tabular}{lcllllll}
		& \multicolumn{7}{c}{Items}                                                                                                                                                        \\
		& \multicolumn{1}{l}{}     & \multicolumn{1}{c}{1}  & \multicolumn{1}{c}{2}  & \multicolumn{1}{c}{...} & \multicolumn{1}{c}{i}  & \multicolumn{1}{c}{...} & \multicolumn{1}{c}{m}  \\ \cline{3-8} 
		& \multicolumn{1}{c|}{1}   & \multicolumn{1}{l|}{2} & \multicolumn{1}{l|}{}  & \multicolumn{1}{l|}{1}  & \multicolumn{1}{l|}{}  & \multicolumn{1}{l|}{}   & \multicolumn{1}{l|}{3} \\ \cline{3-8} 
		Users & \multicolumn{1}{c|}{2}   & \multicolumn{1}{l|}{4} & \multicolumn{1}{l|}{}  & \multicolumn{1}{l|}{}   & \multicolumn{1}{l|}{5} & \multicolumn{1}{l|}{}   & \multicolumn{1}{l|}{}  \\ \cline{3-8} 
		& \multicolumn{1}{c|}{...} & \multicolumn{1}{l|}{}  & \multicolumn{1}{l|}{}  & \multicolumn{1}{l|}{1}  & \multicolumn{1}{l|}{}  & \multicolumn{1}{l|}{}   & \multicolumn{1}{l|}{4} \\ \cline{3-8} 
		& \multicolumn{1}{c|}{u}   & \multicolumn{1}{l|}{}  & \multicolumn{1}{l|}{4} & \multicolumn{1}{l|}{}   & \multicolumn{1}{l|}{5} & \multicolumn{1}{l|}{}   & \multicolumn{1}{l|}{1} \\ \cline{3-8} 
		& \multicolumn{1}{l|}{}    & \multicolumn{1}{l|}{2} & \multicolumn{1}{l|}{}  & \multicolumn{1}{l|}{}   & \multicolumn{1}{l|}{}  & \multicolumn{1}{l|}{3}  & \multicolumn{1}{l|}{}  \\ \cline{3-8} 
		& \multicolumn{1}{l|}{n}   & \multicolumn{1}{l|}{}  & \multicolumn{1}{l|}{4} & \multicolumn{1}{l|}{}   & \multicolumn{1}{l|}{3} & \multicolumn{1}{l|}{}   & \multicolumn{1}{l|}{}  \\ \cline{3-8} 
	\end{tabular}
\end{table}

\subsubsection{Nutzerinformation}
Damit ein \textit{Recommender System} einem Nutzer Vorschläge bereitstellen kann, benötigt es Nutzerinformationen. Das Design des jeweiligen Systems hängt auch, wie oben beschrieben, von der Art der Information und von der Art der Beschaffung dieser ab.

\paragraph{Explizite Nutzerinformation}
Bei der expliziten Methode muss der Nutzer individuelle Informationen aktiv über sich preisgeben. Dies kann über konkrete Fragestellungen zu beispielsweise Geburtsdatum, Geschlecht oder Interessen geschehen. Diese Art der Information beschreiben einen Nutzer konkret. 

\noindent
Eine andere Art der Information sind Bewertungen von Objekten. Diese lassen sich beispielsweise Intervall basiert darstellen. Hierbei werden geordnete Zahlen in einem Intervall als Indikator genutzt, ob ein Objekt gut oder schlecht war - zum Beispiel eine Bewertung eines Produktes von 0 bis 5 Sternen bei Amazon. Diese Information beschreiben die Vorlieben eines Nutzers konkret.

\noindent
Je größer diese Skala ist, desto differenzierter ist auch das Meinungsbild, da jeder Nutzer sich genau ausdrücken kann. Jedoch desto komplizierter und unübersichtlich wird auch das Bewertungsverfahren an sich, da man einen zu großen Entscheidungsraum für den Nutzer darbietet.

\paragraph{Implizite Nutzerinformation}
Um implizit Nutzerinformationen zu erfassen, muss ein System die Verhaltensmuster seiner Kunden als Daten abspeichern. Beispielsweise könnte das System von YouTube erfassen, ob Videos frühzeitig abgebrochen oder ganz angeschaut werden. Anklicken von Webseiten und die darauf verbrachte Zeit könnte ebenfalls als Bewertung gespeichert und zur Generierung von Vorschlägen genutzt werden.

\subsubsection{Content-based filtering}
Unter \textit{content-based filtering} versteht man das Betrachten von Ähnlichkeiten zwischen Objekten anhand von Schlüsselwörtern (Eigenschaften) und daraus dann das Vorhersagen der Nutzer-Objekt Kombination für ein bestimmtes Objekt. 
Nimmt man an, Film 1 und Film 2 haben ähnliche Eigenschaften (gleiches Genre, gleiche Schauspieler, ...) und Nutzer A mag Film 1, so wird das System Film 2 vorschlagen.

\noindent
Das System ist also unabhängig von anderen Nutzerdaten, da die Vorschläge nur auf Präferenzen eines einzelnen Nutzers basieren. Dies bietet im Hinblick auf eine App auch gute Skalierungs"-möglich"-keiten. Zudem kann auf Nischen-Präferenzen gut eingegangen werden, da nicht mit anderen Nutzerdaten verglichen wird, sondern nur ein Nutzer für sich betrachtet wird.

\noindent
Gleichzeitig schlagen \textit{content-based filtering} Systeme aber eher offensichtliche Objekte vor, da Nutzer oft unzureichend genaue "Beschreibungen", also Vorlieben mit sich bringen. Dadurch, dass nur basierend auf Schlüsselwörter neue Objekte vorgeschlagen und andere Nutzerwertungen nicht miteinbezogen werden, sind die Vorschläge sehr wahrscheinlich oftmals ähnlich bis gleich - man "verfängt" sich quasi in eine Richtung \cite{aggarwal2016}.

\subsubsection{Collaborative Filtering}
Unter \textit{collaborative filtering} versteht man das Betrachten von Ähnlichkeiten im Verhalten von Nutzern anhand von Bewertungen und Prä"-fer"-enzen, bzw. anhand der Ähn"-lich"-keiten von Objekten.

\noindent
Generell wird bei \cite{aggarwal2016} in zwei Typen unterschieden:

\begin{enumerate}		
	\item \textit{Memory-based Methoden}: Es wird, wie oben beschrieben, aus gesammelten Daten Ähn"-lich"-keit herausgearbeitet und Nutzer-Objekt Kombinationen durch eben diese vorhergesagt. Daher wird dieser Typ auch \textit{neighborhood-based collaborative filtering} genannt. Man unterscheidet weiter in:
	\begin{enumerate}
		\item \textit{User-based}: Ausgehend von einem Nutzer A werden andere Nutzer mit ähnlichen Nutzer-Objekt Kombinationen gesucht, um Vorhersagen für Bewertungen von A zu treffen. Ähnlichkeitsbeziehungen werden also über die Reihen der Bewertungsmatrix berechnet.
		\item \textit{Item-based}: Hierbei werden ähnliche Objekte gesucht und diese genutzt um die Bewertung eines Nutzers für ein Objekt vorherzusagen. Es werden somit Spalten für die Berechnung der Ähnlichkeitsbeziehungen verwendet.
	\end{enumerate}
	\item \textit{Model-based Methoden}: Machine Learning und Data Mining Methoden werden verwendet um Vorhersagen über Nutzer-Objekt Kombinationen zu treffen. Hierbei sind auch gute Vorhersagen bei niedriger Bewertungsdichte in der Matrix möglich.
\end{enumerate}

\noindent
Vereinfacht gesagt: Wenn Nutzer A ähnliche Bewertungen verteilt wie Nutzer B, und B den Film 1 positiv bewertet hat, wird das System Film 1 auch Nutzer A vorschlagen. Das selbe gilt auch umgekehrt (\textit{Item-based}).

\noindent
Diese Art leidet sehr unter dem \textit{sparsity} Problem, also dass die Nutzer zu wenige Bewertungen von Objekten ausüben. Daher sind Vorhersagen über Ähnlichkeit von Nutzern aufgrund unzureichender Datensätze nicht sinnvoll möglich. Dieses Problem wird \textit{Cold-Start Problem} genannt.

\subsubsection{Ähnlichkeit von Objekten und Nutzern}
Sowohl bei \textit{collaborative filtering}, als auch bei \textit{content-based filtering} wird jedes Objekt und jeder Nutzer als ein Vektor im Vektorraum-Modell $E = \mathbb{R}^d$ (englisch \textit{embedding space}) erfasst. Sind Objekte beispielsweise ähnlich, haben sie eine geringe Distanz voneinander. 

\noindent
Ähnlichkeitsfunktionen sind Funktionen $s : E \times E  \rightarrow \mathbb{R}$ welche aus zwei Vektoren beispielsweise von einem Objekt $q \in E$ und einem Nutzer $x \in E$ ein Skalar berechnen, welches die Ähnlichkeit dieser zwei beschreibt $s(q,x)$.

\noindent
Hierfür werden mindestens eine der folgenden Funktionen verwendet:
\begin{itemize}
	\item Cosinus-Funktion
	\item Skalarprodukt
	\item Euklidischer Abstand
\end{itemize} 

\paragraph{Cosinus-Funktion}
Hier wird einfach der Winkel zwischen beiden Vektoren berechnet: $s(q,x) = \cos(q,x)$

\paragraph{Skalarprodukt}
Je größer das Skalarprodukt, desto ähnlicher sind sich die Vektoren. $s(q,x) = q \circ x = \sum_{i=1}^{d}q_i x_i$ 

\paragraph{Euklidischer Abstand}
$s(q,x) = ||q-x|| = [\sum_{i=1}^{d}(q_i - x_i)^2]^\frac{1}{2}$





\url{https://dl.acm.org/doi/pdf/10.1145/3383313.3412488}
\url{http://www.microlinkcolleges.net/elib/files/undergraduate/Photography/504703.pdf}